% !TeX spellcheck = es_ES
\documentclass[oneside,11pt,parskip=full]{scrbook}

% Incluir sólo algunos capítulos/ficheros. Útil para trabajar sólo en lo que interesa en cada momento.
%\includeonly{teoria-conjuntos-difusos,logica-difusa}

% Utilizar UTF-8 como codificación de entrada
\usepackage[utf8]{inputenc}

% Traducción al español de palabras como "Bibiliografía","Figura" etc.
\usepackage[spanish, es-tabla]{babel}

% Fuentes
\usepackage[T1]{fontenc}
\usepackage{palatino}
\addtokomafont{disposition}{\rmfamily}
\addtokomafont{title}{\rmfamily}

% Márgenes
\usepackage[margin=1in]{geometry}

% Paquetes para generar la página de título
\usepackage[pdftex]{graphicx}
\usepackage{epstopdf,epsfig}

% Texto de pruebas
\usepackage{lipsum}

% Código fuente
\usepackage{listings}
\usepackage{amsthm}
\usepackage{amssymb}

% Índice
\usepackage{makeidx}

% Enumeraciones
\usepackage{enumitem}

% Gráficos
\usepackage{tikz}
\usetikzlibrary{shapes,shadows,arrows}
\usepackage{pgfplots}
\pgfplotsset{compat=newest}
\usepackage{amsmath}

% Tablas
\usepackage{longtable}
\usepackage{multirow}
\usepackage{hhline}

%Algoritmos
\usepackage[linesnumbered,ruled]{algorithm2e}
\renewcommand*{\algorithmcfname}{Algoritmo}
\let\oldnl\nl% Store \nl in \oldnl
\newcommand{\nonl}{\renewcommand{\nl}{\let\nl\oldnl}}% Remove line number for one line

% Float: poner las figuras en un punto exacto
\usepackage{float}
\usepackage{subcaption}

\usepackage[bookmarks]{hyperref}

% Código MATLAB
\usepackage[framed,numbered,autolinebreaks]{mcode}
\usepackage{listingsutf8}
% Transliterar acentos
\lstset{
    literate={á}{{\'a}}1 {é}{{\'e}}1 {í}{{\'i}}1 {ó}{{\'o}}1 {ú}{{\'u}}1 
    			  {Á}{{\'A}}1 {É}{{\'E}}1 {Í}{{\'I}}1 {Ó}{{\'O}}1 {Ú}{{\'U}}1
    			  {ü}{{\"{u}}}1 {Ü}{{\"{U}}}1 
    			  {ñ}{{\~n}}1,
}

% Comandos
\newcommand{\HRule}{\rule{\linewidth}{0.5mm}}

% Nombre de sección en cursiva
\renewcommand*{\headfont}{\normalfont\itshape}
% Nº de página en negrita
\renewcommand*{\pnumfont}{\normalfont\bfseries}

% Nombre para el caption de código fuente
\renewcommand{\lstlistingname}{Código}

% Teoremas, definiciones, ejemplos
\newtheorem{theorem}{Teorema}[chapter]
\newtheorem{example}{Ejemplo}[chapter]
\newtheorem{proposition}{Proposición}[chapter]
\newtheorem{definition}{Definición}[chapter]

% Estilos del header y el footer
\usepackage{scrpage2}
\pagestyle{scrheadings}
\ihead{}
\chead{}
\ohead[]{\headmark}
\ifoot[]{}
\cfoot[]{}
\ofoot[\pagemark]{\pagemark}

% Información general del documento
\title{Documento de ejemplo}
\author{Mikel Pintor}
\date{Febrero 2013}
\titlehead{Un draft de ejemplo escrito en LaTeX}
\publishers{Escuela de ingenieros industriales y de telecomunicaciones\\Universidad de Navarra}

\makeindex

\begin{document}

\begin{titlepage}
\begin{center}

% Logos de la universidad y la escuela
\begin{minipage}[t]{0.15\textwidth}
\includegraphics[width=\linewidth]{./images/logo_upna}
\end{minipage}
\hfill
\begin{minipage}[t]{0.15\textwidth}
\includegraphics[width=\linewidth]{./images/logo_escuela}
\end{minipage}\\[2cm]

\textsc{\LARGE ESCUELA TÉCNICA SUPERIOR DE INGENIEROS INDUSTRIALES Y DE TELECOMUNICACIÓN}\\[1.5cm]

\textsc{\Large Ingeniería Informática}\\[0.5cm]

% Título
\HRule \\[0.4cm]
{ \huge \bfseries Aplicación de un método de interpolación, basado en índices de solapamiento, a la detección de riesgos ambientales\\[0.4cm] }

\HRule \\[1.5cm]

% Autor y tutor
\vfill
\hfill
\begin{minipage}{0.4\textwidth}
\begin{flushright} \large
\emph{Autor:} Mikel \textsc{Pintor}\\
\emph{Tutores:} Humberto \textsc{Bustince}\\
Fco. Javier \textsc{Fernández}
{\large Pamplona,23 de julio de 2014}
\end{flushright}
\end{minipage}
\vfill

\end{center}
\end{titlepage}
\frontmatter
\tableofcontents
\listoftables
\listoffigures

\chapter{Introducción}
\label{cha:introduccion}
\clearpage
\lipsum

\mainmatter

% Capítulos
% !TeX spellcheck = es_ES
\chapter{Teoría de conjuntos difusos}
\label{cha:teoria-conjuntos-difusos}
\section{Conjuntos difusos}

\subsection{Definición y propiedades}
En esta sección se presentan algunas definiciones y teoremas que forman la base de toda la teoría de lógica difusa y de los teoremas y métodos presentados en este trabajo.

\begin{definition}
Sea \emph{X} un conjunto clásico de objetos, denominado \emph{universo}, cuyos elementos son denotados como \emph{x}. La pertenencia a un subconjunto clásico \emph{A} de \emph{X} se expresa normalmente por medio de una función de pertenencia $\mu_{A}:X\rightarrow\{0,1\}$ tal que:
\begin{equation}
\mu_{A}(x)=\begin{cases} 1 & \mbox{iff } \mbox{x}\in A, \\ 0 & \mbox{iff } \mbox{x}\notin A \end{cases}
\end{equation}
(se utiliza \emph{iff} como abreviatura de "si y sólo si")
\end{definition}

De esta forma, en un conjunto clásico, se dice que un elemento \emph{x} pertenece al conjunto A ($\mu_{A}(x)=1$) o no pertenece ($\mu_{A}(x)=0$). 
\begin{example}
\normalfont
Un ejemplo de conjunto clásico es el conjunto de todos los números enteros pares, cuya función de pertenencia $\mu_{A}:\mathbb{Z}\rightarrow\{0,1\}$ podría expresarse como:
\begin{equation}
\mu_{A}(x)=\begin{cases} 1 & \mbox{iff } \mbox{x}\bmod 2=0, \\ 0 & \mbox{iff } \mbox{x}\bmod 2\ne 0 \end{cases}
\end{equation}
\end{example}
Si el valor de salida de la función de pertenencia puede ser cualquier número en el intervalo [0,1] (y no sólo los valores discretos \{0,1\}) se dice que el conjunto \emph{A} es \emph{difuso} y que la función $\mu_{A}(x)$ es el \emph{grado de pertenencia} de \emph{x} a \emph{A}. En este tipo de conjuntos, cuanto más próximo sea el valor de $\mu_{A}(x)$ a 1, más pertenece \emph{x} a \emph{A}.

El concepto de conjunto difuso fue introducido por L.A. Zadeh en 1965 \cite{Zadeh65} y se define como:
\begin{definition}
Dado un conjunto de referencia (o universo) \emph{U}, un conjunto difuso \emph{A} sobre \emph{U} es un conjunto tal que:\\
\begin{equation}
\{(u_{i},\mu_{A}(u_{i}))\arrowvert u_{i} \in U\}
\end{equation}
donde \begin{math}\emph{A}:\emph{U}\rightarrow[0,1]\end{math} es la \emph{función de pertenencia}\index{función de pertenencia} (o \emph{grado de pertenencia}) de \emph{A}.
\end{definition}

\begin{example}
\normalfont
Supongamos que el universo \emph{U} está compuesto por los valores razonables de altura (medida en centímetros) de una persona adulta, tal que:

\begin{equation}
\emph{U} = \{150,151,\ldots,229,230\}
\end{equation}

Se puede definir sobre \emph{U} el conjunto difuso \emph{VT}, para representar el concepto de altura \emph{"Muy alta"}, especificando su función de pertenencia \emph{$\mu_{VT}$}, de forma que los valores de altura \emph{$\mu_{i}$} considerados como \emph{"muy altos"} hagan que el valor de dicha función de pertenencia \emph{$\mu_{VT}(\mu_{i})$} sea más próximo a 1 (pertenecen más al conjunto).\\
\\
Se puede elegir cualquier tipo de función para definir al conjunto difuso, siempre que se ajuste a las necesidades del problema que se pretende resolver. En este ejemplo se va a definir la función de pertenencia \emph{$\mu_{VT}$} como una función sigmoidal tal que:
\begin{equation}
\mu_{VT}(\mu_{i}) = \frac{1}{1 + e^{-0.4(\mu_{i}-190)}}
\end{equation}
La función de pertenencia $\mu_{VT}(\mu_{i})$ se representa en la figura \ref{fig:fuzzyset-verytall-example}. Se puede observar que alturas que no podrían considerarse muy altas como 150, 160, 170,etc. hacen que la función de pertenencia  $\mu_{VT}$ sea cero (no pertenencen al conjunto). Algunos valores del conjunto difuso \emph{VT}:
\begin{equation}
VT = \{(150,0),(160,0),(170,0),\ldots,(190,0.5),(195,0.88),(200,1),(210,1),\ldots\}
\end{equation}
\end{example}

\begin{figure}[t]
	\centering
	\newlength\figureheight 
	\newlength\figurewidth
	\setlength\figureheight{4.5cm}
	\setlength\figurewidth{12cm}
	% This file was created by matlab2tikz v0.4.7 (commit 06bbbef116b887fd29db8b1bac523c404c7f7694) running on MATLAB 8.0.
% Copyright (c) 2008--2014, Nico Schlömer <nico.schloemer@gmail.com>
% All rights reserved.
% Minimal pgfplots version: 1.3
% 
\begin{tikzpicture}

\begin{axis}[%
width=\figurewidth,
height=\figureheight,
scale only axis,
xmin=150,
xmax=230,
xlabel={$\mu{}_{\text{i}}\text{ - Altura en cm}$},
ymin=-0.1,
ymax=1.1,
ylabel={$\mu{}_{\text{VT}}\text{(}\mu{}_{\text{i}}\text{)}$},
legend style={at={(0.97,0.03)},anchor=south east,draw=black,fill=white,legend cell align=left}
]
\addplot [color=blue,solid,line width=1.5pt]
  table[row sep=crcr]{150	1.12535162055095e-07\\
150.1	1.17127808604502e-07\\
150.2	1.21907884570381e-07\\
150.3	1.26883039088593e-07\\
150.4	1.32061233461842e-07\\
150.5	1.37450753899427e-07\\
150.6	1.43060224776898e-07\\
150.7	1.4889862243685e-07\\
150.8	1.54975289552946e-07\\
150.9	1.61299950080123e-07\\
151	1.67882724814952e-07\\
151.1	1.74734147591e-07\\
151.2	1.81865182135148e-07\\
151.3	1.89287239611824e-07\\
151.4	1.97012196883211e-07\\
151.5	2.05052415514691e-07\\
151.6	2.13420761555885e-07\\
151.7	2.22130626128981e-07\\
151.8	2.31195946857281e-07\\
151.9	2.40631230168244e-07\\
152	2.50451574506755e-07\\
152.1	2.6067269449571e-07\\
152.2	2.71310946082621e-07\\
152.3	2.8238335271246e-07\\
152.4	2.93907632568622e-07\\
152.5	3.05902226925625e-07\\
152.6	3.18386329658865e-07\\
152.7	3.313799179587e-07\\
152.8	3.44903784297974e-07\\
152.9	3.58979569704136e-07\\
153	3.73629798389248e-07\\
153.1	3.888779137932e-07\\
153.2	4.04748316097899e-07\\
153.3	4.21266401272407e-07\\
153.4	4.38458601711502e-07\\
153.5	4.56352428532765e-07\\
153.6	4.74976515599755e-07\\
153.7	4.94360665341816e-07\\
153.8	5.14535896443798e-07\\
153.9	5.35534493481965e-07\\
154	5.57390058585609e-07\\
154.1	5.80137565206936e-07\\
154.2	6.03813414085313e-07\\
154.3	6.28455491495431e-07\\
154.4	6.54103229872536e-07\\
154.5	6.80797670911849e-07\\
154.6	7.0858153124299e-07\\
154.7	7.37499270784616e-07\\
154.8	7.67597163888589e-07\\
154.9	7.98923373387501e-07\\
155	8.31528027664132e-07\\
155.1	8.65463300866014e-07\\
155.2	9.00783496393575e-07\\
155.3	9.37545133795406e-07\\
155.4	9.75807039209627e-07\\
155.5	1.01563043949625e-06\\
155.6	1.05707906021093e-06\\
155.7	1.10021922757701e-06\\
155.8	1.14511997461899e-06\\
155.9	1.1918531516272e-06\\
156	1.24049354113058e-06\\
156.1	1.29111897756123e-06\\
156.2	1.3438104718025e-06\\
156.3	1.3986523408198e-06\\
156.4	1.45573234258132e-06\\
156.5	1.51514181648505e-06\\
156.6	1.57697582951616e-06\\
156.7	1.64133332836911e-06\\
156.8	1.70831729777757e-06\\
156.9	1.77803492530552e-06\\
157	1.85059777286345e-06\\
157.1	1.92612195522365e-06\\
157.2	2.00472832582054e-06\\
157.3	2.08654267013323e-06\\
157.4	2.17169590695948e-06\\
157.5	2.26032429790357e-06\\
157.6	2.35256966541272e-06\\
157.7	2.44857961971114e-06\\
157.8	2.54850779499489e-06\\
157.9	2.65251409526506e-06\\
158	2.76076495019305e-06\\
158.1	2.87343358142687e-06\\
158.2	2.9907002797647e-06\\
158.3	3.11275269363916e-06\\
158.4	3.23978612937351e-06\\
158.5	3.37200386369078e-06\\
158.6	3.50961746897491e-06\\
158.7	3.65284715180485e-06\\
158.8	3.80192210530298e-06\\
158.9	3.95708087586131e-06\\
159	4.11857174483263e-06\\
159.1	4.28665312579647e-06\\
159.2	4.46159397803589e-06\\
159.3	4.64367423688635e-06\\
159.4	4.8331852616446e-06\\
159.5	5.03043030175512e-06\\
159.6	5.23572498201847e-06\\
159.7	5.44939780759849e-06\\
159.8	5.67179068963591e-06\\
159.9	5.90325949230864e-06\\
160	6.14417460221472e-06\\
160.1	6.39492152098717e-06\\
160.2	6.65590148208973e-06\\
160.3	6.92753209277934e-06\\
160.4	7.21024800226174e-06\\
160.5	7.50450159711e-06\\
160.6	7.81076372505625e-06\\
160.7	8.12952444831543e-06\\
160.8	8.46129382764526e-06\\
160.9	8.80660273839554e-06\\
161	9.16600371985332e-06\\
161.1	9.54007185923986e-06\\
161.2	9.92940571177426e-06\\
161.3	1.03346282582745e-05\\
161.4	1.07563879018261e-05\\
161.5	1.11953595051133e-05\\
161.6	1.16522454700696e-05\\
161.7	1.21277768615743e-05\\
161.8	1.26227145769914e-05\\
161.9	1.31378505634194e-05\\
162	1.36740090845997e-05\\
162.1	1.42320480395057e-05\\
162.2	1.48128603347223e-05\\
162.3	1.54173753128082e-05\\
162.4	1.60465602389226e-05\\
162.5	1.67014218480952e-05\\
162.6	1.73830079556069e-05\\
162.7	1.8092409133059e-05\\
162.8	1.88307604528061e-05\\
162.9	1.95992433035389e-05\\
163	2.03990872799214e-05\\
163.1	2.12315721492952e-05\\
163.2	2.2098029898597e-05\\
163.3	2.29998468647563e-05\\
163.4	2.39384659519744e-05\\
163.5	2.49153889394299e-05\\
163.6	2.593217888309e-05\\
163.7	2.69904626154658e-05\\
163.8	2.80919333473023e-05\\
163.9	2.92383533753526e-05\\
164	3.04315569005653e-05\\
164.1	3.16734529611747e-05\\
164.2	3.296602848538e-05\\
164.3	3.43113514684834e-05\\
164.4	3.57115742795526e-05\\
164.5	3.71689371028894e-05\\
164.6	3.86857715197865e-05\\
164.7	4.02645042362891e-05\\
164.8	4.19076609629042e-05\\
164.9	4.36178704524425e-05\\
165	4.53978687024344e-05\\
165.1	4.725050332881e-05\\
165.2	4.91787381178212e-05\\
165.3	5.11856577634542e-05\\
165.4	5.32744727978788e-05\\
165.5	5.54485247227949e-05\\
165.6	5.77112913498369e-05\\
165.7	6.00663923585472e-05\\
165.8	6.25175950807636e-05\\
165.9	6.5068820520623e-05\\
166	6.77241496197701e-05\\
166.1	7.04878297777251e-05\\
166.2	7.33642816377876e-05\\
166.3	7.6358106149268e-05\\
166.4	7.94740919172621e-05\\
166.5	8.27172228516664e-05\\
166.6	8.60926861275657e-05\\
166.7	8.96058804696507e-05\\
166.8	9.32624247738134e-05\\
166.9	9.70681670795998e-05\\
167	0.000101029193907773\\
167.1	0.000105151839977773\\
167.2	0.000109442698320505\\
167.3	0.000113908630802462\\
167.4	0.000118556779077877\\
167.5	0.000123394575986232\\
167.6	0.00012842975741318\\
167.7	0.000133670374633634\\
167.8	0.000139124807156559\\
167.9	0.00014480177609175\\
168	0.000150710358059757\\
168.1	0.000156859999666871\\
168.2	0.000163260532568056\\
168.3	0.000169922189141575\\
168.4	0.000176855618800013\\
168.5	0.000184071904963424\\
168.6	0.000191582582721282\\
168.7	0.00019939965721107\\
168.8	0.000207535622742382\\
168.9	0.000216003482696582\\
169	0.000224816770233295\\
169.1	0.000233989569836176\\
169.2	0.000243536539731751\\
169.3	0.000253472935216452\\
169.4	0.000263814632928313\\
169.5	0.000274578156101332\\
169.6	0.000285780700841879\\
169.7	0.000297440163468182\\
169.8	0.000309575168955517\\
169.9	0.000322205100531344\\
170	0.000335350130466478\\
170.1	0.00034903125211006\\
170.2	0.000363270313218076\\
170.3	0.00037809005062704\\
170.4	0.000393514126326484\\
170.5	0.00040956716498605\\
170.6	0.000426274792995\\
170.7	0.000443663679074348\\
170.8	0.000461761576524077\\
170.9	0.000480597367170262\\
171	0.000500201107079564\\
171.1	0.000520604074110941\\
171.2	0.000541838817377267\\
171.3	0.00056393920869224\\
171.4	0.000586940496080804\\
171.5	0.000610879359434401\\
171.6	0.000635793968395226\\
171.7	0.000661724042557041\\
171.8	0.000688710914073259\\
171.9	0.000716797592766405\\
172	0.000746028833836697\\
172.1	0.000776451208270841\\
172.2	0.000808113176056168\\
172.3	0.000841065162308884\\
172.4	0.000875359636429239\\
172.5	0.000911051194400645\\
172.6	0.000948196644353723\\
172.7	0.000986855095520903\\
172.8	0.00102708805071153\\
172.9	0.00106895950244197\\
173	0.00111253603286032\\
173.1	0.00115788691760961\\
173.2	0.00120508423377894\\
173.3	0.00125420297209689\\
173.4	0.00130532115352677\\
173.5	0.00135851995042896\\
173.6	0.00141388381246065\\
173.7	0.00147150059738952\\
173.8	0.00153146170700329\\
173.9	0.00159386222830301\\
174	0.00165880108017442\\
174.1	0.00172638116573715\\
174.2	0.00179670953057844\\
174.3	0.00186989752708406\\
174.4	0.00194606098508556\\
174.5	0.00202532038904988\\
174.6	0.00210780106204325\\
174.7	0.00219363335670874\\
174.8	0.00228295285350307\\
174.9	0.00237590056644499\\
175	0.00247262315663477\\
175.1	0.00257327315381036\\
175.2	0.0026780091862131\\
175.3	0.00278699621904229\\
175.4	0.00290040580178421\\
175.5	0.00301841632470842\\
175.6	0.00314121328482942\\
175.7	0.0032689895616386\\
175.8	0.00340194570291728\\
175.9	0.00354029022094651\\
176	0.00368423989943599\\
176.1	0.00383402011149744\\
176.2	0.00398986514899408\\
176.3	0.00415201856360066\\
176.4	0.00432073351991193\\
176.5	0.00449627316094118\\
176.6	0.00467891098635069\\
176.7	0.00486893124375862\\
176.8	0.00506662933346628\\
176.9	0.00527231222694868\\
177	0.0054862988994504\\
177.1	0.00570892077702333\\
177.2	0.00594052219834034\\
177.3	0.00618146089161117\\
177.4	0.00643210846691865\\
177.5	0.00669285092428486\\
177.6	0.00696408917776207\\
177.7	0.00724623959583139\\
177.8	0.00753973455837343\\
177.9	0.00784502303045565\\
178	0.00816257115315989\\
178.1	0.00849286285164433\\
178.2	0.00883640046060767\\
178.3	0.00919370536728814\\
178.4	0.00956531867209169\\
178.5	0.00995180186690432\\
178.6	0.0103537375310917\\
178.7	0.0107717300451416\\
178.8	0.0112064063218429\\
178.9	0.0116584165548331\\
179	0.0121284349842742\\
179.1	0.0126171606793387\\
179.2	0.0131253183371027\\
179.3	0.0136536590973509\\
179.4	0.0142029613726912\\
179.5	0.0147740316932731\\
179.6	0.0153677055652755\\
179.7	0.0159848483422025\\
179.8	0.0166263561078817\\
179.9	0.0172931565699055\\
180	0.0179862099620916\\
180.1	0.0187065099543546\\
180.2	0.0194550845681929\\
180.3	0.0202329970957855\\
180.4	0.0210413470204683\\
180.5	0.0218812709361305\\
180.6	0.0227539434628039\\
180.7	0.0236605781554611\\
180.8	0.0246024284027395\\
180.9	0.0255807883120077\\
181	0.0265969935768659\\
181.1	0.0276524223228231\\
181.2	0.0287484959265398\\
181.3	0.0298866798036364\\
181.4	0.0310684841596675\\
181.5	0.0322954646984505\\
181.6	0.0335692232814824\\
181.7	0.0348914085317367\\
181.8	0.0362637163746485\\
181.9	0.037687890508606\\
182	0.0391657227967644\\
182.1	0.0406990535714664\\
182.2	0.0422897718420336\\
182.3	0.0439398153961415\\
182.4	0.0456511707844438\\
182.5	0.0474258731775668\\
182.6	0.0492660060840265\\
182.7	0.0511737009170914\\
182.8	0.0531511363980639\\
182.9	0.0552005377829344\\
183	0.0573241758988687\\
183.1	0.0595243659765014\\
183.2	0.0618034662635883\\
183.3	0.0641638764051742\\
183.4	0.0666080355750908\\
183.5	0.0691384203433468\\
183.6	0.0717575422637512\\
183.7	0.0744679451660277\\
183.8	0.0772722021366602\\
183.9	0.0801729121728125\\
184	0.0831726964939224\\
184.1	0.0862741944959165\\
184.2	0.0894800593335611\\
184.3	0.0927929531171574\\
184.4	0.096215541710693\\
184.5	0.0997504891196851\\
184.6	0.103400451458249\\
184.7	0.107168070486528\\
184.8	0.111055966711408\\
184.9	0.11506673204555\\
185	0.119202922022118\\
185.1	0.123467047565224\\
185.2	0.127861566319081\\
185.3	0.132388873542066\\
185.4	0.13705129257546\\
185.5	0.141851064900488\\
185.6	0.146790339801382\\
185.7	0.151871163656659\\
185.8	0.157095468885453\\
185.9	0.162465062580696\\
186	0.167981614866076\\
186.1	0.173646647019005\\
186.2	0.179461519407326\\
186.3	0.185427419292983\\
186.4	0.191545348561468\\
186.5	0.197816111441418\\
186.6	0.204240302284091\\
186.7	0.210818293477746\\
186.8	0.217550223576888\\
186.9	0.224435985730927\\
187	0.231475216500982\\
187.1	0.238667285157089\\
187.2	0.246011283551051\\
187.3	0.253506016662339\\
187.4	0.261149993915751\\
187.5	0.268941421369995\\
187.6	0.27687819487561\\
187.7	0.284957894299009\\
187.8	0.293177778906433\\
187.9	0.301534783997462\\
188	0.310025518872388\\
188.1	0.318646266210974\\
188.2	0.327392982932239\\
188.3	0.336261302595648\\
188.4	0.345246539393681\\
188.5	0.354343693774205\\
188.6	0.363547459718433\\
188.7	0.372852233686803\\
188.8	0.382252125230752\\
188.9	0.391740969253486\\
189	0.401312339887548\\
189.1	0.410959565941334\\
189.2	0.420675747851249\\
189.3	0.430453776060772\\
189.4	0.440286350732808\\
189.5	0.450166002687522\\
189.6	0.460085115444434\\
189.7	0.470035948235427\\
189.8	0.480010659844419\\
189.9	0.490001333120035\\
190	0.5\\
190.1	0.509998666879965\\
190.2	0.519989340155581\\
190.3	0.529964051764573\\
190.4	0.539914884555566\\
190.5	0.549833997312478\\
190.6	0.559713649267192\\
190.7	0.569546223939228\\
190.8	0.579324252148751\\
190.9	0.589040434058666\\
191	0.598687660112452\\
191.1	0.608259030746514\\
191.2	0.617747874769248\\
191.3	0.627147766313197\\
191.4	0.636452540281567\\
191.5	0.645656306225795\\
191.6	0.654753460606319\\
191.7	0.663738697404352\\
191.8	0.672607017067761\\
191.9	0.681353733789026\\
192	0.689974481127613\\
192.1	0.698465216002538\\
192.2	0.706822221093567\\
192.3	0.715042105700991\\
192.4	0.72312180512439\\
192.5	0.731058578630005\\
192.6	0.738850006084249\\
192.7	0.746493983337661\\
192.8	0.753988716448949\\
192.9	0.761332714842911\\
193	0.768524783499018\\
193.1	0.775564014269073\\
193.2	0.782449776423112\\
193.3	0.789181706522254\\
193.4	0.795759697715909\\
193.5	0.802183888558582\\
193.6	0.808454651438532\\
193.7	0.814572580707017\\
193.8	0.820538480592674\\
193.9	0.826353352980995\\
194	0.832018385133924\\
194.1	0.837534937419304\\
194.2	0.842904531114547\\
194.3	0.848128836343341\\
194.4	0.853209660198618\\
194.5	0.858148935099512\\
194.6	0.86294870742454\\
194.7	0.867611126457934\\
194.8	0.872138433680919\\
194.9	0.876532952434776\\
195	0.880797077977882\\
195.1	0.88493326795445\\
195.2	0.888944033288592\\
195.3	0.892831929513472\\
195.4	0.896599548541751\\
195.5	0.900249510880315\\
195.6	0.903784458289307\\
195.7	0.907207046882843\\
195.8	0.910519940666439\\
195.9	0.913725805504084\\
196	0.916827303506078\\
196.1	0.919827087827187\\
196.2	0.92272779786334\\
196.3	0.925532054833972\\
196.4	0.928242457736249\\
196.5	0.930861579656653\\
196.6	0.933391964424909\\
196.7	0.935836123594826\\
196.8	0.938196533736412\\
196.9	0.940475634023499\\
197	0.942675824101131\\
197.1	0.944799462217066\\
197.2	0.946848863601936\\
197.3	0.948826299082909\\
197.4	0.950733993915973\\
197.5	0.952574126822433\\
197.6	0.954348829215556\\
197.7	0.956060184603859\\
197.8	0.957710228157966\\
197.9	0.959300946428534\\
198	0.960834277203236\\
198.1	0.962312109491394\\
198.2	0.963736283625351\\
198.3	0.965108591468263\\
198.4	0.966430776718518\\
198.5	0.96770453530155\\
198.6	0.968931515840333\\
198.7	0.970113320196364\\
198.8	0.97125150407346\\
198.9	0.972347577677177\\
199	0.973403006423134\\
199.1	0.974419211687992\\
199.2	0.97539757159726\\
199.3	0.976339421844539\\
199.4	0.977246056537196\\
199.5	0.978118729063869\\
199.6	0.978958652979532\\
199.7	0.979767002904215\\
199.8	0.980544915431807\\
199.9	0.981293490045645\\
200	0.982013790037908\\
200.1	0.982706843430095\\
200.2	0.983373643892118\\
200.3	0.984015151657797\\
200.4	0.984632294434724\\
200.5	0.985225968306727\\
200.6	0.985797038627309\\
200.7	0.986346340902649\\
200.8	0.986874681662897\\
200.9	0.987382839320661\\
201	0.987871565015726\\
201.1	0.988341583445167\\
201.2	0.988793593678157\\
201.3	0.989228269954858\\
201.4	0.989646262468908\\
201.5	0.990048198133096\\
201.6	0.990434681327908\\
201.7	0.990806294632712\\
201.8	0.991163599539392\\
201.9	0.991507137148356\\
202	0.99183742884684\\
202.1	0.992154976969544\\
202.2	0.992460265441627\\
202.3	0.992753760404169\\
202.4	0.993035910822238\\
202.5	0.993307149075715\\
202.6	0.993567891533081\\
202.7	0.993818539108389\\
202.8	0.99405947780166\\
202.9	0.994291079222977\\
203	0.994513701100549\\
203.1	0.994727687773051\\
203.2	0.994933370666534\\
203.3	0.995131068756241\\
203.4	0.995321089013649\\
203.5	0.995503726839059\\
203.6	0.995679266480088\\
203.7	0.995847981436399\\
203.8	0.996010134851006\\
203.9	0.996165979888503\\
204	0.996315760100564\\
204.1	0.996459709779054\\
204.2	0.996598054297083\\
204.3	0.996731010438361\\
204.4	0.996858786715171\\
204.5	0.996981583675292\\
204.6	0.997099594198216\\
204.7	0.997213003780958\\
204.8	0.997321990813787\\
204.9	0.99742672684619\\
205	0.997527376843365\\
205.1	0.997624099433555\\
205.2	0.997717047146497\\
205.3	0.997806366643291\\
205.4	0.997892198937957\\
205.5	0.99797467961095\\
205.6	0.998053939014914\\
205.7	0.998130102472916\\
205.8	0.998203290469422\\
205.9	0.998273618834263\\
206	0.998341198919826\\
206.1	0.998406137771697\\
206.2	0.998468538292997\\
206.3	0.99852849940261\\
206.4	0.998586116187539\\
206.5	0.998641480049571\\
206.6	0.998694678846473\\
206.7	0.998745797027903\\
206.8	0.998794915766221\\
206.9	0.99884211308239\\
207	0.99888746396714\\
207.1	0.998931040497558\\
207.2	0.998972911949289\\
207.3	0.999013144904479\\
207.4	0.999051803355646\\
207.5	0.999088948805599\\
207.6	0.999124640363571\\
207.7	0.999158934837691\\
207.8	0.999191886823944\\
207.9	0.999223548791729\\
208	0.999253971166163\\
208.1	0.999283202407234\\
208.2	0.999311289085927\\
208.3	0.999338275957443\\
208.4	0.999364206031605\\
208.5	0.999389120640566\\
208.6	0.999413059503919\\
208.7	0.999436060791308\\
208.8	0.999458161182623\\
208.9	0.999479395925889\\
209	0.999499798892921\\
209.1	0.99951940263283\\
209.2	0.999538238423476\\
209.3	0.999556336320926\\
209.4	0.999573725207005\\
209.5	0.999590432835014\\
209.6	0.999606485873673\\
209.7	0.999621909949373\\
209.8	0.999636729686782\\
209.9	0.99965096874789\\
210	0.999664649869534\\
210.1	0.999677794899469\\
210.2	0.999690424831044\\
210.3	0.999702559836532\\
210.4	0.999714219299158\\
210.5	0.999725421843899\\
210.6	0.999736185367072\\
210.7	0.999746527064784\\
210.8	0.999756463460268\\
210.9	0.999766010430164\\
211	0.999775183229767\\
211.1	0.999783996517303\\
211.2	0.999792464377258\\
211.3	0.999800600342789\\
211.4	0.999808417417279\\
211.5	0.999815928095037\\
211.6	0.9998231443812\\
211.7	0.999830077810858\\
211.8	0.999836739467432\\
211.9	0.999843140000333\\
212	0.99984928964194\\
212.1	0.999855198223908\\
212.2	0.999860875192843\\
212.3	0.999866329625366\\
212.4	0.999871570242587\\
212.5	0.999876605424014\\
212.6	0.999881443220922\\
212.7	0.999886091369197\\
212.8	0.99989055730168\\
212.9	0.999894848160022\\
213	0.999898970806092\\
213.1	0.99990293183292\\
213.2	0.999906737575226\\
213.3	0.99991039411953\\
213.4	0.999913907313873\\
213.5	0.999917282777148\\
213.6	0.999920525908083\\
213.7	0.999923641893851\\
213.8	0.999926635718362\\
213.9	0.999929512170222\\
214	0.99993227585038\\
214.1	0.999934931179479\\
214.2	0.999937482404919\\
214.3	0.999939933607641\\
214.4	0.99994228870865\\
214.5	0.999944551475277\\
214.6	0.999946725527202\\
214.7	0.999948814342237\\
214.8	0.999950821261882\\
214.9	0.999952749496671\\
215	0.999954602131298\\
215.1	0.999956382129548\\
215.2	0.999958092339037\\
215.3	0.999959735495764\\
215.4	0.99996131422848\\
215.5	0.999962831062897\\
215.6	0.99996428842572\\
215.7	0.999965688648532\\
215.8	0.999967033971515\\
215.9	0.999968326547039\\
216	0.999969568443099\\
216.1	0.999970761646625\\
216.2	0.999971908066653\\
216.3	0.999973009537385\\
216.4	0.999974067821117\\
216.5	0.999975084611061\\
216.6	0.999976061534048\\
216.7	0.999977000153135\\
216.8	0.999977901970102\\
216.9	0.999978768427851\\
217	0.99997960091272\\
217.1	0.999980400756696\\
217.2	0.999981169239547\\
217.3	0.999981907590867\\
217.4	0.999982616992044\\
217.5	0.999983298578152\\
217.6	0.999983953439761\\
217.7	0.999984582624687\\
217.8	0.999985187139665\\
217.9	0.99998576795196\\
218	0.999986325990915\\
218.1	0.999986862149437\\
218.2	0.999987377285423\\
218.3	0.999987872223139\\
218.4	0.99998834775453\\
218.5	0.999988804640495\\
218.6	0.999989243612098\\
218.7	0.999989665371742\\
218.8	0.999990070594288\\
218.9	0.999990459928141\\
219	0.99999083399628\\
219.1	0.999991193397262\\
219.2	0.999991538706172\\
219.3	0.999991870475552\\
219.4	0.999992189236275\\
219.5	0.999992495498403\\
219.6	0.999992789751998\\
219.7	0.999993072467907\\
219.8	0.999993344098518\\
219.9	0.999993605078479\\
220	0.999993855825398\\
220.1	0.999994096740508\\
220.2	0.99999432820931\\
220.3	0.999994550602192\\
220.4	0.999994764275018\\
220.5	0.999994969569698\\
220.6	0.999995166814738\\
220.7	0.999995356325763\\
220.8	0.999995538406022\\
220.9	0.999995713346874\\
221	0.999995881428255\\
221.1	0.999996042919124\\
221.2	0.999996198077895\\
221.3	0.999996347152848\\
221.4	0.999996490382531\\
221.5	0.999996627996136\\
221.6	0.999996760213871\\
221.7	0.999996887247306\\
221.8	0.99999700929972\\
221.9	0.999997126566419\\
222	0.99999723923505\\
222.1	0.999997347485905\\
222.2	0.999997451492205\\
222.3	0.99999755142038\\
222.4	0.999997647430335\\
222.5	0.999997739675702\\
222.6	0.999997828304093\\
222.7	0.99999791345733\\
222.8	0.999997995271674\\
222.9	0.999998073878045\\
223	0.999998149402227\\
223.1	0.999998221965075\\
223.2	0.999998291682702\\
223.3	0.999998358666672\\
223.4	0.999998423024171\\
223.5	0.999998484858183\\
223.6	0.999998544267658\\
223.7	0.999998601347659\\
223.8	0.999998656189528\\
223.9	0.999998708881022\\
224	0.999998759506459\\
224.1	0.999998808146848\\
224.2	0.999998854880025\\
224.3	0.999998899780772\\
224.4	0.99999894292094\\
224.5	0.999998984369561\\
224.6	0.999999024192961\\
224.7	0.999999062454866\\
224.8	0.999999099216504\\
224.9	0.999999134536699\\
225	0.999999168471972\\
225.1	0.999999201076627\\
225.2	0.999999232402836\\
225.3	0.999999262500729\\
225.4	0.999999291418469\\
225.5	0.999999319202329\\
225.6	0.99999934589677\\
225.7	0.999999371544508\\
225.8	0.999999396186586\\
225.9	0.999999419862435\\
226	0.999999442609941\\
226.1	0.999999464465506\\
226.2	0.999999485464104\\
226.3	0.999999505639335\\
226.4	0.999999525023484\\
226.5	0.999999543647571\\
226.6	0.999999561541398\\
226.7	0.999999578733599\\
226.8	0.999999595251684\\
226.9	0.999999611122086\\
227	0.999999626370202\\
227.1	0.99999964102043\\
227.2	0.999999655096216\\
227.3	0.999999668620082\\
227.4	0.99999968161367\\
227.5	0.999999694097773\\
227.6	0.999999706092367\\
227.7	0.999999717616647\\
227.8	0.999999728689054\\
227.9	0.999999739327305\\
228	0.999999749548426\\
228.1	0.99999975936877\\
228.2	0.999999768804053\\
228.3	0.999999777869374\\
228.4	0.999999786579238\\
228.5	0.999999794947585\\
228.6	0.999999802987803\\
228.7	0.99999981071276\\
228.8	0.999999818134818\\
228.9	0.999999825265853\\
229	0.999999832117275\\
229.1	0.99999983870005\\
229.2	0.999999845024711\\
229.3	0.999999851101378\\
229.4	0.999999856939775\\
229.5	0.999999862549246\\
229.6	0.999999867938767\\
229.7	0.999999873116961\\
229.8	0.999999878092116\\
229.9	0.999999882872191\\
230	0.999999887464838\\
};
\addlegendentry{$\mu{}_{\text{VT}}\text{(}\mu{}_{\text{i}}\text{)}$};

\addplot [color=green,line width=1.5pt,mark size=4.0pt,only marks,mark=asterisk,mark options={solid},forget plot]
  table[row sep=crcr]{195	0.88\\
};
\node[right, inner sep=0mm, text=black]
at (axis cs:195,0.88,0) {$\text{  }\leftarrow\text{ (}\mu{}_{\text{i}}\text{=195; }\mu{}_{\text{VT}}\text{(195)=0.8808)}$};
\end{axis}
\end{tikzpicture}%
	\caption{Función de pertenencia del conjunto difuso \emph{"Muy Alto"} ($\mu_{VT}(\mu_{i})$).}
	\label{fig:fuzzyset-verytall-example}
\end{figure}

\subsection{Variables lingüísticas}
Una variable lingüística se puede definir de manera informal como una variable que puede tomar palabras del lenguaje natural como valores. Para formular palabras y utilizarlas como valores de variables lingüísticas se utilizan conjuntos difusos (por medio de sus correspondientes funciones de pertenencia). Así pues, una variable lingüística se puede definir como \cite{wang1997}:

\begin{definition}\label{def:informal-lang-variable}
Una \emph{variable lingüística} es un tipo de variable que puede tomar palabras como valores. Las palabras son caracterizadas por conjuntos difusos definidos en el mismo universo en el que la variable es definida.
\end{definition}

\begin{example}\label{ex:lang-variable}
\normalfont
Supongamos que la temperatura medida por un termómetro en grados celsius (ºC) es una variable $x$ que puede tomar valores enteros en el intervalo $[0,100]$. Este intervalo de valores posibles es lo que se define como el \emph{universo de discurso} o \emph{universo de referencia} $U$. Podemos definir sobre $U$ los conjuntos difusos \emph{``baja''},\emph{``media''} y \emph{``alta''} para modelar los conceptos de temperatura alta, media y baja respectivamente. Si consideramos que $x$ es una variable lingüística entonces $x$ puede tomar los valores \emph{``baja''},\emph{``media''} y \emph{``alta''}. Es decir, podemos decir que: \emph{``x es baja''},\emph{``x es media''} o \emph{``x es alta''}, o lo que es lo mismo \emph{``Temperatura es baja''},\emph{``Temperatura es media''} o \emph{``Temperatura es alta''}. En la figura \ref{fig:example-temp-lang-variable} se representan las funciones de pertenencia para los conjuntos difusos \emph{``Temperatura baja''},\emph{``Temperatura media''} y \emph{``Temperatura alta''}.
\end{example}

\begin{figure}[t]
	\centering
	\setlength\figureheight{4.5cm}
	\setlength\figurewidth{12cm}
	\input{figures/example-temp-lang-variable.tikz}
	\caption{Ejemplo \ref{ex:lang-variable}: Funciones de pertenencia para los valores \emph{``baja''} ($\mu_{Baja}$),\emph{``media''} ($\mu_{Media}$) y \emph{``alta''} ($\mu_{Alta}$) de la variable lingüística \emph{Temperatura}.}
	\label{fig:example-temp-lang-variable}
\end{figure}

Una definición más formal de \emph{variable lingüística} (equivalente a la definición \ref{def:informal-lang-variable}), dada por Zadeh \cite{Zadeh75}, es la siguiente:

\begin{definition}\label{def:formal-lang-variable}
\normalfont
Una \emph{variable lingüística} se caracteriza por la tupla $(X,T,U,M)$, donde:
\begin{itemize}
  \item $X$ es el nombre de la variable lingüística. En el ejemplo \ref{ex:lang-variable},``Temperatura en ºC''.
  \item $T$ es el conjunto de valores lingüísticos que $X$ puede tomar. En el ejemplo \ref{ex:lang-variable}, $T$ = \{baja, media, alta\}
  \item $U$ es el universo de referencia del que cada valor lingüístico toma sus valores cuantitativos (escalares). En el ejemplo \ref{ex:lang-variable}, $U = [0,100]$
  \item $M$ es la regla semántica que relaciona cada valor lingüístico en $T$ con un conjunto difuso en $U$. En el ejemplo \ref{ex:lang-variable}, $M$ relaciona los valores ``baja'', ``media'' y ``alta'' con las funciones de pertenencia representadas en la figura \ref{fig:example-temp-lang-variable}.
\end{itemize}
\end{definition}
\subsection{Operaciones sobre conjuntos difusos}
En esta sección se definen algunas operaciones básicas sobre conjuntos difusos. Un conjunto difuso compuesto a partir de dos conjuntos difusos sobre el mismo universo \emph{U} puede definirse por su función de pertenencia tal que:
\begin{definition}
Dada una función \begin{math}\emph{F}:[0,1]^2\rightarrow[0,1]\end{math} y dos conjuntos difusos \emph{A} y \emph{B} definidos sobre el mismo universo \emph{U}, \begin{math}A,B\in FS(U)\end{math}, denotamos como \begin{math}F(A,B)\end{math} el conjunto difuso sobre \emph{U} cuya función de pertenencia viene dada por:
\begin{equation}
\mu_{F(A,B)}(u_{i}) = F(A(u_{i}),B(u_{i}))
\end{equation}
\end{definition}
De esta forma, se pueden definir las operaciones de unión ($\cup$) y de intersección ($\cap$) clásicas sobre los subconjuntos difusos \emph{A} y \emph{B} sobre \emph{U} \cite{dubois1980} tal que:
\begin{equation}
\forall\mu_{i}\in U,\quad\mu_{A\cup B}(u_{i}) = max(\mu_{A}(u_{i}),\mu_{B}(u_{i}))
\end{equation}
\begin{equation}
\forall\mu_{i}\in U,\quad\mu_{A\cap B}(u_{i}) = min(\mu_{A}(u_{i}),\mu_{B}(u_{i}))
\end{equation}
donde $\mu_{A\cup B}$ y $\mu_{A\cap B}$ son las funciones de pertenencia de $A\cup B$ (unión) y $A\cap B$ (intersección) respectivamente.
\begin{definition}
El \emph{complementario} \emph{$\overline{A}$} de un conjunto difuso \emph{A} sobre \emph{U} puede definirse por su función de pertenencia \cite{Zadeh65}, tal que:
\begin{equation}
\forall \mu_{i} \in U, \quad \mu_{\overline{A}}(\mu_{i}) = 1 - \mu_{A}(\mu_{i})
\end{equation}
\end{definition}


\section{T-normas y operadores de agregación}\label{sec:t-norms-aggregation-operator}
Dada una función \begin{math}\emph{F}:[0,1]^2\rightarrow[0,1]\end{math} y dos conjuntos difusos \emph{A} y \emph{B} definidos sobre el mismo universo \emph{U}, \begin{math}A,B\in FS(U)\end{math}, denotamos como \begin{math}F(A,B)\end{math} el conjunto difuso sobre \emph{U} cuya función de pertenencia viene dada por:
\begin{equation}
\mu_{F(A,B)}(u_{i}) = F(A(u_{i}),B(u_{i}))
\end{equation}
Una clase importante de este tipo de funciones son las llamadas t-normas y t-conormas (normas triangulares) \cite{klement2000triangular}.
\begin{definition}\label{def:tnorma}
Una t-norma es una operación binaria \emph{T} en el intervalo $[0,1]$ que es conmutativa, asociativa, monótona y tiene el valor \emph{1} como elemento neutro. Es decir, una función $\emph{T} : [0,1]^2 \rightarrow [0,1]$ tal que $\forall x,y,z \in [0,1]$:
\begin{enumerate}[label=(T\arabic*),ref=(T\arabic*)]
   \item $T(x,y) = T(y,x)$ (Conmutatividad)\label{T1}
   \item $T(x,T(y,z)) = T(T(x,y),z)$ (Asociatividad)\label{T2}
   \item $T(x,y) \leq T(x,z)$ cuando $y \leq z$ (Monotonía)\label{T3}
   \item $T(x,1) = x$ (Elemento neutro)\label{T4}
  \end{enumerate}
\end{definition}
Estas propiedades son suficientes para garantizar que las t-normas generalizan la conjunción clásica ($x\bigwedge y$) cuando se aplican a valores booleanos ($T(0,0)=0$,$T(0,1)=T(1,0)=0$ y $T(1,1)=1$).
Algunos ejemplos prominentes de t-normas son los siguientes:
\begin{itemize}
	\item \bfseries Mínimo o t-norma de Gödel: $T_{G}(x,y) = min\{x,y\}$
	\item \bfseries Producto: $T_{P}(x,y) = xy$
	\item \bfseries \L{}ukasiewicz: $T_{L}(x,y) = max\{x+y-1,0\}$
	\item \bfseries T-norma drástica: \mdseries $T_{D}(x,y) = \begin{cases} y & if\quad x=1, \\ x & if\quad y=1, \\0 & \text{en cualquier otro caso}. \end{cases}$
\end{itemize}
\begin{figure}[H]
	\centering
	\begin{subfigure}[b]{0.45\textwidth}
		\setlength\figureheight{4.5cm}
		\setlength\figurewidth{6cm}
		\input{figures/t-norms/min-3dplot.tikz}
		\caption{Mínimo}
		\label{fig:t-norms-min}
	\end{subfigure}
	\qquad
	\begin{subfigure}[b]{0.45\textwidth}
		\setlength\figureheight{4.5cm}
		\setlength\figurewidth{6cm}
		\input{figures/t-norms/prod-3dplot.tikz}
		\caption{Producto}
		\label{fig:t-norms-prod}
	\end{subfigure}
	
	\vspace{1 cm}
	\begin{subfigure}[b]{0.45\textwidth}
		\setlength\figureheight{4.5cm}
		\setlength\figurewidth{6cm}
		\input{figures/t-norms/luka-3dplot.tikz}
		\caption{\L{}ukasiewicz}
		\label{fig:t-norms-lukasiewicz}
	\end{subfigure}
	\qquad
	\begin{subfigure}[b]{0.45\textwidth}
			\setlength\figureheight{4.5cm}
			\setlength\figurewidth{6cm}
			\input{figures/t-norms/drastic-tnorm-3dplot.tikz}
			\caption{T-norma drástica}
			\label{fig:t-norms-drastic}
	\end{subfigure}
	\label{fig:t-norms}
	\caption{Representación gráfica de las t-normas \emph{Mínimo} (\ref{fig:t-norms-min}), \emph{Producto} (\ref{fig:t-norms-prod}), \emph{\L{}ukasiewicz} (\ref{fig:t-norms-lukasiewicz}) y la \emph{t-norma drástica} (\ref{fig:t-norms-drastic}).}
\end{figure}

\begin{definition}
Una t-conorma es una operación binaria \emph{S} en el intervalo $[0,1]$ que es conmutativa, asociativa, monotona y tiene el valor \emph{0} como elemento neutro. Es decir, una función $\emph{S} : [0,1]^2 \rightarrow [0,1]$ tal que $\forall x,y,z \in [0,1]$:
\begin{enumerate}[label=(S\arabic*),ref=(S\arabic*)]
   \item $S(x,y) = S(y,x)$ (Conmutatividad)\label{S1}
   \item $S(x,S(y,z)) = S(S(x,y),z)$ (Asociatividad)\label{S2}
   \item $S(x,y) \leq S(x,z)$ cuando $y \leq z$ (Monotonía)\label{S3}
   \item $S(x,0) = x$ (Elemento neutro)\label{S4}
  \end{enumerate}
\end{definition}
De la misma forma que las t-normas (definición \ref{def:tnorma}) generalizan la conjunción clásica, las t-conormas generalizan la disyunción ($x\bigvee y$). A continuación se incluyen algunos ejemplos de t-conormas:
\begin{itemize}
	\item \bfseries Máximo: $S_{max}(x,y) = max\{x,y\}$
	\item \bfseries Suma probabilística: $S_{P}(x,y) = x + y - xy$
	\item \bfseries Suma acotada: $S_{B}(x,y) = min\{x+y,1\}$
	\item \bfseries T-conorma drástica: \mdseries $S_{D}(x,y) = \begin{cases} y & if\quad x=0, \\ x & if\quad y=0, \\1 & \text{en cualquier otro caso}. \end{cases}$
\end{itemize}
\begin{figure}[H]
	\centering
	\begin{subfigure}[b]{0.45\textwidth}
		\setlength\figureheight{4.5cm}
		\setlength\figurewidth{6cm}
		\input{figures/t-conorms/max-3dplot.tikz}
		\caption{Máximo}
		\label{fig:t-conorms-max}
	\end{subfigure}
	\qquad
	\begin{subfigure}[b]{0.45\textwidth}
		\setlength\figureheight{4.5cm}
		\setlength\figurewidth{6cm}
		% This file was created by matlab2tikz v0.4.7 (commit 4d7ae5c4fd0932fb51051d86111bc23ed23e4580) running on MATLAB 8.0.
% Copyright (c) 2008--2014, Nico Schlömer <nico.schloemer@gmail.com>
% All rights reserved.
% Minimal pgfplots version: 1.3
% 
\begin{tikzpicture}

\begin{axis}[%
width=\figurewidth,
height=\figureheight,
view={-37.5}{30},
scale only axis,
xmin=0,
xmax=1,
xlabel={x},
xmajorgrids,
ymin=0,
ymax=1,
ylabel={y},
ymajorgrids,
zmin=0,
zmax=1,
zlabel={$\text{S}_{\text{P}}\text{(x,y)}$},
zmajorgrids,
axis x line*=bottom,
axis y line*=left,
axis z line*=left
]

\addplot3[%
surf,
shader=faceted,
draw=black,
colormap/jet,
mesh/rows=11]
table[row sep=crcr,header=false] {0	0	0\\
0	0.1	0.1\\
0	0.2	0.2\\
0	0.3	0.3\\
0	0.4	0.4\\
0	0.5	0.5\\
0	0.6	0.6\\
0	0.7	0.7\\
0	0.8	0.8\\
0	0.9	0.9\\
0	1	1\\
0.1	0	0.1\\
0.1	0.1	0.19\\
0.1	0.2	0.28\\
0.1	0.3	0.37\\
0.1	0.4	0.46\\
0.1	0.5	0.55\\
0.1	0.6	0.64\\
0.1	0.7	0.73\\
0.1	0.8	0.82\\
0.1	0.9	0.91\\
0.1	1	1\\
0.2	0	0.2\\
0.2	0.1	0.28\\
0.2	0.2	0.36\\
0.2	0.3	0.44\\
0.2	0.4	0.52\\
0.2	0.5	0.6\\
0.2	0.6	0.68\\
0.2	0.7	0.76\\
0.2	0.8	0.84\\
0.2	0.9	0.92\\
0.2	1	1\\
0.3	0	0.3\\
0.3	0.1	0.37\\
0.3	0.2	0.44\\
0.3	0.3	0.51\\
0.3	0.4	0.58\\
0.3	0.5	0.65\\
0.3	0.6	0.72\\
0.3	0.7	0.79\\
0.3	0.8	0.86\\
0.3	0.9	0.93\\
0.3	1	1\\
0.4	0	0.4\\
0.4	0.1	0.46\\
0.4	0.2	0.52\\
0.4	0.3	0.58\\
0.4	0.4	0.64\\
0.4	0.5	0.7\\
0.4	0.6	0.76\\
0.4	0.7	0.82\\
0.4	0.8	0.88\\
0.4	0.9	0.94\\
0.4	1	1\\
0.5	0	0.5\\
0.5	0.1	0.55\\
0.5	0.2	0.6\\
0.5	0.3	0.65\\
0.5	0.4	0.7\\
0.5	0.5	0.75\\
0.5	0.6	0.8\\
0.5	0.7	0.85\\
0.5	0.8	0.9\\
0.5	0.9	0.95\\
0.5	1	1\\
0.6	0	0.6\\
0.6	0.1	0.64\\
0.6	0.2	0.68\\
0.6	0.3	0.72\\
0.6	0.4	0.76\\
0.6	0.5	0.8\\
0.6	0.6	0.84\\
0.6	0.7	0.88\\
0.6	0.8	0.92\\
0.6	0.9	0.96\\
0.6	1	1\\
0.7	0	0.7\\
0.7	0.1	0.73\\
0.7	0.2	0.76\\
0.7	0.3	0.79\\
0.7	0.4	0.82\\
0.7	0.5	0.85\\
0.7	0.6	0.88\\
0.7	0.7	0.91\\
0.7	0.8	0.94\\
0.7	0.9	0.97\\
0.7	1	1\\
0.8	0	0.8\\
0.8	0.1	0.82\\
0.8	0.2	0.84\\
0.8	0.3	0.86\\
0.8	0.4	0.88\\
0.8	0.5	0.9\\
0.8	0.6	0.92\\
0.8	0.7	0.94\\
0.8	0.8	0.96\\
0.8	0.9	0.98\\
0.8	1	1\\
0.9	0	0.9\\
0.9	0.1	0.91\\
0.9	0.2	0.92\\
0.9	0.3	0.93\\
0.9	0.4	0.94\\
0.9	0.5	0.95\\
0.9	0.6	0.96\\
0.9	0.7	0.97\\
0.9	0.8	0.98\\
0.9	0.9	0.99\\
0.9	1	1\\
1	0	1\\
1	0.1	1\\
1	0.2	1\\
1	0.3	1\\
1	0.4	1\\
1	0.5	1\\
1	0.6	1\\
1	0.7	1\\
1	0.8	1\\
1	0.9	1\\
1	1	1\\
};
\end{axis}
\end{tikzpicture}%
		\caption{Suma probabilística}
		\label{fig:t-conorms-probabilistic-sum}
	\end{subfigure}
	
	\vspace{1 cm}
	\begin{subfigure}[b]{0.45\textwidth}
		\setlength\figureheight{4.5cm}
		\setlength\figurewidth{6cm}
		\input{figures/t-conorms/bounded-sum-3dplot.tikz}
		\caption{Suma acotada}
		\label{fig:t-conorms-bounded-sum}
	\end{subfigure}
	\qquad
	\begin{subfigure}[b]{0.45\textwidth}
			\setlength\figureheight{4.5cm}
			\setlength\figurewidth{6cm}
			\input{figures/t-conorms/drastic-tconorm-3dplot.tikz}
			\caption{T-conorma drástica}
			\label{fig:t-conorms-drastic}
	\end{subfigure}
	\label{fig:t-conorms}
	\caption{Representación gráfica de las t-conormas \emph{Máximo} (\ref{fig:t-conorms-max}), \emph{Suma probabilística} (\ref{fig:t-conorms-probabilistic-sum}), \emph{Suma acotada} (\ref{fig:t-conorms-bounded-sum})  y la \emph{t-conorma drástica} (\ref{fig:t-conorms-drastic}).}
\end{figure}
Otra clase importante de funciones son los \emph{operadores de agregación} \index{operador de agregación}\cite{calvo2002aggregation, beliakov2007aggregation}.
\begin{definition}
Una función $\emph{M} : [a,b]^{n} \rightarrow [a,b]$ es un operador de agregación si es monótona y no decreciente en cada una de sus componentes y además cumple que $M(a, a, \cdots,a) = a$ y $M(b, b, \cdots,b) = b$.
\end{definition}
\begin{definition}
Una función de agregación \emph{M} se dice que es una media si cumple que:
\begin{equation}\label{def:mean-aggregation-operator}
min(x) = min(x_{1},\cdots,x_{n})\leq \emph{M}(x_{1},\cdots,x_{n}) \leq max(x_{1},\cdots,x_{n}) = max(x)
\end{equation}
\end{definition}
A continuación se incluyen algunos de los operadores de agregación más comunes:
\begin{itemize}
	\item \bfseries Media aritmética: $M(x_{1},x_{2},\cdots,x_{n}) = \frac{1}{n}\sum\limits_{i=1}^{n}x_{i}$
	\item \bfseries Media geométrica: $M(x_{1},x_{2},\cdots,x_{n}) = (\prod\limits_{i=1}^{n}x_{i})^{\frac{1}{n}}$
	\item \bfseries Media ponderada: $M_{w_{1},w_{2},\cdots,w_{n}}(x_{1},x_{2},\cdots,x_{n}) =\sum\limits_{i=1}^{n}(w_{i}\cdot x_{i})$ \normalfont tal que $0 \leq w_{i} \leq 1$ y $\sum\limits_{i=1}^{n}w_{i} = 1$.
	\item \bfseries Mediana: \normalfont Se toma el elemento central del conjunto ordenado de argumentos.
	\item \bfseries Máximo: $M(x_{1},x_{2},\cdots,x_{n}) = max(x_{1},x_{2},\cdots,x_{n})$
	\item \bfseries Mínimo: $M(x_{1},x_{2},\cdots,x_{n}) = min(x_{1},x_{2},\cdots,x_{n})$
\end{itemize}

\section{Funciones de solapamiento}\label{sec:overlap-functions}
En esta sección se presenta el concepto de \emph{función de solapamiento}\index{función de solapamiento} \cite{bustince2009overlap,bustince2010overlapfunctions,bustince2012pairwisecomparisons,jurio2013propertiesoverlap}, que tendrá una gran importancia en el desarrollo de este trabajo, ya que a partir de estas funciones se construyen los índices de solapamiento, que serán utilizados en el método de interpolación presentado.
\subsection{Definición de función de solapamiento y teorema de construcción.}
\begin{definition}
Una función de solapamiento $G_{O} : [0,1]^{2} \rightarrow [0,1]$ cumple:
\begin{enumerate}[label=(G\arabic*),ref=(G\arabic*)]
   \item $G_{O}(x,y) = G_{O}(y,x) \;\; \forall \; x,y \in [0,1]$ \label{G1}
   \item $G_{O}(x,y) = 0$ si y sólo si $x \cdot y = 0$ \label{G2}
   \item $G_{O}(x,y) = 1$ si y sólo si $x \cdot y = 1$ \label{G3}
   \item $G_{O}$ es creciente \label{G4}
   \item $G_{O}$ es continua \label{G5}
\end{enumerate}
\end{definition}
Las funciones de solapamiento generalizan los operadores de intersección tales como el mínimo o, en general, las t-normas (definidas en \ref{def:tnorma}). Además, las funciones de solapamiento son casos particulares de los operadores de agregación sin divisores por cero o divisores por uno. Algunos ejemplos de funciones de solapamiento son:\\
\\
$G_{O}(x,y) = min(x,y)$\\
$G_{O}(x,y) = \sqrt{xy}$\\
$G_{O}(x,y) = min(x^{k}y,xy^{k}), k \in ]0,1[$\\
\\
En \cite{bustince2009overlap} se presenta un teorema que proporciona tanto una caracterización como un método de construcción para funciones de solapamiento:
\begin{theorem}\label{th:overlap-function-construction}
Una función $G_{O} : [0,1]^{2} \rightarrow [0,1]$ es una función de solapamiento si y sólo si puede ser expresada como:
\begin{equation}
G_{O}(x,y) = \frac{f(x,y)}{f(x,y) + h(x,y)}
\end{equation}
para cualesquiera $f,h: [0,1]^{2} \rightarrow [0,1]$ tales que:
\begin{enumerate}[label=(\arabic*),ref=(\arabic*)]
   \item f y h son simétricas;
   \item f es no decreciente y h es no creciente;
   \item $f(x,y) = 0$ si y sólo si $min(x.y) = 0$;
   \item $h(x,y) = 0$ si y sólo si $min(x.y) = 1$;
   \item f y h son continuas
\end{enumerate}
\end{theorem}
\begin{example}
A continuación se presentan algunos ejemplos de construcción de funciones de solapamiento utilizando el teorema  \ref{th:overlap-function-construction}:
\begin{enumerate}[label=(\arabic*),ref=(\arabic*)]
	\item Si $f(x,y)=min(x,y)$ y $h(x,y) = max(1-x,1-y)$ entonces: 
	\begin{equation}
	G_{O}(x,y) = min(x,y)
	\end{equation} es una función de solapamiento.
	\item Si $f(x,y)=\sqrt{x \times y}$ y $h(x,y) = max(1-x,1-y)$ entonces: 
	\begin{equation}G_{O}(x,y) = \frac{\sqrt{x \times y}}{\sqrt{x \times y} + max(1-x,1-y)}
	\end{equation} es una función de solapamiento.
	\item Si $f(x,y)=\sqrt{x \times y}$ y $h(x,y) = 1 - x \times y $ entonces: 
	\begin{equation}G_{O}(x,y) = \frac{\sqrt{x \times y}}{\sqrt{x \times y} + 1 - x \times y}
	\end{equation} es una función de solapamiento.
\end{enumerate}
\end{example}
\begin{figure}[t]
	\centering
	\begin{subfigure}[b]{0.45\textwidth}
		\setlength\figureheight{4.5cm}
		\setlength\figurewidth{6cm}
		\input{figures/overlap-functions/overlap-min-3dplot.tikz}
		\caption{$G_{O}(x,y) = min(x,y)$}
		\label{fig:overlap-functions-min}
	\end{subfigure}
	\qquad
	\begin{subfigure}[b]{0.45\textwidth}
		\setlength\figureheight{4.5cm}
		\setlength\figurewidth{6cm}
		\input{figures/overlap-functions/overlap-sqrt-max-3dplot.tikz}
		\caption{$G_{O}(x,y) = \frac{\sqrt{x \times y}}{\sqrt{x \times y} + max(1-x,1-y)}$}
		\label{fig:overlap-functions-sqrt-max}
	\end{subfigure}
	
	\vspace{1 cm}
	\begin{subfigure}[b]{0.45\textwidth}
		\setlength\figureheight{4.5cm}
		\setlength\figurewidth{6cm}
		\input{figures/overlap-functions/overlap-sqrt-2-3dplot.tikz}
		\caption{$G_{O}(x,y) = \frac{\sqrt{x \times y}}{\sqrt{x \times y} + 1 - x \times y}$}
		\label{fig:overlap-functions-sqrt-2}
	\end{subfigure}
	\caption{Representación gráfica de algunas funciones de solapamiento.}
	\label{fig:overlap-functions-examples}
\end{figure}
En la figura \ref{fig:overlap-functions-examples} se pueden ver las representaciones gráficas de las funciones de solapamiento del ejemplo anterior.

\subsection{Caso particular: t-normas.}
Por la definición \ref{def:tnorma} sabemos que una t-norma es una función conmutativa, asociativa, y creciente $T:[0,1]^2 \rightarrow [0,1]$ tal que $T(x,1) = x$ para todo $x \in [0,1]$. Bajo ciertas condiciones, las t-normas también cumplen la definición de función de solapamiento \cite{bustince2009overlap}. 
\begin{theorem}
Si una t-norma \emph{T} es una función de solapamiento, entonces \emph{T} es uno de los siguientes 3 tipos:
\begin{enumerate}[label=(\arabic*),ref=(\arabic*)]
	\item $T = \min$
	\item T es estricta (continua y estrictamente monótona)
	\item T es la suma ordinal de la familia $\{([a_{m},b_{m}],T_{m})\}$, siendo $T_{m}$ todas las t-normas continuas y arquimedeanas tales que si para algún $m_{0}$ tenemos que $a_{m_{0}} = 0$ entonces $T_{m_{0}}$ es necesariamente una t-norma estricta.
\end{enumerate}
\end{theorem}
\begin{proposition}
Sea $G_{O}$ una función de solapamiento tal que $G_{O}(x,G_{O}(y,z)) = G_{O}(y,G_{O}(x,z))$, entonces \emph{T} es una t-norma \cite{bustince2013overlap}.
\end{proposition}
\section{Índices de solapamiento}\label{sec:overlap-indexes}
\subsection{Definición y propiedades}
En esta sección se presenta la definición de índice de solapamiento y se estudian algunas de sus propiedades más importantes. Los índices de solapamiento tienen una importancia vital en el desarrollo de este trabajo, ya que son la base del método estudiado. 

El concepto de solapamiento aplicado a conjuntos difusos fue introducido por Zadeh en 1978 \cite{zadeh1978}.

\begin{definition}
Sea $A,B \in FS(U)$. La consistencia entre A y B se define como:
\begin{equation}\label{eq:zadeh-consistency}
O_{Z}(A,B) = sup_{i=1}^{n}(min(A(u_{i}),B(u_{i})))
\end{equation}
\end{definition}

Para conjuntos de referencia finitos, el supremo es en realidad el máximo elemento del conjunto.

En 1982 Dubois y Prade \cite{dubois2000} presentan la siguiente definición para el índice de solapamiento:

\begin{definition}
Un índice de solapamiento es una función $O : FS(U) \times FS(U) \rightarrow [0,1]$ tal que:
\begin{enumerate}[label=(O\arabic*),ref=(O\arabic*)]
\item $O(A,B) = 0$ si y sólo si A y B son completamente disjuntos. \label{DP01}
\item $O(A,B) = 1$, si ($A(u_{i}) = 0$ o $B(u_{i}) = 0$) o ($A(u_{i}) = 1$ o $B(u_{i}) = 1$) \label{DPO2}
\item $O(A,B) = O(B,A)$ \label{DPO3}
\item Si $B \leq C$, entonces $O(A,B) \leq O(A,C)$ \label{DPO4}
\end{enumerate}
\label{def:dubois-overlap-index}
\end{definition}
La condición \ref{DPO2} en esta definición presenta la ventaja de que, si \emph{A} no es difuso, entonces $O(A,A) = 1$.

Por esta razón Dubois, Ostasiewicz y Prade imponen en \cite{dubois2000} las siguientes condiciones:
\begin{enumerate}[label=(\arabic*),ref=(\arabic*)]
\item Para todos los conjuntos difusos tales que $A(u_{i}) \le 1$ y $A(u_{i}) \le 1$ para cualquier $u_{i} \in U$, \ref{DPO2} debe ignorarse.
\item El índice ROC (\cite{dubois2000}) no satisface \ref{DPO2}.
\end{enumerate}
Debido a estas consideraciones, normalmente sólo se imponen las condiciones \ref{DP01}, \ref{DPO3}, \ref{DPO4} de la definición \ref{def:dubois-overlap-index} a los índices de solapamiento.
En \cite{bustince2013overlap} se propone la siguiente definición de índice de solapamiento:
\begin{definition}\label{def:bustince-overlap-index}
Un índice de solapamiento es una función $O : FS(U) \times FS(U) \rightarrow [0,1]$ tal que:
\begin{enumerate}[label=(O\arabic*),ref=(O\arabic*)]
\item $O(A,B) = 0$ si y sólo si A y B tienen soportes disjuntos, es decir, $A(u_{i})B(u_{i}) = 0$ para todo $u_{i} \in U$\label{BO1}
\item $O(A,B) = O(B,A)$\label{BO2}
\item Si $B \leq C$, entonces $O(A,B) \leq O(A,C)$\label{BO3}
\end{enumerate}
Un índice de solapamiento normal es un índice \emph{O} tal que:
\begin{enumerate}[label=(O4),ref=(O4)]
\item Si existe un $u_{i} \in U$ tal que $A(u_{i}) = B(u_{i}) = 1$, entonces $O(A,B) = 1$\label{BO4}
\end{enumerate}
\end{definition}
\subsection{Construcción de índices de solapamiento}\label{sec:overlap-index-construction}
En esta sección se presenta el método de construcción de índices de solapamiento propuesto por \cite{bustince2013overlap} que se basa en operadores de agregación (sección \ref{sec:t-norms-aggregation-operator}) y funciones de solapamiento (sección \ref{sec:overlap-functions}).
\begin{theorem}
Sea $M : [0,1]^{2} \rightarrow [0,1]$ una función de agregación tal que $M(x_{1},\cdots,x_{n}) = 0$ si y sólo si $x_{1} = \cdots = x_{n} = 0$. Sea $G_{O} : [0,1]^{2} \rightarrow [0,1]$ una función de solapamiento. Entonces, la función $O: F(U) \times F(U) \rightarrow [0,1]$ definida como:
\begin{equation}\label{eq:construction-overlap-index}
O(A,B) = M(G_{O}(A(u_{1}),B(u_{1})),\cdots,G_{O}(A(u_{n}),B(u_{n})))
\end{equation}
es un índice de solapamiento en el sentido de la definición \ref{def:bustince-overlap-index}. Recíprocamente, si $G_{O}$ es una función de solapamiento y $M:[0,1]^{n} \rightarrow [0,1]$ es un operador de agregación tal que O definido por la ecuación \ref{eq:construction-overlap-index} es un índice de solapamiento, entonces $M(x_{1},\cdots,x_{n}) = 0$ si y sólo si $x_{1} = \cdots = x_{n} = 0$.
\end{theorem}


% !TeX spellcheck = es_ES
\chapter{Lógica difusa}
\label{cha:logica-difusa}
En esta sección se introducen las bases de la lógica difusa y los mecanismos de inferencia difusa. Además se describen algunos de los métodos clásicos de inferencia difusa basada en reglas.

\section{¿Qué es la lógica difusa?}
La lógica difusa fue introducida por Lofti A. Zadeh en 1965. Desde entonces se ha aplicado en multitud de escenarios tales como control de procesos industriales, medicina, electrónica y otros tipos de sistemas expertos. En general, la mayoría de aplicaciones de la lógica difusa son el área del control.

La motivación para el desarrollo de una lógica difusa fue expresada por Zadeh (1984, \cite{Zadeh1984}) de la siguiente manera:
\begin{quote}
\emph{``La habilidad de la mente humana para razonar en términos difusos es realmente una gran ventaja. Aunque una gran cantidad de información es captada por los sentidos en una situación determinada, de alguna manera la mente humana es capaz de descartar la mayoría de esa información y concentrarse sólo en aquello que es relevante.''}
\end{quote}
El objetivo principal de la lógica difusa es intentar dotar a las máquinas de un sistema de razonamiento aproximado similar al de los humanos, que pueda lidiar con imprecisiones y términos inexactos. Para ello, la lógica difusa se basa en la utilización de variables lingüísticas expresadas en lenguaje natural, que forman la base del razonamiento aproximado. 

Por medio del uso de variables lingüísticas se pueden modelar conceptos como \emph{``muy alto''} o \emph{``bastante caliente''} y establecer relaciones entre ellos, expresadas de forma matemática y algorítmica. Los sistemas de lógica difusa tratan la imprecisión de las entradas mediante el uso de variables lingüísticas (expresadas como conjuntos difusos), que transforman dichos valores de entrada en números difusos.

\section{Razonamiento aproximado}
El razonamiento aproximado utiliza conjuntos difusos y lógica difusa para modelar la forma de discurrir y razonar del ser humano. El razonamiento aproximado no posee la precisión y exactitud de la lógica clásica, pero es más efectivo a la hora de lidiar con imprecisiones, conceptos vagos o sistemas complejos y/o pobremente definidos.

En este proyecto se va a implementar una forma de razonamiento aproximado que utiliza un mecanismo de inferencia difusa basado en el modus ponens generalizado, que es una extensión del modus ponens clásico.

\subsection{Modus ponens clásico}
El modus ponens clásico es una proposición compuesta muy conocida en la lógica clásica. Tiene la forma:

\begin{equation}
((p \wedge (p \rightarrow q)) \rightarrow q)
\end{equation}

Para inferir el valor de verdad de \emph{q} a partir del de \emph{p} utilizando el modus ponens clásico, se utiliza una \emph{regla} de inferencia, expresada de forma simbólica utilizando el \emph{silogismo}:

\begin{equation}
\begin{array}{rl}
\text{Premisa 1:} & p \\
\text{Premisa 2:} & p \to q \\
\cline{2-2}
\text{Conclusión:} & q
\end{array}
\label{eq:inference-rule}
\end{equation}

que puede expresarse como: si la proposición \emph{p} es verdadera (premisa 1) y la proposición $p \to q$ es verdadera (premisa 2), entonces \emph{q} es verdadera (conclusión). Esto permite obtener el valor de verdad de \emph{q} a partir del de \emph{p} utilizando una regla de inferencia. Es decir, por medio de esta regla de inferencia sabemos que, si \emph{p} es verdadero, entonces \emph{q} es también verdadero.

La regla de inferencia \ref{eq:inference-rule} puede expresarse de manera más detallada como:

\begin{equation}
\begin{array}{rl}
\text{Premisa 1:} & x \text{ es } A \\
\text{Premisa 2:} & \text{IF } x \text{ es } A \text{ THEN } y \text{ es } B \\
\cline{2-2}
\text{Conclusión:} & y \text{ es } B
\end{array}
\label{eq:inference-rule-detail}
\end{equation}

donde $p = x \text{ es } A$ y $q = y \text{ es } B$.

\begin{example}
Un ejemplo de la regla de inferencia \ref{eq:inference-rule-detail} es la regla de divisibilidad por 3: \emph{``Ùn número entero es divisible por 3 si la suma de sus dígitos es múltiplo de 3''}. Puede expresarse mediante el siguiente silogismo:

\begin{equation}
\normalfont
\begin{array}{rl}
\text{Premisa 1:} & \text{\emph{La suma de los dígitos de un número es múltiplo de 3}} \\
\text{Premisa 2:} & \text{IF \emph{La suma de los dígitos de un número es múltiplo de 3} THEN \emph{El número es divisible por 3}} \\
\cline{2-2}
\text{Conclusión:} & \text{\emph{El número es divisible por 3}}
\end{array}
\label{eq:inference-rule-example}
\end{equation}

En este ejemplo se utilizan conceptos exactos (lógica clásica). Un número pertenece o no pertence al conjunto de números cuya suma de cifras es múltiplo de 3. Por lo tanto la proposición \emph{``La suma de los dígitos de un número es múltiplo de 3''} sólo puede tomar los valores \emph{Verdadero} (1) o \emph{Falso} (0). 
\end{example}

\subsection{Modus ponens generalizado}

El \emph{modus ponens generalizado} es, como su nombre índica, una generalización del modus ponens clásico (definido en la sección anterior) utilizado en la lógica difusa. Tiene la forma:

\begin{equation}
\begin{array}{rl}
\text{Premisa 1:} & p' \\
\text{Premisa 2:} & p \to q \\
\cline{2-2}
\text{Conclusión:} & q'
\end{array}
\label{eq:fuzzy-inference-rule}
\end{equation}

o lo que es lo mismo:

\begin{equation}
\begin{array}{rl}
\text{Premisa 1:} & x \text{ es } A' \\
\text{Premisa 2:} & \text{IF } x \text{ es } A \text{ THEN } y \text{ es } B \\
\cline{2-2}
\text{Conclusión:} & y \text{ es } B'
\end{array}
\label{eq:fuzzy-inference-rule-detail}
\end{equation}

En este caso las proposiciones $p'=x \text{ es } A'$, $p=x \text{ es } A$, $q=y \text{ es } B$, $q'=y \text{ es } B'$ vienen caracterizadas por los conjuntos difusos $A'$,$A$,$B$ y $B'$,que representan conceptos difusos. Esta regla de inferencia puede interpretarse de la siguiente manera: si $p \to q$ y tenemos $p'$ (aproximadamente $p$) entonces tenemos $q'$ (aproximadamente q). El modus ponens generalizado permite, por tanto, utilizar conceptos difusos y obtener el ``grado de verdad'' de la conclusión a partir de las premisas.

En la siguiente sección se define el concepto de \emph{regla difusa} (p.e. $\text{IF } x \text{ es } A \text{ THEN } y \text{ es } B$), que constituye una de las premisas en el modus ponens generalizado. En las secciones \ref{sec:mamdani-controller} y \ref{sec:interpolation-method} se definen los métodos de inferencia difusa que se van a estudiar en este proyecto. Estos métodos utilizan reglas difusas y el modus ponens generalizado para obtener las salidas ($B'$) a partir de las premisas ($A'$).

\section{Reglas difusas}
Una \emph{regla difusa IF-THEN}\footnote{Podría traducirse como SI-ENTONCES.} puede definirse de manera informal como:

\begin{definition}
Una regla difusa IF-THEN es una estructura condicional de la forma:
\begin{equation}
\text{IF [antecedentes] THEN [consecuentes]}
\end{equation},
donde [antecedentes] y [consecuentes] son proposiciones difusas.
\end{definition}

Las proposiciones difusas pueden ser \emph{simples} o \emph{compuestas}. Una proposición difusa simple es un predicado único formado por una variable lingüística \emph{x} y un valor \emph{A} de dicha variable (un conjunto difuso). Es decir, \emph{A} es un conjunto difuso (valor lingüístico) definido sobre el mismo universo de referencia \emph{U} de \emph{x}. Por ejemplo si la variable \emph{x} es la \emph{``Temperatura medida en ºC''} del ejemplo \ref{ex:lang-variable}, entonces las siguientes son proposiciones difusas sobre \emph{x}:

\begin{equation}
\text{$x$ es B}
\end{equation}
\begin{equation}
\text{$x$ es M}
\end{equation}
\begin{equation}
\text{$x$ es A}
\end{equation}

donde \emph{B}, \emph{M} y \emph{A} son los valores lingüísticos ``baja'', ``media'' y ``alta'' respectivamente. Estos valores lingüísticos vienen definidos por conjuntos difusos con funciones de pertenencia $\mu_{Baja}$, $\mu_{Media}$ y $\mu_{Alta}$.

Las proposiciones difusas simples se pueden combinar para formar predicados más complejos. Para ello se utilizan normalmente los conectivos \emph{AND} (intersección difusa) y \emph{OR} (unión difusa). Por ejemplo, se puede definir la condición \emph{``Temperatura es Alta Ó Temperatura es Media''} con la siguiente proposición difusa:

\begin{equation}
\text{$x$ es $A$ OR $x$ es $M$}
\end{equation}

donde $A$ y $M$ son los valores lingüísticos \emph{``alta''} ($\mu_{Alta}$) y \emph{``media''} ($\mu_{Media}$) respectivamente. En la misma proposición difusa se pueden utilizar variables diferentes (generalmente es así) definidas normalmente sobre universos de referencia diferentes. El consecuente generalmente suele ser una proposición difusa simple formado por una variable \emph{y} sobre \emph{V}. Por ello, se puede definir el concepto de \emph{regla difusa IF-THEN} de manera más formal como:

\begin{definition}
Una \emph{regla difusa IF-THEN} es una estructura condicional de la forma:

\begin{equation}
\text{IF $x_{1}$ es $A_{1}$ $\odot$ $x_{2}$ es $A_{2}$ $\odot$ \ldots $\odot$ $x_{n}$ es $A_{n}$ THEN $y$ es $B$}
\end{equation}
donde $x_{1} \in U_{i}$,\ldots, $x_{n} \in U_{n}$,$y \in V$ son variables lingüísticas y $\odot$ algún conectivo difuso (AND,OR,AND NOT,etc.).
\end{definition}

\section{Sistemas difusos basados en reglas}\label{sec:sistemas-difusos-basados-en-reglas}
Un enfoque muy utilizado a la hora de modelar un sistema de lógica difusa es la utilización de \emph{reglas} expresadas en un lenguaje muy próximo al natural, mediante variables lingüísticas. Estas reglas habitualmente son formuladas por un experto humano aunque pueden ser derivadas también de datos numéricos. A este tipo de sistemas difusos se les denomina \emph{sistemas difusos basados en reglas}. Los métodos estudiados en este trabajo pertenecen a este tipo de sistemas.

Existen principalmente 3 tipos de sistemas difusos \cite{wang1997}: sistemas difusos puros, sistemas difusos tipo Takagi-Sugeno-Kang (TKG) y sistemas difusos con fusificador y defusificador. A continuación se realiza una breve descripción de las características de estos sistemas.

En la figura \ref{fig:pure-fuzzy-system} se representa el esquema básico de un sistema difuso puro. En los sistemas difusos puros tanto las entradas como la salida del sistema son conjuntos difusos. El sistema difuso está dotado de un \emph{conjunto de reglas difusas} \emph{IF-THEN}, que forman la base de conocimiento sobre el dominio del problema. El \emph{sistema de inferencia difusa} combina estas reglas \emph{IF-THEN} y transforma los conjuntos difusos de entrada sobre el universo $U \subset R^{n}$ ($x_{1}.\ldots,x_{n}$) en un conjunto difuso sobre $V \subset R$ ($y$), utilizando principios de lógica difusa (modus ponens).

\begin{figure}[tb]
	\centering
	\tikzstyle{block} = [draw, fill=blue!20, rectangle, minimum height=3em, minimum width=6em]
\tikzstyle{input} = [coordinate]
\tikzstyle{output} = [coordinate]

\tikzset{
    state/.style={
    	   fill=white,
           rectangle,
           rounded corners,
           draw=black, very thick,
           minimum height=4em,
           minimum width=15em,
           inner sep=2pt,
           text centered,
           drop shadow
           },
}

\tikzstyle{materia}=[draw, fill=blue!20, text width=6.0em, text centered,
  minimum height=1.5em,drop shadow]


\shorthandoff{>}
\begin{tikzpicture}[auto, node distance=2cm,>=latex']
\node [input, name=input] {};
\node [state, right of=input,node distance=7cm,align=left] (inference-engine) {Sistema de inferencia difusa};
\node [state, above of=inference-engine] (rule-base) {Conjunto de reglas difusas};
\node [output, right of=inference-engine,node distance=7cm] (output) {};

\draw [->] (input) -- node {$x \in FS(U)$} (inference-engine);
\draw [->] (rule-base) -- node {} (inference-engine);
\draw [->] (inference-engine) -- node {$y \in FS(V)$} (output);
\end{tikzpicture}
	\caption{Diagrama de un sistema difuso puro.}
	\label{fig:pure-fuzzy-system}
\end{figure}

El principal problema de los sistemas difusos puros es que tanto las entradas como las salidas son conjuntos difusos, mientras que en aplicaciones de control industrial se utilizan generalmente valores escalares reales. Para solucionar este problema, Takagi, Sugeno y Kang propusieron un nuevo método simplificado que utiliza entradas y salidas escalares \cite{takagisugeno1985}\cite{sugenokang1988}.

El método de Takagi-Sugeno-Kang utiliza reglas de la forma:

\begin{equation}
R_{i}: \text{IF }x_{1}\text{ es }A_{1i}\text{, }x_{2}\text{ es }A_{2i}\text{ , \ldots , }x_{n}\text{ es }A_{ni}\text{ THEN } y = f_{i}(x_{1},x_{2},\ldots,x_{n})
\end{equation}

donde $A_{ij}$ representa las funciones de pertenencia asociadas a las variables lingüísticas utilizadas en los antecedentes y $f_{i}(x_{1},x_{2},\ldots,x_{n})$ representa los consecuentes. Normalmente $f_{i}$ es un polinomio en las variables de entrada, pero puede ser cualquier tipo de función mientras pueda describir la salida del modelo de forma apropiada, teniendo en cuenta las entradas. 

Como se puede ver, la principal diferencia entre un sistema difuso puro y el sistema de Takagi-Sugeno-Kang es que el consecuente de la regla (\emph{THEN}) es sustituido por una simple fórmula matemática. Este cambio supone que el método para combinar las diferentes reglas para proporcionar una salida es más sencillo. De hecho, en el método de Takagi-Sugeno-Kang se utiliza la media ponderada de los consecuentes obtenidos de las diferentes reglas.

El principal problema del método de Takagi-Sugeno-Kang es que el consecuente de las reglas es una función matemática y por lo tanto se pierde cierta capacidad de modelar el sistema utilizando el lenguaje natural. Para solucionar estos problemas, se puede modificar el sistema difuso puro, añadiendo un fusificador a la entrada de las variables y un defusificador a la salida.

\begin{figure}[tb]
	\centering
	\input{figures/fuzzy-system.tikz}
	\caption{Diagrama de un sistema difuso con fusificador y defusificador.}
	\label{fig:fuzzy-system}
\end{figure}

En la figura \ref{fig:fuzzy-system} se representa el esquema básico de un sistema difuso con fusificador y defusificador. El fusificador transforma las entradas escalares en $U$ del sistema en conjuntos difusos. Una vez realizada la inferencia difusa se obtiene un conjunto difuso como salida. El defusificador transforma el conjunto difuso de salida en una variable escalar en $V$. Existen multitud de métodos para realizar estas transformaciones y en siguientes secciones se detallarán algunos de ellos. El sistema difuso presentado en este trabajo se basa en este último tipo de sistema.

\subsection{Fusificadores}\label{sec:fusificadores}

En la sección \ref{sec:sistemas-difusos-basados-en-reglas} se ha definido un tipo de sistema difuso que utiliza un concepto llamado \emph{fusificador} para transformar las entradas escalares del sistema en conjuntos difusos. Un \emph{fusificador} puede definirse formalmente como:

\begin{definition}\label{def:fusificador}
Un \emph{fusificador} es una función $F:\mathbb{R} \rightarrow FS(U)$ que toma un valor escalar real $x^{*}$ en el universo $U$, $x^{*}\in U \subset \mathbb{R}$, y lo transforma en un conjunto difuso $A'$ sobre $U$.
\end{definition}

Para que el conjunto de salida represente de manera lo más fiel posible al valor de entrada, es necesario imponer algunas restricciones a la hora de construir el fusificador. La principal es que la función de pertenencia de $A'$ debe tener un valor alto en el punto $x^*$ (generalmente máximo, $\mu_{A'}(x^*) = \max(\mu_{A'}(\mu_{i})),\forall \mu_{i} \in U$). Además, si se considera que existe ruido o imprecisión en los valores entrada (por ejemplo imprecisiones provocadas por instrumentos de medida) se puede utilizar una función de pertenencia cuyo valor sea alto en los puntos próximos a $x^*$ y bajo (o incluso cero) en los más alejados.

Algunos de los fusificadores más utilizados son los siguientes \cite{wang1997}:

\begin{itemize}
\item\bfseries Fusificador singleton: \normalfont El \emph{fusificador singleton} (fig. \ref{fig:fuzzifier-singleton}) transforma un valor escalar real $x^* \in U$ en un conjunto difuso $A'$ en $U$, de forma que la función de pertenencia de $A'$ tiene valor 1 en el punto $x^*$ y 0 en el resto de puntos de $U$:
\begin{equation}
\mu_{A'}(x) = \begin{cases} 1 & \mbox{si } \mbox{x}=x^*, \\ 0 & \mbox{en cualquier otro caso} \end{cases}
\end{equation}
Este tipo de fusificador tiene la ventaja de ser simple y facilitar los cálculos realizados en el motor de inferencia. Sin embargo, tiene la desventaja de que no puede lidiar con ruido o imprecisiones en los valores de entrada. 
\item\bfseries Fusificador gaussiano: \normalfont El \emph{fusificador gaussiano} (fig. \ref{fig:fuzzifier-gaussian}) transforma un valor escalar real $x^* \in U$ en un conjunto difuso $A'$ en $U$, utilizando una función de pertenencia Gaussiana:
\begin{equation}
\mu_{A'}(x) = e^{-(\frac{x_1 - x_1^*}{a_1})^2}\star\ldots\star e^{-(\frac{x_n - x_n^*}{a_n})^2}
\end{equation}
donde $a_i$ son parámetros positivos y $\star$ es una t-norma, generalmente el producto algebraico o el mínimo. Este tipo de fusificador es adecuado cuando existe ruido o imprecisiones en los datos de entrada.
\item\bfseries Fusificador triangular: \normalfont El \emph{fusificador triangular} (fig. \ref{fig:fuzzifier-triangular}) transforma un valor escalar real $x^* \in U$ en un conjunto difuso $A'$ en $U$, utilizando una función de pertenencia triangular:
\begin{equation}
\mu_{A'}(x) = \begin{cases} (1 - \frac{|x_1 - x_1^*|}{b_1})\star\ldots\star  (1 - \frac{|x_n - x_n^*|}{b_n}) & \mbox{si } |x_i - x_i^*| \leq b_i, i = 1,2,\ldots,n, \\ 0 & \mbox{en cualquier otro caso} \end{cases}
\end{equation}
donde $b_i$ son parámetros positivos y $\star$ es una t-norma, generalmente el producto algebraico o el mínimo. Al igual que el fusificador gaussiano este tipo de fusificador es adecuado cuando existe ruido o imprecisiones en los datos de entrada.
\end{itemize}

\begin{figure}[H]
	\centering
	\begin{subfigure}[t]{\textwidth}
		\setlength\figureheight{4cm}
		\setlength\figurewidth{12cm}
		\input{figures/fuzzifiers/fuzzifier_singleton.tikz}
		\caption{Fusificador singleton.}
		\label{fig:fuzzifier-singleton}
	\end{subfigure}
	
	\vspace{1 cm}
	\begin{subfigure}[t]{\textwidth}
		\setlength\figureheight{4cm}
		\setlength\figurewidth{12cm}
		\input{figures/fuzzifiers/fuzzifier_gaussian.tikz}
		\caption{Fusificador gaussiano.}
		\label{fig:fuzzifier-gaussian}
	\end{subfigure}
	
	\vspace{1 cm}
	\begin{subfigure}[t]{\textwidth}
		\setlength\figureheight{4cm}
		\setlength\figurewidth{12cm}
		\input{figures/fuzzifiers/fuzzifier_triangular.tikz}
		\caption{Fusificador triangular.}
		\label{fig:fuzzifier-triangular}
	\end{subfigure}
		\caption{Representación de las funciones de pertenencia obtenidas con los fusificadores \emph{singleton} (\ref{fig:fuzzifier-singleton}), \emph{gaussiano} (\ref{fig:fuzzifier-gaussian}) y \emph{triangular} (\ref{fig:fuzzifier-triangular}).}
		\label{fig:fuzzifiers}
\end{figure}

\subsection{Defusificadores}\label{sec:defusificadores}
El último paso en el sistema difuso definido en \ref{sec:sistemas-difusos-basados-en-reglas} es la \emph{defusificación}, es decir, transformar el conjunto difuso, obtenido al aplicar las reglas de inferencia, en un valor escalar. Puede decirse que un defusificador realiza la operación complementaria a un fusificador (descritos en la sección \ref{sec:fusificadores}). Así pues, un defusificador puede definirse como \cite{wang1997}:

\begin{definition}
Un \emph{defusificador} es una función que transforma un conjunto difuso $B'$ en $V \subset \mathbb{R}$ (la salida del sistema de inferencia difusa) en un valor escalar $y^* \in V$.
\end{definition}

La misión de un defusificador es obtener el punto $y^*$ en $V$ que mejor representa al conjunto difuso $B'$. En este sentido, los defusificadores cumplen un rol similar a los operadores de agregación (por ejemplo la media aritmética). Se pueden utilizar diversas funciones como defusificadores, aunque generalmente se eligen aquellas que satisfacen las siguientes condiciones \cite{hellendoorn93}:

\begin{itemize}
\item\bfseries Continuidad: \normalfont Un cambio pequeño en la entrada del sistema difuso no debería causar una gran variación en la salida.
\item\bfseries Desambiguación: \normalfont El método de defusificación debe devolver un único valor para $y^*$. Es decir, no debe existir ambigüedad a la hora de seleccionar el valor para $y^*$.
\item\bfseries Plausibilidad: \normalfont El punto $y^*$ debe representar al conjunto difuso $B'$ de forma intuitiva. Es decir, el punto $y^*$ debe estar aproximadamente en el centro del soporte de $B'$ y tener un grado de pertenencia alto en $B'$.
\item\bfseries Simplicidad computacional: \normalfont La función debe ser lo más sencilla posible de calcular. Esto es especialmente importante en controladores difusos que operan en tiempo real.
\item\bfseries Método de ponderado: \normalfont El método seleccionado debe ponderar los diferentes conjuntos de salida. Este criterio depende del ámbito del problema.
\end{itemize}
A continuación se presentan algunos de los métodos de defusificación más utilizados:

\begin{itemize}
  \item\bfseries Centroide: \normalfont También conocido como \emph{centro de gravedad}, el método del \emph{centroide} fue propuesto por Sugeno en 1985 y es uno de los métodos más utilizados \cite{lee1990}. Este método obtiene el valor $y^*$ como el centro del área cubierta por la función de pertenencia de $B'$. El método del centro de gravedad puede definirse como:
  \begin{equation}
	y^* = \frac{\sum y \mu_{B'}(y)}{\sum \mu_{B'}(y)}
  \end{equation}
  Este método tiene la ventaja de que proporciona un valor plausible de forma intuitiva.
  \item\bfseries Bisector: \normalfont El método del bisector obtiene el punto por el que pasa la línea que divide la región delimitada por la función de pertenencia de $B'$ y el eje de abscisas, en dos subsecciones de igual área:
  \begin{equation}
	y^*\text{ tal que: } \sum\limits_{y=y_0}^{y=y^*}\mu_{B'}(y) = \sum\limits_{y=y^*}^{y=y_n}\mu_{B'}(y)
  \end{equation}
  El valor obtenido con este método coincide en ocasiones con el método del \emph{centroide} y también es computacionalmente costoso.
  \item\bfseries Máximo: \normalfont El defusificador del máximo obtiene el punto $y^*$ en el que la función de pertenencia $\mu_{B'}(y)$ alcanza su máximo valor. Es decir:
  \begin{equation}
	y^*\text{ tal que: } \mu_{B'}(y^*) = \max(\mu_{B'}(y_{i}))\qquad ,\forall y_{i} \in V
  \end{equation}
  Este método tiene la ventaja de ser fácil de implementar y poco costoso de calcular. Sin embargo, puede producir resultados ambiguos, puesto que pueden existir varios puntos en los que $\mu_{B'}(y)$ alcanza un valor máximo. En el caso de que existan varios puntos con valores máximos es necesario imponer algún criterio para seleccionar uno de ellos. Si se construye el conjunto:
  \begin{equation}
	hgt(B') = \{y \in V | \mu_{B'}(y) = \sup\limits_{y \in V}\mu_{B'}(y)\}
  \end{equation}
  es decir, el conjunto de todos los puntos en $V$ donde $\mu_{B'}(y)$ alcanza su valor máximo. El defusificador del máximo obtiene un elemento arbitrario $y^*$ del conjunto $hgt(B')$, es decir:
  \begin{equation}
	y^* = \text{ cualquier punto en } hgt(B')
  \end{equation}
  \item\bfseries Menor de máximos (SOM, \emph{smallest of maximum}): \normalfont Dado el conjunto $hgt(B')$ con todos los puntos donde $\mu_{B'}(y)$ alcanza su valor máximo, el defusificador \emph{menor de máximos} obtiene el menor punto $y^*$ del conjunto tal que:
  \begin{equation}
	y^* = \inf\{y \in hgt(B')\}
  \end{equation}
  \item\bfseries Mayor de máximos (LOM, \emph{largest of maximum}): \normalfont Dado el conjunto $hgt(B')$ con todos los puntos donde $\mu_{B'}(y)$ alcanza su valor máximo, el defusificador \emph{mayor de máximos} obtiene el mayor punto $y^*$ del conjunto tal que:
   \begin{equation}
  	y^* = \sup\{y \in hgt(B')\}
    \end{equation}
   \item\bfseries Media de máximos (MOM, \emph{mean of maximum}): \normalfont Dado el conjunto $hgt(B')$ con todos los puntos donde $\mu_{B'}(y)$ alcanza su valor máximo, el defusificador \emph{media de máximos} obtiene el punto medio $y^*$ del conjunto tal que:
   \begin{equation}
     	y^* = \frac{ \inf\{y \in hgt(B')\} + \sup\{y \in hgt(B')\}}{2}
   \end{equation}
\end{itemize}
En la figura \ref{fig:defuzzifiers} se presenta una función de pertenencia y los resultados de aplicar los defusificadores definidos anteriormente. En este caso, los resultados obtenidos con el método del \emph{centroide} y el \emph{bisector} coinciden, aunque no siempre tiene por qué ser así.
\begin{figure}[t]
	\centering
	\setlength\figureheight{5.5cm}
	\setlength\figurewidth{12cm}
	% This file was created by matlab2tikz v0.4.7 (commit e06548f86e222baeac9b1532d97c4b747a61abe5) running on MATLAB 8.0.
% Copyright (c) 2008--2014, Nico Schlömer <nico.schloemer@gmail.com>
% All rights reserved.
% Minimal pgfplots version: 1.3
% 
%
% defining custom colors
\definecolor{mycolor1}{rgb}{0.00000,0.75000,0.75000}%
\definecolor{mycolor2}{rgb}{0.75000,0.00000,0.75000}%
%
\begin{tikzpicture}

\begin{axis}[%
width=\figurewidth,
height=\figureheight,
scale only axis,
xmin=0,
xmax=100,
xtick={0,10,20,30,40,50,60,70,80,90,100},
xticklabels={\empty},
ymin=0,
ymax=1,
axis x line*=bottom,
axis y line*=left,
legend style={draw=black,fill=white,legend cell align=left}
]
\addplot [color=black,solid,line width=1.2pt]
  table[row sep=crcr]{0	0\\
1	0.06\\
2	0.12\\
3	0.18\\
4	0.24\\
5	0.3\\
6	0.36\\
7	0.42\\
8	0.48\\
9	0.54\\
10	0.6\\
11	0.6\\
12	0.6\\
13	0.6\\
14	0.6\\
15	0.6\\
16	0.6\\
17	0.6\\
18	0.6\\
19	0.6\\
20	0.6\\
21	0.6\\
22	0.6\\
23	0.6\\
24	0.6\\
25	0.6\\
26	0.6\\
27	0.6\\
28	0.6\\
29	0.633333333333333\\
30	0.666666666666667\\
31	0.7\\
32	0.733333333333333\\
33	0.766666666666667\\
34	0.8\\
35	0.833333333333333\\
36	0.866666666666667\\
37	0.9\\
38	0.933333333333333\\
39	0.966666666666667\\
40	1\\
41	1\\
42	1\\
43	1\\
44	1\\
45	1\\
46	1\\
47	1\\
48	1\\
49	1\\
50	1\\
51	1\\
52	1\\
53	1\\
54	1\\
55	1\\
56	1\\
57	1\\
58	1\\
59	1\\
60	1\\
61	0.95\\
62	0.9\\
63	0.85\\
64	0.8\\
65	0.75\\
66	0.7\\
67	0.65\\
68	0.6\\
69	0.55\\
70	0.5\\
71	0.45\\
72	0.4\\
73	0.4\\
74	0.4\\
75	0.4\\
76	0.4\\
77	0.4\\
78	0.4\\
79	0.4\\
80	0.4\\
81	0.4\\
82	0.4\\
83	0.4\\
84	0.4\\
85	0.4\\
86	0.4\\
87	0.4\\
88	0.4\\
89	0.4\\
90	0.4\\
91	0.36\\
92	0.32\\
93	0.28\\
94	0.24\\
95	0.2\\
96	0.16\\
97	0.12\\
98	0.08\\
99	0.04\\
100	0\\
};
\addlegendentry{$\mu\text{(x)}$};

\addplot [color=blue,line width=1.2pt,mark size=4.5pt,only marks,mark=asterisk,mark options={solid}]
  table[row sep=crcr]{47	1\\
};
\addlegendentry{centroid};

\addplot [color=black!50!green,line width=1.2pt,mark size=4.5pt,only marks,mark=+,mark options={solid}]
  table[row sep=crcr]{47	1\\
};
\addlegendentry{bisector};

\addplot [color=red,line width=1.2pt,mark size=3.2pt,only marks,mark=square,mark options={solid}]
  table[row sep=crcr]{50	1\\
};
\addlegendentry{mom};

\addplot [color=mycolor1,line width=1.2pt,mark size=3.0pt,only marks,mark=triangle,mark options={solid,rotate=180}]
  table[row sep=crcr]{40	1\\
};
\addlegendentry{som};

\addplot [color=mycolor2,line width=1.2pt,mark size=3.0pt,only marks,mark=triangle,mark options={solid}]
  table[row sep=crcr]{60	1\\
};
\addlegendentry{lom};

\end{axis}
\end{tikzpicture}%
	\caption{Resultado de aplicar los defusificadores \emph{centroide} (\textasteriskcentered), \emph{bisector}(+), \emph{media de máximos} ($\square$), \emph{menor de máximos}($\triangledown$) y \emph{mayor de máximos}($\vartriangle$).}
	\label{fig:defuzzifiers}
\end{figure}

\section{El controlador de Mamdani}\label{sec:mamdani-controller}
Uno de los métodos de inferencia difusa más utilizado es el llamado \emph{controlador de Mamdani} \cite{Mamdani1975}, propuesto por Mamdani y Assilian en 1975 para realizar el control de un motor de vapor a partir de un conjunto de reglas obtenidas de operadores humanos experimentados.

El método de Mamdani utiliza reglas difusas de la forma:

\begin{equation}
R_{i}: \text{IF }\chi_{1}\text{ es }A_{i1}\text{, }\chi_{2}\text{ es }A_{i2}\text{ , \ldots , }\chi_{n}\text{ es }A_{im}\text{ THEN } y \text{ es } B_i
\end{equation}

donde $\chi_1,\ldots,\chi_m$ son variables lingüísticas cuyos valores son conjuntos difusos $A_{ij} \subset FS(U_j)$ con $j\in\{1,\ldots,m\}$. El consecuente $y$ es también una variable lingüística sobre $Y$ ($B_i \subset FS(Y)$). Las entradas del sistema son valores escalares $x_1 \in U_1,\ldots,x_m \in U_m$. 

Cada regla se evalúa por separado y se obtiene el grado de pertenencia de cada entrada al conjunto difuso de su correspondiente antecedente, utilizando el operador mínimo para realizar la conjunción AND de los antecedentes ($k_i = \min(\mu_{Ai1}(x_1),\ldots,\mu_{Aim}(x_m))$). A este valor $k_i$ se le denomina normalmente \emph{grado de activación} de la regla. El resultado de evaluar cada regla es un conjunto difuso sobre $Y$ ($B_i'$), que se calcula ``truncando'' el consecuente con el grado de activación ($B_i' = \{(y,\min(B_i(y),k_i))|y \in Y\}$). En el caso de múltiples reglas, el conjunto difuso de salida se calcula combinando los conjuntos difusos obtenidos en cada regla utilizando el máximo ($B'(y) = \max\limits_{i=1}^{n}(B_i')$).

\begin{algorithm}
\caption{Método de Mamdani}
\label{algo:mamdani}
\DontPrintSemicolon
\KwIn{Un conjunto de reglas $R_{i}$ ($i \in \{1,\ldots,n\}$) con varios antecedentes $A_{ij}$ ($j \in \{1,\ldots,m\}$)  y entradas escalares $x_1,\ldots,x_m$.}
\KwOut{\emph{B'}.}
\For{$i \in \{1,\ldots,n\}$} {
Calcular $k_i = \min(\mu_{Ai1}(x_1),\ldots,\mu_{Aim}(x_m))$ \\
Calcular $B_i' = \{(y,\min(B_i(y),k_i))|y \in Y\}$
}
Construir $B' = \{(y, B'(y))|y \in Y\}$ dado por: \\
\centering
\nonl $B'(y) = \max\limits_{i=1}^{n}(B_i')$.\\
\Return{$B'$}\;
\end{algorithm}

La principal ventaja del método de Mamdani es que es relativamente sencillo de implementar y computacionalmente eficiente. El método de Larsen es similar, pero utilizando el producto en vez del mínimo como operador de implicación \cite{larsen1980}.

\section{Método de interpolación basado en índices de solapamiento}\label{sec:interpolation-method}
En esta sección se introduce un nuevo método para resolver sistemas basados en reglas que generaliza el método clásico de interpolación , utilizando para ello índices de solapamiento \cite{bustince2013overlap}.\\
\\
Históricamente el método más utilizado para resolver sistemas basados en reglas, es decir, para calcular el consecuente \emph{B'}, era el método de interpolación \cite{klir1987}. En este método se utiliza la consistencia de Zadeh \emph{$O_{Z}$} \cite{zadeh1978}, dada en la ecuación \ref{eq:zadeh-consistency}. Los pasos a seguir son:

\begin{algorithm}
\DontPrintSemicolon
\KwIn{Un conjunto de reglas $R_{j}$, con $j \in \{1,\ldots,n\}$, un hecho \emph{A'} y el índice de consistencia $O_{Z}$ (ec. \ref{eq:zadeh-consistency}).}
\KwOut{\emph{B'}.}
\vspace{0.4 cm}
\For{$j \in \{1,\ldots,n\}$} {
Calcular $O_{Z}(A',A_{j}) = \max\limits_{x \in X}(\min(A'(x),A_{j}(x))) $
}
Construir $B' = \{(y, B'(y))|y \in Y\}$ dado por: \\
\centering
\nonl $B'(y) = \max\limits_{j=1}^{n}(\min(B_{j}(y),O_{Z}(A', A_{j})))$.\\
\Return{$B'$}\;
\caption{Método de interpolación}
\label{algo:interpolation-method}
\end{algorithm}
Dado que el índice de consistencia $O_{Z}$ es un índice de solapamiento que cumple la definición \ref{def:dubois-overlap-index}, el algoritmo \ref{algo:interpolation-method} se puede generalizar para utilizar cualquier índice de solapamiento:

\begin{algorithm}
\DontPrintSemicolon
\KwIn{Un conjunto de reglas $R_{j}$, con $j \in \{1,\ldots,n\}$ y un hecho \emph{A'}.}
\KwOut{\emph{B'}.}
\vspace{0.4 cm}
Seleccionar un operador de agregación $M_{1}$, una función de solapamiento $G_{O}$ y un índice de solapamiento O.\\
\For{$j \in \{1,\ldots,n\}$} {
Calcular $O(A', A_{j})$
}
Construir $B' = \{(y, B'(y))|y \in Y\}$ dado por: \\
\centering
\nonl $B'(y) = \overset{n}{\underset{j=1}{M_{1}}}(G_{O}(B_{j}(y),O(A', A_{j})))$.\\
\Return{$B'$}\;
\caption{Método de interpolación generalizado}
\label{algo:overlap-interpolation-method}
\end{algorithm}

Si en el algoritmo \ref{algo:overlap-interpolation-method} utilizamos $M_{1} = \max$, $G_{O} = \min$ y $O = O_{Z}$ entonces se recupera el algoritmo \ref{algo:interpolation-method}.

En el caso de que cada regla tenga varios antecedentes, se puede generalizar el algoritmo \ref{algo:overlap-interpolation-method} de la siguiente manera:

\begin{algorithm}
\DontPrintSemicolon
\KwIn{Un conjunto de reglas $R_{j}$ con varios antecedentes, con $j \in \{1,\ldots,n\}$ y un hecho \emph{A'}.}
\KwOut{\emph{B'}.}
\vspace{0.4 cm}
Seleccionar un operador de agregación $M$, una t-norma \emph{T} y un índice de solapamiento \emph{O}.\\
\For{$i =1 \to n$} {
Calcular $O(A'_{i}, A_{i1}),\ldots,O(A'_{m}, A_{im})$\\
Calcular $k_{i} = T(O(A'_{i}, A_{i1}),\ldots,O(A'_{m}, A_{im}))$\\
Construir sobre el universo de referencia \emph{Y} el conjunto $K_{i} = \{(y,k_{i})|y \in Y\}$
}
Construir $B' = \{(y, B'(y))|y \in Y\}$ dado por: \\
\centering
\nonl $B'(y) = \overset{n}{\underset{i=1}{M}}(\min(K_{i},B_{i}))$.\\
\Return{$B'$}\;
\caption{Método de interpolación generalizado para reglas con varios antecedentes}
\label{algo:multi-overlap-interpolation-method}
\end{algorithm}



% !TeX spellcheck = es_ES
\chapter{Detección de incendios forestales}
\label{cha:deteccion-incendios-forestales}
En este capítulo se presenta una aplicación práctica de los métodos de inferencia difusa basados en reglas, descritos en los capítulos anteriores (REFERENCIA). El objetivo es definir un sistema basado en reglas capaz de determinar el riesgo de incendios forestales a partir de mediciones de algunas magnitudes tales como la temperatura, el humo, la humedad etc. Estas magnitudes constituirán las entradas del sistema de inferencia que obtendrá una estimación cuantitativa del riesgo del incendio forestal.

\section{Red de sensores inalámbricos}
El primer paso para la determinación del riesgo de un incendio forestal es la medición y toma de datos ambientales relacionados con dicho incendio. Para ello se puede desplegar una red de sensores inalámbricos (\emph{Wireless Sensor Networks}, WSN), que han sido desarrolladas y utilizadas en una gran variedad de aplicaciones en áreas tales como automoción \cite{hsin2007}, defensa, medicina, agricultura \cite{tao2008}\cite{hwang2010}, etc. 

Algunas aplicaciones de este tipo de sistemas relacionadas con los riesgos ambientales incluyen, por ejemplo, el control de movimientos de personas y ganado, monitorización de factores ambientales que afectan a la calidad de los cultivos, detección de incendios forestales, mediciones meteorológicas y detección de inundaciones etc. \cite{Akyildiz2002}

Las redes de sensores inalámbricos están diseñadas para monitorizar y controlar eventos en lugares que presentan riesgos ambientales tales como bosques, terrenos montañosos etc. Por esta razón se diseña este tipo de sistemas de forma que sean lo más autónomos posible. Esto incluye, por ejemplo, la utilización de energías renovables (típicamente energía solar), que posibilitan que los sensores desplegados en el terreno sean totalmente auto-suficientes.

El diseño y despliegue de una red de sensores inalámbricos queda totalmente fuera del alcance de este proyecto y por tanto no se va a entrar en ningún tipo de detalle técnico sobre estos sistemas. El objetivo de este proyecto es evaluar algoritmos de inferencia difusa sobre los valores que serían entregados al sistema de decisión en una aplicación real . 

Las mediciones realizadas por la red de sensores constituyen las entradas del sistema de inferencia y decisión. Estas entradas son, originalmente, valores escalares que son transformados por el sistema de lógica difusa en valores difusos (por medio de variables lingüísticas). Esto permite realizar, sobre estas mediciones, un proceso de razonamiento aproximado que puede tratar mejor las imprecisiones, en comparación con utilizar dichos valores escalares directamente.


\section{Magnitudes medidas y variables lingüísticas}

Las entradas del sistema difuso son valores escalares de magnitudes que describen el incendio forestal, tales como la temperatura o la luminosidad, medidas por la red de sensores. Para cada una de estas magnitudes se define una variable lingüística con tres valores posibles (definidas por sus correspondientes funciones de pertenencia). Así pues, los valores escalares de entrada serán transformados en conjuntos difusos que posteriormente serán comparados con las variables lingüísticas según las reglas definidas, para obtener como salida el riesgo de incendio (que es también una variable lingüística). Estas variables lingüísticas así como sus correspondientes universos (rangos de valores posibles) son:

\begin{enumerate}[label=($\chi_\arabic*$),ref=(X\arabic*)]
   \item \bfseries Temperatura: \normalfont medida en grados centígrados (0ºC a 120ºC).
   \item \bfseries Humo: \normalfont medida en partes por millón (0 a 100ppm).
   \item \bfseries Luz: \normalfont medida en lux (0 a 1000 lux).
   \item \bfseries Humedad: \normalfont medida en partes por millón (0 a 100ppm).
   \item \bfseries Distancia: \normalfont medida en metros (0 a 80m).
\end{enumerate}

Para cada una de estas variables lingüísticas se definen tres valores posibles: \emph{Baja} (L,\emph{Low}), \emph{Media} (M, \emph{Medium}) y \emph{Alta} (H, \emph{High}) que los valores tienen el sentido de \emph{Cerca} (C, \emph{Close}), \emph{Media} (M, \emph{Medium}) y \emph{Lejos} (F, \emph{Far}) respectivamente.

La salida del sistema viene dada por la variable lingüística $y = $ \emph{``Riesgo de incendio''}, que determina el riesgo de incendio forestal en una escala del 0 al 100 (\%). Esta variable lingüística puede tomar los valores: \emph{Muy bajo} (VL, \emph{Very Low}), \emph{Bajo} (L, \emph{Low}), \emph{Medio} (M, \emph{Medium}), \emph{Alto} (H, \emph{High}) y \emph{Muy Alto} (VH, \emph{Very High}). 

En la figura \ref{fig:fire-detection-lang-variables} se representan las variables lingüística descritas anteriormente. Como se puede ver, se han elegido funciones de pertenencia lineales para modelar los valores de las variables lingüísticas.

\begin{figure}[t]
	\centering
	\begin{subfigure}[b]{0.45\textwidth}
		\setlength\figureheight{2.5cm}
		\setlength\figurewidth{6cm}
		\input{figures/fire-detection-lang-variables/temp_lang_variable.tikz}
		\caption{$\chi_1$ - Temperatura (ºC)}
		\label{fig:temp-lang-variable}
	\end{subfigure}
	\qquad
	\begin{subfigure}[b]{0.45\textwidth}
		\setlength\figureheight{2.5cm}
		\setlength\figurewidth{6cm}
		\input{figures/fire-detection-lang-variables/smoke_lang_variable.tikz}
		\caption{$\chi_2$ - Humo (ppm)}
		\label{fig:smoke-lang-variable}
	\end{subfigure}
	
	\vspace{1 cm}
	\begin{subfigure}[b]{0.45\textwidth}
		\setlength\figureheight{2.5cm}
		\setlength\figurewidth{6cm}
		\input{figures/fire-detection-lang-variables/light_lang_variable.tikz}
		\caption{$\chi_3$ - Luz (lux)}
		\label{fig:light-lang-variable}
	\end{subfigure}
	\qquad
	\begin{subfigure}[b]{0.45\textwidth}
		\setlength\figureheight{2.5cm}
		\setlength\figurewidth{6cm}
		\input{figures/fire-detection-lang-variables/humidity_lang_variable.tikz}
		\caption{$\chi_4$ - Humedad (ppm)}
		\label{fig:humidity-lang-variable}
	\end{subfigure}
	
	\vspace{1 cm}
	\begin{subfigure}[b]{0.45\textwidth}
		\setlength\figureheight{2.5cm}
		\setlength\figurewidth{6cm}
		\input{figures/fire-detection-lang-variables/distance_lang_variable.tikz}
		\caption{$\chi_5$ - Distancia (m)}
		\label{fig:distance-lang-variable}
	\end{subfigure}
	\qquad
	\begin{subfigure}[b]{0.45\textwidth}
		\setlength\figureheight{2.5cm}
		\setlength\figurewidth{6cm}
		\input{figures/fire-detection-lang-variables/threat_lang_variable.tikz}
		\caption{$y$ - Riesgo de incendio (\%)}
		\label{fig:threat-lang-variable}
	\end{subfigure}
	\caption{Variables lingüísticas utilizadas en la determinación de riesgo de incendios.}
	\label{fig:fire-detection-lang-variables}
\end{figure}

\section{Conjunto de reglas}




\chapter{Implementación en MATLAB}
En este capítulo se incluye el código fuente de la implementación en MATLAB de los métodos estudiados en este proyecto.

\section{General}
En esta sección se incluyen algunas funciones de propósito general que han sido implementadas en el desarrollo de este proyecto.

\subsection{T-normas y operadores de agregación}
MATLAB dispone de una colección importante de funciones predefinidas para calcular t-normas y operadores de agregación. En este proyecto se han utilizado las siguientes:

\begin{itemize}
\item Media aritmética: \lstinline|mean()|.
\item Máximo: \lstinline|max()|.
\item Producto: \lstinline|prod()|.
\item Mínimo: \lstinline|min()|.
\item Media geométrica: \lstinline|geomean()|.
\item Media armónica: \lstinline|harmmean()|.
\end{itemize}

Además se han implementado sendas funciones para calcular las t-normas del seno y de Einstein:

\lstinputlisting[label=lst:sinmean, caption=T-norma del seno (\lstinline|functions/sinmean.m|),inputencoding=cp1252]{./matlab/functions/sinmean.m}

\lstinputlisting[label=lst:einsteinmean, caption=T-norma de Einstein (\lstinline|functions/einsteinmean.m|),inputencoding=cp1252]{./matlab/functions/einsteinmean.m}

Como se puede ver en la función \lstinline|einsteinmean()| se ha utilizado una función anónima (\lstinline|@(x)(1-x)|), que es una funcionalidad muy útil de MATLAB, inspirada en los lenguajes funcionales, que permite declarar funciones \emph{lambda} o \emph{anónimas} (sin nombre) y utilizarlas como argumentos de otras funciones (en este caso de \lstinline|arrayfun()|, que aplica dicha función anónima a todos los elementos de un vector). Esta característica del lenguaje de MATLAB se utilizará también a la hora de definir índices de solapamiento.

\section{Detección de incendios forestales}
\subsection{Variables lingüísticas}
Como se ha definido en \ref{def:formal-lang-variable}, una variable lingüística es una variable que puede tomar conjuntos difusos como valores. Así mismo, un conjunto difuso viene definido por su función de pertenencia. Por tanto, se debe definir una función para cada valor de cada variable lingüística utilizada. 

Generalmente, en MATLAB, cada función se define en un fichero separado que únicamente contiene el código de dicha función. En este caso y dado que se utilizan 5 variables lingüísticas para los antecedentes con 3 valores posibles cada una y una variable de salida para el consecuente con 5 valores posibles, habría que crear 20 ficheros para definir estas funciones. Aunque es perfectamente posible y válido hacerlo de esta manera, resulta tedioso manejar tantos ficheros. 

Una solución adecuada al problema es definir cada variable lingüística en un fichero en el que se implementan las funciones de pertenencia de sus posibles valores. Para ello se puede crear una clase para cada variable lingüística y definir la función de pertenencia de cada valor como un método estático de la misma. Esta forma de definir las variables lingüísticas tiene la ventaja de que todo el código asociado a la variable está en un mismo fichero. Además la forma de utilizar las funciones resulta muy cómoda y expresiva.

Por ejemplo, para la variable lingüística $\chi_1$ = \emph{Temperatura} se ha definido la clase \lstinline|temp| en el fichero \lstinline|lang_variables/temp.m|. Esta clase tiene los siguientes métodos estáticos:
\begin{itemize}
\item \lstinline|get_x()| : Devuelve el universo de referencia para la variable (en este caso el vector $[0,100]$).
\item \lstinline|low(t)| : Función de pertenencia para el valor ``Bajo''.
\item \lstinline|medium(t)| : Función de pertenencia para el valor ``Medio''.
\item \lstinline|high(t)| : Función de pertenencia para el valor ``Alto''.
\end{itemize}

De esta forma, para obtener el grado de pertenencia de un valor \lstinline|t| al conjunto \emph{``Temperatura Media''} basta con hacer:

\begin{lstlisting}
v = temp.medium(t);
\end{lstlisting}

Además es posible obtener un puntero a estas funciones utilizando el operador \lstinline|@|:

\begin{lstlisting}
fh = @temp.medium;
v = fh(t);
\end{lstlisting}

La variable \lstinline|fh| es un puntero a la función \lstinline|temp.medium()| y puede utilizarse de la misma forma que ésta (línea 2). Además la variable \lstinline|fh| se puede utilizar de forma similar a otros tipos de variables, es decir, se pueden crear vectores de punteros de funciones, pasarlos como parámetros a otras funciones etc. Esta propiedad se utilizará de forma intensiva a la hora de implementar los métodos estudiados en este proyecto y el conjunto de reglas.

A continuación se incluyen las definiciones de las variables lingüísticas utilizadas en la aplicación práctica de detección de incendios forestales:

\lstinputlisting[caption=$\chi_1$ - Temperatura (\lstinline|lang_variables/temp.m|),inputencoding=cp1252]{./matlab/lang_variables/temp.m}

\lstinputlisting[caption=$\chi_2$ - Humo (\lstinline|lang_variables/smoke.m|), inputencoding=cp1252]{./matlab/lang_variables/smoke.m}

\lstinputlisting[caption=$\chi_3$ - Luz (\lstinline|lang_variables/llight.m|),inputencoding=cp1252]{./matlab/lang_variables/llight.m}

\lstinputlisting[caption=$\chi_4$ - Humedad (\lstinline|lang_variables/humidity.m|),inputencoding=cp1252]{./matlab/lang_variables/humidity.m}

\lstinputlisting[caption=$\chi_5$ - Distancia (\lstinline|lang_variables/distance.m|),inputencoding=cp1252]{./matlab/lang_variables/distance.m}

\lstinputlisting[caption=$y$ - Riesgo (\lstinline|lang_variables/threat.m|),inputencoding=cp1252]{./matlab/lang_variables/threat.m}

\subsection{Conjunto de reglas}

El conjunto de reglas se ha definido en la función \lstinline|fire_detection_rules()| (\lstinline|fire_detection_rules.m|), de forma que sea sencillo obtener y utilizar este conjunto en otros scripts y funciones. Esta función devuelve un vector de estructuras donde cada una tiene los siguientes campos:
\begin{itemize}
\item \lstinline|R(i).n|: Nº de regla.
\item \lstinline|R(i).A|: Vector con las funciones de pertenencia de los valores (punteros a funciones) de cada uno de los antecedentes de la regla.
\item \lstinline|R(i).B|: Puntero a función de pertenencia del valor del consecuente.
\end{itemize}

Por ejemplo para definir un conjunto de reglas con una única regla tal que:
\begin{multline}
\text{IF \emph{Temperatura} es ``Baja'' AND }  \text{ \emph{Humo} es ``Alto'' AND } \text{ \emph{Luz} es ``Baja''} \\
  \text{AND \emph{Humedad} es ``Alta'' AND }  \text{ \emph{Distancia} es ``Media'' THEN }  \text{ \emph{Riesgo} es ``Bajo'' }
\end{multline}
hay que construir una estructura de la siguiente forma:

\begin{lstlisting}
R(1).n = 1;
R(1).A = {@temp.low, @smoke.high, @llight.low, @humidity.high, @distance.medium};
R(1).B = @threat.low;
\end{lstlisting}

Este tipo de estructura tiene la ventaja de que es sencilla de implementar y de entender. Sin embargo, para conjuntos de reglas grandes (como es el caso) resulta tedioso tener que indicar el índice para cada regla y repetir una y otra vez el nombre del conjunto (\lstinline|R|) o de los campos. 

Por esta razón se ha optado por una forma de definir el conjunto de reglas más concisa y clara. El conjunto de reglas se define como una matriz, en la que cada fila corresponde a una regla y cada columna es un puntero a la función de pertenencia del valor de la variable lingüística que ocupa esa posición. Las primeras $n-1$ columnas de cada fila corresponden a los antecedentes de la regla, y la última al consecuente.

En la función  \lstinline|fire_detection_rules| (código \ref{matlab_fire_detection_rules}) se define el conjunto de reglas en la matriz \lstinline|rules| (líneas 2-335) de esta forma. Dado que el resto de funciones implementadas esperan un conjunto de reglas definido como un vector de estructuras, se transforma esta matriz en dicho vector en las líneas 340-344. De esta forma, el conjunto de reglas es muy sencillo de leer y modificar y se asemeja mucho a la tabla \ref{tab:fire-detection-rule-set}.

\lstset{linewidth=18cm}
\lstinputlisting[label=matlab_fire_detection_rules, caption=Conjunto de reglas (\lstinline|fire_detection_rules.m|) ,inputencoding=cp1252]{./matlab/fire_detection_rules.m}

\chapter{Conclusiones}
\label{cha:conclusiones}
\begin{tabbing}
AAAA\=AAAA\=AAAA\kill\\
\>que\\
\>\>tal?\\
\end{tabbing}
\begin{theorem}
If \emph{proposition}\index{proposition|textbf} $P$ is a tautology then $\sim P$ is a contradiction, and conversely.
\end{theorem}
\begin{theorem}[Tautologies and Contradictions]
content...
\end{theorem}

Esto es una cita \cite{bolourchi2013}




\backmatter

% Bibliografía
\bibliographystyle{ieeetr}
\bibliography{bibliography}

% Glosario de términos
\printindex

\end{document}
