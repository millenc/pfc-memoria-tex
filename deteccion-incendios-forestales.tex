% !TeX spellcheck = es_ES
\chapter{Detección de incendios forestales}
\label{cha:deteccion-incendios-forestales}
En este capítulo se presenta una aplicación práctica de los métodos de inferencia difusa basados en reglas, descritos en los capítulos anteriores (REFERENCIA). El objetivo es definir un sistema basado en reglas capaz de determinar el riesgo de incendios forestales a partir de mediciones de algunas magnitudes tales como la temperatura, el humo, la humedad etc. Estas magnitudes constituirán las entradas del sistema de inferencia que obtendrá una estimación cuantitativa del riesgo del incendio forestal.

\section{Red de sensores inalámbricos}
El primer paso para la determinación del riesgo de un incendio forestal es la medición y toma de datos ambientales relacionados con dicho incendio. Para ello se puede desplegar una red de sensores inalámbricos (\emph{Wireless Sensor Networks}, WSN), que han sido desarrolladas y utilizadas en una gran variedad de aplicaciones en áreas tales como automoción \cite{hsin2007}, defensa, medicina, agricultura \cite{tao2008}\cite{hwang2010}, etc. 

Algunas aplicaciones de este tipo de sistemas relacionadas con los riesgos ambientales incluyen, por ejemplo, el control de movimientos de personas y ganado, monitorización de factores ambientales que afectan a la calidad de los cultivos, detección de incendios forestales, mediciones meteorológicas y detección de inundaciones etc. \cite{Akyildiz2002}

Las redes de sensores inalámbricos están diseñadas para monitorizar y controlar eventos en lugares que presentan riesgos ambientales tales como bosques, terrenos montañosos etc. Por esta razón se diseña este tipo de sistemas de forma que sean lo más autónomos posible. Esto incluye, por ejemplo, la utilización de energías renovables (típicamente energía solar), que posibilitan que los sensores desplegados en el terreno sean totalmente auto-suficientes.

El diseño y despliegue de una red de sensores inalámbricos queda totalmente fuera del alcance de este proyecto y por tanto no se va a entrar en ningún tipo de detalle técnico sobre estos sistemas. El objetivo de este proyecto es evaluar algoritmos de inferencia difusa sobre los valores que serían entregados al sistema de decisión en una aplicación real . 

Las mediciones realizadas por la red de sensores constituyen las entradas del sistema de inferencia y decisión. Estas entradas son, originalmente, valores escalares que son transformados por el sistema de lógica difusa en valores difusos (por medio de variables lingüísticas). Esto permite realizar, sobre estas mediciones, un proceso de razonamiento aproximado que puede tratar mejor las imprecisiones, en comparación con utilizar dichos valores escalares directamente.


\section{Magnitudes medidas y variables lingüísticas}

Las entradas del sistema difuso son valores escalares de magnitudes que describen el incendio forestal, tales como la temperatura o la luminosidad, medidas por la red de sensores. Para cada una de estas magnitudes se define una variable lingüística con tres valores posibles (definidas por sus correspondientes funciones de pertenencia). Así pues, los valores escalares de entrada serán transformados en conjuntos difusos que posteriormente serán comparados con las variables lingüísticas según las reglas definidas, para obtener como salida el riesgo de incendio (que es también una variable lingüística). Estas variables lingüísticas así como sus correspondientes universos (rangos de valores posibles) son:

\begin{enumerate}[label=($\chi_\arabic*$),ref=(X\arabic*)]
   \item \bfseries Temperatura: \normalfont medida en grados centígrados (0ºC a 120ºC).
   \item \bfseries Humo: \normalfont medida en partes por millón (0 a 100ppm).
   \item \bfseries Luz: \normalfont medida en lux (0 a 1000 lux).
   \item \bfseries Humedad: \normalfont medida en partes por millón (0 a 100ppm).
   \item \bfseries Distancia: \normalfont medida en metros (0 a 80m).
\end{enumerate}

Para cada una de estas variables lingüísticas se definen tres valores posibles: \emph{Baja} (L,\emph{Low}), \emph{Media} (M, \emph{Medium}) y \emph{Alta} (H, \emph{High}) que los valores tienen el sentido de \emph{Cerca} (C, \emph{Close}), \emph{Media} (M, \emph{Medium}) y \emph{Lejos} (F, \emph{Far}) respectivamente.

La salida del sistema viene dada por la variable lingüística $y = $ \emph{``Riesgo de incendio''}, que determina el riesgo de incendio forestal en una escala del 0 al 100 (\%). Esta variable lingüística puede tomar los valores: \emph{Muy bajo} (VL, \emph{Very Low}), \emph{Bajo} (L, \emph{Low}), \emph{Medio} (M, \emph{Medium}), \emph{Alto} (H, \emph{High}) y \emph{Muy Alto} (VH, \emph{Very High}). 

En la figura \ref{fig:fire-detection-lang-variables} se representan las variables lingüística descritas anteriormente. Como se puede ver, se han elegido funciones de pertenencia lineales para modelar los valores de las variables lingüísticas.

\begin{figure}[t]
	\centering
	\begin{subfigure}[b]{0.45\textwidth}
		\setlength\figureheight{2.5cm}
		\setlength\figurewidth{6cm}
		\input{figures/fire-detection-lang-variables/temp_lang_variable.tikz}
		\caption{$\chi_1$ - Temperatura (ºC)}
		\label{fig:temp-lang-variable}
	\end{subfigure}
	\qquad
	\begin{subfigure}[b]{0.45\textwidth}
		\setlength\figureheight{2.5cm}
		\setlength\figurewidth{6cm}
		\input{figures/fire-detection-lang-variables/smoke_lang_variable.tikz}
		\caption{$\chi_2$ - Humo (ppm)}
		\label{fig:smoke-lang-variable}
	\end{subfigure}
	
	\vspace{1 cm}
	\begin{subfigure}[b]{0.45\textwidth}
		\setlength\figureheight{2.5cm}
		\setlength\figurewidth{6cm}
		\input{figures/fire-detection-lang-variables/light_lang_variable.tikz}
		\caption{$\chi_3$ - Luz (lux)}
		\label{fig:light-lang-variable}
	\end{subfigure}
	\qquad
	\begin{subfigure}[b]{0.45\textwidth}
		\setlength\figureheight{2.5cm}
		\setlength\figurewidth{6cm}
		\input{figures/fire-detection-lang-variables/humidity_lang_variable.tikz}
		\caption{$\chi_4$ - Humedad (ppm)}
		\label{fig:humidity-lang-variable}
	\end{subfigure}
	
	\vspace{1 cm}
	\begin{subfigure}[b]{0.45\textwidth}
		\setlength\figureheight{2.5cm}
		\setlength\figurewidth{6cm}
		\input{figures/fire-detection-lang-variables/distance_lang_variable.tikz}
		\caption{$\chi_5$ - Distancia (m)}
		\label{fig:distance-lang-variable}
	\end{subfigure}
	\qquad
	\begin{subfigure}[b]{0.45\textwidth}
		\setlength\figureheight{2.5cm}
		\setlength\figurewidth{6cm}
		\input{figures/fire-detection-lang-variables/threat_lang_variable.tikz}
		\caption{$y$ - Riesgo de incendio (\%)}
		\label{fig:threat-lang-variable}
	\end{subfigure}
	\caption{Variables lingüísticas utilizadas en la determinación de riesgo de incendios.}
	\label{fig:fire-detection-lang-variables}
\end{figure}

\section{Conjunto de reglas}

Una vez definidas las variables lingüísticas, es necesario construir un conjunto de reglas, que forman la base de conocimiento sobre el problema y permiten, mediante el mecanismo de inferencia, transformar las entradas en salidas. En este caso particular las reglas tienen la forma:

\begin{equation}
\text{IF }\chi_1  \text{ es }A_1\text{ AND }\chi_2 \text{ es }A_2\text{ AND }\chi_3 \text{ es }A_3\text{ AND }\chi_4  \text{ es }A_4\text{ AND }\chi_5 \text{ es }A_5\text{ THEN }y\text{ es }B
\end{equation}

Donde $\chi_1,\ldots,\chi_n$ son las variables de entrada del sistema difuso (definidas en la sección anterior) e $y$ es la salida (riesgo de incendio). Todas las reglas definidas en este sistema harán uso de las 5 variables lingüísticas conectadas mediante el operador de conjunción (AND).

\begin{center}
	\def\arraystretch{1.5}
    \begin{longtable}{| l | l | l | l | l | l |}
    \hline
    $\chi_1=$ Temp. &  $\chi_2=$ Humo &  $\chi_3=$ Luz &  $\chi_4=$ Humedad &  $\chi_5=$ Distancia &  $y=$ Riesgo \\ \hline
    L & L & L & H & H & VL \\ \hline 
    L & L & L & H & M & VL \\ \hline 
    L & L & L & H & L & VL \\ \hline 
    L & L & L & M & H & VL \\ \hline 
    L & L & L & M & M & VL \\ \hline 
    L & L & L & M & L & L \\ \hline 
    L & L & L & L & H & VL \\ \hline 
    L & L & L & L & M & L \\ \hline 
    L & L & L & L & L & L \\ \hline 
    L & L & M & H & H & VL \\ \hline 
    L & L & M & H & M & L \\ \hline 
    L & L & M & H & L & L \\ \hline 
    L & L & M & M & H & VL \\ \hline 
    L & L & M & M & M & L \\ \hline 
    L & L & M & M & L & L \\ \hline 
    L & L & M & L & H & VL \\ \hline 
    L & L & M & L & M & L \\ \hline 
    L & L & M & L & L & L \\ \hline 
    L & L & H & H & H & L \\ \hline 
    L & L & H & H & M & L \\ \hline 
    L & L & H & H & L & L \\ \hline 
    L & L & H & M & H & L \\ \hline 
    L & L & H & M & M & L \\ \hline 
    L & L & H & M & L & M \\ \hline 
    L & L & H & L & H & L \\ \hline 
    L & L & H & L & M & M \\ \hline 
    L & L & H & L & L & M \\ \hline 
    L & M & L & H & H & VL \\ \hline 
    L & M & L & H & M & VL \\ \hline 
    L & M & L & H & L & L \\ \hline 
    L & M & L & M & H & VL \\ \hline 
    L & M & L & M & M & L \\ \hline 
    L & M & L & M & L & L \\ \hline 
    L & M & L & L & H & L \\ \hline 
    L & M & L & L & M & L \\ \hline 
    L & M & L & L & L & M \\ \hline 
    L & M & M & H & H & L \\ \hline 
    L & M & M & H & M & L \\ \hline 
    L & M & M & H & L & M \\ \hline 
    L & M & M & M & H & L \\ \hline 
    L & M & M & M & M & L \\ \hline 
    L & M & M & M & L & M \\ \hline 
    L & M & M & L & H & L \\ \hline 
    L & M & M & L & M & L \\ \hline 
    L & M & M & L & L & M \\ \hline 
    L & M & H & H & H & L \\ \hline 
    L & M & H & H & M & M \\ \hline 
    L & M & H & H & L & M \\ \hline 
    L & M & H & M & H & L \\ \hline 
    L & M & H & M & M & M \\ \hline 
    L & M & H & M & L & M \\ \hline 
    L & M & H & L & H & L \\ \hline 
    L & M & H & L & M & M \\ \hline 
    L & M & H & L & L & H \\ \hline 
    L & H & L & H & H & L \\ \hline 
    L & H & L & H & M & L \\ \hline 
    L & H & L & H & L & M \\ \hline 
    L & H & L & M & H & L \\ \hline 
    L & H & L & M & M & L \\ \hline 
    L & H & L & M & L & M \\ \hline 
    L & H & L & L & H & L \\ \hline 
    L & H & L & L & M & M \\ \hline 
    L & H & L & L & L & M \\ \hline 
    L & H & M & H & H & L \\ \hline 
    L & H & M & H & M & M \\ \hline 
    L & H & M & H & L & M \\ \hline 
    L & H & M & M & H & L \\ \hline 
    L & H & M & M & M & M \\ \hline 
    L & H & M & M & L & M \\ \hline 
    L & H & M & L & H & L \\ \hline 
    L & H & M & L & M & M \\ \hline 
    L & H & M & L & L & M \\ \hline 
    L & H & H & H & H & M \\ \hline 
    L & H & H & H & M & M \\ \hline 
    L & H & H & H & L & M \\ \hline 
    L & H & H & M & H & M \\ \hline 
    L & H & H & M & M & M \\ \hline 
    L & H & H & M & L & H \\ \hline 
    L & H & H & L & H & M \\ \hline 
    L & H & H & L & M & H \\ \hline 
    L & H & H & L & L & H \\ \hline 
    M & L & L & H & H & VL \\ \hline 
    M & L & L & H & M & VL \\ \hline 
    M & L & L & H & L & L \\ \hline 
    M & L & L & M & H & VL \\ \hline 
    M & L & L & M & M & L \\ \hline 
    M & L & L & M & L & L \\ \hline 
    M & L & L & L & H & L \\ \hline 
    M & L & L & L & M & L \\ \hline 
    M & L & L & L & L & M \\ \hline 
    M & L & M & H & H & L \\ \hline 
    M & L & M & H & M & L \\ \hline 
    M & L & M & H & L & M \\ \hline 
    M & L & M & M & H & L \\ \hline 
    M & L & M & M & M & L \\ \hline 
    M & L & M & M & L & M \\ \hline 
    M & L & M & L & H & L \\ \hline 
    M & L & M & L & M & L \\ \hline 
    M & L & M & L & L & M \\ \hline 
    M & L & H & H & H & L \\ \hline 
    M & L & H & H & M & M \\ \hline 
    M & L & H & H & L & M \\ \hline 
    M & L & H & M & H & L \\ \hline 
    M & L & H & M & M & M \\ \hline 
    M & L & H & M & L & M \\ \hline 
    M & L & H & L & H & L \\ \hline 
    M & L & H & L & M & M \\ \hline 
    M & L & H & L & L & H \\ \hline 
    M & M & L & H & H & L \\ \hline 
    M & M & L & H & M & L \\ \hline 
    M & M & L & H & L & M \\ \hline 
    M & M & L & M & H & L \\ \hline 
    M & M & L & M & M & L \\ \hline 
    M & M & L & M & L & M \\ \hline 
    M & M & L & L & H & L \\ \hline 
    M & M & L & L & M & M \\ \hline 
    M & M & L & L & L & M \\ \hline 
    M & M & M & H & H & L \\ \hline 
    M & M & M & H & M & M \\ \hline 
    M & M & M & H & L & M \\ \hline 
    M & M & M & M & H & L \\ \hline 
    M & M & M & M & M & M \\ \hline 
    M & M & M & M & L & M \\ \hline 
    M & M & M & L & H & M \\ \hline 
    M & M & M & L & M & M \\ \hline 
    M & M & M & L & L & M \\ \hline 
    M & M & H & H & H & M \\ \hline 
    M & M & H & H & M & M \\ \hline 
    M & M & H & H & L & H \\ \hline 
    M & M & H & M & H & M \\ \hline 
    M & M & H & M & M & M \\ \hline 
    M & M & H & M & L & H \\ \hline 
    M & M & H & L & H & M \\ \hline 
    M & M & H & L & M & H \\ \hline 
    M & M & H & L & L & H \\ \hline 
    M & H & L & H & H & L \\ \hline 
    M & H & L & H & M & M \\ \hline 
    M & H & L & H & L & M \\ \hline 
    M & H & L & M & H & L \\ \hline 
    M & H & L & M & M & M \\ \hline 
    M & H & L & M & L & M \\ \hline 
    M & H & L & L & H & M \\ \hline 
    M & H & L & L & M & M \\ \hline 
    M & H & L & L & L & H \\ \hline 
    M & H & M & H & H & M \\ \hline 
    M & H & M & H & M & M \\ \hline 
    M & H & M & H & L & M \\ \hline 
    M & H & M & M & H & M \\ \hline 
    M & H & M & M & M & M \\ \hline 
    M & H & M & M & L & H \\ \hline 
    M & H & M & L & H & M \\ \hline 
    M & H & M & L & M & H \\ \hline 
    M & H & M & L & L & H \\ \hline 
    M & H & H & H & H & M \\ \hline 
    M & H & H & H & M & M \\ \hline 
    M & H & H & H & L & H \\ \hline 
    M & H & H & M & H & M \\ \hline 
    M & H & H & M & M & H \\ \hline 
    M & H & H & M & L & H \\ \hline 
    M & H & H & L & H & H \\ \hline 
    M & H & H & L & M & H \\ \hline 
    M & H & H & L & L & VH \\ \hline 
    H & L & L & H & H & L \\ \hline 
    H & L & L & H & M & L \\ \hline 
    H & L & L & H & L & M \\ \hline 
    H & L & L & M & H & L \\ \hline 
    H & L & L & M & M & L \\ \hline 
    H & L & L & M & L & M \\ \hline 
    H & L & L & L & H & L \\ \hline 
    H & L & L & L & M & M \\ \hline 
    H & L & L & L & L & M \\ \hline 
    H & L & M & H & H & L \\ \hline 
    H & L & M & H & M & M \\ \hline 
    H & L & M & H & L & M \\ \hline 
    H & L & M & M & H & L \\ \hline 
    H & L & M & M & M & M \\ \hline 
    H & L & M & M & L & M \\ \hline 
    H & L & M & L & H & L \\ \hline 
    H & L & M & L & M & M \\ \hline 
    H & L & M & L & L & M \\ \hline 
    H & L & H & H & H & M \\ \hline 
    H & L & H & H & M & M \\ \hline 
    H & L & H & H & L & M \\ \hline 
    H & L & H & M & H & M \\ \hline 
    H & L & H & M & M & M \\ \hline 
    H & L & H & M & L & H \\ \hline 
    H & L & H & L & H & M \\ \hline 
    H & L & H & L & M & H \\ \hline 
    H & L & H & L & L & H \\ \hline 
    H & M & L & H & H & L \\ \hline 
    H & M & L & H & M & M \\ \hline 
    H & M & L & H & L & M \\ \hline 
    H & M & L & M & H & L \\ \hline 
    H & M & L & M & M & M \\ \hline 
    H & M & L & M & L & M \\ \hline 
    H & M & L & L & H & M \\ \hline 
    H & M & L & L & M & M \\ \hline 
    H & M & L & L & L & H \\ \hline 
    H & M & M & H & H & M \\ \hline 
    H & M & M & H & M & M \\ \hline 
    H & M & M & H & L & H \\ \hline 
    H & M & M & M & H & M \\ \hline 
    H & M & M & M & M & H \\ \hline 
    H & M & M & M & L & H \\ \hline 
    H & M & M & L & H & M \\ \hline 
    H & M & M & L & M & H \\ \hline 
    H & M & M & L & L & H \\ \hline 
    H & M & H & H & H & H \\ \hline 
    H & M & H & H & M & H \\ \hline 
    H & M & H & H & L & H \\ \hline 
    H & M & H & M & H & H \\ \hline 
    H & M & H & M & M & H \\ \hline 
    H & M & H & M & L & VH \\ \hline 
    H & M & H & L & H & H \\ \hline 
    H & M & H & L & M & VH \\ \hline 
    H & M & H & L & L & VH \\ \hline 
    H & H & L & H & H & M \\ \hline 
    H & H & L & H & M & M \\ \hline 
    H & H & L & H & L & H \\ \hline 
    H & H & L & M & H & M \\ \hline 
    H & H & L & M & M & M \\ \hline 
    H & H & L & M & L & H \\ \hline 
    H & H & L & L & H & H \\ \hline 
    H & H & L & L & M & H \\ \hline 
    H & H & L & L & L & H \\ \hline 
    H & H & M & H & H & M \\ \hline 
    H & H & M & H & M & H \\ \hline 
    H & H & M & H & L & H \\ \hline 
    H & H & M & M & H & H \\ \hline 
    H & H & M & M & M & H \\ \hline 
    H & H & M & M & L & H \\ \hline 
    H & H & M & L & H & H \\ \hline 
    H & H & M & L & M & H \\ \hline 
    H & H & M & L & L & VH \\ \hline 
    H & H & H & H & H & H \\ \hline 
    H & H & H & H & M & H \\ \hline 
    H & H & H & H & L & VH \\ \hline 
    H & H & H & M & H & H \\ \hline 
    H & H & H & M & M & VH \\ \hline 
    H & H & H & M & L & VH \\ \hline 
    H & H & H & L & H & VH \\ \hline 
    H & H & H & L & M & VH \\ \hline 
    H & H & H & L & L & VH \\
    \hline
    \caption{Conjunto de reglas del sistema difuso de detección de incendios forestales.}
    \label{tab:fire-detection-rule-set}
    \end{longtable}
\end{center}

\section{Resultados y comparación de métodos}
En esta sección se presentan los resultados obtenidos al aplicar los métodos descritos en secciones anteriores y se realiza una comparativa de los mismos, usando diferentes t-normas e índices de solapamiento (en el caso del algoritmo \ref{algo:multi-overlap-interpolation-method}). 

\subsection{Método de Mamdani}
El primer método estudiado es el \emph{método de Mamdani}, presentado en la sección \ref{sec:mamdani-controller} (algoritmo \ref{algo:mamdani}). En la figura \ref{fig:mamdani-fire-detection-example} y en la tabla \ref{tab:fire-detection-mamdani-example} se presentan los resultados de aplicar el método de Mamdani, utilizando el conjunto de reglas definidas en la tabla \ref{tab:fire-detection-rule-set} a diferentes entradas. Las entradas del sistema son valores escalares para cada una de las variables lingüísticas:

\begin{center}
	%\def\arraystretch{1.5}
    \begin{longtable}{| c | c | c | c | c | c | c | c | c  | c  | c |}
    \hline
    \multirow{2}{*}{\textbf{Fig.}} & \multirow{2}{*}{\textbf{Temp.}} & \multirow{2}{*}{\textbf{Humo}} & \multirow{2}{*}{\textbf{Luz}}& \multirow{2}{*}{\textbf{Hum.}} & \multirow{2}{*}{\textbf{Dist.}} &  \multicolumn{5}{|c|}{\textbf{Riesgo (\%)}} \\ 
    \cline{7-11}
    & & & & & & \textbf{cent.  (\textasteriskcentered)} & \textbf{bis. (+)} & \textbf{som ($\triangledown$)} & \textbf{mom ($\square$)} & \textbf{lom ($\vartriangle$)}  \\ 
    \hline
    \ref{fig:fire-detection-mamdani-low} & 25 & 0 & 200 & 20 & 70 & 20 & 18 & 0 & 6 & 12 \\ 
    \hline
    \ref{fig:fire-detection-mamdani-low-medium} & 30 & 20 & 500 & 50 & 40 & 35 & 33 & 10 & 25 & 40 \\
    \hline
    \ref{fig:fire-detection-mamdani-medium} & 40 & 50 & 500 & 30 & 40 & 43 & 45 & 38 & 50 & 62 \\
    \hline
    \ref{fig:fire-detection-mamdani-high} & 100 & 90 & 900 & 0 & 20 & 80 & 82 & 84 & 92 & 100 \\
    \hline
    \caption{Resultados de aplicar el método de Mamdani a la detección de incendios forestales. Las gráficas correspondientes se muestran en la figura \ref{fig:mamdani-fire-detection-example}.}
    \end{longtable}
    \label{tab:fire-detection-mamdani-example}
\end{center}

Los valores de salida (riesgo (\%)) se obtienen aplicando los desdifusificadores definidos en la sección \ref{sec:desdifusificadores} al conjunto difuso obtenido al aplicar el método de Mamdani sobre las entradas (representados en la figura \ref{fig:mamdani-fire-detection-example}). 

Se puede comprobar que los resultados obtenidos con el método de Mamdani parecen ser intuitivamente correctos. En el primer ejemplo (fig. \ref{fig:fire-detection-mamdani-low}) las entradas sugieren un riesgo de incendio bajo, debido a la baja temperatura medida, el humo inexistente y la baja luminosidad. Aplicando el método de Mamdani efectivamente se obtiene un nivel de riesgo bajo (entre 0\% y 20\%). 

Como se puede ver, en general, los valores obtenidos con el desdifusificador \emph{centroide} y el \emph{bisector} son similares y pueden diferir ampliamente de los obtenidos con los métodos basados en el máximo. En este ejemplo se puede comprobar además que los desdifusificadores basados en el máximo pueden no dar resultados intuitivamente acertados, ya que el método del menor de los máximos da un resultado del 0\% de riesgo en el primer ejemplo, a pesar de que las entradas no son mínimas.

A medida que aumentan la temperatura, el humo y la luz y disminuyen la humedad y la distancia el riesgo de incendio aumenta (figuras \ref{fig:fire-detection-mamdani-medium} y \ref{fig:fire-detection-mamdani-high}), lo cual es coherente con lo expresado en el conjunto de reglas.

\begin{figure}[t]
	\centering
	\begin{subfigure}[b]{0.45\textwidth}
		\setlength\figureheight{3cm}
		\setlength\figurewidth{6cm}
		% This file was created by matlab2tikz v0.4.7 running on MATLAB 8.0.
% Copyright (c) 2008--2014, Nico Schlömer <nico.schloemer@gmail.com>
% All rights reserved.
% Minimal pgfplots version: 1.3
% 
%
% defining custom colors
\definecolor{mycolor1}{rgb}{0.00000,1.00000,1.00000}%
\definecolor{mycolor2}{rgb}{1.00000,0.00000,1.00000}%
%
\begin{tikzpicture}

\begin{axis}[%
width=\figurewidth,
height=\figureheight,
scale only axis,
xmin=0,
xmax=100,
xlabel={Riesgo (\%)},
ymin=0,
ymax=0.5,
axis x line*=bottom,
axis y line*=left
]
\addplot [color=black,solid,line width=1.2pt,forget plot]
  table[row sep=crcr]{0	0.5\\
1	0.5\\
2	0.5\\
3	0.5\\
4	0.5\\
5	0.5\\
6	0.5\\
7	0.5\\
8	0.5\\
9	0.5\\
10	0.5\\
11	0.5\\
12	0.5\\
13	0.48\\
14	0.44\\
15	0.4\\
16	0.36\\
17	0.32\\
18	0.3\\
19	0.3\\
20	0.3\\
21	0.3\\
22	0.3\\
23	0.3\\
24	0.3\\
25	0.3\\
26	0.3\\
27	0.3\\
28	0.3\\
29	0.3\\
30	0.3\\
31	0.3\\
32	0.3\\
33	0.3\\
34	0.3\\
35	0.3\\
36	0.3\\
37	0.3\\
38	0.3\\
39	0.3\\
40	0.3\\
41	0.3\\
42	0.3\\
43	0.28\\
44	0.24\\
45	0.2\\
46	0.16\\
47	0.12\\
48	0.08\\
49	0.04\\
50	0\\
51	0\\
52	0\\
53	0\\
54	0\\
55	0\\
56	0\\
57	0\\
58	0\\
59	0\\
60	0\\
61	0\\
62	0\\
63	0\\
64	0\\
65	0\\
66	0\\
67	0\\
68	0\\
69	0\\
70	0\\
71	0\\
72	0\\
73	0\\
74	0\\
75	0\\
76	0\\
77	0\\
78	0\\
79	0\\
80	0\\
81	0\\
82	0\\
83	0\\
84	0\\
85	0\\
86	0\\
87	0\\
88	0\\
89	0\\
90	0\\
91	0\\
92	0\\
93	0\\
94	0\\
95	0\\
96	0\\
97	0\\
98	0\\
99	0\\
100	0\\
};
\addplot [color=blue,line width=1.2pt,mark size=4.5pt,only marks,mark=asterisk,mark options={solid},forget plot]
  table[row sep=crcr]{20	0.3\\
};
\addplot [color=green,line width=1.2pt,mark size=4.5pt,only marks,mark=+,mark options={solid},forget plot]
  table[row sep=crcr]{18	0.3\\
};
\addplot [color=red,line width=1.2pt,mark size=3.2pt,only marks,mark=square,mark options={solid},forget plot]
  table[row sep=crcr]{6	0.5\\
};
\addplot [color=mycolor1,line width=1.2pt,mark size=3.0pt,only marks,mark=triangle,mark options={solid,rotate=180},forget plot]
  table[row sep=crcr]{0	0.5\\
};
\addplot [color=mycolor2,line width=1.2pt,mark size=3.0pt,only marks,mark=triangle,mark options={solid},forget plot]
  table[row sep=crcr]{12	0.5\\
};
\end{axis}
\end{tikzpicture}%
		\caption{Riesgo bajo.}\label{fig:fire-detection-mamdani-low}
	\end{subfigure}
	\qquad
	\begin{subfigure}[b]{0.45\textwidth}
		\setlength\figureheight{3cm}
		\setlength\figurewidth{6cm}
		\input{figures/fire-detection-results/mamdani/mamdani-T30_S20_L500_H50_D40.tikz}
		\caption{Riesgo bajo-medio.}\label{fig:fire-detection-mamdani-low-medium}
	\end{subfigure}
	
	\vspace{1cm}
	\begin{subfigure}[b]{0.45\textwidth}
		\setlength\figureheight{3cm}
		\setlength\figurewidth{6cm}
		% This file was created by matlab2tikz v0.4.7 running on MATLAB 8.0.
% Copyright (c) 2008--2014, Nico Schlömer <nico.schloemer@gmail.com>
% All rights reserved.
% Minimal pgfplots version: 1.3
% 
%
% defining custom colors
\definecolor{mycolor1}{rgb}{0.00000,1.00000,1.00000}%
\definecolor{mycolor2}{rgb}{1.00000,0.00000,1.00000}%
%
\begin{tikzpicture}

\begin{axis}[%
width=\figurewidth,
height=\figureheight,
scale only axis,
xmin=0,
xmax=100,
xlabel={Riesgo (\%)},
ymin=0,
ymax=0.5,
axis x line*=bottom,
axis y line*=left
]
\addplot [color=black,solid,line width=1.2pt,forget plot]
  table[row sep=crcr]{0	0\\
1	0.04\\
2	0.08\\
3	0.12\\
4	0.16\\
5	0.2\\
6	0.2\\
7	0.2\\
8	0.2\\
9	0.2\\
10	0.2\\
11	0.2\\
12	0.2\\
13	0.2\\
14	0.2\\
15	0.2\\
16	0.2\\
17	0.2\\
18	0.2\\
19	0.2\\
20	0.2\\
21	0.2\\
22	0.2\\
23	0.2\\
24	0.2\\
25	0.2\\
26	0.2\\
27	0.2\\
28	0.2\\
29	0.2\\
30	0.2\\
31	0.24\\
32	0.28\\
33	0.32\\
34	0.36\\
35	0.4\\
36	0.44\\
37	0.48\\
38	0.5\\
39	0.5\\
40	0.5\\
41	0.5\\
42	0.5\\
43	0.5\\
44	0.5\\
45	0.5\\
46	0.5\\
47	0.5\\
48	0.5\\
49	0.5\\
50	0.5\\
51	0.5\\
52	0.5\\
53	0.5\\
54	0.5\\
55	0.5\\
56	0.5\\
57	0.5\\
58	0.5\\
59	0.5\\
60	0.5\\
61	0.5\\
62	0.5\\
63	0.48\\
64	0.44\\
65	0.4\\
66	0.36\\
67	0.32\\
68	0.28\\
69	0.24\\
70	0.2\\
71	0.16\\
72	0.12\\
73	0.08\\
74	0.04\\
75	0\\
76	0\\
77	0\\
78	0\\
79	0\\
80	0\\
81	0\\
82	0\\
83	0\\
84	0\\
85	0\\
86	0\\
87	0\\
88	0\\
89	0\\
90	0\\
91	0\\
92	0\\
93	0\\
94	0\\
95	0\\
96	0\\
97	0\\
98	0\\
99	0\\
100	0\\
};
\addplot [color=blue,line width=1.2pt,mark size=4.5pt,only marks,mark=asterisk,mark options={solid},forget plot]
  table[row sep=crcr]{43	0.5\\
};
\addplot [color=green,line width=1.2pt,mark size=4.5pt,only marks,mark=+,mark options={solid},forget plot]
  table[row sep=crcr]{45	0.5\\
};
\addplot [color=red,line width=1.2pt,mark size=3.2pt,only marks,mark=square,mark options={solid},forget plot]
  table[row sep=crcr]{50	0.5\\
};
\addplot [color=mycolor1,line width=1.2pt,mark size=3.0pt,only marks,mark=triangle,mark options={solid,rotate=180},forget plot]
  table[row sep=crcr]{38	0.5\\
};
\addplot [color=mycolor2,line width=1.2pt,mark size=3.0pt,only marks,mark=triangle,mark options={solid},forget plot]
  table[row sep=crcr]{62	0.5\\
};
\end{axis}
\end{tikzpicture}%
		\caption{Riesgo medio.}\label{fig:fire-detection-mamdani-medium}
	\end{subfigure}
	\qquad
	\begin{subfigure}[b]{0.45\textwidth}
		\setlength\figureheight{3cm}
		\setlength\figurewidth{6cm}
		% This file was created by matlab2tikz v0.4.7 running on MATLAB 8.0.
% Copyright (c) 2008--2014, Nico Schlömer <nico.schloemer@gmail.com>
% All rights reserved.
% Minimal pgfplots version: 1.3
% 
%
% defining custom colors
\definecolor{mycolor1}{rgb}{0.00000,1.00000,1.00000}%
\definecolor{mycolor2}{rgb}{1.00000,0.00000,1.00000}%
%
\begin{tikzpicture}

\begin{axis}[%
width=\figurewidth,
height=\figureheight,
scale only axis,
xmin=0,
xmax=100,
xlabel={Riesgo (\%)},
ymin=0,
ymax=0.35,
axis x line*=bottom,
axis y line*=left
]
\addplot [color=black,solid,line width=1.2pt,forget plot]
  table[row sep=crcr]{0	0\\
1	0\\
2	0\\
3	0\\
4	0\\
5	0\\
6	0\\
7	0\\
8	0\\
9	0\\
10	0\\
11	0\\
12	0\\
13	0\\
14	0\\
15	0\\
16	0\\
17	0\\
18	0\\
19	0\\
20	0\\
21	0\\
22	0\\
23	0\\
24	0\\
25	0\\
26	0\\
27	0\\
28	0\\
29	0\\
30	0\\
31	0\\
32	0\\
33	0\\
34	0\\
35	0\\
36	0\\
37	0\\
38	0\\
39	0\\
40	0\\
41	0\\
42	0\\
43	0\\
44	0\\
45	0\\
46	0\\
47	0\\
48	0\\
49	0\\
50	0\\
51	0.04\\
52	0.08\\
53	0.12\\
54	0.16\\
55	0.2\\
56	0.2\\
57	0.2\\
58	0.2\\
59	0.2\\
60	0.2\\
61	0.2\\
62	0.2\\
63	0.2\\
64	0.2\\
65	0.2\\
66	0.2\\
67	0.2\\
68	0.2\\
69	0.2\\
70	0.2\\
71	0.2\\
72	0.2\\
73	0.2\\
74	0.2\\
75	0.2\\
76	0.2\\
77	0.2\\
78	0.2\\
79	0.2\\
80	0.2\\
81	0.24\\
82	0.28\\
83	0.32\\
84	0.333333333333333\\
85	0.333333333333333\\
86	0.333333333333333\\
87	0.333333333333333\\
88	0.333333333333333\\
89	0.333333333333333\\
90	0.333333333333333\\
91	0.333333333333333\\
92	0.333333333333333\\
93	0.333333333333333\\
94	0.333333333333333\\
95	0.333333333333333\\
96	0.333333333333333\\
97	0.333333333333333\\
98	0.333333333333333\\
99	0.333333333333333\\
100	0.333333333333333\\
};
\addplot [color=blue,line width=1.2pt,mark size=4.5pt,only marks,mark=asterisk,mark options={solid},forget plot]
  table[row sep=crcr]{80	0.2\\
};
\addplot [color=green,line width=1.2pt,mark size=4.5pt,only marks,mark=+,mark options={solid},forget plot]
  table[row sep=crcr]{82	0.28\\
};
\addplot [color=red,line width=1.2pt,mark size=3.2pt,only marks,mark=square,mark options={solid},forget plot]
  table[row sep=crcr]{92	0.333333333333333\\
};
\addplot [color=mycolor1,line width=1.2pt,mark size=3.0pt,only marks,mark=triangle,mark options={solid,rotate=180},forget plot]
  table[row sep=crcr]{84	0.333333333333333\\
};
\addplot [color=mycolor2,line width=1.2pt,mark size=3.0pt,only marks,mark=triangle,mark options={solid},forget plot]
  table[row sep=crcr]{100	0.333333333333333\\
};
\end{axis}
\end{tikzpicture}%
		\caption{Riesgo alto.}\label{fig:fire-detection-mamdani-high}
	\end{subfigure}
	\caption{Método de Mamdani aplicado a la detección de incendios forestales.}
	\label{fig:mamdani-fire-detection-example}
\end{figure}

\subsection{Método de interpolación basada índices de solapamiento}
Para ejecutar el algoritmo \ref{algo:multi-overlap-interpolation-method} es necesario seleccionar una t-norma(\emph{T}), un operador de agregación(\emph{M}) y un índice de solapamiento (\emph{O}). En este estudio se va a utilizar la media aritmética como operador de agregación. En cuanto a las t-normas, se van a utilizar las siguientes:

\begin{itemize}
	\item \bfseries Mínimo: $T_{min}(x_{1},x_{2},\cdots,x_{n}) = \min\{x_{1},x_{2},\cdots,x_{n}\}$
	\item \bfseries Producto: $T_{prod}(x_{1},x_{2},\cdots,x_{n}) = x_{1}*x_{2} * \cdots * x_{n}$
	\item \bfseries Media geométrica: $T_{geo}(x_{1},x_{2},\cdots,x_{n}) = (\prod\limits_{i=1}^{n}x_{i})^{\frac{1}{n}}$
	\item \bfseries Media armónica: $T_{harm}(x_{1},x_{2},\cdots,x_{n}) = \frac{n}{\sum_{i=1}^{n}\frac{1}{x_i}}$
	\item \bfseries Seno: $T_{sin}(x_{1},x_{2},\cdots,x_{n}) = \frac{1}{n}\sin(\frac{\pi}{2}(x_1 * x_2 * \ldots * x_n))^{\frac{1}{4}}$
	\item \bfseries Einstein: $T_{einstein}(x_{1},x_{2},\cdots,x_{n}) = \frac{(x_{1}*x_{2} * \cdots * x_{n})}{1 + (1 - x_{1} )* \ldots * (1 - x_{n} )}$
\end{itemize}

Además se van a utilizar los siguientes índices de solapamiento:

\begin{itemize}
	\item \bfseries Media de productos: $O_{\pi}(A,B) = \frac{1}{n}\sum_{i=1}^{n}\mu_A(x_i)*\mu_B(x_i)$
	\item \bfseries Media de mínimos: $O_{avgmin}(A,B) = \frac{1}{n}\sum_{i=1}^{n}\min(\mu_A(x_i),\mu_B(x_i))$
	\item \bfseries Máximo de mínimos: $O_{Z}(A,B) = \max\limits_{i=1}^{n}(\min(\mu_A(x_i),\mu_B(x_i)))$
	\item \bfseries Media de raíces cuadradas: $O_{\sqrt{\text{ }}}(A,B) =  \frac{1}{n}\sum_{i=1}^{n}\sqrt{\mu_A(x_i)*\mu_B(x_i)}$
	\item \bfseries Media del seno:  $O_{sin}(A, B) = \frac{1}{n}\sum_{i=1}^{n}\sin(\frac{\pi}{2}(\mu_A(x_i)*\mu_B(x_i)))^{\frac{1}{4}}$
\end{itemize}

Así por ejemplo, si tomamos $M = \text{Media aritmética}$, $T = T_{min}$ y $O = O_Z$ y aplicamos el método de interpolación(algoritmo \ref{algo:multi-overlap-interpolation-method}) a las entradas de la tabla \ref{tab:fire-detection-mamdani-example} se obtienen los resultados mostrados en la tabla \ref{tab:fire-detection-interpolation-example} y la figura \ref{fig:interpolation-fire-detection-example}:

\begin{center}
	%\def\arraystretch{1.5}
    \begin{longtable}{| c | c | c | c | c | c | c | c | c  | c  | c |}
    \hline
    \multirow{2}{*}{\textbf{Fig.}} & \multirow{2}{*}{\textbf{Temp.}} & \multirow{2}{*}{\textbf{Humo}} & \multirow{2}{*}{\textbf{Luz}}& \multirow{2}{*}{\textbf{Hum.}} & \multirow{2}{*}{\textbf{Dist.}} &  \multicolumn{5}{|c|}{\textbf{Riesgo (\%)}} \\ 
    \cline{7-11}
    & & & & & & \textbf{cent.  (\textasteriskcentered)} & \textbf{bis. (+)} & \textbf{som ($\triangledown$)} & \textbf{mom ($\square$)} & \textbf{lom ($\vartriangle$)}  \\ 
    \hline
    \ref{fig:fire-detection-interpolation-low} & 25 & 0 & 200 & 20 & 70 & 18 & 15 & 8 & 10 & 12  \\ 
    \hline
    \ref{fig:fire-detection-interpolation-low-medium} & 30 & 20 & 500 & 50 & 40 & 30 & 30 & 32 & 36 & 40 \\
    \hline
    \ref{fig:fire-detection-interpolation-medium} & 40 & 50 & 500 & 30 & 40 & 41 & 41 & 38 & 42 & 45 \\
    \hline
    \ref{fig:fire-detection-interpolation-high} & 100 & 90 & 900 & 0 & 20 & 85 & 87 & 84 & 90 & 95 \\
    \hline
    \caption{Resultados de aplicar el método de interpolación con $M = \text{Media aritmética}$, $T = T_{min}$ y $O = O_Z$  a la detección de incendios forestales. Las gráficas correspondientes se muestran en la figura \ref{fig:interpolation-fire-detection-example}.}
    \end{longtable}
    \label{tab:fire-detection-interpolation-example}
\end{center}

\begin{figure}[t]
	\centering
	\begin{subfigure}[b]{0.45\textwidth}
		\setlength\figureheight{3cm}
		\setlength\figurewidth{6cm}
		% This file was created by matlab2tikz v0.4.7 running on MATLAB 8.0.
% Copyright (c) 2008--2014, Nico Schlömer <nico.schloemer@gmail.com>
% All rights reserved.
% Minimal pgfplots version: 1.3
% 
%
% defining custom colors
\definecolor{mycolor1}{rgb}{0.00000,1.00000,1.00000}%
\definecolor{mycolor2}{rgb}{1.00000,0.00000,1.00000}%
%
\begin{tikzpicture}

\begin{axis}[%
width=\figurewidth,
height=\figureheight,
scale only axis,
xmin=0,
xmax=100,
xlabel={Riesgo (\%)},
ymin=0,
ymax=0.01,
axis x line*=bottom,
axis y line*=left
]
\addplot [color=black,solid,line width=1.2pt,forget plot]
  table[row sep=crcr]{0	0.00617283950617284\\
1	0.00666666666666667\\
2	0.0071604938271605\\
3	0.00765432098765432\\
4	0.00814814814814815\\
5	0.00864197530864198\\
6	0.0091358024691358\\
7	0.00938271604938272\\
8	0.00946502057613169\\
9	0.00946502057613169\\
10	0.00946502057613169\\
11	0.00946502057613169\\
12	0.00946502057613169\\
13	0.00938271604938272\\
14	0.00921810699588477\\
15	0.00905349794238683\\
16	0.00888888888888889\\
17	0.00872427983539095\\
18	0.008559670781893\\
19	0.00823045267489712\\
20	0.00740740740740741\\
21	0.0065843621399177\\
22	0.00576131687242798\\
23	0.00493827160493827\\
24	0.00411522633744856\\
25	0.00329218106995885\\
26	0.00329218106995885\\
27	0.00329218106995885\\
28	0.00329218106995885\\
29	0.00329218106995885\\
30	0.00329218106995885\\
31	0.00329218106995885\\
32	0.00329218106995885\\
33	0.00329218106995885\\
34	0.00329218106995885\\
35	0.00329218106995885\\
36	0.00329218106995885\\
37	0.00329218106995885\\
38	0.00329218106995885\\
39	0.00329218106995885\\
40	0.00329218106995885\\
41	0.00329218106995885\\
42	0.00329218106995885\\
43	0.00320987654320988\\
44	0.00296296296296296\\
45	0.00246913580246914\\
46	0.00197530864197531\\
47	0.00148148148148148\\
48	0.000987654320987654\\
49	0.000493827160493828\\
50	0\\
51	0\\
52	0\\
53	0\\
54	0\\
55	0\\
56	0\\
57	0\\
58	0\\
59	0\\
60	0\\
61	0\\
62	0\\
63	0\\
64	0\\
65	0\\
66	0\\
67	0\\
68	0\\
69	0\\
70	0\\
71	0\\
72	0\\
73	0\\
74	0\\
75	0\\
76	0\\
77	0\\
78	0\\
79	0\\
80	0\\
81	0\\
82	0\\
83	0\\
84	0\\
85	0\\
86	0\\
87	0\\
88	0\\
89	0\\
90	0\\
91	0\\
92	0\\
93	0\\
94	0\\
95	0\\
96	0\\
97	0\\
98	0\\
99	0\\
100	0\\
};
\addplot [color=blue,line width=1.2pt,mark size=4.5pt,only marks,mark=asterisk,mark options={solid},forget plot]
  table[row sep=crcr]{18	0.008559670781893\\
};
\addplot [color=green,line width=1.2pt,mark size=4.5pt,only marks,mark=+,mark options={solid},forget plot]
  table[row sep=crcr]{15	0.00905349794238683\\
};
\addplot [color=red,line width=1.2pt,mark size=3.2pt,only marks,mark=square,mark options={solid},forget plot]
  table[row sep=crcr]{10	0.00946502057613169\\
};
\addplot [color=mycolor1,line width=1.2pt,mark size=3.0pt,only marks,mark=triangle,mark options={solid,rotate=180},forget plot]
  table[row sep=crcr]{8	0.00946502057613169\\
};
\addplot [color=mycolor2,line width=1.2pt,mark size=3.0pt,only marks,mark=triangle,mark options={solid},forget plot]
  table[row sep=crcr]{12	0.00946502057613169\\
};
\end{axis}
\end{tikzpicture}%
		\caption{Riesgo bajo.}\label{fig:fire-detection-interpolation-low}
	\end{subfigure}
	\qquad
	\begin{subfigure}[b]{0.45\textwidth}
		\setlength\figureheight{3cm}
		\setlength\figurewidth{6cm}
		\input{figures/fire-detection-results/interpolation/example/interpolation-O-Oz_Tmin_Mmean--T30_S20_L500_H50_D40.tikz}
		\caption{Riesgo bajo-medio.}\label{fig:fire-detection-interpolation-low-medium}
	\end{subfigure}
	
	\vspace{1cm}
	\begin{subfigure}[b]{0.45\textwidth}
		\setlength\figureheight{3cm}
		\setlength\figurewidth{6cm}
		% This file was created by matlab2tikz v0.4.7 running on MATLAB 8.0.
% Copyright (c) 2008--2014, Nico Schlömer <nico.schloemer@gmail.com>
% All rights reserved.
% Minimal pgfplots version: 1.3
% 
%
% defining custom colors
\definecolor{mycolor1}{rgb}{0.00000,1.00000,1.00000}%
\definecolor{mycolor2}{rgb}{1.00000,0.00000,1.00000}%
%
\begin{tikzpicture}

\begin{axis}[%
width=\figurewidth,
height=\figureheight,
scale only axis,
xmin=0,
xmax=100,
xlabel={Riesgo (\%)},
ymin=0,
ymax=0.005,
axis x line*=bottom,
axis y line*=left
]
\addplot [color=black,solid,line width=1.2pt,forget plot]
  table[row sep=crcr]{0	0\\
1	0.000329218106995885\\
2	0.00065843621399177\\
3	0.000987654320987654\\
4	0.00131687242798354\\
5	0.00164609053497942\\
6	0.00164609053497942\\
7	0.00164609053497942\\
8	0.00164609053497942\\
9	0.00164609053497942\\
10	0.00164609053497942\\
11	0.00164609053497942\\
12	0.00164609053497942\\
13	0.00164609053497942\\
14	0.00164609053497942\\
15	0.00164609053497942\\
16	0.00164609053497942\\
17	0.00164609053497942\\
18	0.00164609053497942\\
19	0.00164609053497942\\
20	0.00164609053497942\\
21	0.00164609053497942\\
22	0.00164609053497942\\
23	0.00164609053497942\\
24	0.00164609053497942\\
25	0.00164609053497942\\
26	0.00197530864197531\\
27	0.00230452674897119\\
28	0.00263374485596708\\
29	0.00296296296296296\\
30	0.00329218106995885\\
31	0.00362139917695473\\
32	0.00382716049382716\\
33	0.0039917695473251\\
34	0.00415637860082304\\
35	0.00432098765432099\\
36	0.00448559670781893\\
37	0.00465020576131687\\
38	0.00473251028806584\\
39	0.00473251028806584\\
40	0.00473251028806584\\
41	0.00473251028806584\\
42	0.00473251028806584\\
43	0.00473251028806584\\
44	0.00473251028806584\\
45	0.00473251028806584\\
46	0.00440329218106996\\
47	0.00407407407407407\\
48	0.00374485596707819\\
49	0.0034156378600823\\
50	0.00308641975308642\\
51	0.00308641975308642\\
52	0.00308641975308642\\
53	0.00308641975308642\\
54	0.00308641975308642\\
55	0.00308641975308642\\
56	0.00308641975308642\\
57	0.00308641975308642\\
58	0.00308641975308642\\
59	0.00308641975308642\\
60	0.00308641975308642\\
61	0.00308641975308642\\
62	0.00308641975308642\\
63	0.00300411522633745\\
64	0.00283950617283951\\
65	0.00267489711934156\\
66	0.00251028806584362\\
67	0.00234567901234568\\
68	0.00218106995884774\\
69	0.00197530864197531\\
70	0.00164609053497942\\
71	0.00131687242798354\\
72	0.000987654320987654\\
73	0.000658436213991769\\
74	0.000329218106995885\\
75	0\\
76	0\\
77	0\\
78	0\\
79	0\\
80	0\\
81	0\\
82	0\\
83	0\\
84	0\\
85	0\\
86	0\\
87	0\\
88	0\\
89	0\\
90	0\\
91	0\\
92	0\\
93	0\\
94	0\\
95	0\\
96	0\\
97	0\\
98	0\\
99	0\\
100	0\\
};
\addplot [color=blue,line width=1.2pt,mark size=4.5pt,only marks,mark=asterisk,mark options={solid},forget plot]
  table[row sep=crcr]{41	0.00473251028806584\\
};
\addplot [color=green,line width=1.2pt,mark size=4.5pt,only marks,mark=+,mark options={solid},forget plot]
  table[row sep=crcr]{41	0.00473251028806584\\
};
\addplot [color=red,line width=1.2pt,mark size=3.2pt,only marks,mark=square,mark options={solid},forget plot]
  table[row sep=crcr]{42	0.00473251028806584\\
};
\addplot [color=mycolor1,line width=1.2pt,mark size=3.0pt,only marks,mark=triangle,mark options={solid,rotate=180},forget plot]
  table[row sep=crcr]{38	0.00473251028806584\\
};
\addplot [color=mycolor2,line width=1.2pt,mark size=3.0pt,only marks,mark=triangle,mark options={solid},forget plot]
  table[row sep=crcr]{45	0.00473251028806584\\
};
\end{axis}
\end{tikzpicture}%
		\caption{Riesgo medio.}\label{fig:fire-detection-interpolation-medium}
	\end{subfigure}
	\qquad
	\begin{subfigure}[b]{0.45\textwidth}
		\setlength\figureheight{3cm}
		\setlength\figurewidth{6cm}
		% This file was created by matlab2tikz v0.4.7 running on MATLAB 8.0.
% Copyright (c) 2008--2014, Nico Schlömer <nico.schloemer@gmail.com>
% All rights reserved.
% Minimal pgfplots version: 1.3
% 
%
% defining custom colors
\definecolor{mycolor1}{rgb}{0.00000,1.00000,1.00000}%
\definecolor{mycolor2}{rgb}{1.00000,0.00000,1.00000}%
%
\begin{tikzpicture}

\begin{axis}[%
width=\figurewidth,
height=\figureheight,
scale only axis,
xmin=0,
xmax=100,
xlabel={Riesgo (\%)},
ymin=0,
ymax=0.0045,
axis x line*=bottom,
axis y line*=left
]
\addplot [color=black,solid,line width=1.2pt,forget plot]
  table[row sep=crcr]{0	0\\
1	0\\
2	0\\
3	0\\
4	0\\
5	0\\
6	0\\
7	0\\
8	0\\
9	0\\
10	0\\
11	0\\
12	0\\
13	0\\
14	0\\
15	0\\
16	0\\
17	0\\
18	0\\
19	0\\
20	0\\
21	0\\
22	0\\
23	0\\
24	0\\
25	0\\
26	0\\
27	0\\
28	0\\
29	0\\
30	0\\
31	0\\
32	0\\
33	0\\
34	0\\
35	0\\
36	0\\
37	0\\
38	0\\
39	0\\
40	0\\
41	0\\
42	0\\
43	0\\
44	0\\
45	0\\
46	0\\
47	0\\
48	0\\
49	0\\
50	0\\
51	0.000164609053497942\\
52	0.000329218106995885\\
53	0.000493827160493827\\
54	0.00065843621399177\\
55	0.000823045267489712\\
56	0.000823045267489712\\
57	0.000823045267489712\\
58	0.000823045267489712\\
59	0.000823045267489712\\
60	0.000823045267489712\\
61	0.000823045267489712\\
62	0.000823045267489712\\
63	0.000823045267489712\\
64	0.000823045267489712\\
65	0.000823045267489712\\
66	0.000823045267489712\\
67	0.000823045267489712\\
68	0.000823045267489712\\
69	0.000823045267489712\\
70	0.000823045267489712\\
71	0.000823045267489712\\
72	0.000823045267489712\\
73	0.000823045267489712\\
74	0.000823045267489712\\
75	0.000823045267489712\\
76	0.00131687242798354\\
77	0.00181069958847737\\
78	0.00230452674897119\\
79	0.00279835390946502\\
80	0.00329218106995885\\
81	0.00362139917695473\\
82	0.00395061728395062\\
83	0.0042798353909465\\
84	0.00438957475994513\\
85	0.00438957475994513\\
86	0.00438957475994513\\
87	0.00438957475994513\\
88	0.00438957475994513\\
89	0.00438957475994513\\
90	0.00438957475994513\\
91	0.00438957475994513\\
92	0.00438957475994513\\
93	0.00438957475994513\\
94	0.00438957475994513\\
95	0.00438957475994513\\
96	0.00422496570644719\\
97	0.00406035665294925\\
98	0.0038957475994513\\
99	0.00373113854595336\\
100	0.00356652949245542\\
};
\addplot [color=blue,line width=1.2pt,mark size=4.5pt,only marks,mark=asterisk,mark options={solid},forget plot]
  table[row sep=crcr]{85	0.00438957475994513\\
};
\addplot [color=green,line width=1.2pt,mark size=4.5pt,only marks,mark=+,mark options={solid},forget plot]
  table[row sep=crcr]{87	0.00438957475994513\\
};
\addplot [color=red,line width=1.2pt,mark size=3.2pt,only marks,mark=square,mark options={solid},forget plot]
  table[row sep=crcr]{90	0.00438957475994513\\
};
\addplot [color=mycolor1,line width=1.2pt,mark size=3.0pt,only marks,mark=triangle,mark options={solid,rotate=180},forget plot]
  table[row sep=crcr]{84	0.00438957475994513\\
};
\addplot [color=mycolor2,line width=1.2pt,mark size=3.0pt,only marks,mark=triangle,mark options={solid},forget plot]
  table[row sep=crcr]{95	0.00438957475994513\\
};
\end{axis}
\end{tikzpicture}%
		\caption{Riesgo alto.}\label{fig:fire-detection-interpolation-high}
	\end{subfigure}
	\caption{Método de interpolación aplicado a la detección de incendios forestales.}
	\label{fig:interpolation-fire-detection-example}
\end{figure}

En este caso, los resultados son bastante similares a los obtenidos con el método de Mamdani, a pesar de que se obtienen conjuntos de salida con características diferentes. Los valores obtenidos con los desdifusificadores \emph{centroide} y \emph{bisector} son bastante parecidos a los obtenidos con el método de Mamdani, mientras que en los desdifusificadores basados en el máximo(\emph{som}, \emph{mom} y \emph{lom}) las diferencias son más notables.