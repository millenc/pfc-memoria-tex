% !TeX spellcheck = es_ES
\chapter{Detección de incendios forestales}
\label{cha:deteccion-incendios-forestales}
En este capítulo se presenta una aplicación práctica de los métodos de inferencia difusa basados en reglas, descritos en los capítulos anteriores (REFERENCIA). El objetivo es definir un sistema basado en reglas capaz de determinar el riesgo de incendios forestales a partir de mediciones de algunas magnitudes tales como la temperatura, el humo, la humedad etc. Estas magnitudes constituirán las entradas del sistema de inferencia que obtendrá una estimación cuantitativa del riesgo del incendio forestal.

\section{Red de sensores inalámbricos}
El primer paso para la determinación del riesgo de un incendio forestal es la medición y toma de datos ambientales relacionados con dicho incendio. Para ello se puede desplegar una red de sensores inalámbricos (\emph{Wireless Sensor Networks}, WSN), que han sido desarrolladas y utilizadas en una gran variedad de aplicaciones en áreas tales como automoción \cite{hsin2007}, defensa, medicina, agricultura \cite{tao2008}\cite{hwang2010}, etc. 

Algunas aplicaciones de este tipo de sistemas relacionadas con los riesgos ambientales incluyen, por ejemplo, el control de movimientos de personas y ganado, monitorización de factores ambientales que afectan a la calidad de los cultivos, detección de incendios forestales, mediciones meteorológicas y detección de inundaciones etc. \cite{Akyildiz2002}

Las redes de sensores inalámbricos están diseñadas para monitorizar y controlar eventos en lugares que presentan riesgos ambientales tales como bosques, terrenos montañosos etc. Por esta razón se diseña este tipo de sistemas de forma que sean lo más autónomos posible. Esto incluye, por ejemplo, la utilización de energías renovables (típicamente energía solar), que posibilitan que los sensores desplegados en el terreno sean totalmente auto-suficientes.

El diseño y despliegue de una red de sensores inalámbricos queda totalmente fuera del alcance de este proyecto y por tanto no se va a entrar en ningún tipo de detalle técnico sobre estos sistemas. El objetivo de este proyecto es evaluar algoritmos de inferencia difusa sobre los valores que serían entregados al sistema de decisión en una aplicación real . 

Las mediciones realizadas por la red de sensores constituyen las entradas del sistema de inferencia y decisión. Estas entradas son, originalmente, valores escalares que son transformados por el sistema de lógica difusa en valores difusos (por medio de variables lingüísticas). Esto permite realizar, sobre estas mediciones, un proceso de razonamiento aproximado que puede tratar mejor las imprecisiones, en comparación con utilizar dichos valores escalares directamente.


\section{Magnitudes medidas y variables lingüísticas}

Las entradas del sistema difuso son valores escalares de magnitudes que describen el incendio forestal, tales como la temperatura o la luminosidad, medidas por la red de sensores. Para cada una de estas magnitudes se define una variable lingüística con tres valores posibles (definidas por sus correspondientes funciones de pertenencia). Así pues, los valores escalares de entrada serán transformados en conjuntos difusos que posteriormente serán comparados con las variables lingüísticas según las reglas definidas, para obtener como salida el riesgo de incendio (que es también una variable lingüística). Estas variables lingüísticas así como sus correspondientes universos (rangos de valores posibles) son:

\begin{enumerate}[label=($\chi_\arabic*$),ref=(X\arabic*)]
   \item \bfseries Temperatura: \normalfont medida en grados centígrados (0ºC a 120ºC).
   \item \bfseries Humo: \normalfont medida en partes por millón (0 a 100ppm).
   \item \bfseries Luz: \normalfont medida en lux (0 a 1000 lux).
   \item \bfseries Humedad: \normalfont medida en partes por millón (0 a 100ppm).
   \item \bfseries Distancia: \normalfont medida en metros (0 a 80m).
\end{enumerate}

Para cada una de estas variables lingüísticas se definen tres valores posibles: \emph{Baja} (L,\emph{Low}), \emph{Media} (M, \emph{Medium}) y \emph{Alta} (H, \emph{High}) que los valores tienen el sentido de \emph{Cerca} (C, \emph{Close}), \emph{Media} (M, \emph{Medium}) y \emph{Lejos} (F, \emph{Far}) respectivamente.

La salida del sistema viene dada por la variable lingüística $y = $ \emph{``Riesgo de incendio''}, que determina el riesgo de incendio forestal en una escala del 0 al 100 (\%). Esta variable lingüística puede tomar los valores: \emph{Muy bajo} (VL, \emph{Very Low}), \emph{Bajo} (L, \emph{Low}), \emph{Medio} (M, \emph{Medium}), \emph{Alto} (H, \emph{High}) y \emph{Muy Alto} (VH, \emph{Very High}). 

En la figura \ref{fig:fire-detection-lang-variables} se representan las variables lingüística descritas anteriormente. Como se puede ver, se han elegido funciones de pertenencia lineales para modelar los valores de las variables lingüísticas.

\begin{figure}[t]
	\centering
	\begin{subfigure}[b]{0.45\textwidth}
		\setlength\figureheight{2.5cm}
		\setlength\figurewidth{6cm}
		\input{figures/fire-detection-lang-variables/temp_lang_variable.tikz}
		\caption{$\chi_1$ - Temperatura (ºC)}
		\label{fig:temp-lang-variable}
	\end{subfigure}
	\qquad
	\begin{subfigure}[b]{0.45\textwidth}
		\setlength\figureheight{2.5cm}
		\setlength\figurewidth{6cm}
		\input{figures/fire-detection-lang-variables/smoke_lang_variable.tikz}
		\caption{$\chi_2$ - Humo (ppm)}
		\label{fig:smoke-lang-variable}
	\end{subfigure}
	
	\vspace{1 cm}
	\begin{subfigure}[b]{0.45\textwidth}
		\setlength\figureheight{2.5cm}
		\setlength\figurewidth{6cm}
		\input{figures/fire-detection-lang-variables/light_lang_variable.tikz}
		\caption{$\chi_3$ - Luz (lux)}
		\label{fig:light-lang-variable}
	\end{subfigure}
	\qquad
	\begin{subfigure}[b]{0.45\textwidth}
		\setlength\figureheight{2.5cm}
		\setlength\figurewidth{6cm}
		\input{figures/fire-detection-lang-variables/humidity_lang_variable.tikz}
		\caption{$\chi_4$ - Humedad (ppm)}
		\label{fig:humidity-lang-variable}
	\end{subfigure}
	
	\vspace{1 cm}
	\begin{subfigure}[b]{0.45\textwidth}
		\setlength\figureheight{2.5cm}
		\setlength\figurewidth{6cm}
		\input{figures/fire-detection-lang-variables/distance_lang_variable.tikz}
		\caption{$\chi_5$ - Distancia (m)}
		\label{fig:distance-lang-variable}
	\end{subfigure}
	\qquad
	\begin{subfigure}[b]{0.45\textwidth}
		\setlength\figureheight{2.5cm}
		\setlength\figurewidth{6cm}
		\input{figures/fire-detection-lang-variables/threat_lang_variable.tikz}
		\caption{$y$ - Riesgo de incendio (\%)}
		\label{fig:threat-lang-variable}
	\end{subfigure}
	\caption{Variables lingüísticas utilizadas en la determinación de riesgo de incendios.}
	\label{fig:fire-detection-lang-variables}
\end{figure}

\section{Conjunto de reglas}


