% !TeX spellcheck = es_ES
\chapter{Conclusiones}
En este trabajo se ha presentado un nuevo método de inferencia difusa basado en índices de solapamiento. En los sistemas difusos basados en reglas, las entradas del sistema se comparan con los valores de las variables lingüísticas definidas como antecedentes en dichas reglas. De alguna forma, hay que calcular ``cómo de coincidentes'' son las entradas a dichos antecedentes. Los índices de solapamiento dan, precisamente, una medida del solapamiento entre dos conjuntos difusos y, por tanto, basar un método de inferencia en ellos resulta algo natural.

La principal ventaja de este método, frente a otros métodos clásicos como el controlador de Mamdani, es que es una generalización del método de interpolación y permite la utilización y combinación de diferentes índices de solapamiento, t-normas y operadores de agregación. Una gran ventaja de este método es que está construido a partir de funciones bien conocidas y ampliamente estudiadas como las t-normas, los operadores de agregación y las funciones de solapamiento.

En la sección \ref{sec:fire-detection-results} se ha presentado una comparativa de los resultados obtenidos con el método basado en índices de solapamiento y el método clásico de Mamdani aplicados al caso práctico de la detección y determinación de riesgos ambientales. Como se puede comprobar en dicha comparativa, los resultados obtenidos con ambos métodos son intuitivamente correctos y en general bastante similares entre sí. 

A pesar de ello, se puede comprobar que, dependiendo de las diferentes t-normas e índices de solapamiento elegidos, los resultados obtenidos pueden variar ligeramente. En general, el método de interpolación basado en índices de solapamiento produce mejores resultados, dado que la combinación convexa de índices de solapamiento es también un índice de solapamiento. En una aplicación real, el método de interpolación basado en índices de solapamiento tiene la ventaja de que permite probar diferentes combinaciones de funciones y utilizar aquellas que mejores resultados ofrezcan (comparando con los resultados esperados por un experto). Gracias al teorema de construcción de índices de solapamiento (ec. \ref{eq:construction-overlap-index}), se pueden probar multitud de índices diferentes, construidos a partir de funciones muy conocidas y estudiadas como los operadores de agregación y las funciones de solapamiento.

Otra ventaja del método de interpolación basado en índices de solapamiento es que, a pesar de proporcionar una gran flexibilidad y potencia, es un método relativamente sencillo de entender e implementar en cualquier lenguaje de programación (en este caso se ha implementado en MATLAB).

\chapter{Líneas futuras}
En este trabajo se ha presentado un nuevo método de interpolación basado en índices de solapamiento, propuesto en \cite{bustince2013overlap}. En ese mismo artículo se propone también un algoritmo similar basado en índices de solapamiento para problemas de clasificación. En dicho algoritmo se utilizan diferentes operadores de agregación para agregar ciertos conjuntos difusos obtenidos en el proceso de clasificación. Para obtener un resultado único, se utilizan funciones penalty \cite{Bustince2014} para determinar qué combinación de operadores de agregación produce resultados más similares a los valores agregados.

Esta misma idea se puede aplicar al método de interpolación basado en índices de solapamiento. Como se puede ver en la línea 7 del algoritmo \ref{algo:multi-overlap-interpolation-method},  se utiliza un operador de agregación para agregar los conjuntos difusos obtenidos en cada regla. En este paso se podrían utilizar diferentes operadores de agregación y determinar mediante funciones penalty qué operador o combinación de operadores de agregación produce los resultados más similares a los conjuntos agregados. Sería interesante comprobar si esta aplicación de funciones penalty produce mejores resultados y compararlos con el método de interpolación ``básico'' presentado en este trabajo.

En el ámbito de los sistemas difusos basados en reglas existen otros métodos y técnicas dignas de mencionar y que sería interesante estudiar en futuros trabajos. Uno de los campos más importante e interesante es el \emph{aprendizaje de reglas difusas}. Habitualmente los conjuntos de reglas se construyen utilizando conocimiento experto sobre el problema. Sin embargo, también es posible generar dicho conjunto de reglas mediante procesos de aprendizaje, similares a los de las redes neuronales. En estos procesos de aprendizaje se utilizan ejemplos numéricos en los que se indican las entradas y sus correspondientes salidas. A partir de estos ejemplos es posible generar reglas difusas que transforman las entradas del sistema en las salidas esperadas.

En \cite{Alcala99} y \cite{Serrurier2007} se presentan algunos métodos y técnicas para el aprendizaje de reglas difusas. Otro tipo de técnicas interesantes relacionadas con el aprendizaje de reglas son aquellos relacionadas con la reducción y optimización de los conjuntos de reglas aprendidos. Una propiedad de estos métodos de aprendizaje es que tienden a generar conjuntos de reglas ``sobredimensionados'', es decir, con reglas redundantes que pueden degradar el rendimiento del sistema difuso. Por ello existen multitud de estudios y trabajos relacionados con la reducción y optimización de conjuntos de reglas que pueden ser interesantes de cara a futuros trabajos.