% !TeX spellcheck = es_ES
\chapter{Conclusiones}
En este trabajo se ha presentado un nuevo método de lógica difusa basada en reglas, cuya principal característica es que está basado en índices de solapamiento. En los sistemas difusos basados en reglas, las entradas del sistema se comparan con los valores de las variables lingüísticas definidas como antecedentes en dichas reglas. De alguna forma, hay que calcular ``cómo de parecidas'' son las entradas a dichos antecedentes. Los índices de solapamiento dan, precisamente, una medida de la similitud entre dos conjuntos difusos y por tanto basar un método de inferencia en ellos resulta algo natural.

La principal ventaja de este método, frente a otros métodos clásicos como el controlador de Mamdani, es que es una generalización del método de interpolación y permite la utilización y combinación de diferentes índices de solapamiento, t-normas y operadores de agregación. Una gran ventaja de este método es que está construido a partir de funciones bien conocidas y ampliamente estudiadas como las t-normas, los operadores de agregación y las funciones de solapamiento.

En la sección \ref{sec:fire-detection-results} se ha presentado una comparativa de los resultados obtenidos con el método basado en índices de solapamiento y el método clásico de Mamdani aplicados al caso práctico de la detección y determinación de riesgos ambientales. Como se puede comprobar en dicha comparativa, los resultados obtenidos con ambos métodos son intuitivamente correctos y en general bastante similares entre sí. 

A pesar de ello, se puede comprobar que, dependiendo de las diferentes t-normas e índices de solapamiento elegidos, los resultados obtenidos pueden variar ligeramente. Sin embargo, en una aplicación real el método de interpolación basado en índices de solapamiento tiene la ventaja de que permite probar diferentes combinaciones de funciones y utilizar aquellas que mejores resultados ofrezcan (comparando con los resultados esperados por un experto). Gracias al teorema de construcción de índices de solapamiento (ec. \ref{eq:construction-overlap-index}) además se pueden probar multitud de índices diferentes, construidos a partir de funciones muy conocidas y estudiadas como los operadores de agregación y las funciones de solapamiento.

Otra ventaja del método de interpolación basado en índices de solapamiento es que, a pesar de proporcionar una gran flexibilidad y potencia, es un método relativamente sencillo de implementar en cualquier lenguaje de programación (en este caso se ha implementado en MATLAB). Esta propiedad y el hecho de que sea también muy sencillo definir conjuntos de reglas a partir de conocimiento experto hacen que los sistemas difusos basados en reglas sean opciones muy a tener en cuenta, especialmente en el ámbito del control.