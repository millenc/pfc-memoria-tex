% !TeX spellcheck = es_ES
\documentclass[oneside,11pt,parskip=full]{scrbook}

% Incluir sólo algunos capítulos/ficheros. Útil para trabajar sólo en lo que interesa en cada momento.
\includeonly{introduccion,conclusiones}

% Utilizar UTF-8 como codificación de entrada
\usepackage[utf8]{inputenc}

% Fuentes
\usepackage[T1]{fontenc}
\usepackage[math]{anttor}
%\usepackage{libris}
%\usepackage{palatino}
\usepackage{stix}

% Márgenes
\usepackage[margin=1in]{geometry}

% Paquetes para generar la página de título
\usepackage[pdftex]{graphicx}

% Texto de pruebas
\usepackage{lipsum}

\newcommand{\HRule}{\rule{\linewidth}{0.5mm}}

% Estilos del header y el footer
\usepackage{scrpage2}
\pagestyle{scrheadings}
\ihead{}
\chead{}
\ohead[]{\headmark}
\ifoot[]{}
\cfoot[]{}
\ofoot[\pagemark]{\pagemark}


% Nombre de sección en cursiva
\renewcommand*{\headfont}{\normalfont\itshape}
% Nº de página en negrita
\renewcommand*{\pnumfont}{\normalfont\bfseries}

% Información general del documento
\title{Documento de ejemplo}
\author{Mikel Pintor}
\date{Febrero 2013}
\titlehead{Un draft de ejemplo escrito en LaTeX}
\publishers{Escuela de ingenieros industriales y de telecomunicaciones\\Universidad de Navarra}

\begin{document}

\begin{titlepage}
\begin{center}

% Logos de la universidad y la escuela
\begin{minipage}[t]{0.15\textwidth}
\includegraphics[width=\linewidth]{./images/logo_upna}
\end{minipage}
\hfill
\begin{minipage}[t]{0.15\textwidth}
\includegraphics[width=\linewidth]{./images/logo_escuela}
\end{minipage}\\[2cm]

\textsc{\LARGE ESCUELA TÉCNICA SUPERIOR DE INGENIEROS INDUSTRIALES Y DE TELECOMUNICACIÓN}\\[1.5cm]

\textsc{\Large Ingeniería Informática}\\[0.5cm]

% Título
\HRule \\[0.4cm]
{ \huge \bfseries Aplicación de un método de interpolación, basado en índices de solapamiento, a la detección de riesgos ambientales\\[0.4cm] }

\HRule \\[1.5cm]

% Autor y tutor
\vfill
\hfill
\begin{minipage}{0.4\textwidth}
\begin{flushright} \large
\emph{Autor:} Mikel \textsc{Pintor}\\
\emph{Tutores:} Humberto \textsc{Bustince}\\
Fco. Javier \textsc{Fernández}
{\large Pamplona,23 de julio de 2014}
\end{flushright}
\end{minipage}
\vfill

\end{center}
\end{titlepage}
\frontmatter
\tableofcontents
\listoffigures
\listoftables

\chapter{Reconocimientos}
\chapter{Objetivos}

\mainmatter

\chapter{Introducción}
\label{cha:introduccion}
\clearpage
\lipsum

\chapter{Conclusiones}
\label{cha:conclusiones}
\begin{tabbing}
AAAA\=AAAA\=AAAA\kill\\
\>que\\
\>\>tal?\\
\end{tabbing}
\begin{theorem}
If \emph{proposition}\index{proposition|textbf} $P$ is a tautology then $\sim P$ is a contradiction, and conversely.
\end{theorem}
\begin{theorem}[Tautologies and Contradictions]
content...
\end{theorem}

Esto es una cita \cite{bolourchi2013}




\backmatter

% Glosario de términos
% Bibliografía

\end{document}
