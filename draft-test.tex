% !TeX spellcheck = es_ES
\documentclass[oneside,11pt,parskip=full]{scrbook}

% Incluir sólo algunos capítulos/ficheros. Útil para trabajar sólo en lo que interesa en cada momento.
\includeonly{introduccion,conclusiones,ejemplo-codigo-fuente-matlab}

% Utilizar UTF-8 como codificación de entrada
\usepackage[utf8]{inputenc}

% Fuentes
\usepackage[T1]{fontenc}
\usepackage[math]{anttor}
%\usepackage{libris}
%\usepackage{palatino}
\usepackage{stix}

% Márgenes
\usepackage[margin=1in]{geometry}

% Paquetes para generar la página de título
\usepackage[pdftex]{graphicx}

% Texto de pruebas
\usepackage{lipsum}

% Código fuente
\usepackage{scrhack}
\usepackage{listings}
\usepackage{amsthm}

% Comandos
\newcommand{\HRule}{\rule{\linewidth}{0.5mm}}

% Nombre de sección en cursiva
\renewcommand*{\headfont}{\normalfont\itshape}
% Nº de página en negrita
\renewcommand*{\pnumfont}{\normalfont\bfseries}

% Teoremas
\newtheorem{theorem}{Theorem}[chapter]


% Estilos del header y el footer
\usepackage{scrpage2}
\pagestyle{scrheadings}
\ihead{}
\chead{}
\ohead[]{\headmark}
\ifoot[]{}
\cfoot[]{}
\ofoot[\pagemark]{\pagemark}

% Información general del documento
\title{Documento de ejemplo}
\author{Mikel Pintor}
\date{Febrero 2013}
\titlehead{Un draft de ejemplo escrito en LaTeX}
\publishers{Escuela de ingenieros industriales y de telecomunicaciones\\Universidad de Navarra}

\begin{document}

\begin{titlepage}
\begin{center}

% Logos de la universidad y la escuela
\begin{minipage}[t]{0.15\textwidth}
\includegraphics[width=\linewidth]{./images/logo_upna}
\end{minipage}
\hfill
\begin{minipage}[t]{0.15\textwidth}
\includegraphics[width=\linewidth]{./images/logo_escuela}
\end{minipage}\\[2cm]

\textsc{\LARGE ESCUELA TÉCNICA SUPERIOR DE INGENIEROS INDUSTRIALES Y DE TELECOMUNICACIÓN}\\[1.5cm]

\textsc{\Large Ingeniería Informática}\\[0.5cm]

% Título
\HRule \\[0.4cm]
{ \huge \bfseries Aplicación de un método de interpolación, basado en índices de solapamiento, a la detección de riesgos ambientales\\[0.4cm] }

\HRule \\[1.5cm]

% Autor y tutor
\vfill
\hfill
\begin{minipage}{0.4\textwidth}
\begin{flushright} \large
\emph{Autor:} Mikel \textsc{Pintor}\\
\emph{Tutores:} Humberto \textsc{Bustince}\\
Fco. Javier \textsc{Fernández}
{\large Pamplona,23 de julio de 2014}
\end{flushright}
\end{minipage}
\vfill

\end{center}
\end{titlepage}
\frontmatter
\tableofcontents
\listoffigures
\listoftables

\chapter{Reconocimientos}
\chapter{Objetivos}

\mainmatter

\chapter{Introducción}
\label{cha:introduccion}
\clearpage
\lipsum

\chapter{Conclusiones}
\label{cha:conclusiones}
\begin{tabbing}
AAAA\=AAAA\=AAAA\kill\\
\>que\\
\>\>tal?\\
\end{tabbing}
\begin{theorem}
If \emph{proposition}\index{proposition|textbf} $P$ is a tautology then $\sim P$ is a contradiction, and conversely.
\end{theorem}
\begin{theorem}[Tautologies and Contradictions]
content...
\end{theorem}

Esto es una cita \cite{bolourchi2013}




\chapter{Ejemplo código fuente MATLAB}
\label{cha:ejemplo-codigo-fuente-matlab}
\lstset{language=MATLAB,basicstyle=\ttfamily,numbers=left}
\begin{lstlisting}
function Class = classify_fact( R, x, O , T, M, P )
%CLASSIFY_FACT Summary of this function goes here
%   Detailed explanation goes here

% Clases utilizadas en las reglas
Cm =  sort(unique([R(:).C]));

% Numero de indices de solapamiento utilizados
S = length(O);
KC = zeros(length(Cm),1);

for C=Cm(1:end)
    % Seleccionar las reglas que clasifican la clase actual
    rules = R([R.C] == C);
    alpha = length(rules);
    if(alpha>0)
        r = length(rules(1).CF);
    else
        r = 0;
    end
    
    K = zeros(r*S,1);
    ki = 1;
    for t=1:r
        for L=1:S
            k = zeros(alpha,1);
            for j=1:alpha
                k(j) = matching_degree(rules(j), x, O{L}, T) * rules(j).CF(t);
            end
            K(ki) = max(k);
            ki=ki+1;
        end
    end
    KcnfrOs(C,:) = K; 
    % KC(C) = M(K);
end

% Algoritmo 5: Seleccionamos la clase cuyo valor sea mayor
% [max_KC, Class] = max(KC);

%Algoritmo 6: Seleccionamos la clase haciendo uso de funciones penalty
Class = class_consensus(KcnfrOs,M,P);

end

function m = matching_degree(R, fact, O, T)

overlaps = zeros(length(R.A),1);

for i=1:length(R.A)
    overlaps(i) = O(fact(i).v(2,:) , arrayfun(R.A{i},fact(i).v(1,:)) );
end
m = T(overlaps);

end
\end{lstlisting}

\backmatter

% Glosario de términos

% Bibliografía
\bibliographystyle{plain}
\bibliography{draft-test}

\end{document}
