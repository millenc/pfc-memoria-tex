% !TeX spellcheck = es_ES
\chapter{Teoría de conjuntos difusos}
\label{cha:teoria-conjuntos-difusos}

En este primer capítulo se introduce la teoría relacionada con conjuntos difusos necesaria para el desarrollo del proyecto. Los conjuntos difusos forman la base de los métodos estudiados en este proyecto y han tenido una importancia capital en el desarrollo de la inteligencia artificial en las últimas decadas, desde que fueran introducidos por L.A. Zadeh en 1965 \cite{Zadeh65}. Por esta razón, es necesario definirlos de forma apropiada y estudiar algunas de sus propiedades fundamentales.

\section{Conjuntos difusos}

En esta sección se define el concepto de conjunto difuso (sec. \ref{sec:fuzzy-set-definition-and-properties}), así como las operaciones más básicas que se pueden realizar sobre ellos (sec. \ref{sec:fuzzy-set-operations}). En la sección \ref{sec:lang-variables} se presenta el concepto de variable lingüística, una de las aplicaciones más importantes de los conjuntos difusos y clave para el desarrollo de la lógica difusa.

\subsection{Definición y propiedades}
\label{sec:fuzzy-set-definition-and-properties}
\begin{definition}
Sea \emph{X} un conjunto clásico de objetos, denominado \emph{universo}, cuyos elementos son denotados como \emph{x}. La pertenencia a un subconjunto clásico \emph{A} de \emph{X} se expresa normalmente por medio de una función de pertenencia $\mu_{A}:X\rightarrow\{0,1\}$ tal que:
\begin{equation}
\mu_{A}(x)=\begin{cases} 1 & \mbox{iff } \mbox{x}\in A, \\ 0 & \mbox{iff } \mbox{x}\notin A \end{cases}
\end{equation}
(se utiliza \emph{iff} como abreviatura de "si y sólo si")
\end{definition}

De esta forma, en un conjunto clásico, se dice que un elemento \emph{x} pertenece al conjunto A ($\mu_{A}(x)=1$) o no pertenece ($\mu_{A}(x)=0$). 
\begin{example}
\normalfont
Un ejemplo de conjunto clásico es el conjunto de todos los números enteros pares, cuya función de pertenencia $\mu_{A}:\mathbb{Z}\rightarrow\{0,1\}$ podría expresarse como:
\begin{equation}
\mu_{A}(x)=\begin{cases} 1 & \mbox{iff } \mbox{x}\bmod 2=0, \\ 0 & \mbox{iff } \mbox{x}\bmod 2\ne 0 \end{cases}
\end{equation}
\end{example}
Si el valor de salida de la función de pertenencia puede ser cualquier número en el intervalo [0,1] (y no sólo los valores discretos \{0,1\}) se dice que el conjunto \emph{A} es \emph{difuso} y que la función $\mu_{A}(x)$ es el \emph{grado de pertenencia} de \emph{x} a \emph{A}. En este tipo de conjuntos, cuanto más próximo sea el valor de $\mu_{A}(x)$ a 1, más pertenece \emph{x} a \emph{A}.

El concepto de conjunto difuso fue introducido por L.A. Zadeh en 1965 \cite{Zadeh65} y se define como:
\begin{definition}
Dado un conjunto de referencia (o universo) \emph{U}, un conjunto difuso \emph{A} sobre \emph{U} es un conjunto tal que:\\
\begin{equation}
\{(u_{i},\mu_{A}(u_{i}))\arrowvert u_{i} \in U\}
\end{equation}
donde \begin{math}\emph{A}:\emph{U}\rightarrow[0,1]\end{math} es la \emph{función de pertenencia}\index{función de pertenencia} (o \emph{grado de pertenencia}) de \emph{A}.
\end{definition}

\begin{example}
\normalfont
Supongamos que el universo \emph{U} está compuesto por los valores razonables de altura (medida en centímetros) de una persona adulta, tal que:

\begin{equation}
\emph{U} = \{150,151,\ldots,229,230\}
\end{equation}

Se puede definir sobre \emph{U} el conjunto difuso \emph{VT}, para representar el concepto de altura \emph{"Muy alta"}, especificando su función de pertenencia \emph{$\mu_{VT}$}, de forma que los valores de altura \emph{$\mu_{i}$} considerados como \emph{"muy altos"} hagan que el valor de dicha función de pertenencia \emph{$\mu_{VT}(\mu_{i})$} sea más próximo a 1 (pertenecen más al conjunto).\\
\\
Se puede elegir cualquier tipo de función para definir al conjunto difuso, siempre que se ajuste a las necesidades del problema que se pretende resolver. En este ejemplo se va a definir la función de pertenencia \emph{$\mu_{VT}$} como una función sigmoidal tal que:
\begin{equation}
\mu_{VT}(\mu_{i}) = \frac{1}{1 + e^{-0.4(\mu_{i}-190)}}
\end{equation}
La función de pertenencia $\mu_{VT}(\mu_{i})$ se representa en la figura \ref{fig:fuzzyset-verytall-example}. Se puede observar que alturas que no podrían considerarse muy altas como 150, 160, 170,etc. hacen que la función de pertenencia  $\mu_{VT}$ sea cero (no pertenencen al conjunto). Algunos valores del conjunto difuso \emph{VT}:
\begin{equation}
VT = \{(150,0),(160,0),(170,0),\ldots,(190,0.5),(195,0.88),(200,1),(210,1),\ldots\}
\end{equation}
\end{example}

\begin{figure}[t]
	\centering
	\newlength\figureheight 
	\newlength\figurewidth
	\setlength\figureheight{4.5cm}
	\setlength\figurewidth{12cm}
	% This file was created by matlab2tikz v0.4.7 (commit 06bbbef116b887fd29db8b1bac523c404c7f7694) running on MATLAB 8.0.
% Copyright (c) 2008--2014, Nico Schlömer <nico.schloemer@gmail.com>
% All rights reserved.
% Minimal pgfplots version: 1.3
% 
\begin{tikzpicture}

\begin{axis}[%
width=\figurewidth,
height=\figureheight,
scale only axis,
xmin=150,
xmax=230,
xlabel={$\mu{}_{\text{i}}\text{ - Altura en cm}$},
ymin=-0.1,
ymax=1.1,
ylabel={$\mu{}_{\text{VT}}\text{(}\mu{}_{\text{i}}\text{)}$},
legend style={at={(0.97,0.03)},anchor=south east,draw=black,fill=white,legend cell align=left}
]
\addplot [color=blue,solid,line width=1.5pt]
  table[row sep=crcr]{150	1.12535162055095e-07\\
150.1	1.17127808604502e-07\\
150.2	1.21907884570381e-07\\
150.3	1.26883039088593e-07\\
150.4	1.32061233461842e-07\\
150.5	1.37450753899427e-07\\
150.6	1.43060224776898e-07\\
150.7	1.4889862243685e-07\\
150.8	1.54975289552946e-07\\
150.9	1.61299950080123e-07\\
151	1.67882724814952e-07\\
151.1	1.74734147591e-07\\
151.2	1.81865182135148e-07\\
151.3	1.89287239611824e-07\\
151.4	1.97012196883211e-07\\
151.5	2.05052415514691e-07\\
151.6	2.13420761555885e-07\\
151.7	2.22130626128981e-07\\
151.8	2.31195946857281e-07\\
151.9	2.40631230168244e-07\\
152	2.50451574506755e-07\\
152.1	2.6067269449571e-07\\
152.2	2.71310946082621e-07\\
152.3	2.8238335271246e-07\\
152.4	2.93907632568622e-07\\
152.5	3.05902226925625e-07\\
152.6	3.18386329658865e-07\\
152.7	3.313799179587e-07\\
152.8	3.44903784297974e-07\\
152.9	3.58979569704136e-07\\
153	3.73629798389248e-07\\
153.1	3.888779137932e-07\\
153.2	4.04748316097899e-07\\
153.3	4.21266401272407e-07\\
153.4	4.38458601711502e-07\\
153.5	4.56352428532765e-07\\
153.6	4.74976515599755e-07\\
153.7	4.94360665341816e-07\\
153.8	5.14535896443798e-07\\
153.9	5.35534493481965e-07\\
154	5.57390058585609e-07\\
154.1	5.80137565206936e-07\\
154.2	6.03813414085313e-07\\
154.3	6.28455491495431e-07\\
154.4	6.54103229872536e-07\\
154.5	6.80797670911849e-07\\
154.6	7.0858153124299e-07\\
154.7	7.37499270784616e-07\\
154.8	7.67597163888589e-07\\
154.9	7.98923373387501e-07\\
155	8.31528027664132e-07\\
155.1	8.65463300866014e-07\\
155.2	9.00783496393575e-07\\
155.3	9.37545133795406e-07\\
155.4	9.75807039209627e-07\\
155.5	1.01563043949625e-06\\
155.6	1.05707906021093e-06\\
155.7	1.10021922757701e-06\\
155.8	1.14511997461899e-06\\
155.9	1.1918531516272e-06\\
156	1.24049354113058e-06\\
156.1	1.29111897756123e-06\\
156.2	1.3438104718025e-06\\
156.3	1.3986523408198e-06\\
156.4	1.45573234258132e-06\\
156.5	1.51514181648505e-06\\
156.6	1.57697582951616e-06\\
156.7	1.64133332836911e-06\\
156.8	1.70831729777757e-06\\
156.9	1.77803492530552e-06\\
157	1.85059777286345e-06\\
157.1	1.92612195522365e-06\\
157.2	2.00472832582054e-06\\
157.3	2.08654267013323e-06\\
157.4	2.17169590695948e-06\\
157.5	2.26032429790357e-06\\
157.6	2.35256966541272e-06\\
157.7	2.44857961971114e-06\\
157.8	2.54850779499489e-06\\
157.9	2.65251409526506e-06\\
158	2.76076495019305e-06\\
158.1	2.87343358142687e-06\\
158.2	2.9907002797647e-06\\
158.3	3.11275269363916e-06\\
158.4	3.23978612937351e-06\\
158.5	3.37200386369078e-06\\
158.6	3.50961746897491e-06\\
158.7	3.65284715180485e-06\\
158.8	3.80192210530298e-06\\
158.9	3.95708087586131e-06\\
159	4.11857174483263e-06\\
159.1	4.28665312579647e-06\\
159.2	4.46159397803589e-06\\
159.3	4.64367423688635e-06\\
159.4	4.8331852616446e-06\\
159.5	5.03043030175512e-06\\
159.6	5.23572498201847e-06\\
159.7	5.44939780759849e-06\\
159.8	5.67179068963591e-06\\
159.9	5.90325949230864e-06\\
160	6.14417460221472e-06\\
160.1	6.39492152098717e-06\\
160.2	6.65590148208973e-06\\
160.3	6.92753209277934e-06\\
160.4	7.21024800226174e-06\\
160.5	7.50450159711e-06\\
160.6	7.81076372505625e-06\\
160.7	8.12952444831543e-06\\
160.8	8.46129382764526e-06\\
160.9	8.80660273839554e-06\\
161	9.16600371985332e-06\\
161.1	9.54007185923986e-06\\
161.2	9.92940571177426e-06\\
161.3	1.03346282582745e-05\\
161.4	1.07563879018261e-05\\
161.5	1.11953595051133e-05\\
161.6	1.16522454700696e-05\\
161.7	1.21277768615743e-05\\
161.8	1.26227145769914e-05\\
161.9	1.31378505634194e-05\\
162	1.36740090845997e-05\\
162.1	1.42320480395057e-05\\
162.2	1.48128603347223e-05\\
162.3	1.54173753128082e-05\\
162.4	1.60465602389226e-05\\
162.5	1.67014218480952e-05\\
162.6	1.73830079556069e-05\\
162.7	1.8092409133059e-05\\
162.8	1.88307604528061e-05\\
162.9	1.95992433035389e-05\\
163	2.03990872799214e-05\\
163.1	2.12315721492952e-05\\
163.2	2.2098029898597e-05\\
163.3	2.29998468647563e-05\\
163.4	2.39384659519744e-05\\
163.5	2.49153889394299e-05\\
163.6	2.593217888309e-05\\
163.7	2.69904626154658e-05\\
163.8	2.80919333473023e-05\\
163.9	2.92383533753526e-05\\
164	3.04315569005653e-05\\
164.1	3.16734529611747e-05\\
164.2	3.296602848538e-05\\
164.3	3.43113514684834e-05\\
164.4	3.57115742795526e-05\\
164.5	3.71689371028894e-05\\
164.6	3.86857715197865e-05\\
164.7	4.02645042362891e-05\\
164.8	4.19076609629042e-05\\
164.9	4.36178704524425e-05\\
165	4.53978687024344e-05\\
165.1	4.725050332881e-05\\
165.2	4.91787381178212e-05\\
165.3	5.11856577634542e-05\\
165.4	5.32744727978788e-05\\
165.5	5.54485247227949e-05\\
165.6	5.77112913498369e-05\\
165.7	6.00663923585472e-05\\
165.8	6.25175950807636e-05\\
165.9	6.5068820520623e-05\\
166	6.77241496197701e-05\\
166.1	7.04878297777251e-05\\
166.2	7.33642816377876e-05\\
166.3	7.6358106149268e-05\\
166.4	7.94740919172621e-05\\
166.5	8.27172228516664e-05\\
166.6	8.60926861275657e-05\\
166.7	8.96058804696507e-05\\
166.8	9.32624247738134e-05\\
166.9	9.70681670795998e-05\\
167	0.000101029193907773\\
167.1	0.000105151839977773\\
167.2	0.000109442698320505\\
167.3	0.000113908630802462\\
167.4	0.000118556779077877\\
167.5	0.000123394575986232\\
167.6	0.00012842975741318\\
167.7	0.000133670374633634\\
167.8	0.000139124807156559\\
167.9	0.00014480177609175\\
168	0.000150710358059757\\
168.1	0.000156859999666871\\
168.2	0.000163260532568056\\
168.3	0.000169922189141575\\
168.4	0.000176855618800013\\
168.5	0.000184071904963424\\
168.6	0.000191582582721282\\
168.7	0.00019939965721107\\
168.8	0.000207535622742382\\
168.9	0.000216003482696582\\
169	0.000224816770233295\\
169.1	0.000233989569836176\\
169.2	0.000243536539731751\\
169.3	0.000253472935216452\\
169.4	0.000263814632928313\\
169.5	0.000274578156101332\\
169.6	0.000285780700841879\\
169.7	0.000297440163468182\\
169.8	0.000309575168955517\\
169.9	0.000322205100531344\\
170	0.000335350130466478\\
170.1	0.00034903125211006\\
170.2	0.000363270313218076\\
170.3	0.00037809005062704\\
170.4	0.000393514126326484\\
170.5	0.00040956716498605\\
170.6	0.000426274792995\\
170.7	0.000443663679074348\\
170.8	0.000461761576524077\\
170.9	0.000480597367170262\\
171	0.000500201107079564\\
171.1	0.000520604074110941\\
171.2	0.000541838817377267\\
171.3	0.00056393920869224\\
171.4	0.000586940496080804\\
171.5	0.000610879359434401\\
171.6	0.000635793968395226\\
171.7	0.000661724042557041\\
171.8	0.000688710914073259\\
171.9	0.000716797592766405\\
172	0.000746028833836697\\
172.1	0.000776451208270841\\
172.2	0.000808113176056168\\
172.3	0.000841065162308884\\
172.4	0.000875359636429239\\
172.5	0.000911051194400645\\
172.6	0.000948196644353723\\
172.7	0.000986855095520903\\
172.8	0.00102708805071153\\
172.9	0.00106895950244197\\
173	0.00111253603286032\\
173.1	0.00115788691760961\\
173.2	0.00120508423377894\\
173.3	0.00125420297209689\\
173.4	0.00130532115352677\\
173.5	0.00135851995042896\\
173.6	0.00141388381246065\\
173.7	0.00147150059738952\\
173.8	0.00153146170700329\\
173.9	0.00159386222830301\\
174	0.00165880108017442\\
174.1	0.00172638116573715\\
174.2	0.00179670953057844\\
174.3	0.00186989752708406\\
174.4	0.00194606098508556\\
174.5	0.00202532038904988\\
174.6	0.00210780106204325\\
174.7	0.00219363335670874\\
174.8	0.00228295285350307\\
174.9	0.00237590056644499\\
175	0.00247262315663477\\
175.1	0.00257327315381036\\
175.2	0.0026780091862131\\
175.3	0.00278699621904229\\
175.4	0.00290040580178421\\
175.5	0.00301841632470842\\
175.6	0.00314121328482942\\
175.7	0.0032689895616386\\
175.8	0.00340194570291728\\
175.9	0.00354029022094651\\
176	0.00368423989943599\\
176.1	0.00383402011149744\\
176.2	0.00398986514899408\\
176.3	0.00415201856360066\\
176.4	0.00432073351991193\\
176.5	0.00449627316094118\\
176.6	0.00467891098635069\\
176.7	0.00486893124375862\\
176.8	0.00506662933346628\\
176.9	0.00527231222694868\\
177	0.0054862988994504\\
177.1	0.00570892077702333\\
177.2	0.00594052219834034\\
177.3	0.00618146089161117\\
177.4	0.00643210846691865\\
177.5	0.00669285092428486\\
177.6	0.00696408917776207\\
177.7	0.00724623959583139\\
177.8	0.00753973455837343\\
177.9	0.00784502303045565\\
178	0.00816257115315989\\
178.1	0.00849286285164433\\
178.2	0.00883640046060767\\
178.3	0.00919370536728814\\
178.4	0.00956531867209169\\
178.5	0.00995180186690432\\
178.6	0.0103537375310917\\
178.7	0.0107717300451416\\
178.8	0.0112064063218429\\
178.9	0.0116584165548331\\
179	0.0121284349842742\\
179.1	0.0126171606793387\\
179.2	0.0131253183371027\\
179.3	0.0136536590973509\\
179.4	0.0142029613726912\\
179.5	0.0147740316932731\\
179.6	0.0153677055652755\\
179.7	0.0159848483422025\\
179.8	0.0166263561078817\\
179.9	0.0172931565699055\\
180	0.0179862099620916\\
180.1	0.0187065099543546\\
180.2	0.0194550845681929\\
180.3	0.0202329970957855\\
180.4	0.0210413470204683\\
180.5	0.0218812709361305\\
180.6	0.0227539434628039\\
180.7	0.0236605781554611\\
180.8	0.0246024284027395\\
180.9	0.0255807883120077\\
181	0.0265969935768659\\
181.1	0.0276524223228231\\
181.2	0.0287484959265398\\
181.3	0.0298866798036364\\
181.4	0.0310684841596675\\
181.5	0.0322954646984505\\
181.6	0.0335692232814824\\
181.7	0.0348914085317367\\
181.8	0.0362637163746485\\
181.9	0.037687890508606\\
182	0.0391657227967644\\
182.1	0.0406990535714664\\
182.2	0.0422897718420336\\
182.3	0.0439398153961415\\
182.4	0.0456511707844438\\
182.5	0.0474258731775668\\
182.6	0.0492660060840265\\
182.7	0.0511737009170914\\
182.8	0.0531511363980639\\
182.9	0.0552005377829344\\
183	0.0573241758988687\\
183.1	0.0595243659765014\\
183.2	0.0618034662635883\\
183.3	0.0641638764051742\\
183.4	0.0666080355750908\\
183.5	0.0691384203433468\\
183.6	0.0717575422637512\\
183.7	0.0744679451660277\\
183.8	0.0772722021366602\\
183.9	0.0801729121728125\\
184	0.0831726964939224\\
184.1	0.0862741944959165\\
184.2	0.0894800593335611\\
184.3	0.0927929531171574\\
184.4	0.096215541710693\\
184.5	0.0997504891196851\\
184.6	0.103400451458249\\
184.7	0.107168070486528\\
184.8	0.111055966711408\\
184.9	0.11506673204555\\
185	0.119202922022118\\
185.1	0.123467047565224\\
185.2	0.127861566319081\\
185.3	0.132388873542066\\
185.4	0.13705129257546\\
185.5	0.141851064900488\\
185.6	0.146790339801382\\
185.7	0.151871163656659\\
185.8	0.157095468885453\\
185.9	0.162465062580696\\
186	0.167981614866076\\
186.1	0.173646647019005\\
186.2	0.179461519407326\\
186.3	0.185427419292983\\
186.4	0.191545348561468\\
186.5	0.197816111441418\\
186.6	0.204240302284091\\
186.7	0.210818293477746\\
186.8	0.217550223576888\\
186.9	0.224435985730927\\
187	0.231475216500982\\
187.1	0.238667285157089\\
187.2	0.246011283551051\\
187.3	0.253506016662339\\
187.4	0.261149993915751\\
187.5	0.268941421369995\\
187.6	0.27687819487561\\
187.7	0.284957894299009\\
187.8	0.293177778906433\\
187.9	0.301534783997462\\
188	0.310025518872388\\
188.1	0.318646266210974\\
188.2	0.327392982932239\\
188.3	0.336261302595648\\
188.4	0.345246539393681\\
188.5	0.354343693774205\\
188.6	0.363547459718433\\
188.7	0.372852233686803\\
188.8	0.382252125230752\\
188.9	0.391740969253486\\
189	0.401312339887548\\
189.1	0.410959565941334\\
189.2	0.420675747851249\\
189.3	0.430453776060772\\
189.4	0.440286350732808\\
189.5	0.450166002687522\\
189.6	0.460085115444434\\
189.7	0.470035948235427\\
189.8	0.480010659844419\\
189.9	0.490001333120035\\
190	0.5\\
190.1	0.509998666879965\\
190.2	0.519989340155581\\
190.3	0.529964051764573\\
190.4	0.539914884555566\\
190.5	0.549833997312478\\
190.6	0.559713649267192\\
190.7	0.569546223939228\\
190.8	0.579324252148751\\
190.9	0.589040434058666\\
191	0.598687660112452\\
191.1	0.608259030746514\\
191.2	0.617747874769248\\
191.3	0.627147766313197\\
191.4	0.636452540281567\\
191.5	0.645656306225795\\
191.6	0.654753460606319\\
191.7	0.663738697404352\\
191.8	0.672607017067761\\
191.9	0.681353733789026\\
192	0.689974481127613\\
192.1	0.698465216002538\\
192.2	0.706822221093567\\
192.3	0.715042105700991\\
192.4	0.72312180512439\\
192.5	0.731058578630005\\
192.6	0.738850006084249\\
192.7	0.746493983337661\\
192.8	0.753988716448949\\
192.9	0.761332714842911\\
193	0.768524783499018\\
193.1	0.775564014269073\\
193.2	0.782449776423112\\
193.3	0.789181706522254\\
193.4	0.795759697715909\\
193.5	0.802183888558582\\
193.6	0.808454651438532\\
193.7	0.814572580707017\\
193.8	0.820538480592674\\
193.9	0.826353352980995\\
194	0.832018385133924\\
194.1	0.837534937419304\\
194.2	0.842904531114547\\
194.3	0.848128836343341\\
194.4	0.853209660198618\\
194.5	0.858148935099512\\
194.6	0.86294870742454\\
194.7	0.867611126457934\\
194.8	0.872138433680919\\
194.9	0.876532952434776\\
195	0.880797077977882\\
195.1	0.88493326795445\\
195.2	0.888944033288592\\
195.3	0.892831929513472\\
195.4	0.896599548541751\\
195.5	0.900249510880315\\
195.6	0.903784458289307\\
195.7	0.907207046882843\\
195.8	0.910519940666439\\
195.9	0.913725805504084\\
196	0.916827303506078\\
196.1	0.919827087827187\\
196.2	0.92272779786334\\
196.3	0.925532054833972\\
196.4	0.928242457736249\\
196.5	0.930861579656653\\
196.6	0.933391964424909\\
196.7	0.935836123594826\\
196.8	0.938196533736412\\
196.9	0.940475634023499\\
197	0.942675824101131\\
197.1	0.944799462217066\\
197.2	0.946848863601936\\
197.3	0.948826299082909\\
197.4	0.950733993915973\\
197.5	0.952574126822433\\
197.6	0.954348829215556\\
197.7	0.956060184603859\\
197.8	0.957710228157966\\
197.9	0.959300946428534\\
198	0.960834277203236\\
198.1	0.962312109491394\\
198.2	0.963736283625351\\
198.3	0.965108591468263\\
198.4	0.966430776718518\\
198.5	0.96770453530155\\
198.6	0.968931515840333\\
198.7	0.970113320196364\\
198.8	0.97125150407346\\
198.9	0.972347577677177\\
199	0.973403006423134\\
199.1	0.974419211687992\\
199.2	0.97539757159726\\
199.3	0.976339421844539\\
199.4	0.977246056537196\\
199.5	0.978118729063869\\
199.6	0.978958652979532\\
199.7	0.979767002904215\\
199.8	0.980544915431807\\
199.9	0.981293490045645\\
200	0.982013790037908\\
200.1	0.982706843430095\\
200.2	0.983373643892118\\
200.3	0.984015151657797\\
200.4	0.984632294434724\\
200.5	0.985225968306727\\
200.6	0.985797038627309\\
200.7	0.986346340902649\\
200.8	0.986874681662897\\
200.9	0.987382839320661\\
201	0.987871565015726\\
201.1	0.988341583445167\\
201.2	0.988793593678157\\
201.3	0.989228269954858\\
201.4	0.989646262468908\\
201.5	0.990048198133096\\
201.6	0.990434681327908\\
201.7	0.990806294632712\\
201.8	0.991163599539392\\
201.9	0.991507137148356\\
202	0.99183742884684\\
202.1	0.992154976969544\\
202.2	0.992460265441627\\
202.3	0.992753760404169\\
202.4	0.993035910822238\\
202.5	0.993307149075715\\
202.6	0.993567891533081\\
202.7	0.993818539108389\\
202.8	0.99405947780166\\
202.9	0.994291079222977\\
203	0.994513701100549\\
203.1	0.994727687773051\\
203.2	0.994933370666534\\
203.3	0.995131068756241\\
203.4	0.995321089013649\\
203.5	0.995503726839059\\
203.6	0.995679266480088\\
203.7	0.995847981436399\\
203.8	0.996010134851006\\
203.9	0.996165979888503\\
204	0.996315760100564\\
204.1	0.996459709779054\\
204.2	0.996598054297083\\
204.3	0.996731010438361\\
204.4	0.996858786715171\\
204.5	0.996981583675292\\
204.6	0.997099594198216\\
204.7	0.997213003780958\\
204.8	0.997321990813787\\
204.9	0.99742672684619\\
205	0.997527376843365\\
205.1	0.997624099433555\\
205.2	0.997717047146497\\
205.3	0.997806366643291\\
205.4	0.997892198937957\\
205.5	0.99797467961095\\
205.6	0.998053939014914\\
205.7	0.998130102472916\\
205.8	0.998203290469422\\
205.9	0.998273618834263\\
206	0.998341198919826\\
206.1	0.998406137771697\\
206.2	0.998468538292997\\
206.3	0.99852849940261\\
206.4	0.998586116187539\\
206.5	0.998641480049571\\
206.6	0.998694678846473\\
206.7	0.998745797027903\\
206.8	0.998794915766221\\
206.9	0.99884211308239\\
207	0.99888746396714\\
207.1	0.998931040497558\\
207.2	0.998972911949289\\
207.3	0.999013144904479\\
207.4	0.999051803355646\\
207.5	0.999088948805599\\
207.6	0.999124640363571\\
207.7	0.999158934837691\\
207.8	0.999191886823944\\
207.9	0.999223548791729\\
208	0.999253971166163\\
208.1	0.999283202407234\\
208.2	0.999311289085927\\
208.3	0.999338275957443\\
208.4	0.999364206031605\\
208.5	0.999389120640566\\
208.6	0.999413059503919\\
208.7	0.999436060791308\\
208.8	0.999458161182623\\
208.9	0.999479395925889\\
209	0.999499798892921\\
209.1	0.99951940263283\\
209.2	0.999538238423476\\
209.3	0.999556336320926\\
209.4	0.999573725207005\\
209.5	0.999590432835014\\
209.6	0.999606485873673\\
209.7	0.999621909949373\\
209.8	0.999636729686782\\
209.9	0.99965096874789\\
210	0.999664649869534\\
210.1	0.999677794899469\\
210.2	0.999690424831044\\
210.3	0.999702559836532\\
210.4	0.999714219299158\\
210.5	0.999725421843899\\
210.6	0.999736185367072\\
210.7	0.999746527064784\\
210.8	0.999756463460268\\
210.9	0.999766010430164\\
211	0.999775183229767\\
211.1	0.999783996517303\\
211.2	0.999792464377258\\
211.3	0.999800600342789\\
211.4	0.999808417417279\\
211.5	0.999815928095037\\
211.6	0.9998231443812\\
211.7	0.999830077810858\\
211.8	0.999836739467432\\
211.9	0.999843140000333\\
212	0.99984928964194\\
212.1	0.999855198223908\\
212.2	0.999860875192843\\
212.3	0.999866329625366\\
212.4	0.999871570242587\\
212.5	0.999876605424014\\
212.6	0.999881443220922\\
212.7	0.999886091369197\\
212.8	0.99989055730168\\
212.9	0.999894848160022\\
213	0.999898970806092\\
213.1	0.99990293183292\\
213.2	0.999906737575226\\
213.3	0.99991039411953\\
213.4	0.999913907313873\\
213.5	0.999917282777148\\
213.6	0.999920525908083\\
213.7	0.999923641893851\\
213.8	0.999926635718362\\
213.9	0.999929512170222\\
214	0.99993227585038\\
214.1	0.999934931179479\\
214.2	0.999937482404919\\
214.3	0.999939933607641\\
214.4	0.99994228870865\\
214.5	0.999944551475277\\
214.6	0.999946725527202\\
214.7	0.999948814342237\\
214.8	0.999950821261882\\
214.9	0.999952749496671\\
215	0.999954602131298\\
215.1	0.999956382129548\\
215.2	0.999958092339037\\
215.3	0.999959735495764\\
215.4	0.99996131422848\\
215.5	0.999962831062897\\
215.6	0.99996428842572\\
215.7	0.999965688648532\\
215.8	0.999967033971515\\
215.9	0.999968326547039\\
216	0.999969568443099\\
216.1	0.999970761646625\\
216.2	0.999971908066653\\
216.3	0.999973009537385\\
216.4	0.999974067821117\\
216.5	0.999975084611061\\
216.6	0.999976061534048\\
216.7	0.999977000153135\\
216.8	0.999977901970102\\
216.9	0.999978768427851\\
217	0.99997960091272\\
217.1	0.999980400756696\\
217.2	0.999981169239547\\
217.3	0.999981907590867\\
217.4	0.999982616992044\\
217.5	0.999983298578152\\
217.6	0.999983953439761\\
217.7	0.999984582624687\\
217.8	0.999985187139665\\
217.9	0.99998576795196\\
218	0.999986325990915\\
218.1	0.999986862149437\\
218.2	0.999987377285423\\
218.3	0.999987872223139\\
218.4	0.99998834775453\\
218.5	0.999988804640495\\
218.6	0.999989243612098\\
218.7	0.999989665371742\\
218.8	0.999990070594288\\
218.9	0.999990459928141\\
219	0.99999083399628\\
219.1	0.999991193397262\\
219.2	0.999991538706172\\
219.3	0.999991870475552\\
219.4	0.999992189236275\\
219.5	0.999992495498403\\
219.6	0.999992789751998\\
219.7	0.999993072467907\\
219.8	0.999993344098518\\
219.9	0.999993605078479\\
220	0.999993855825398\\
220.1	0.999994096740508\\
220.2	0.99999432820931\\
220.3	0.999994550602192\\
220.4	0.999994764275018\\
220.5	0.999994969569698\\
220.6	0.999995166814738\\
220.7	0.999995356325763\\
220.8	0.999995538406022\\
220.9	0.999995713346874\\
221	0.999995881428255\\
221.1	0.999996042919124\\
221.2	0.999996198077895\\
221.3	0.999996347152848\\
221.4	0.999996490382531\\
221.5	0.999996627996136\\
221.6	0.999996760213871\\
221.7	0.999996887247306\\
221.8	0.99999700929972\\
221.9	0.999997126566419\\
222	0.99999723923505\\
222.1	0.999997347485905\\
222.2	0.999997451492205\\
222.3	0.99999755142038\\
222.4	0.999997647430335\\
222.5	0.999997739675702\\
222.6	0.999997828304093\\
222.7	0.99999791345733\\
222.8	0.999997995271674\\
222.9	0.999998073878045\\
223	0.999998149402227\\
223.1	0.999998221965075\\
223.2	0.999998291682702\\
223.3	0.999998358666672\\
223.4	0.999998423024171\\
223.5	0.999998484858183\\
223.6	0.999998544267658\\
223.7	0.999998601347659\\
223.8	0.999998656189528\\
223.9	0.999998708881022\\
224	0.999998759506459\\
224.1	0.999998808146848\\
224.2	0.999998854880025\\
224.3	0.999998899780772\\
224.4	0.99999894292094\\
224.5	0.999998984369561\\
224.6	0.999999024192961\\
224.7	0.999999062454866\\
224.8	0.999999099216504\\
224.9	0.999999134536699\\
225	0.999999168471972\\
225.1	0.999999201076627\\
225.2	0.999999232402836\\
225.3	0.999999262500729\\
225.4	0.999999291418469\\
225.5	0.999999319202329\\
225.6	0.99999934589677\\
225.7	0.999999371544508\\
225.8	0.999999396186586\\
225.9	0.999999419862435\\
226	0.999999442609941\\
226.1	0.999999464465506\\
226.2	0.999999485464104\\
226.3	0.999999505639335\\
226.4	0.999999525023484\\
226.5	0.999999543647571\\
226.6	0.999999561541398\\
226.7	0.999999578733599\\
226.8	0.999999595251684\\
226.9	0.999999611122086\\
227	0.999999626370202\\
227.1	0.99999964102043\\
227.2	0.999999655096216\\
227.3	0.999999668620082\\
227.4	0.99999968161367\\
227.5	0.999999694097773\\
227.6	0.999999706092367\\
227.7	0.999999717616647\\
227.8	0.999999728689054\\
227.9	0.999999739327305\\
228	0.999999749548426\\
228.1	0.99999975936877\\
228.2	0.999999768804053\\
228.3	0.999999777869374\\
228.4	0.999999786579238\\
228.5	0.999999794947585\\
228.6	0.999999802987803\\
228.7	0.99999981071276\\
228.8	0.999999818134818\\
228.9	0.999999825265853\\
229	0.999999832117275\\
229.1	0.99999983870005\\
229.2	0.999999845024711\\
229.3	0.999999851101378\\
229.4	0.999999856939775\\
229.5	0.999999862549246\\
229.6	0.999999867938767\\
229.7	0.999999873116961\\
229.8	0.999999878092116\\
229.9	0.999999882872191\\
230	0.999999887464838\\
};
\addlegendentry{$\mu{}_{\text{VT}}\text{(}\mu{}_{\text{i}}\text{)}$};

\addplot [color=green,line width=1.5pt,mark size=4.0pt,only marks,mark=asterisk,mark options={solid},forget plot]
  table[row sep=crcr]{195	0.88\\
};
\node[right, inner sep=0mm, text=black]
at (axis cs:195,0.88,0) {$\text{  }\leftarrow\text{ (}\mu{}_{\text{i}}\text{=195; }\mu{}_{\text{VT}}\text{(195)=0.8808)}$};
\end{axis}
\end{tikzpicture}%
	\caption{Función de pertenencia del conjunto difuso \emph{"Muy Alto"} ($\mu_{VT}(\mu_{i})$).}
	\label{fig:fuzzyset-verytall-example}
\end{figure}

\subsection{Variables lingüísticas}
\label{sec:lang-variables}
Una variable lingüística se puede definir de manera informal como una variable que puede tomar palabras del lenguaje natural como valores. Para formular palabras y utilizarlas como valores de variables lingüísticas se utilizan conjuntos difusos (por medio de sus correspondientes funciones de pertenencia). Así pues, una variable lingüística se puede definir como \cite{wang1997}:

\begin{definition}\label{def:informal-lang-variable}
Una \emph{variable lingüística} es un tipo de variable que puede tomar palabras como valores. Las palabras son caracterizadas por conjuntos difusos definidos en el mismo universo en el que la variable es definida.
\end{definition}

\begin{example}\label{ex:lang-variable}
\normalfont
Supongamos que la temperatura medida por un termómetro en grados celsius (ºC) es una variable $x$ que puede tomar valores enteros en el intervalo $[0,100]$. Este intervalo de valores posibles es lo que se define como el \emph{universo de discurso} o \emph{universo de referencia} $U$. Podemos definir sobre $U$ los conjuntos difusos \emph{``baja''},\emph{``media''} y \emph{``alta''} para modelar los conceptos de temperatura alta, media y baja respectivamente. Si consideramos que $x$ es una variable lingüística entonces $x$ puede tomar los valores \emph{``baja''},\emph{``media''} y \emph{``alta''}. Es decir, podemos decir que: \emph{``x es baja''},\emph{``x es media''} o \emph{``x es alta''}, o lo que es lo mismo \emph{``Temperatura es baja''},\emph{``Temperatura es media''} o \emph{``Temperatura es alta''}. En la figura \ref{fig:example-temp-lang-variable} se representan las funciones de pertenencia para los conjuntos difusos \emph{``Temperatura baja''},\emph{``Temperatura media''} y \emph{``Temperatura alta''}.
\end{example}

\begin{figure}[t]
	\centering
	\setlength\figureheight{4.5cm}
	\setlength\figurewidth{12cm}
	% This file was created by matlab2tikz v0.4.7 (commit 05e36af552d2a8253e2edb4c73807088323ff197) running on MATLAB 8.0.
% Copyright (c) 2008--2014, Nico Schlömer <nico.schloemer@gmail.com>
% All rights reserved.
% Minimal pgfplots version: 1.3
% 
\begin{tikzpicture}

\begin{axis}[%
width=\figurewidth,
height=\figureheight,
clip=false,
scale only axis,
xmin=0,
xmax=100,
xlabel={x - Temperatura},
ymin=0,
ymax=1
]
\addplot [color=black,solid,line width=1.5pt,forget plot]
  table[row sep=crcr]{0	1\\
1	0.975\\
2	0.95\\
3	0.925\\
4	0.9\\
5	0.875\\
6	0.85\\
7	0.825\\
8	0.8\\
9	0.775\\
10	0.75\\
11	0.725\\
12	0.7\\
13	0.675\\
14	0.65\\
15	0.625\\
16	0.6\\
17	0.575\\
18	0.55\\
19	0.525\\
20	0.5\\
21	0.475\\
22	0.45\\
23	0.425\\
24	0.4\\
25	0.375\\
26	0.35\\
27	0.325\\
28	0.3\\
29	0.275\\
30	0.25\\
31	0.225\\
32	0.2\\
33	0.175\\
34	0.15\\
35	0.125\\
36	0.1\\
37	0.075\\
38	0.05\\
39	0.025\\
40	0\\
};
\addplot [color=black,solid,line width=1.5pt,forget plot]
  table[row sep=crcr]{10	0\\
11	0.025\\
12	0.05\\
13	0.075\\
14	0.1\\
15	0.125\\
16	0.15\\
17	0.175\\
18	0.2\\
19	0.225\\
20	0.25\\
21	0.275\\
22	0.3\\
23	0.325\\
24	0.35\\
25	0.375\\
26	0.4\\
27	0.425\\
28	0.45\\
29	0.475\\
30	0.5\\
31	0.525\\
32	0.55\\
33	0.575\\
34	0.6\\
35	0.625\\
36	0.65\\
37	0.675\\
38	0.7\\
39	0.725\\
40	0.75\\
41	0.775\\
42	0.8\\
43	0.825\\
44	0.85\\
45	0.875\\
46	0.9\\
47	0.925\\
48	0.95\\
49	0.975\\
50	1\\
51	0.975\\
52	0.95\\
53	0.925\\
54	0.9\\
55	0.875\\
56	0.85\\
57	0.825\\
58	0.8\\
59	0.775\\
60	0.75\\
61	0.725\\
62	0.7\\
63	0.675\\
64	0.65\\
65	0.625\\
66	0.6\\
67	0.575\\
68	0.55\\
69	0.525\\
70	0.5\\
71	0.475\\
72	0.45\\
73	0.425\\
74	0.4\\
75	0.375\\
76	0.35\\
77	0.325\\
78	0.3\\
79	0.275\\
80	0.25\\
81	0.225\\
82	0.2\\
83	0.175\\
84	0.15\\
85	0.125\\
86	0.1\\
87	0.075\\
88	0.05\\
89	0.025\\
90	0\\
};
\addplot [color=black,solid,line width=1.5pt,forget plot]
  table[row sep=crcr]{60	0\\
61	0.025\\
62	0.05\\
63	0.075\\
64	0.1\\
65	0.125\\
66	0.15\\
67	0.175\\
68	0.2\\
69	0.225\\
70	0.25\\
71	0.275\\
72	0.3\\
73	0.325\\
74	0.35\\
75	0.375\\
76	0.4\\
77	0.425\\
78	0.45\\
79	0.475\\
80	0.5\\
81	0.525\\
82	0.55\\
83	0.575\\
84	0.6\\
85	0.625\\
86	0.65\\
87	0.675\\
88	0.7\\
89	0.725\\
90	0.75\\
91	0.775\\
92	0.8\\
93	0.825\\
94	0.85\\
95	0.875\\
96	0.9\\
97	0.925\\
98	0.95\\
99	0.975\\
100	1\\
};
\node[right, inner sep=0mm, text=black]
at (axis cs:1,1.05,0) {$\text{Baja (}\mu{}_{\text{Baja}}\text{(x))}$};
\node[right, inner sep=0mm, text=black]
at (axis cs:40,1.05,0) {$\text{Media (}\mu{}_{\text{Media}}\text{(x))}$};
\node[right, inner sep=0mm, text=black]
at (axis cs:80,1.05,0) {$\text{Alta (}\mu{}_{\text{Alta}}\text{(x))}$};
\end{axis}
\end{tikzpicture}%
	\caption{Ejemplo \ref{ex:lang-variable}: Funciones de pertenencia para los valores \emph{``baja''} ($\mu_{Baja}$),\emph{``media''} ($\mu_{Media}$) y \emph{``alta''} ($\mu_{Alta}$) de la variable lingüística \emph{Temperatura}.}
	\label{fig:example-temp-lang-variable}
\end{figure}

Una definición más formal de \emph{variable lingüística} (equivalente a la definición \ref{def:informal-lang-variable}), dada por Zadeh \cite{Zadeh75}, es la siguiente:

\begin{definition}\label{def:formal-lang-variable}
\normalfont
Una \emph{variable lingüística} se caracteriza por la tupla $(X,T,U,M)$, donde:
\begin{itemize}
  \item $X$ es el nombre de la variable lingüística. En el ejemplo \ref{ex:lang-variable},``Temperatura en ºC''.
  \item $T$ es el conjunto de valores lingüísticos que $X$ puede tomar. En el ejemplo \ref{ex:lang-variable}, $T$ = \{baja, media, alta\}
  \item $U$ es el universo de referencia del que cada valor lingüístico toma sus valores cuantitativos (escalares). En el ejemplo \ref{ex:lang-variable}, $U = [0,100]$
  \item $M$ es la regla semántica que relaciona cada valor lingüístico en $T$ con un conjunto difuso en $U$. En el ejemplo \ref{ex:lang-variable}, $M$ relaciona los valores ``baja'', ``media'' y ``alta'' con las funciones de pertenencia representadas en la figura \ref{fig:example-temp-lang-variable}.
\end{itemize}
\end{definition}
\subsection{Operaciones sobre conjuntos difusos}
\label{sec:fuzzy-set-operations}
En esta sección se definen algunas operaciones básicas sobre conjuntos difusos. Un conjunto difuso compuesto a partir de dos conjuntos difusos sobre el mismo universo \emph{U} puede definirse por su función de pertenencia tal que:
\begin{definition}
Dada una función \begin{math}\emph{F}:[0,1]^2\rightarrow[0,1]\end{math} y dos conjuntos difusos \emph{A} y \emph{B} definidos sobre el mismo universo \emph{U}, \begin{math}A,B\in FS(U)\end{math}, denotamos como \begin{math}F(A,B)\end{math} el conjunto difuso sobre \emph{U} cuya función de pertenencia viene dada por:
\begin{equation}
\mu_{F(A,B)}(u_{i}) = F(A(u_{i}),B(u_{i}))
\end{equation}
\end{definition}
De esta forma, se pueden definir las operaciones de unión ($\cup$) y de intersección ($\cap$) clásicas sobre los subconjuntos difusos \emph{A} y \emph{B} sobre \emph{U} \cite{dubois1980} tal que:
\begin{equation}
\forall\mu_{i}\in U,\quad\mu_{A\cup B}(u_{i}) = max(\mu_{A}(u_{i}),\mu_{B}(u_{i}))
\end{equation}
\begin{equation}
\forall\mu_{i}\in U,\quad\mu_{A\cap B}(u_{i}) = min(\mu_{A}(u_{i}),\mu_{B}(u_{i}))
\end{equation}
donde $\mu_{A\cup B}$ y $\mu_{A\cap B}$ son las funciones de pertenencia de $A\cup B$ (unión) y $A\cap B$ (intersección) respectivamente.
\begin{definition}
El \emph{complementario} \emph{$\overline{A}$} de un conjunto difuso \emph{A} sobre \emph{U} puede definirse por su función de pertenencia \cite{Zadeh65}, tal que:
\begin{equation}
\forall \mu_{i} \in U, \quad \mu_{\overline{A}}(\mu_{i}) = 1 - \mu_{A}(\mu_{i})
\end{equation}
\end{definition}


\section{T-normas y operadores de agregación}\label{sec:t-norms-aggregation-operator}
Dada una función \begin{math}\emph{F}:[0,1]^2\rightarrow[0,1]\end{math} y dos conjuntos difusos \emph{A} y \emph{B} definidos sobre el mismo universo \emph{U}, \begin{math}A,B\in FS(U)\end{math}, denotamos como \begin{math}F(A,B)\end{math} el conjunto difuso sobre \emph{U} cuya función de pertenencia viene dada por:
\begin{equation}
\mu_{F(A,B)}(u_{i}) = F(A(u_{i}),B(u_{i}))
\end{equation}
Una clase importante de este tipo de funciones son las llamadas t-normas y t-conormas (normas triangulares) \cite{klement2000triangular}.
\begin{definition}\label{def:tnorma}
Una t-norma es una operación binaria \emph{T} en el intervalo $[0,1]$ que es conmutativa, asociativa, monótona y tiene el valor \emph{1} como elemento neutro. Es decir, una función $\emph{T} : [0,1]^2 \rightarrow [0,1]$ tal que $\forall x,y,z \in [0,1]$:
\begin{enumerate}[label=(T\arabic*),ref=(T\arabic*)]
   \item $T(x,y) = T(y,x)$ (Conmutatividad)\label{T1}
   \item $T(x,T(y,z)) = T(T(x,y),z)$ (Asociatividad)\label{T2}
   \item $T(x,y) \leq T(x,z)$ cuando $y \leq z$ (Monotonía)\label{T3}
   \item $T(x,1) = x$ (Elemento neutro)\label{T4}
  \end{enumerate}
\end{definition}
Estas propiedades son suficientes para garantizar que las t-normas generalizan la conjunción clásica ($x\bigwedge y$) cuando se aplican a valores booleanos ($T(0,0)=0$,$T(0,1)=T(1,0)=0$ y $T(1,1)=1$).
Algunos ejemplos prominentes de t-normas son los siguientes:
\begin{itemize}
	\item \bfseries Mínimo o t-norma de Gödel: $T_{G}(x,y) = min\{x,y\}$
	\item \bfseries Producto: $T_{P}(x,y) = xy$
	\item \bfseries \L{}ukasiewicz: $T_{L}(x,y) = max\{x+y-1,0\}$
	\item \bfseries T-norma drástica: \mdseries $T_{D}(x,y) = \begin{cases} y & if\quad x=1, \\ x & if\quad y=1, \\0 & \text{en cualquier otro caso}. \end{cases}$
\end{itemize}
\begin{figure}[H]
	\centering
	\begin{subfigure}[b]{0.45\textwidth}
		\setlength\figureheight{4.5cm}
		\setlength\figurewidth{6cm}
		% This file was created by matlab2tikz v0.4.7 (commit 4d7ae5c4fd0932fb51051d86111bc23ed23e4580) running on MATLAB 8.0.
% Copyright (c) 2008--2014, Nico Schlömer <nico.schloemer@gmail.com>
% All rights reserved.
% Minimal pgfplots version: 1.3
% 
\begin{tikzpicture}

\begin{axis}[%
width=\figurewidth,
height=\figureheight,
view={-37.5}{30},
scale only axis,
xmin=0,
xmax=1,
xlabel={x},
xmajorgrids,
ymin=0,
ymax=1,
ylabel={y},
ymajorgrids,
zmin=0,
zmax=1,
zlabel={$\text{T}_{\text{G}}\text{(x,y)}$},
zmajorgrids,
axis x line*=bottom,
axis y line*=left,
axis z line*=left
]

\addplot3[%
surf,
shader=faceted,
draw=black,
colormap/jet,
mesh/rows=11]
table[row sep=crcr,header=false] {0	0	0\\
0	0.1	0\\
0	0.2	0\\
0	0.3	0\\
0	0.4	0\\
0	0.5	0\\
0	0.6	0\\
0	0.7	0\\
0	0.8	0\\
0	0.9	0\\
0	1	0\\
0.1	0	0\\
0.1	0.1	0.1\\
0.1	0.2	0.1\\
0.1	0.3	0.1\\
0.1	0.4	0.1\\
0.1	0.5	0.1\\
0.1	0.6	0.1\\
0.1	0.7	0.1\\
0.1	0.8	0.1\\
0.1	0.9	0.1\\
0.1	1	0.1\\
0.2	0	0\\
0.2	0.1	0.1\\
0.2	0.2	0.2\\
0.2	0.3	0.2\\
0.2	0.4	0.2\\
0.2	0.5	0.2\\
0.2	0.6	0.2\\
0.2	0.7	0.2\\
0.2	0.8	0.2\\
0.2	0.9	0.2\\
0.2	1	0.2\\
0.3	0	0\\
0.3	0.1	0.1\\
0.3	0.2	0.2\\
0.3	0.3	0.3\\
0.3	0.4	0.3\\
0.3	0.5	0.3\\
0.3	0.6	0.3\\
0.3	0.7	0.3\\
0.3	0.8	0.3\\
0.3	0.9	0.3\\
0.3	1	0.3\\
0.4	0	0\\
0.4	0.1	0.1\\
0.4	0.2	0.2\\
0.4	0.3	0.3\\
0.4	0.4	0.4\\
0.4	0.5	0.4\\
0.4	0.6	0.4\\
0.4	0.7	0.4\\
0.4	0.8	0.4\\
0.4	0.9	0.4\\
0.4	1	0.4\\
0.5	0	0\\
0.5	0.1	0.1\\
0.5	0.2	0.2\\
0.5	0.3	0.3\\
0.5	0.4	0.4\\
0.5	0.5	0.5\\
0.5	0.6	0.5\\
0.5	0.7	0.5\\
0.5	0.8	0.5\\
0.5	0.9	0.5\\
0.5	1	0.5\\
0.6	0	0\\
0.6	0.1	0.1\\
0.6	0.2	0.2\\
0.6	0.3	0.3\\
0.6	0.4	0.4\\
0.6	0.5	0.5\\
0.6	0.6	0.6\\
0.6	0.7	0.6\\
0.6	0.8	0.6\\
0.6	0.9	0.6\\
0.6	1	0.6\\
0.7	0	0\\
0.7	0.1	0.1\\
0.7	0.2	0.2\\
0.7	0.3	0.3\\
0.7	0.4	0.4\\
0.7	0.5	0.5\\
0.7	0.6	0.6\\
0.7	0.7	0.7\\
0.7	0.8	0.7\\
0.7	0.9	0.7\\
0.7	1	0.7\\
0.8	0	0\\
0.8	0.1	0.1\\
0.8	0.2	0.2\\
0.8	0.3	0.3\\
0.8	0.4	0.4\\
0.8	0.5	0.5\\
0.8	0.6	0.6\\
0.8	0.7	0.7\\
0.8	0.8	0.8\\
0.8	0.9	0.8\\
0.8	1	0.8\\
0.9	0	0\\
0.9	0.1	0.1\\
0.9	0.2	0.2\\
0.9	0.3	0.3\\
0.9	0.4	0.4\\
0.9	0.5	0.5\\
0.9	0.6	0.6\\
0.9	0.7	0.7\\
0.9	0.8	0.8\\
0.9	0.9	0.9\\
0.9	1	0.9\\
1	0	0\\
1	0.1	0.1\\
1	0.2	0.2\\
1	0.3	0.3\\
1	0.4	0.4\\
1	0.5	0.5\\
1	0.6	0.6\\
1	0.7	0.7\\
1	0.8	0.8\\
1	0.9	0.9\\
1	1	1\\
};
\end{axis}
\end{tikzpicture}%
		\caption{Mínimo}
		\label{fig:t-norms-min}
	\end{subfigure}
	\qquad
	\begin{subfigure}[b]{0.45\textwidth}
		\setlength\figureheight{4.5cm}
		\setlength\figurewidth{6cm}
		% This file was created by matlab2tikz v0.4.7 (commit 4d7ae5c4fd0932fb51051d86111bc23ed23e4580) running on MATLAB 8.0.
% Copyright (c) 2008--2014, Nico Schlömer <nico.schloemer@gmail.com>
% All rights reserved.
% Minimal pgfplots version: 1.3
% 
\begin{tikzpicture}

\begin{axis}[%
width=\figurewidth,
height=\figureheight,
view={-37.5}{30},
scale only axis,
xmin=0,
xmax=1,
xlabel={x},
xmajorgrids,
ymin=0,
ymax=1,
ylabel={y},
ymajorgrids,
zmin=0,
zmax=1,
zlabel={$\text{T}_{\text{P}}\text{(x,y)}$},
zmajorgrids,
axis x line*=bottom,
axis y line*=left,
axis z line*=left
]

\addplot3[%
surf,
shader=faceted,
draw=black,
colormap/jet,
mesh/rows=11]
table[row sep=crcr,header=false] {0	0	0\\
0	0.1	0\\
0	0.2	0\\
0	0.3	0\\
0	0.4	0\\
0	0.5	0\\
0	0.6	0\\
0	0.7	0\\
0	0.8	0\\
0	0.9	0\\
0	1	0\\
0.1	0	0\\
0.1	0.1	0.01\\
0.1	0.2	0.02\\
0.1	0.3	0.03\\
0.1	0.4	0.04\\
0.1	0.5	0.05\\
0.1	0.6	0.06\\
0.1	0.7	0.07\\
0.1	0.8	0.08\\
0.1	0.9	0.09\\
0.1	1	0.1\\
0.2	0	0\\
0.2	0.1	0.02\\
0.2	0.2	0.04\\
0.2	0.3	0.06\\
0.2	0.4	0.08\\
0.2	0.5	0.1\\
0.2	0.6	0.12\\
0.2	0.7	0.14\\
0.2	0.8	0.16\\
0.2	0.9	0.18\\
0.2	1	0.2\\
0.3	0	0\\
0.3	0.1	0.03\\
0.3	0.2	0.06\\
0.3	0.3	0.09\\
0.3	0.4	0.12\\
0.3	0.5	0.15\\
0.3	0.6	0.18\\
0.3	0.7	0.21\\
0.3	0.8	0.24\\
0.3	0.9	0.27\\
0.3	1	0.3\\
0.4	0	0\\
0.4	0.1	0.04\\
0.4	0.2	0.08\\
0.4	0.3	0.12\\
0.4	0.4	0.16\\
0.4	0.5	0.2\\
0.4	0.6	0.24\\
0.4	0.7	0.28\\
0.4	0.8	0.32\\
0.4	0.9	0.36\\
0.4	1	0.4\\
0.5	0	0\\
0.5	0.1	0.05\\
0.5	0.2	0.1\\
0.5	0.3	0.15\\
0.5	0.4	0.2\\
0.5	0.5	0.25\\
0.5	0.6	0.3\\
0.5	0.7	0.35\\
0.5	0.8	0.4\\
0.5	0.9	0.45\\
0.5	1	0.5\\
0.6	0	0\\
0.6	0.1	0.06\\
0.6	0.2	0.12\\
0.6	0.3	0.18\\
0.6	0.4	0.24\\
0.6	0.5	0.3\\
0.6	0.6	0.36\\
0.6	0.7	0.42\\
0.6	0.8	0.48\\
0.6	0.9	0.54\\
0.6	1	0.6\\
0.7	0	0\\
0.7	0.1	0.07\\
0.7	0.2	0.14\\
0.7	0.3	0.21\\
0.7	0.4	0.28\\
0.7	0.5	0.35\\
0.7	0.6	0.42\\
0.7	0.7	0.49\\
0.7	0.8	0.56\\
0.7	0.9	0.63\\
0.7	1	0.7\\
0.8	0	0\\
0.8	0.1	0.08\\
0.8	0.2	0.16\\
0.8	0.3	0.24\\
0.8	0.4	0.32\\
0.8	0.5	0.4\\
0.8	0.6	0.48\\
0.8	0.7	0.56\\
0.8	0.8	0.64\\
0.8	0.9	0.72\\
0.8	1	0.8\\
0.9	0	0\\
0.9	0.1	0.09\\
0.9	0.2	0.18\\
0.9	0.3	0.27\\
0.9	0.4	0.36\\
0.9	0.5	0.45\\
0.9	0.6	0.54\\
0.9	0.7	0.63\\
0.9	0.8	0.72\\
0.9	0.9	0.81\\
0.9	1	0.9\\
1	0	0\\
1	0.1	0.1\\
1	0.2	0.2\\
1	0.3	0.3\\
1	0.4	0.4\\
1	0.5	0.5\\
1	0.6	0.6\\
1	0.7	0.7\\
1	0.8	0.8\\
1	0.9	0.9\\
1	1	1\\
};
\end{axis}
\end{tikzpicture}%
		\caption{Producto}
		\label{fig:t-norms-prod}
	\end{subfigure}
	
	\vspace{1 cm}
	\begin{subfigure}[b]{0.45\textwidth}
		\setlength\figureheight{4.5cm}
		\setlength\figurewidth{6cm}
		% This file was created by matlab2tikz v0.4.7 (commit 4d7ae5c4fd0932fb51051d86111bc23ed23e4580) running on MATLAB 8.0.
% Copyright (c) 2008--2014, Nico Schlömer <nico.schloemer@gmail.com>
% All rights reserved.
% Minimal pgfplots version: 1.3
% 
\begin{tikzpicture}

\begin{axis}[%
width=\figurewidth,
height=\figureheight,
view={-37.5}{30},
scale only axis,
xmin=0,
xmax=1,
xlabel={x},
xmajorgrids,
ymin=0,
ymax=1,
ylabel={y},
ymajorgrids,
zmin=0,
zmax=1,
zlabel={$\text{T}_{\text{L}}\text{(x,y)}$},
zmajorgrids,
axis x line*=bottom,
axis y line*=left,
axis z line*=left
]

\addplot3[%
surf,
shader=faceted,
draw=black,
colormap/jet,
mesh/rows=11]
table[row sep=crcr,header=false] {0	0	0\\
0	0.1	0\\
0	0.2	0\\
0	0.3	0\\
0	0.4	0\\
0	0.5	0\\
0	0.6	0\\
0	0.7	0\\
0	0.8	0\\
0	0.9	0\\
0	1	0\\
0.1	0	0\\
0.1	0.1	0\\
0.1	0.2	0\\
0.1	0.3	0\\
0.1	0.4	0\\
0.1	0.5	0\\
0.1	0.6	0\\
0.1	0.7	0\\
0.1	0.8	0\\
0.1	0.9	0\\
0.1	1	0.1\\
0.2	0	0\\
0.2	0.1	0\\
0.2	0.2	0\\
0.2	0.3	0\\
0.2	0.4	0\\
0.2	0.5	0\\
0.2	0.6	0\\
0.2	0.7	0\\
0.2	0.8	0\\
0.2	0.9	0.1\\
0.2	1	0.2\\
0.3	0	0\\
0.3	0.1	0\\
0.3	0.2	0\\
0.3	0.3	0\\
0.3	0.4	0\\
0.3	0.5	0\\
0.3	0.6	0\\
0.3	0.7	0\\
0.3	0.8	0.1\\
0.3	0.9	0.2\\
0.3	1	0.3\\
0.4	0	0\\
0.4	0.1	0\\
0.4	0.2	0\\
0.4	0.3	0\\
0.4	0.4	0\\
0.4	0.5	0\\
0.4	0.6	0\\
0.4	0.7	0.1\\
0.4	0.8	0.2\\
0.4	0.9	0.3\\
0.4	1	0.4\\
0.5	0	0\\
0.5	0.1	0\\
0.5	0.2	0\\
0.5	0.3	0\\
0.5	0.4	0\\
0.5	0.5	0\\
0.5	0.6	0.1\\
0.5	0.7	0.2\\
0.5	0.8	0.3\\
0.5	0.9	0.4\\
0.5	1	0.5\\
0.6	0	0\\
0.6	0.1	0\\
0.6	0.2	0\\
0.6	0.3	0\\
0.6	0.4	0\\
0.6	0.5	0.1\\
0.6	0.6	0.2\\
0.6	0.7	0.3\\
0.6	0.8	0.4\\
0.6	0.9	0.5\\
0.6	1	0.6\\
0.7	0	0\\
0.7	0.1	0\\
0.7	0.2	0\\
0.7	0.3	0\\
0.7	0.4	0.1\\
0.7	0.5	0.2\\
0.7	0.6	0.3\\
0.7	0.7	0.4\\
0.7	0.8	0.5\\
0.7	0.9	0.6\\
0.7	1	0.7\\
0.8	0	0\\
0.8	0.1	0\\
0.8	0.2	0\\
0.8	0.3	0.1\\
0.8	0.4	0.2\\
0.8	0.5	0.3\\
0.8	0.6	0.4\\
0.8	0.7	0.5\\
0.8	0.8	0.6\\
0.8	0.9	0.7\\
0.8	1	0.8\\
0.9	0	0\\
0.9	0.1	0\\
0.9	0.2	0.1\\
0.9	0.3	0.2\\
0.9	0.4	0.3\\
0.9	0.5	0.4\\
0.9	0.6	0.5\\
0.9	0.7	0.6\\
0.9	0.8	0.7\\
0.9	0.9	0.8\\
0.9	1	0.9\\
1	0	0\\
1	0.1	0.1\\
1	0.2	0.2\\
1	0.3	0.3\\
1	0.4	0.4\\
1	0.5	0.5\\
1	0.6	0.6\\
1	0.7	0.7\\
1	0.8	0.8\\
1	0.9	0.9\\
1	1	1\\
};
\end{axis}
\end{tikzpicture}%
		\caption{\L{}ukasiewicz}
		\label{fig:t-norms-lukasiewicz}
	\end{subfigure}
	\qquad
	\begin{subfigure}[b]{0.45\textwidth}
			\setlength\figureheight{4.5cm}
			\setlength\figurewidth{6cm}
			% This file was created by matlab2tikz v0.4.7 (commit 730e9c094bba45cf576cc1ba69976f02cfab8e0c) running on MATLAB 8.0.
% Copyright (c) 2008--2014, Nico Schlömer <nico.schloemer@gmail.com>
% All rights reserved.
% Minimal pgfplots version: 1.3
% 
\begin{tikzpicture}

\begin{axis}[%
width=\figurewidth,
height=\figureheight,
view={-37.5}{30},
scale only axis,
xmin=0,
xmax=1,
xlabel={x},
xmajorgrids,
ymin=0,
ymax=1,
ylabel={y},
ymajorgrids,
zmin=0,
zmax=1,
zlabel={$\text{T}_{\text{D}}\text{(x,y)}$},
zmajorgrids,
axis x line*=bottom,
axis y line*=left,
axis z line*=left
]

\addplot3[%
surf,
shader=faceted,
draw=black,
colormap/jet,
mesh/rows=11]
table[row sep=crcr,header=false] {0	0	0\\
0	0.1	0\\
0	0.2	0\\
0	0.3	0\\
0	0.4	0\\
0	0.5	0\\
0	0.6	0\\
0	0.7	0\\
0	0.8	0\\
0	0.9	0\\
0	1	0\\
0.1	0	0\\
0.1	0.1	0\\
0.1	0.2	0\\
0.1	0.3	0\\
0.1	0.4	0\\
0.1	0.5	0\\
0.1	0.6	0\\
0.1	0.7	0\\
0.1	0.8	0\\
0.1	0.9	0\\
0.1	1	0.1\\
0.2	0	0\\
0.2	0.1	0\\
0.2	0.2	0\\
0.2	0.3	0\\
0.2	0.4	0\\
0.2	0.5	0\\
0.2	0.6	0\\
0.2	0.7	0\\
0.2	0.8	0\\
0.2	0.9	0\\
0.2	1	0.2\\
0.3	0	0\\
0.3	0.1	0\\
0.3	0.2	0\\
0.3	0.3	0\\
0.3	0.4	0\\
0.3	0.5	0\\
0.3	0.6	0\\
0.3	0.7	0\\
0.3	0.8	0\\
0.3	0.9	0\\
0.3	1	0.3\\
0.4	0	0\\
0.4	0.1	0\\
0.4	0.2	0\\
0.4	0.3	0\\
0.4	0.4	0\\
0.4	0.5	0\\
0.4	0.6	0\\
0.4	0.7	0\\
0.4	0.8	0\\
0.4	0.9	0\\
0.4	1	0.4\\
0.5	0	0\\
0.5	0.1	0\\
0.5	0.2	0\\
0.5	0.3	0\\
0.5	0.4	0\\
0.5	0.5	0\\
0.5	0.6	0\\
0.5	0.7	0\\
0.5	0.8	0\\
0.5	0.9	0\\
0.5	1	0.5\\
0.6	0	0\\
0.6	0.1	0\\
0.6	0.2	0\\
0.6	0.3	0\\
0.6	0.4	0\\
0.6	0.5	0\\
0.6	0.6	0\\
0.6	0.7	0\\
0.6	0.8	0\\
0.6	0.9	0\\
0.6	1	0.6\\
0.7	0	0\\
0.7	0.1	0\\
0.7	0.2	0\\
0.7	0.3	0\\
0.7	0.4	0\\
0.7	0.5	0\\
0.7	0.6	0\\
0.7	0.7	0\\
0.7	0.8	0\\
0.7	0.9	0\\
0.7	1	0.7\\
0.8	0	0\\
0.8	0.1	0\\
0.8	0.2	0\\
0.8	0.3	0\\
0.8	0.4	0\\
0.8	0.5	0\\
0.8	0.6	0\\
0.8	0.7	0\\
0.8	0.8	0\\
0.8	0.9	0\\
0.8	1	0.8\\
0.9	0	0\\
0.9	0.1	0\\
0.9	0.2	0\\
0.9	0.3	0\\
0.9	0.4	0\\
0.9	0.5	0\\
0.9	0.6	0\\
0.9	0.7	0\\
0.9	0.8	0\\
0.9	0.9	0\\
0.9	1	0.9\\
1	0	0\\
1	0.1	0.1\\
1	0.2	0.2\\
1	0.3	0.3\\
1	0.4	0.4\\
1	0.5	0.5\\
1	0.6	0.6\\
1	0.7	0.7\\
1	0.8	0.8\\
1	0.9	0.9\\
1	1	1\\
};
\end{axis}
\end{tikzpicture}%
			\caption{T-norma drástica}
			\label{fig:t-norms-drastic}
	\end{subfigure}
	\label{fig:t-norms}
	\caption{Representación gráfica de las t-normas \emph{Mínimo} (\ref{fig:t-norms-min}), \emph{Producto} (\ref{fig:t-norms-prod}), \emph{\L{}ukasiewicz} (\ref{fig:t-norms-lukasiewicz}) y la \emph{t-norma drástica} (\ref{fig:t-norms-drastic}).}
\end{figure}

\begin{definition}
Una t-conorma es una operación binaria \emph{S} en el intervalo $[0,1]$ que es conmutativa, asociativa, monotona y tiene el valor \emph{0} como elemento neutro. Es decir, una función $\emph{S} : [0,1]^2 \rightarrow [0,1]$ tal que $\forall x,y,z \in [0,1]$:
\begin{enumerate}[label=(S\arabic*),ref=(S\arabic*)]
   \item $S(x,y) = S(y,x)$ (Conmutatividad)\label{S1}
   \item $S(x,S(y,z)) = S(S(x,y),z)$ (Asociatividad)\label{S2}
   \item $S(x,y) \leq S(x,z)$ cuando $y \leq z$ (Monotonía)\label{S3}
   \item $S(x,0) = x$ (Elemento neutro)\label{S4}
  \end{enumerate}
\end{definition}
De la misma forma que las t-normas (definición \ref{def:tnorma}) generalizan la conjunción clásica, las t-conormas generalizan la disyunción ($x\bigvee y$). A continuación se incluyen algunos ejemplos de t-conormas:
\begin{itemize}
	\item \bfseries Máximo: $S_{max}(x,y) = max\{x,y\}$
	\item \bfseries Suma probabilística: $S_{P}(x,y) = x + y - xy$
	\item \bfseries Suma acotada: $S_{B}(x,y) = min\{x+y,1\}$
	\item \bfseries T-conorma drástica: \mdseries $S_{D}(x,y) = \begin{cases} y & if\quad x=0, \\ x & if\quad y=0, \\1 & \text{en cualquier otro caso}. \end{cases}$
\end{itemize}
\begin{figure}[H]
	\centering
	\begin{subfigure}[b]{0.45\textwidth}
		\setlength\figureheight{4.5cm}
		\setlength\figurewidth{6cm}
		% This file was created by matlab2tikz v0.4.7 (commit 4d7ae5c4fd0932fb51051d86111bc23ed23e4580) running on MATLAB 8.0.
% Copyright (c) 2008--2014, Nico Schlömer <nico.schloemer@gmail.com>
% All rights reserved.
% Minimal pgfplots version: 1.3
% 
\begin{tikzpicture}

\begin{axis}[%
width=\figurewidth,
height=\figureheight,
view={-37.5}{30},
scale only axis,
xmin=0,
xmax=1,
xlabel={x},
xmajorgrids,
ymin=0,
ymax=1,
ylabel={y},
ymajorgrids,
zmin=0,
zmax=1,
zlabel={$\text{S}_{\text{max}}\text{(x,y)}$},
zmajorgrids,
axis x line*=bottom,
axis y line*=left,
axis z line*=left
]

\addplot3[%
surf,
shader=faceted,
draw=black,
colormap/jet,
mesh/rows=11]
table[row sep=crcr,header=false] {0	0	0\\
0	0.1	0.1\\
0	0.2	0.2\\
0	0.3	0.3\\
0	0.4	0.4\\
0	0.5	0.5\\
0	0.6	0.6\\
0	0.7	0.7\\
0	0.8	0.8\\
0	0.9	0.9\\
0	1	1\\
0.1	0	0.1\\
0.1	0.1	0.1\\
0.1	0.2	0.2\\
0.1	0.3	0.3\\
0.1	0.4	0.4\\
0.1	0.5	0.5\\
0.1	0.6	0.6\\
0.1	0.7	0.7\\
0.1	0.8	0.8\\
0.1	0.9	0.9\\
0.1	1	1\\
0.2	0	0.2\\
0.2	0.1	0.2\\
0.2	0.2	0.2\\
0.2	0.3	0.3\\
0.2	0.4	0.4\\
0.2	0.5	0.5\\
0.2	0.6	0.6\\
0.2	0.7	0.7\\
0.2	0.8	0.8\\
0.2	0.9	0.9\\
0.2	1	1\\
0.3	0	0.3\\
0.3	0.1	0.3\\
0.3	0.2	0.3\\
0.3	0.3	0.3\\
0.3	0.4	0.4\\
0.3	0.5	0.5\\
0.3	0.6	0.6\\
0.3	0.7	0.7\\
0.3	0.8	0.8\\
0.3	0.9	0.9\\
0.3	1	1\\
0.4	0	0.4\\
0.4	0.1	0.4\\
0.4	0.2	0.4\\
0.4	0.3	0.4\\
0.4	0.4	0.4\\
0.4	0.5	0.5\\
0.4	0.6	0.6\\
0.4	0.7	0.7\\
0.4	0.8	0.8\\
0.4	0.9	0.9\\
0.4	1	1\\
0.5	0	0.5\\
0.5	0.1	0.5\\
0.5	0.2	0.5\\
0.5	0.3	0.5\\
0.5	0.4	0.5\\
0.5	0.5	0.5\\
0.5	0.6	0.6\\
0.5	0.7	0.7\\
0.5	0.8	0.8\\
0.5	0.9	0.9\\
0.5	1	1\\
0.6	0	0.6\\
0.6	0.1	0.6\\
0.6	0.2	0.6\\
0.6	0.3	0.6\\
0.6	0.4	0.6\\
0.6	0.5	0.6\\
0.6	0.6	0.6\\
0.6	0.7	0.7\\
0.6	0.8	0.8\\
0.6	0.9	0.9\\
0.6	1	1\\
0.7	0	0.7\\
0.7	0.1	0.7\\
0.7	0.2	0.7\\
0.7	0.3	0.7\\
0.7	0.4	0.7\\
0.7	0.5	0.7\\
0.7	0.6	0.7\\
0.7	0.7	0.7\\
0.7	0.8	0.8\\
0.7	0.9	0.9\\
0.7	1	1\\
0.8	0	0.8\\
0.8	0.1	0.8\\
0.8	0.2	0.8\\
0.8	0.3	0.8\\
0.8	0.4	0.8\\
0.8	0.5	0.8\\
0.8	0.6	0.8\\
0.8	0.7	0.8\\
0.8	0.8	0.8\\
0.8	0.9	0.9\\
0.8	1	1\\
0.9	0	0.9\\
0.9	0.1	0.9\\
0.9	0.2	0.9\\
0.9	0.3	0.9\\
0.9	0.4	0.9\\
0.9	0.5	0.9\\
0.9	0.6	0.9\\
0.9	0.7	0.9\\
0.9	0.8	0.9\\
0.9	0.9	0.9\\
0.9	1	1\\
1	0	1\\
1	0.1	1\\
1	0.2	1\\
1	0.3	1\\
1	0.4	1\\
1	0.5	1\\
1	0.6	1\\
1	0.7	1\\
1	0.8	1\\
1	0.9	1\\
1	1	1\\
};
\end{axis}
\end{tikzpicture}%
		\caption{Máximo}
		\label{fig:t-conorms-max}
	\end{subfigure}
	\qquad
	\begin{subfigure}[b]{0.45\textwidth}
		\setlength\figureheight{4.5cm}
		\setlength\figurewidth{6cm}
		% This file was created by matlab2tikz v0.4.7 (commit 4d7ae5c4fd0932fb51051d86111bc23ed23e4580) running on MATLAB 8.0.
% Copyright (c) 2008--2014, Nico Schlömer <nico.schloemer@gmail.com>
% All rights reserved.
% Minimal pgfplots version: 1.3
% 
\begin{tikzpicture}

\begin{axis}[%
width=\figurewidth,
height=\figureheight,
view={-37.5}{30},
scale only axis,
xmin=0,
xmax=1,
xlabel={x},
xmajorgrids,
ymin=0,
ymax=1,
ylabel={y},
ymajorgrids,
zmin=0,
zmax=1,
zlabel={$\text{S}_{\text{P}}\text{(x,y)}$},
zmajorgrids,
axis x line*=bottom,
axis y line*=left,
axis z line*=left
]

\addplot3[%
surf,
shader=faceted,
draw=black,
colormap/jet,
mesh/rows=11]
table[row sep=crcr,header=false] {0	0	0\\
0	0.1	0.1\\
0	0.2	0.2\\
0	0.3	0.3\\
0	0.4	0.4\\
0	0.5	0.5\\
0	0.6	0.6\\
0	0.7	0.7\\
0	0.8	0.8\\
0	0.9	0.9\\
0	1	1\\
0.1	0	0.1\\
0.1	0.1	0.19\\
0.1	0.2	0.28\\
0.1	0.3	0.37\\
0.1	0.4	0.46\\
0.1	0.5	0.55\\
0.1	0.6	0.64\\
0.1	0.7	0.73\\
0.1	0.8	0.82\\
0.1	0.9	0.91\\
0.1	1	1\\
0.2	0	0.2\\
0.2	0.1	0.28\\
0.2	0.2	0.36\\
0.2	0.3	0.44\\
0.2	0.4	0.52\\
0.2	0.5	0.6\\
0.2	0.6	0.68\\
0.2	0.7	0.76\\
0.2	0.8	0.84\\
0.2	0.9	0.92\\
0.2	1	1\\
0.3	0	0.3\\
0.3	0.1	0.37\\
0.3	0.2	0.44\\
0.3	0.3	0.51\\
0.3	0.4	0.58\\
0.3	0.5	0.65\\
0.3	0.6	0.72\\
0.3	0.7	0.79\\
0.3	0.8	0.86\\
0.3	0.9	0.93\\
0.3	1	1\\
0.4	0	0.4\\
0.4	0.1	0.46\\
0.4	0.2	0.52\\
0.4	0.3	0.58\\
0.4	0.4	0.64\\
0.4	0.5	0.7\\
0.4	0.6	0.76\\
0.4	0.7	0.82\\
0.4	0.8	0.88\\
0.4	0.9	0.94\\
0.4	1	1\\
0.5	0	0.5\\
0.5	0.1	0.55\\
0.5	0.2	0.6\\
0.5	0.3	0.65\\
0.5	0.4	0.7\\
0.5	0.5	0.75\\
0.5	0.6	0.8\\
0.5	0.7	0.85\\
0.5	0.8	0.9\\
0.5	0.9	0.95\\
0.5	1	1\\
0.6	0	0.6\\
0.6	0.1	0.64\\
0.6	0.2	0.68\\
0.6	0.3	0.72\\
0.6	0.4	0.76\\
0.6	0.5	0.8\\
0.6	0.6	0.84\\
0.6	0.7	0.88\\
0.6	0.8	0.92\\
0.6	0.9	0.96\\
0.6	1	1\\
0.7	0	0.7\\
0.7	0.1	0.73\\
0.7	0.2	0.76\\
0.7	0.3	0.79\\
0.7	0.4	0.82\\
0.7	0.5	0.85\\
0.7	0.6	0.88\\
0.7	0.7	0.91\\
0.7	0.8	0.94\\
0.7	0.9	0.97\\
0.7	1	1\\
0.8	0	0.8\\
0.8	0.1	0.82\\
0.8	0.2	0.84\\
0.8	0.3	0.86\\
0.8	0.4	0.88\\
0.8	0.5	0.9\\
0.8	0.6	0.92\\
0.8	0.7	0.94\\
0.8	0.8	0.96\\
0.8	0.9	0.98\\
0.8	1	1\\
0.9	0	0.9\\
0.9	0.1	0.91\\
0.9	0.2	0.92\\
0.9	0.3	0.93\\
0.9	0.4	0.94\\
0.9	0.5	0.95\\
0.9	0.6	0.96\\
0.9	0.7	0.97\\
0.9	0.8	0.98\\
0.9	0.9	0.99\\
0.9	1	1\\
1	0	1\\
1	0.1	1\\
1	0.2	1\\
1	0.3	1\\
1	0.4	1\\
1	0.5	1\\
1	0.6	1\\
1	0.7	1\\
1	0.8	1\\
1	0.9	1\\
1	1	1\\
};
\end{axis}
\end{tikzpicture}%
		\caption{Suma probabilística}
		\label{fig:t-conorms-probabilistic-sum}
	\end{subfigure}
	
	\vspace{1 cm}
	\begin{subfigure}[b]{0.45\textwidth}
		\setlength\figureheight{4.5cm}
		\setlength\figurewidth{6cm}
		% This file was created by matlab2tikz v0.4.7 (commit 4d7ae5c4fd0932fb51051d86111bc23ed23e4580) running on MATLAB 8.0.
% Copyright (c) 2008--2014, Nico Schlömer <nico.schloemer@gmail.com>
% All rights reserved.
% Minimal pgfplots version: 1.3
% 
\begin{tikzpicture}

\begin{axis}[%
width=\figurewidth,
height=\figureheight,
view={-37.5}{30},
scale only axis,
xmin=0,
xmax=1,
xlabel={x},
xmajorgrids,
ymin=0,
ymax=1,
ylabel={y},
ymajorgrids,
zmin=0,
zmax=1,
zlabel={$\text{S}_{\text{B}}\text{(x,y)}$},
zmajorgrids,
axis x line*=bottom,
axis y line*=left,
axis z line*=left
]

\addplot3[%
surf,
shader=faceted,
draw=black,
colormap/jet,
mesh/rows=11]
table[row sep=crcr,header=false] {0	0	0\\
0	0.1	0.1\\
0	0.2	0.2\\
0	0.3	0.3\\
0	0.4	0.4\\
0	0.5	0.5\\
0	0.6	0.6\\
0	0.7	0.7\\
0	0.8	0.8\\
0	0.9	0.9\\
0	1	1\\
0.1	0	0.1\\
0.1	0.1	0.2\\
0.1	0.2	0.3\\
0.1	0.3	0.4\\
0.1	0.4	0.5\\
0.1	0.5	0.6\\
0.1	0.6	0.7\\
0.1	0.7	0.8\\
0.1	0.8	0.9\\
0.1	0.9	1\\
0.1	1	1\\
0.2	0	0.2\\
0.2	0.1	0.3\\
0.2	0.2	0.4\\
0.2	0.3	0.5\\
0.2	0.4	0.6\\
0.2	0.5	0.7\\
0.2	0.6	0.8\\
0.2	0.7	0.9\\
0.2	0.8	1\\
0.2	0.9	1\\
0.2	1	1\\
0.3	0	0.3\\
0.3	0.1	0.4\\
0.3	0.2	0.5\\
0.3	0.3	0.6\\
0.3	0.4	0.7\\
0.3	0.5	0.8\\
0.3	0.6	0.9\\
0.3	0.7	1\\
0.3	0.8	1\\
0.3	0.9	1\\
0.3	1	1\\
0.4	0	0.4\\
0.4	0.1	0.5\\
0.4	0.2	0.6\\
0.4	0.3	0.7\\
0.4	0.4	0.8\\
0.4	0.5	0.9\\
0.4	0.6	1\\
0.4	0.7	1\\
0.4	0.8	1\\
0.4	0.9	1\\
0.4	1	1\\
0.5	0	0.5\\
0.5	0.1	0.6\\
0.5	0.2	0.7\\
0.5	0.3	0.8\\
0.5	0.4	0.9\\
0.5	0.5	1\\
0.5	0.6	1\\
0.5	0.7	1\\
0.5	0.8	1\\
0.5	0.9	1\\
0.5	1	1\\
0.6	0	0.6\\
0.6	0.1	0.7\\
0.6	0.2	0.8\\
0.6	0.3	0.9\\
0.6	0.4	1\\
0.6	0.5	1\\
0.6	0.6	1\\
0.6	0.7	1\\
0.6	0.8	1\\
0.6	0.9	1\\
0.6	1	1\\
0.7	0	0.7\\
0.7	0.1	0.8\\
0.7	0.2	0.9\\
0.7	0.3	1\\
0.7	0.4	1\\
0.7	0.5	1\\
0.7	0.6	1\\
0.7	0.7	1\\
0.7	0.8	1\\
0.7	0.9	1\\
0.7	1	1\\
0.8	0	0.8\\
0.8	0.1	0.9\\
0.8	0.2	1\\
0.8	0.3	1\\
0.8	0.4	1\\
0.8	0.5	1\\
0.8	0.6	1\\
0.8	0.7	1\\
0.8	0.8	1\\
0.8	0.9	1\\
0.8	1	1\\
0.9	0	0.9\\
0.9	0.1	1\\
0.9	0.2	1\\
0.9	0.3	1\\
0.9	0.4	1\\
0.9	0.5	1\\
0.9	0.6	1\\
0.9	0.7	1\\
0.9	0.8	1\\
0.9	0.9	1\\
0.9	1	1\\
1	0	1\\
1	0.1	1\\
1	0.2	1\\
1	0.3	1\\
1	0.4	1\\
1	0.5	1\\
1	0.6	1\\
1	0.7	1\\
1	0.8	1\\
1	0.9	1\\
1	1	1\\
};
\end{axis}
\end{tikzpicture}%
		\caption{Suma acotada}
		\label{fig:t-conorms-bounded-sum}
	\end{subfigure}
	\qquad
	\begin{subfigure}[b]{0.45\textwidth}
			\setlength\figureheight{4.5cm}
			\setlength\figurewidth{6cm}
			% This file was created by matlab2tikz v0.4.7 (commit 730e9c094bba45cf576cc1ba69976f02cfab8e0c) running on MATLAB 8.0.
% Copyright (c) 2008--2014, Nico Schlömer <nico.schloemer@gmail.com>
% All rights reserved.
% Minimal pgfplots version: 1.3
% 
\begin{tikzpicture}

\begin{axis}[%
width=\figurewidth,
height=\figureheight,
view={-37.5}{30},
scale only axis,
xmin=0,
xmax=1,
xlabel={x},
xmajorgrids,
ymin=0,
ymax=1,
ylabel={y},
ymajorgrids,
zmin=0,
zmax=1,
zlabel={$\text{S}_{\text{D}}\text{(x,y)}$},
zmajorgrids,
axis x line*=bottom,
axis y line*=left,
axis z line*=left
]

\addplot3[%
surf,
shader=faceted,
draw=black,
colormap/jet,
mesh/rows=11]
table[row sep=crcr,header=false] {0	0	0\\
0	0.1	0.1\\
0	0.2	0.2\\
0	0.3	0.3\\
0	0.4	0.4\\
0	0.5	0.5\\
0	0.6	0.6\\
0	0.7	0.7\\
0	0.8	0.8\\
0	0.9	0.9\\
0	1	1\\
0.1	0	0.1\\
0.1	0.1	1\\
0.1	0.2	1\\
0.1	0.3	1\\
0.1	0.4	1\\
0.1	0.5	1\\
0.1	0.6	1\\
0.1	0.7	1\\
0.1	0.8	1\\
0.1	0.9	1\\
0.1	1	1\\
0.2	0	0.2\\
0.2	0.1	1\\
0.2	0.2	1\\
0.2	0.3	1\\
0.2	0.4	1\\
0.2	0.5	1\\
0.2	0.6	1\\
0.2	0.7	1\\
0.2	0.8	1\\
0.2	0.9	1\\
0.2	1	1\\
0.3	0	0.3\\
0.3	0.1	1\\
0.3	0.2	1\\
0.3	0.3	1\\
0.3	0.4	1\\
0.3	0.5	1\\
0.3	0.6	1\\
0.3	0.7	1\\
0.3	0.8	1\\
0.3	0.9	1\\
0.3	1	1\\
0.4	0	0.4\\
0.4	0.1	1\\
0.4	0.2	1\\
0.4	0.3	1\\
0.4	0.4	1\\
0.4	0.5	1\\
0.4	0.6	1\\
0.4	0.7	1\\
0.4	0.8	1\\
0.4	0.9	1\\
0.4	1	1\\
0.5	0	0.5\\
0.5	0.1	1\\
0.5	0.2	1\\
0.5	0.3	1\\
0.5	0.4	1\\
0.5	0.5	1\\
0.5	0.6	1\\
0.5	0.7	1\\
0.5	0.8	1\\
0.5	0.9	1\\
0.5	1	1\\
0.6	0	0.6\\
0.6	0.1	1\\
0.6	0.2	1\\
0.6	0.3	1\\
0.6	0.4	1\\
0.6	0.5	1\\
0.6	0.6	1\\
0.6	0.7	1\\
0.6	0.8	1\\
0.6	0.9	1\\
0.6	1	1\\
0.7	0	0.7\\
0.7	0.1	1\\
0.7	0.2	1\\
0.7	0.3	1\\
0.7	0.4	1\\
0.7	0.5	1\\
0.7	0.6	1\\
0.7	0.7	1\\
0.7	0.8	1\\
0.7	0.9	1\\
0.7	1	1\\
0.8	0	0.8\\
0.8	0.1	1\\
0.8	0.2	1\\
0.8	0.3	1\\
0.8	0.4	1\\
0.8	0.5	1\\
0.8	0.6	1\\
0.8	0.7	1\\
0.8	0.8	1\\
0.8	0.9	1\\
0.8	1	1\\
0.9	0	0.9\\
0.9	0.1	1\\
0.9	0.2	1\\
0.9	0.3	1\\
0.9	0.4	1\\
0.9	0.5	1\\
0.9	0.6	1\\
0.9	0.7	1\\
0.9	0.8	1\\
0.9	0.9	1\\
0.9	1	1\\
1	0	1\\
1	0.1	1\\
1	0.2	1\\
1	0.3	1\\
1	0.4	1\\
1	0.5	1\\
1	0.6	1\\
1	0.7	1\\
1	0.8	1\\
1	0.9	1\\
1	1	1\\
};
\end{axis}
\end{tikzpicture}%
			\caption{T-conorma drástica}
			\label{fig:t-conorms-drastic}
	\end{subfigure}
	\label{fig:t-conorms}
	\caption{Representación gráfica de las t-conormas \emph{Máximo} (\ref{fig:t-conorms-max}), \emph{Suma probabilística} (\ref{fig:t-conorms-probabilistic-sum}), \emph{Suma acotada} (\ref{fig:t-conorms-bounded-sum})  y la \emph{t-conorma drástica} (\ref{fig:t-conorms-drastic}).}
\end{figure}
Otra clase importante de funciones son los \emph{operadores de agregación} \index{operador de agregación}\cite{calvo2002aggregation, beliakov2007aggregation}.
\begin{definition}
Una función $\emph{M} : [a,b]^{n} \rightarrow [a,b]$ es un operador de agregación si es monótona y no decreciente en cada una de sus componentes y además cumple que $M(a, a, \cdots,a) = a$ y $M(b, b, \cdots,b) = b$.
\end{definition}
\begin{definition}
Una función de agregación \emph{M} se dice que es una media si cumple que:
\begin{equation}\label{def:mean-aggregation-operator}
min(x) = min(x_{1},\cdots,x_{n})\leq \emph{M}(x_{1},\cdots,x_{n}) \leq max(x_{1},\cdots,x_{n}) = max(x)
\end{equation}
\end{definition}
A continuación se incluyen algunos de los operadores de agregación más comunes:
\begin{itemize}
	\item \bfseries Media aritmética: $M(x_{1},x_{2},\cdots,x_{n}) = \frac{1}{n}\sum\limits_{i=1}^{n}x_{i}$
	\item \bfseries Media geométrica: $M(x_{1},x_{2},\cdots,x_{n}) = (\prod\limits_{i=1}^{n}x_{i})^{\frac{1}{n}}$
	\item \bfseries Media ponderada: $M_{w_{1},w_{2},\cdots,w_{n}}(x_{1},x_{2},\cdots,x_{n}) =\sum\limits_{i=1}^{n}(w_{i}\cdot x_{i})$ \normalfont tal que $0 \leq w_{i} \leq 1$ y $\sum\limits_{i=1}^{n}w_{i} = 1$.
	\item \bfseries Mediana: \normalfont Se toma el elemento central del conjunto ordenado de argumentos.
	\item \bfseries Máximo: $M(x_{1},x_{2},\cdots,x_{n}) = max(x_{1},x_{2},\cdots,x_{n})$
	\item \bfseries Mínimo: $M(x_{1},x_{2},\cdots,x_{n}) = min(x_{1},x_{2},\cdots,x_{n})$
\end{itemize}

\section{Funciones de solapamiento}\label{sec:overlap-functions}
En esta sección se presenta el concepto de \emph{función de solapamiento}\index{función de solapamiento} \cite{bustince2009overlap,bustince2010overlapfunctions,bustince2012pairwisecomparisons,jurio2013propertiesoverlap}, que tendrá una gran importancia en el desarrollo de este trabajo, ya que a partir de estas funciones se construyen los índices de solapamiento, que serán utilizados en el método de interpolación presentado.
\subsection{Definición de función de solapamiento y teorema de construcción.}
\begin{definition}
Una función de solapamiento $G_{O} : [0,1]^{2} \rightarrow [0,1]$ cumple:
\begin{enumerate}[label=(G\arabic*),ref=(G\arabic*)]
   \item $G_{O}(x,y) = G_{O}(y,x) \;\; \forall \; x,y \in [0,1]$ \label{G1}
   \item $G_{O}(x,y) = 0$ si y sólo si $x \cdot y = 0$ \label{G2}
   \item $G_{O}(x,y) = 1$ si y sólo si $x \cdot y = 1$ \label{G3}
   \item $G_{O}$ es creciente \label{G4}
   \item $G_{O}$ es continua \label{G5}
\end{enumerate}
\end{definition}
Las funciones de solapamiento son casos particulares de los operadores de agregación sin divisores por cero o divisores por uno. Algunos ejemplos de funciones de solapamiento son:\\
\\
$G_{O}(x,y) = min(x,y)$\\
$G_{O}(x,y) = \sqrt{xy}$\\
$G_{O}(x,y) = min(x^{k}y,xy^{k}), k \in ]0,1[$\\
\\
En \cite{bustince2009overlap} se presenta un teorema que proporciona tanto una caracterización como un método de construcción para funciones de solapamiento:
\begin{theorem}\label{th:overlap-function-construction}
Una función $G_{O} : [0,1]^{2} \rightarrow [0,1]$ es una función de solapamiento si y sólo si puede ser expresada como:
\begin{equation}
G_{O}(x,y) = \frac{f(x,y)}{f(x,y) + h(x,y)}
\end{equation}
para cualesquiera $f,h: [0,1]^{2} \rightarrow [0,1]$ tales que:
\begin{enumerate}[label=(\arabic*),ref=(\arabic*)]
   \item f y h son simétricas;
   \item f es no decreciente y h es no creciente;
   \item $f(x,y) = 0$ si y sólo si $min(x.y) = 0$;
   \item $h(x,y) = 0$ si y sólo si $min(x.y) = 1$;
   \item f y h son continuas
\end{enumerate}
\end{theorem}
\begin{example}
A continuación se presentan algunos ejemplos de construcción de funciones de solapamiento utilizando el teorema  \ref{th:overlap-function-construction}:
\begin{enumerate}[label=(\arabic*),ref=(\arabic*)]
	\item Si $f(x,y)=min(x,y)$ y $h(x,y) = max(1-x,1-y)$ entonces: 
	\begin{equation}
	G_{O}(x,y) = min(x,y)
	\end{equation} es una función de solapamiento.
	\item Si $f(x,y)=\sqrt{x \times y}$ y $h(x,y) = max(1-x,1-y)$ entonces: 
	\begin{equation}G_{O}(x,y) = \frac{\sqrt{x \times y}}{\sqrt{x \times y} + max(1-x,1-y)}
	\end{equation} es una función de solapamiento.
	\item Si $f(x,y)=\sqrt{x \times y}$ y $h(x,y) = 1 - x \times y $ entonces: 
	\begin{equation}G_{O}(x,y) = \frac{\sqrt{x \times y}}{\sqrt{x \times y} + 1 - x \times y}
	\end{equation} es una función de solapamiento.
\end{enumerate}
\end{example}
\begin{figure}[t]
	\centering
	\begin{subfigure}[b]{0.45\textwidth}
		\setlength\figureheight{4.5cm}
		\setlength\figurewidth{6cm}
		% This file was created by matlab2tikz v0.4.7 (commit 730e9c094bba45cf576cc1ba69976f02cfab8e0c) running on MATLAB 8.0.
% Copyright (c) 2008--2014, Nico Schlömer <nico.schloemer@gmail.com>
% All rights reserved.
% Minimal pgfplots version: 1.3
% 
\begin{tikzpicture}

\begin{axis}[%
width=\figurewidth,
height=\figureheight,
view={-37.5}{30},
scale only axis,
xmin=0,
xmax=1,
xlabel={x},
xmajorgrids,
ymin=0,
ymax=1,
ylabel={y},
ymajorgrids,
zmin=0,
zmax=1,
zlabel={$\text{G}_{\text{O}}\text{(x,y)}$},
zmajorgrids,
axis x line*=bottom,
axis y line*=left,
axis z line*=left
]

\addplot3[%
surf,
shader=faceted,
draw=black,
colormap/jet,
mesh/rows=11]
table[row sep=crcr,header=false] {0	0	0\\
0	0.1	0\\
0	0.2	0\\
0	0.3	0\\
0	0.4	0\\
0	0.5	0\\
0	0.6	0\\
0	0.7	0\\
0	0.8	0\\
0	0.9	0\\
0	1	0\\
0.1	0	0\\
0.1	0.1	0.1\\
0.1	0.2	0.1\\
0.1	0.3	0.1\\
0.1	0.4	0.1\\
0.1	0.5	0.1\\
0.1	0.6	0.1\\
0.1	0.7	0.1\\
0.1	0.8	0.1\\
0.1	0.9	0.1\\
0.1	1	0.1\\
0.2	0	0\\
0.2	0.1	0.1\\
0.2	0.2	0.2\\
0.2	0.3	0.2\\
0.2	0.4	0.2\\
0.2	0.5	0.2\\
0.2	0.6	0.2\\
0.2	0.7	0.2\\
0.2	0.8	0.2\\
0.2	0.9	0.2\\
0.2	1	0.2\\
0.3	0	0\\
0.3	0.1	0.1\\
0.3	0.2	0.2\\
0.3	0.3	0.3\\
0.3	0.4	0.3\\
0.3	0.5	0.3\\
0.3	0.6	0.3\\
0.3	0.7	0.3\\
0.3	0.8	0.3\\
0.3	0.9	0.3\\
0.3	1	0.3\\
0.4	0	0\\
0.4	0.1	0.1\\
0.4	0.2	0.2\\
0.4	0.3	0.3\\
0.4	0.4	0.4\\
0.4	0.5	0.4\\
0.4	0.6	0.4\\
0.4	0.7	0.4\\
0.4	0.8	0.4\\
0.4	0.9	0.4\\
0.4	1	0.4\\
0.5	0	0\\
0.5	0.1	0.1\\
0.5	0.2	0.2\\
0.5	0.3	0.3\\
0.5	0.4	0.4\\
0.5	0.5	0.5\\
0.5	0.6	0.5\\
0.5	0.7	0.5\\
0.5	0.8	0.5\\
0.5	0.9	0.5\\
0.5	1	0.5\\
0.6	0	0\\
0.6	0.1	0.1\\
0.6	0.2	0.2\\
0.6	0.3	0.3\\
0.6	0.4	0.4\\
0.6	0.5	0.5\\
0.6	0.6	0.6\\
0.6	0.7	0.6\\
0.6	0.8	0.6\\
0.6	0.9	0.6\\
0.6	1	0.6\\
0.7	0	0\\
0.7	0.1	0.1\\
0.7	0.2	0.2\\
0.7	0.3	0.3\\
0.7	0.4	0.4\\
0.7	0.5	0.5\\
0.7	0.6	0.6\\
0.7	0.7	0.7\\
0.7	0.8	0.7\\
0.7	0.9	0.7\\
0.7	1	0.7\\
0.8	0	0\\
0.8	0.1	0.1\\
0.8	0.2	0.2\\
0.8	0.3	0.3\\
0.8	0.4	0.4\\
0.8	0.5	0.5\\
0.8	0.6	0.6\\
0.8	0.7	0.7\\
0.8	0.8	0.8\\
0.8	0.9	0.8\\
0.8	1	0.8\\
0.9	0	0\\
0.9	0.1	0.1\\
0.9	0.2	0.2\\
0.9	0.3	0.3\\
0.9	0.4	0.4\\
0.9	0.5	0.5\\
0.9	0.6	0.6\\
0.9	0.7	0.7\\
0.9	0.8	0.8\\
0.9	0.9	0.9\\
0.9	1	0.9\\
1	0	0\\
1	0.1	0.1\\
1	0.2	0.2\\
1	0.3	0.3\\
1	0.4	0.4\\
1	0.5	0.5\\
1	0.6	0.6\\
1	0.7	0.7\\
1	0.8	0.8\\
1	0.9	0.9\\
1	1	1\\
};
\end{axis}
\end{tikzpicture}%
		\caption{$G_{O}(x,y) = min(x,y)$}
		\label{fig:overlap-functions-min}
	\end{subfigure}
	\qquad
	\begin{subfigure}[b]{0.45\textwidth}
		\setlength\figureheight{4.5cm}
		\setlength\figurewidth{6cm}
		% This file was created by matlab2tikz v0.4.7 (commit 921463bb921f990ffc6a7f597361910de95829d7) running on MATLAB 8.0.
% Copyright (c) 2008--2014, Nico Schlömer <nico.schloemer@gmail.com>
% All rights reserved.
% Minimal pgfplots version: 1.3
% 
\begin{tikzpicture}

\begin{axis}[%
width=\figurewidth,
height=\figureheight,
view={-37.5}{30},
scale only axis,
xmin=0,
xmax=1,
xlabel={x},
xmajorgrids,
ymin=0,
ymax=1,
ylabel={y},
ymajorgrids,
zmin=0,
zmax=1,
zlabel={$\text{G}_{\text{O}}\text{(x,y)}$},
zmajorgrids,
axis x line*=bottom,
axis y line*=left,
axis z line*=left
]

\addplot3[%
surf,
shader=faceted,
draw=black,
colormap/jet,
mesh/rows=11]
table[row sep=crcr,header=false] {0	0	0\\
0	0.1	0\\
0	0.2	0\\
0	0.3	0\\
0	0.4	0\\
0	0.5	0\\
0	0.6	0\\
0	0.7	0\\
0	0.8	0\\
0	0.9	0\\
0	1	0\\
0.1	0	0\\
0.1	0.1	0.1\\
0.1	0.2	0.14142135623731\\
0.1	0.3	0.173205080756888\\
0.1	0.4	0.2\\
0.1	0.5	0.223606797749979\\
0.1	0.6	0.244948974278318\\
0.1	0.7	0.264575131106459\\
0.1	0.8	0.282842712474619\\
0.1	0.9	0.3\\
0.1	1	0.316227766016838\\
0.2	0	0\\
0.2	0.1	0.14142135623731\\
0.2	0.2	0.2\\
0.2	0.3	0.244948974278318\\
0.2	0.4	0.282842712474619\\
0.2	0.5	0.316227766016838\\
0.2	0.6	0.346410161513775\\
0.2	0.7	0.374165738677394\\
0.2	0.8	0.4\\
0.2	0.9	0.424264068711929\\
0.2	1	0.447213595499958\\
0.3	0	0\\
0.3	0.1	0.173205080756888\\
0.3	0.2	0.244948974278318\\
0.3	0.3	0.3\\
0.3	0.4	0.346410161513776\\
0.3	0.5	0.387298334620742\\
0.3	0.6	0.424264068711929\\
0.3	0.7	0.458257569495584\\
0.3	0.8	0.489897948556636\\
0.3	0.9	0.519615242270663\\
0.3	1	0.547722557505166\\
0.4	0	0\\
0.4	0.1	0.2\\
0.4	0.2	0.282842712474619\\
0.4	0.3	0.346410161513776\\
0.4	0.4	0.4\\
0.4	0.5	0.447213595499958\\
0.4	0.6	0.489897948556636\\
0.4	0.7	0.529150262212918\\
0.4	0.8	0.565685424949238\\
0.4	0.9	0.6\\
0.4	1	0.632455532033676\\
0.5	0	0\\
0.5	0.1	0.223606797749979\\
0.5	0.2	0.316227766016838\\
0.5	0.3	0.387298334620742\\
0.5	0.4	0.447213595499958\\
0.5	0.5	0.5\\
0.5	0.6	0.547722557505166\\
0.5	0.7	0.591607978309962\\
0.5	0.8	0.632455532033676\\
0.5	0.9	0.670820393249937\\
0.5	1	0.707106781186548\\
0.6	0	0\\
0.6	0.1	0.244948974278318\\
0.6	0.2	0.346410161513775\\
0.6	0.3	0.424264068711929\\
0.6	0.4	0.489897948556636\\
0.6	0.5	0.547722557505166\\
0.6	0.6	0.6\\
0.6	0.7	0.648074069840786\\
0.6	0.8	0.692820323027551\\
0.6	0.9	0.734846922834953\\
0.6	1	0.774596669241483\\
0.7	0	0\\
0.7	0.1	0.264575131106459\\
0.7	0.2	0.374165738677394\\
0.7	0.3	0.458257569495584\\
0.7	0.4	0.529150262212918\\
0.7	0.5	0.591607978309962\\
0.7	0.6	0.648074069840786\\
0.7	0.7	0.7\\
0.7	0.8	0.748331477354788\\
0.7	0.9	0.793725393319377\\
0.7	1	0.836660026534076\\
0.8	0	0\\
0.8	0.1	0.282842712474619\\
0.8	0.2	0.4\\
0.8	0.3	0.489897948556636\\
0.8	0.4	0.565685424949238\\
0.8	0.5	0.632455532033676\\
0.8	0.6	0.692820323027551\\
0.8	0.7	0.748331477354788\\
0.8	0.8	0.8\\
0.8	0.9	0.848528137423857\\
0.8	1	0.894427190999916\\
0.9	0	0\\
0.9	0.1	0.3\\
0.9	0.2	0.424264068711929\\
0.9	0.3	0.519615242270663\\
0.9	0.4	0.6\\
0.9	0.5	0.670820393249937\\
0.9	0.6	0.734846922834953\\
0.9	0.7	0.793725393319377\\
0.9	0.8	0.848528137423857\\
0.9	0.9	0.9\\
0.9	1	0.948683298050514\\
1	0	0\\
1	0.1	0.316227766016838\\
1	0.2	0.447213595499958\\
1	0.3	0.547722557505166\\
1	0.4	0.632455532033676\\
1	0.5	0.707106781186548\\
1	0.6	0.774596669241483\\
1	0.7	0.836660026534076\\
1	0.8	0.894427190999916\\
1	0.9	0.948683298050514\\
1	1	1\\
};
\end{axis}
\end{tikzpicture}%
		\caption{$G_{O}(x,y) = \sqrt{x \cdot y}$}
		\label{fig:overlap-functions-sqrt}
	\end{subfigure}	
	
	\vspace{1 cm}
	\begin{subfigure}[b]{0.45\textwidth}
		\setlength\figureheight{4.5cm}
		\setlength\figurewidth{6cm}
		% This file was created by matlab2tikz v0.4.7 (commit 730e9c094bba45cf576cc1ba69976f02cfab8e0c) running on MATLAB 8.0.
% Copyright (c) 2008--2014, Nico Schlömer <nico.schloemer@gmail.com>
% All rights reserved.
% Minimal pgfplots version: 1.3
% 
\begin{tikzpicture}

\begin{axis}[%
width=\figurewidth,
height=\figureheight,
view={-37.5}{30},
scale only axis,
xmin=0,
xmax=1,
xlabel={x},
xmajorgrids,
ymin=0,
ymax=1,
ylabel={y},
ymajorgrids,
zmin=0,
zmax=1,
zlabel={$\text{G}_{\text{O}}\text{(x,y)}$},
zmajorgrids,
axis x line*=bottom,
axis y line*=left,
axis z line*=left
]

\addplot3[%
surf,
shader=faceted,
draw=black,
colormap/jet,
mesh/rows=11]
table[row sep=crcr,header=false] {0	0	0\\
0	0.1	0\\
0	0.2	0\\
0	0.3	0\\
0	0.4	0\\
0	0.5	0\\
0	0.6	0\\
0	0.7	0\\
0	0.8	0\\
0	0.9	0\\
0	1	0\\
0.1	0	0\\
0.1	0.1	0.1\\
0.1	0.2	0.13579648178934\\
0.1	0.3	0.161390477796409\\
0.1	0.4	0.181818181818182\\
0.1	0.5	0.19900804996708\\
0.1	0.6	0.213938769133981\\
0.1	0.7	0.227185970264612\\
0.1	0.8	0.239121152365969\\
0.1	0.9	0.25\\
0.1	1	0.260007027345288\\
0.2	0	0\\
0.2	0.1	0.13579648178934\\
0.2	0.2	0.2\\
0.2	0.3	0.234412378314921\\
0.2	0.4	0.261203874963741\\
0.2	0.5	0.283300394099019\\
0.2	0.6	0.302169479251962\\
0.2	0.7	0.318665181883831\\
0.2	0.8	0.333333333333333\\
0.2	0.9	0.346546206455528\\
0.2	1	0.358570173636287\\
0.3	0	0\\
0.3	0.1	0.161390477796409\\
0.3	0.2	0.234412378314921\\
0.3	0.3	0.3\\
0.3	0.4	0.331046251512548\\
0.3	0.5	0.356202453630939\\
0.3	0.6	0.377370477736613\\
0.3	0.7	0.39564392373896\\
0.3	0.8	0.41171425595858\\
0.3	0.9	0.426048498133928\\
0.3	1	0.438977843440086\\
0.4	0	0\\
0.4	0.1	0.181818181818182\\
0.4	0.2	0.261203874963741\\
0.4	0.3	0.331046251512548\\
0.4	0.4	0.4\\
0.4	0.5	0.427050983124842\\
0.4	0.6	0.449489742783178\\
0.4	0.7	0.468626966596886\\
0.4	0.8	0.48528137423857\\
0.4	0.9	0.5\\
0.4	1	0.513167019494862\\
0.5	0	0\\
0.5	0.1	0.19900804996708\\
0.5	0.2	0.283300394099019\\
0.5	0.3	0.356202453630939\\
0.5	0.4	0.427050983124842\\
0.5	0.5	0.5\\
0.5	0.6	0.522774424948339\\
0.5	0.7	0.541960108450192\\
0.5	0.8	0.558481559887747\\
0.5	0.9	0.572949016875158\\
0.5	1	0.585786437626905\\
0.6	0	0\\
0.6	0.1	0.213938769133981\\
0.6	0.2	0.302169479251962\\
0.6	0.3	0.377370477736613\\
0.6	0.4	0.449489742783178\\
0.6	0.5	0.522774424948339\\
0.6	0.6	0.6\\
0.6	0.7	0.618347584860329\\
0.6	0.8	0.633974596215561\\
0.6	0.9	0.647529554910575\\
0.6	1	0.659457573416833\\
0.7	0	0\\
0.7	0.1	0.227185970264612\\
0.7	0.2	0.318665181883831\\
0.7	0.3	0.39564392373896\\
0.7	0.4	0.468626966596886\\
0.7	0.5	0.541960108450192\\
0.7	0.6	0.618347584860329\\
0.7	0.7	0.7\\
0.7	0.8	0.713830971901199\\
0.7	0.9	0.725708114822568\\
0.7	1	0.736068839409471\\
0.8	0	0\\
0.8	0.1	0.239121152365969\\
0.8	0.2	0.333333333333333\\
0.8	0.3	0.41171425595858\\
0.8	0.4	0.48528137423857\\
0.8	0.5	0.558481559887747\\
0.8	0.6	0.633974596215561\\
0.8	0.7	0.713830971901199\\
0.8	0.8	0.8\\
0.8	0.9	0.809256430169454\\
0.8	1	0.817256002368443\\
0.9	0	0\\
0.9	0.1	0.25\\
0.9	0.2	0.346546206455528\\
0.9	0.3	0.426048498133928\\
0.9	0.4	0.5\\
0.9	0.5	0.572949016875158\\
0.9	0.6	0.647529554910575\\
0.9	0.7	0.725708114822568\\
0.9	0.8	0.809256430169454\\
0.9	0.9	0.9\\
0.9	1	0.90464232606174\\
1	0	0\\
1	0.1	0.260007027345288\\
1	0.2	0.358570173636287\\
1	0.3	0.438977843440086\\
1	0.4	0.513167019494862\\
1	0.5	0.585786437626905\\
1	0.6	0.659457573416833\\
1	0.7	0.736068839409471\\
1	0.8	0.817256002368443\\
1	0.9	0.90464232606174\\
1	1	1\\
};
\end{axis}
\end{tikzpicture}%
		\caption{$G_{O}(x,y) = \frac{\sqrt{x \times y}}{\sqrt{x \times y} + max(1-x,1-y)}$}
		\label{fig:overlap-functions-sqrt-max}
	\end{subfigure}
	\qquad
	\begin{subfigure}[b]{0.45\textwidth}
		\setlength\figureheight{4.5cm}
		\setlength\figurewidth{6cm}
		% This file was created by matlab2tikz v0.4.7 (commit 730e9c094bba45cf576cc1ba69976f02cfab8e0c) running on MATLAB 8.0.
% Copyright (c) 2008--2014, Nico Schlömer <nico.schloemer@gmail.com>
% All rights reserved.
% Minimal pgfplots version: 1.3
% 
\begin{tikzpicture}

\begin{axis}[%
width=\figurewidth,
height=\figureheight,
view={-37.5}{30},
scale only axis,
xmin=0,
xmax=1,
xlabel={x},
xmajorgrids,
ymin=0,
ymax=1,
ylabel={y},
ymajorgrids,
zmin=0,
zmax=1,
zlabel={$\text{G}_{\text{O}}\text{(x,y)}$},
zmajorgrids,
axis x line*=bottom,
axis y line*=left,
axis z line*=left
]

\addplot3[%
surf,
shader=faceted,
draw=black,
colormap/jet,
mesh/rows=11]
table[row sep=crcr,header=false] {0	0	0\\
0	0.1	0\\
0	0.2	0\\
0	0.3	0\\
0	0.4	0\\
0	0.5	0\\
0	0.6	0\\
0	0.7	0\\
0	0.8	0\\
0	0.9	0\\
0	1	0\\
0.1	0	0\\
0.1	0.1	0.0917431192660551\\
0.1	0.2	0.12610902712948\\
0.1	0.3	0.151508319611572\\
0.1	0.4	0.172413793103448\\
0.1	0.5	0.190529569340153\\
0.1	0.6	0.206716896335137\\
0.1	0.7	0.221480528279038\\
0.1	0.8	0.235145218523812\\
0.1	0.9	0.247933884297521\\
0.1	1	0.260007027345288\\
0.2	0	0\\
0.2	0.1	0.12610902712948\\
0.2	0.2	0.172413793103448\\
0.2	0.3	0.206716896335137\\
0.2	0.4	0.235145218523812\\
0.2	0.5	0.260007027345288\\
0.2	0.6	0.282458652402388\\
0.2	0.7	0.30317300744256\\
0.2	0.8	0.32258064516129\\
0.2	0.9	0.340975906465844\\
0.2	1	0.358570173636287\\
0.3	0	0\\
0.3	0.1	0.151508319611572\\
0.3	0.2	0.206716896335137\\
0.3	0.3	0.247933884297521\\
0.3	0.4	0.282458652402388\\
0.3	0.5	0.313019361445643\\
0.3	0.6	0.340975906465844\\
0.3	0.7	0.367117797395584\\
0.3	0.8	0.391950358125128\\
0.3	0.9	0.415820185840944\\
0.3	1	0.438977843440086\\
0.4	0	0\\
0.4	0.1	0.172413793103448\\
0.4	0.2	0.235145218523812\\
0.4	0.3	0.282458652402388\\
0.4	0.4	0.32258064516129\\
0.4	0.5	0.358570173636287\\
0.4	0.6	0.391950358125128\\
0.4	0.7	0.423608174468545\\
0.4	0.8	0.454115793296923\\
0.4	0.9	0.483870967741936\\
0.4	1	0.513167019494862\\
0.5	0	0\\
0.5	0.1	0.190529569340153\\
0.5	0.2	0.260007027345288\\
0.5	0.3	0.313019361445643\\
0.5	0.4	0.358570173636287\\
0.5	0.5	0.4\\
0.5	0.6	0.438977843440086\\
0.5	0.7	0.476485322778966\\
0.5	0.8	0.513167019494862\\
0.5	0.9	0.549483279407015\\
0.5	1	0.585786437626905\\
0.6	0	0\\
0.6	0.1	0.206716896335137\\
0.6	0.2	0.282458652402388\\
0.6	0.3	0.340975906465844\\
0.6	0.4	0.391950358125128\\
0.6	0.5	0.438977843440086\\
0.6	0.6	0.483870967741935\\
0.6	0.7	0.527715783401245\\
0.6	0.8	0.571247290198824\\
0.6	0.9	0.615013445480881\\
0.6	1	0.659457573416833\\
0.7	0	0\\
0.7	0.1	0.221480528279038\\
0.7	0.2	0.30317300744256\\
0.7	0.3	0.367117797395584\\
0.7	0.4	0.423608174468545\\
0.7	0.5	0.476485322778966\\
0.7	0.6	0.527715783401245\\
0.7	0.7	0.578512396694215\\
0.7	0.8	0.629732942041193\\
0.7	0.9	0.682055575890956\\
0.7	1	0.736068839409471\\
0.8	0	0\\
0.8	0.1	0.235145218523812\\
0.8	0.2	0.32258064516129\\
0.8	0.3	0.391950358125128\\
0.8	0.4	0.454115793296923\\
0.8	0.5	0.513167019494862\\
0.8	0.6	0.571247290198824\\
0.8	0.7	0.629732942041193\\
0.8	0.8	0.689655172413793\\
0.8	0.9	0.751889216834975\\
0.8	1	0.817256002368443\\
0.9	0	0\\
0.9	0.1	0.247933884297521\\
0.9	0.2	0.340975906465844\\
0.9	0.3	0.415820185840944\\
0.9	0.4	0.483870967741936\\
0.9	0.5	0.549483279407015\\
0.9	0.6	0.615013445480881\\
0.9	0.7	0.682055575890956\\
0.9	0.8	0.751889216834975\\
0.9	0.9	0.825688073394496\\
0.9	1	0.90464232606174\\
1	0	0\\
1	0.1	0.260007027345288\\
1	0.2	0.358570173636287\\
1	0.3	0.438977843440086\\
1	0.4	0.513167019494862\\
1	0.5	0.585786437626905\\
1	0.6	0.659457573416833\\
1	0.7	0.736068839409471\\
1	0.8	0.817256002368443\\
1	0.9	0.90464232606174\\
1	1	1\\
};
\end{axis}
\end{tikzpicture}%
		\caption{$G_{O}(x,y) = \frac{\sqrt{x \times y}}{\sqrt{x \times y} + 1 - x \times y}$}
		\label{fig:overlap-functions-sqrt-2}
	\end{subfigure}
	\caption{Representación gráfica de algunas funciones de solapamiento.}
	\label{fig:overlap-functions-examples}
\end{figure}
En la figura \ref{fig:overlap-functions-examples} se pueden ver las representaciones gráficas de las funciones de solapamiento del ejemplo anterior.

\subsection{Caso particular: t-normas.}
Por la definición \ref{def:tnorma} sabemos que una t-norma es una función conmutativa, asociativa, y creciente $T:[0,1]^2 \rightarrow [0,1]$ tal que $T(x,1) = x$ para todo $x \in [0,1]$. Bajo ciertas condiciones, las t-normas también cumplen la definición de función de solapamiento \cite{bustince2009overlap}. 
\begin{theorem}
Si una t-norma \emph{T} es una función de solapamiento, entonces \emph{T} es uno de los siguientes 3 tipos:
\begin{enumerate}[label=(\arabic*),ref=(\arabic*)]
	\item $T = \min$
	\item T es estricta (continua y estrictamente monótona)
	\item T es la suma ordinal de la familia $\{([a_{m},b_{m}],T_{m})\}$, siendo $T_{m}$ todas las t-normas continuas y arquimedeanas tales que si para algún $m_{0}$ tenemos que $a_{m_{0}} = 0$ entonces $T_{m_{0}}$ es necesariamente una t-norma estricta.
\end{enumerate}
\end{theorem}
\begin{proposition}
Sea $G_{O}$ una función de solapamiento tal que $G_{O}(x,G_{O}(y,z)) = G_{O}(y,G_{O}(x,z))$, entonces \emph{T} es una t-norma \cite{bustince2013overlap}.
\end{proposition}

\section{Índices de solapamiento}\label{sec:overlap-indexes}
\subsection{Definición y propiedades}
En esta sección se presenta la definición de índice de solapamiento y se estudian algunas de sus propiedades más importantes. Los índices de solapamiento tienen una importancia vital en el desarrollo de este trabajo, ya que son la base del método estudiado. 

El concepto de solapamiento aplicado a conjuntos difusos fue introducido por Zadeh en 1978 \cite{zadeh1978}.

\begin{definition}
Sea $A,B \in FS(U)$. La consistencia entre A y B se define como:
\begin{equation}\label{eq:zadeh-consistency}
O_{Z}(A,B) = sup_{i=1}^{n}(min(A(u_{i}),B(u_{i})))
\end{equation}
\end{definition}

Para conjuntos de referencia finitos, el supremo es en realidad el máximo elemento del conjunto.

En 1982 Dubois y Prade \cite{dubois2000} presentan la siguiente definición para el índice de solapamiento:

\begin{definition}
Un índice de solapamiento es una función $O : FS(U) \times FS(U) \rightarrow [0,1]$ tal que:
\begin{enumerate}[label=(O\arabic*),ref=(O\arabic*)]
\item $O(A,B) = 0$ si y sólo si A y B son completamente disjuntos. \label{DP01}
\item $O(A,B) = 1$, si ($A(u_{i}) = 0$ o $B(u_{i}) = 0$) o ($A(u_{i}) = 1$ o $B(u_{i}) = 1$) \label{DPO2}
\item $O(A,B) = O(B,A)$ \label{DPO3}
\item Si $B \leq C$, entonces $O(A,B) \leq O(A,C)$ \label{DPO4}
\end{enumerate}
\label{def:dubois-overlap-index}
\end{definition}
La condición \ref{DPO2} en esta definición presenta la ventaja de que, si \emph{A} no es difuso, entonces $O(A,A) = 1$.

Por esta razón Dubois, Ostasiewicz y Prade imponen en \cite{dubois2000} las siguientes condiciones:
\begin{enumerate}[label=(\arabic*),ref=(\arabic*)]
\item Para todos los conjuntos difusos tales que $A(u_{i}) \le 1$ y $A(u_{i}) \le 1$ para cualquier $u_{i} \in U$, \ref{DPO2} debe ignorarse.
\item El índice ROC (\cite{dubois2000}) no satisface \ref{DPO2}.
\end{enumerate}
Debido a estas consideraciones, normalmente sólo se imponen las condiciones \ref{DP01}, \ref{DPO3}, \ref{DPO4} de la definición \ref{def:dubois-overlap-index} a los índices de solapamiento.
En \cite{bustince2013overlap} se propone la siguiente definición de índice de solapamiento:
\begin{definition}\label{def:bustince-overlap-index}
Un índice de solapamiento es una función $O : FS(U) \times FS(U) \rightarrow [0,1]$ tal que:
\begin{enumerate}[label=(O\arabic*),ref=(O\arabic*)]
\item $O(A,B) = 0$ si y sólo si A y B tienen soportes disjuntos, es decir, $A(u_{i})B(u_{i}) = 0$ para todo $u_{i} \in U$\label{BO1}
\item $O(A,B) = O(B,A)$\label{BO2}
\item Si $B \leq C$, entonces $O(A,B) \leq O(A,C)$\label{BO3}
\end{enumerate}
Un índice de solapamiento normal es un índice \emph{O} tal que:
\begin{enumerate}[label=(O4),ref=(O4)]
\item Si existe un $u_{i} \in U$ tal que $A(u_{i}) = B(u_{i}) = 1$, entonces $O(A,B) = 1$\label{BO4}
\end{enumerate}
\end{definition}
\subsection{Construcción de índices de solapamiento}\label{sec:overlap-index-construction}
En esta sección se presenta el método de construcción de índices de solapamiento propuesto por \cite{bustince2013overlap} que se basa en operadores de agregación (sección \ref{sec:t-norms-aggregation-operator}) y funciones de solapamiento (sección \ref{sec:overlap-functions}).
\begin{theorem}
Sea $M : [0,1]^{2} \rightarrow [0,1]$ una función de agregación tal que $M(x_{1},\cdots,x_{n}) = 0$ si y sólo si $x_{1} = \cdots = x_{n} = 0$. Sea $G_{O} : [0,1]^{2} \rightarrow [0,1]$ una función de solapamiento. Entonces, la función $O: F(U) \times F(U) \rightarrow [0,1]$ definida como:
\begin{equation}\label{eq:construction-overlap-index}
O(A,B) = M(G_{O}(A(u_{1}),B(u_{1})),\cdots,G_{O}(A(u_{n}),B(u_{n})))
\end{equation}
es un índice de solapamiento en el sentido de la definición \ref{def:bustince-overlap-index}. Recíprocamente, si $G_{O}$ es una función de solapamiento y $M:[0,1]^{n} \rightarrow [0,1]$ es un operador de agregación tal que O definido por la ecuación \ref{eq:construction-overlap-index} es un índice de solapamiento, entonces $M(x_{1},\cdots,x_{n}) = 0$ si y sólo si $x_{1} = \cdots = x_{n} = 0$.
\end{theorem}
