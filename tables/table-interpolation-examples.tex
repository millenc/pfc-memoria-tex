\begin{longtable}{| c | c | c | c | c | c | c | c | c  | c  | c |}
    \hline
    \multirow{2}{*}{\textbf{Fig.}} & \multirow{2}{*}{\textbf{Temp.}} & \multirow{2}{*}{\textbf{Humo}} & \multirow{2}{*}{\textbf{Luz}}& \multirow{2}{*}{\textbf{Hum.}} & \multirow{2}{*}{\textbf{Dist.}} &  \multicolumn{5}{|c|}{\textbf{Riesgo (\%)}} \\ 
    \cline{7-11}
    & & & & & & \textbf{cent.  (\textasteriskcentered)} & \textbf{bis. (+)} & \textbf{som ($\triangledown$)} & \textbf{mom ($\square$)} & \textbf{lom ($\vartriangle$)}  \\ 
    \hline
    \ref{fig:fire-detection-interpolation-low} & 25 & 0 & 200 & 20 & 70 & 18 & 15 & 8 & 10 & 12  \\ 
    \hline
    \ref{fig:fire-detection-interpolation-low-medium} & 30 & 20 & 500 & 50 & 40 & 30 & 30 & 32 & 36 & 40 \\
    \hline
    \ref{fig:fire-detection-interpolation-medium} & 40 & 50 & 500 & 30 & 40 & 41 & 41 & 38 & 42 & 45 \\
    \hline
    \ref{fig:fire-detection-interpolation-medium-high} & 80 & 80 & 700 & 20 & 30 & 69 & 69 & 63 & 66 & 68 \\
    \hline
    \ref{fig:fire-detection-interpolation-high} & 100 & 90 & 900 & 10 & 20 & 85 & 87 & 84 & 90 & 95 \\
    \hline
    \ref{fig:fire-detection-interpolation-very-high} & 120 & 100 & 1000 & 10 & 10 & 91 & 92 & 92 & 96 & 100 \\
    \hline
    \caption{Resultados de aplicar el método de interpolación con $M = \text{Media aritmética}$, $T = T_{min}$ y $O = O_Z$  a la detección de incendios forestales. Las gráficas correspondientes se muestran en la figura \ref{fig:interpolation-fire-detection-example}.}
    \end{longtable}