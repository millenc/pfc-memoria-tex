\chapter{Implementación en MATLAB}
En este capítulo se incluye el código fuente de la implementación en MATLAB de los métodos estudiados en este proyecto.

\section{General}
En esta sección se incluyen algunas funciones de propósito general que han sido implementadas en el desarrollo de este proyecto.

\subsection{T-normas y operadores de agregación}
MATLAB dispone de una colección importante de funciones predefinidas para calcular t-normas y operadores de agregación. En este proyecto se han utilizado las siguientes:

\begin{itemize}
\item Media aritmética: \lstinline|mean()|.
\item Máximo: \lstinline|max()|.
\item Producto: \lstinline|prod()|.
\item Mínimo: \lstinline|min()|.
\item Media geométrica: \lstinline|geomean()|.
\item Media armónica: \lstinline|harmmean()|.
\end{itemize}

Además se han implementado sendas funciones para calcular las t-normas del seno y de Einstein:

\lstinputlisting[label=lst:sinmean, caption=T-norma del seno (\lstinline|functions/sinmean.m|),inputencoding=cp1252]{./matlab/functions/sinmean.m}

\lstinputlisting[label=lst:einsteinmean, caption=T-norma de Einstein (\lstinline|functions/einsteinmean.m|),inputencoding=cp1252]{./matlab/functions/einsteinmean.m}

Como se puede ver, en la función \lstinline|einsteinmean()| se ha utilizado una función anónima (\lstinline|@(x)(1-x)|), que es una funcionalidad muy útil de MATLAB, inspirada en los lenguajes funcionales, que permite declarar funciones \emph{lambda} o \emph{anónimas} (sin nombre) y utilizarlas como argumentos de otras funciones (en este caso de \lstinline|arrayfun()|, que aplica dicha función anónima a todos los elementos de un vector). Esta característica del lenguaje de MATLAB se utilizará también a la hora de definir índices de solapamiento.

\subsection{Construcción de índices de solapamiento}
Los índices de solapamiento (\ref{sec:overlap-indexes}) juegan un papel crucial en el desarrollo de este proyecto, al ser la base del método de interpolación estudiado (\ref{sec:interpolation-method}). En la sección \ref{sec:overlap-index-construction} se ha presentado un método para construir índices de solapamiento a partir de operadores de agregación y funciones de solapamiento. 

Este mismo concepto de ``construir'' índices de solapamiento a partir de otras funciones se puede implementar en MATLAB. Para ello se va a utilizar una técnica de programación llamada \emph{clausura}\footnote{Conocida generalmente por su término inglés ``closure''.}, que junto con las funciones anónimas, es muy utilizada en los lenguajes funcionales. A grandes rasgos, una clausura es una función de primer orden (puede tratarse como cualquier otra variable de tipo básico) que mantiene el estado de las variables del ámbito en el que fue creada. Así pues, el objetivo es definir una función que tome como parámetros una función de agregación y una de solapamiento y devuelva una nueva función que sea el índice de solapamiento construido a partir de estas funciones recibidas como parámetros.

Para implementar la construcción de índices de solapamiento se ha definido la función \lstinline|make_overlap_index()| (\lstinline|functions/make_overlap_index.m|), que toma como parámetros una función de agregación \lstinline|M| y una de solapamiento \lstinline|Go| y devuelve una nueva función que es el índice de solapamiento construido a partir de \lstinline|M| y 
\lstinline|Go| (según el teorema \ref{sec:overlap-index-construction}):

\lstinputlisting[label=lst:make_overlap_index,caption=Construcción de índices de solapamiento (\lstinline|functions/make_overlap_index.m|).,inputencoding=cp1252]{./matlab/functions/make_overlap_index.m}

El resultado de llamar a esta función es una nueva función que calcula el índice de solapamiento entre dos conjuntos pasados como parámetros \lstinline|A| y \lstinline|B|. Por ejemplo, si queremos construir el índice de solapamiento de Zadeh $O_Z$ (ecuación \ref{eq:zadeh-consistency}) debemos pasar a la función \lstinline|make_overlap_index()| un puntero a la función máximo (\lstinline|@max|) como función de agregación y un puntero a la función mínimo (\lstinline|@min|) como función de solapamiento:

\begin{lstlisting}
Oz = make_overlap_index(@max,@min);
\end{lstlisting}

La función \lstinline|Oz()| ahora puede utilizarse para calcular el índice de solapamiento entre dos conjuntos \lstinline|A| y \lstinline|B| pasados como parámetros:

\begin{lstlisting}
overlap_index = Oz(A, B);
\end{lstlisting}
 
La construcción de los índices de solapamiento utilizados en la comparativa entre el método de interpolación y Mamdani (sec. \ref{sec:fire-detection-interpolation-method}) se ha implementado en la clase \lstinline|O| (\lstinline|functions/O.m|) utilizando métodos estáticos:

\lstinputlisting[label=lst:overlap_indexes,caption=Índices de solapamiento (\lstinline|functions/O.m|).,inputencoding=cp1252]{./matlab/functions/O.m}

De esta forma se puede obtener el índice de solapamiento ya construido llamando a su función correspondiente. Por ejemplo, para obtener la consistencia de Zadeh:

\begin{lstlisting}
Oz = O.maxmin();
\end{lstlisting}

\section{Sistemas difusos basados en reglas}
En esta sección se incluyen las funciones implementadas relacionadas con los sistemas difusos basados en reglas estudiados a lo largo del proyecto.

\subsection{Conjuntos de reglas}\label{sec:matlab-rule-base}
Los conjuntos de reglas juegan un papel fundamental en los sistemas difusos basados en ellas, dado que expresan el conocimiento sobre el dominio del problema y tienen la misión de definir cómo se transforman las entradas del sistema difuso en salidas. Los métodos de Mamdani e interpolación implementados en este proyecto reciben el conjunto de reglas como un parámetro, de forma que pueden ser utilizados con cualquier conjunto de reglas y aplicados a cualquier tipo de problema.

Existen multitud de formas de implementar el conjunto de reglas. En este caso se ha optado por utilizar un vector de estructuras, debido a que el código resultante es sencillo de entender y de modificar. Una estructura en MATLAB es un tipo de variable compuesta que puede tener un número arbitrario de campos con nombre (similar a las estructuras de lenguajes como C). En el caso del conjunto de reglas, cada regla es una estructura \lstinline|R| con los siguientes campos:

\begin{itemize}
\item \lstinline|R.n|: Nº de regla.
\item \lstinline|R.A|: Vector con las funciones de pertenencia de los valores (punteros a funciones) de cada uno de los antecedentes de la regla.
\item \lstinline|R.B|: Función de pertenencia del valor del consecuente.
\end{itemize}
 
Por ejemplo, para definir una regla para el caso de la detección de incendios forestales tal que:
\begin{multline}
\text{IF \emph{Temperatura} es ``Baja'' AND }  \text{ \emph{Humo} es ``Alto'' AND } \text{ \emph{Luz} es ``Baja''} \\
\text{AND \emph{Humedad} es ``Alta'' AND }  \text{ \emph{Distancia} es ``Media'' THEN }  \text{ \emph{Riesgo} es ``Bajo'' }
\end{multline}
hay que construir una estructura de la siguiente forma:
 
\begin{lstlisting}
R.n = 1;
R.A = {@temp.low, @smoke.high, @llight.low, @humidity.high, @distance.medium};
R.B = @threat.low;
\end{lstlisting}
 
Para definir un conjunto con varias reglas basta con añadir índices:
 
\begin{lstlisting}
R(1).n = 1;
R(1).A = {@temp.low, @smoke.high, @llight.low, @humidity.high, @distance.medium};
R(1).B = @threat.low;

R(2).n = 2;
R(2).A = {@temp.low, @smoke.high, @llight.low, @humidity.high, @distance.high};
R(2).B = @threat.low;
\end{lstlisting}

La variable \lstinline|R| resultante es un vector de estructuras y se puede obtener la regla \emph{i-ésima} haciendo uso del índice (como cualquier otro vector): \lstinline|R(i)|.

\subsection{Difusificadores}
Otro componente importante de los sistemas difusos basados en reglas son los difusificadores. Un difusificador es una función que transforma una entrada escalar del sistema en un conjunto difuso. Existen diferente tipos de difusificadores, tal y como se ha estudiado en la sección \ref{sec:difusificadores}. En este proyecto, sin embargo, sólo se ha utilizado el difusificador \emph{singleton} cuya implementación en MATLAB es la siguiente:

\lstinputlisting[caption=Difusificador singleton (\lstinline|functions/singleton_fuzzifier.m|),inputencoding=cp1252]{./matlab/functions/singleton_fuzzifier.m}

\subsection{Desdifusificadores}
El último componente de un sistema difuso basado en reglas es el \emph{desdifusificador}. Un desdifusificador es una función que transforma el conjunto difuso de salida en un valor escalar. En la sección \ref{sec:desdifusificadores} se han establecido las propiedades que debe tener un desdifusificador y se han definido algunos de los más utilizados.

MATLAB, en su toolbox \emph{Fuzzy Logic,} ya tiene implementados los desdifusificadores más utilizados por medio de la función \lstinline|defuzz(x, mf, type)|. Esta función toma como parámetros el universo de referencia del conjunto \lstinline|x|, los grados de pertenencia para los valores del universo de referencia \lstinline|mf| y el tipo de desdifusificador a utilizar \lstinline|type|. Los tipos de desdifusificadores posibles son:

\begin{itemize}
\item \lstinline|centroid|: Centroide.
\item \lstinline|bisector|: Bisector.
\item \lstinline|mom|: Media de máximos.
\item \lstinline|som|: Menor de máximos.
\item \lstinline|lom|: Mayor de máximos.
\end{itemize}


\subsection{Método de Mamdani}
El primero de los métodos estudiados en este proyecto es el método de Mamdani (algoritmo \ref{algo:mamdani}). La implementación de este método es una función que toma como parámetros:
\begin{itemize}
\item \lstinline|R|: Conjunto de reglas tal y como se ha definido en la sección \ref{sec:matlab-rule-base}.
\item \lstinline|fact|: Premisa o valores escalares de entrada al sistema. Un vector con tantas posiciones como antecedentes tienen las reglas (variables lingüísticas), de forma que el valor para el antecedente \emph{i-ésimo} debe estar en \lstinline|fact(i)|.
\item \lstinline|y|: Universo de salida (vector).
\end{itemize}

\lstinputlisting[caption=Método de Mamdani (\lstinline|functions/mamdani.m|),inputencoding=cp1252]{./matlab/functions/mamdani.m}

\subsection{Método de interpolación basado en índices de solapamiento}
El método más importante estudiado en este proyecto es el método de interpolación basado en índices de solapamiento (algoritmo \ref{algo:multi-overlap-interpolation-method}). Este método se ha implementado en la función \lstinline|interpolation(R, fact, y, O, T, M)| (\lstinline|functions/interpolation.m|) que toma como parámetros:

\begin{itemize}
\item \lstinline|R|: Conjunto de reglas tal y como se ha definido en la sección \ref{sec:matlab-rule-base}.
\item \lstinline|fact|: Premisa. Vector de conjuntos difusos con tantas posiciones como antecedentes tienen las reglas (variables lingüísticas), de forma que el valor para el antecedente \emph{i-ésimo} debe estar en \lstinline|fact(i)|.
\item \lstinline|y|: Universo de salida (vector).
\item \lstinline|O|: Índice de solapamiento. Función que puede construirse utilizando el método implementado en \ref{lst:make_overlap_index}.
\item \lstinline|T|: T-norma. Utilizada para calcular el grado de activación de cada regla agregando los índices de solapamiento de cada antecedente.
\item \lstinline|M|: Operador de agregación.
\end{itemize}

El método toma estos parámetros y devuelve el conjunto difuso de salida, resultado de aplicar el conjunto de reglas ( \lstinline|R|) a la entrada (\lstinline|fact|).

\lstinputlisting[label=lst:interpolation,caption=Método de interpolación basado en índices de solapamiento (\lstinline|functions/interpolation.m|),inputencoding=cp1252]{./matlab/functions/interpolation.m}

Como se puede, ver en la línea 23 del código \ref{lst:interpolation} se utiliza una función llamada \lstinline|matching_degree()| (\lstinline|functions/matching_degree.m|) para calcular el grado de activación de cada regla. Esta función toma como parámetros:

\begin{itemize}
\item \lstinline|R|: Regla.
\item \lstinline|fact|: Premisa.
\item \lstinline|O|: Índice de solapamiento.
\item \lstinline|T|: T-norma.
\end{itemize}

Esta función calcula el índice de solapamiento (\lstinline|O|) de cada antecedente de la regla \lstinline|R| con su correspondiente valor en la premisa \lstinline|fact|  y agrega los valores obtenidos utilizando la t-norma \lstinline|T|:

\lstinputlisting[label=lst:matching_degree,caption=Cálculo del grado de activación basado en índices de solapamiento (\lstinline|functions/matching_degree.m|),inputencoding=cp1252]{./matlab/functions/matching_degree.m}

Estos métodos, a pesar de ser correctos y devolver los resultados esperados son mejorables en lo que a rendimiento se refiere. El principal problema de esta implementación es que se recalculan los conjuntos difusos de los antecedentes y el consecuente en cada regla (línea 27 en \ref{lst:interpolation} y línea 16 en \ref{lst:matching_degree}). Este cálculo sólo es necesario hacerlo la primera vez para cada valor de cada variable lingüística y después reutilizarlo.

Por ello se ha implementado una nueva versión de estos métodos que utiliza una caché de conjuntos difusos ya calculados, que se almacenan en una especie de \emph{tabla hash}. La tabla hash se ha implementado utilizando una clausura, de forma que se devuelve una función que mantiene la estructura de la tabla hash en el ámbito y la utiliza en llamadas posteriores. La función \lstinline|fuzzyset_hash()| se ha implementado en \lstinline|functions/fuzzyset_hash.m| de la siguiente manera:

\lstinputlisting[label=lst:fuzzyset_hash,caption=Función para construir una tabla hash de conjuntos difusos,inputencoding=cp1252]{./matlab/functions/fuzzyset_hash.m}

El resultado de llamar a esta función es un puntero a una nueva función (\lstinline|get_fuzzy_set(mf, x)|) que calcula los grados de pertenencia de un conjunto difuso dados su función de pertenencia \lstinline|mf| y su universo de referencia \lstinline|x|. En caso de que este cálculo haya sido realizado previamente, devuelve el resultado anterior, para evitar trabajo innecesario. 

Esta función hace uso de la estructura \lstinline|hash| (línea 12), que es una variable que ha quedado en su ámbito al crearse la función. Esta variable permite guardar los cálculos realizados en llamadas anteriores a la función y mantener el estado de la tabla hash mientras la función exista. La tabla hash de conjuntos se indexa utilizando el nombre de la función de pertenencia (obtenido mediante la función \lstinline|func2str()|).

Una vez implementada esta función, se puede utilizar en los métodos \lstinline|interpolation()| y \lstinline|matching_degree()| para ganar rendimiento:

\lstinputlisting[label=lst:hashed_interpolation,caption=Método de interpolación basado en índices de solapamiento (\lstinline|functions/hashed_interpolation.m|),inputencoding=cp1252]{./matlab/functions/hashed_interpolation.m}

\lstinputlisting[label=lst:hashed_matching_degree,caption=Cálculo del grado de activación basado en índices de solapamiento (\lstinline|functions/hashed_matching_degree.m|),inputencoding=cp1252]{./matlab/functions/hashed_matching_degree.m}

A pesar de que esta función es muy simple, la ganancia de rendimiento al utilizar esta técnica es notable.
 
\section{Detección de incendios forestales}
\subsection{Variables lingüísticas}
Como se ha definido en \ref{def:formal-lang-variable}, una variable lingüística es una variable que puede tomar conjuntos difusos como valores. Así mismo, un conjunto difuso viene definido por su función de pertenencia. Por tanto, se debe definir una función para cada valor de cada variable lingüística utilizada. 

Generalmente, en MATLAB, cada función se define en un fichero separado que únicamente contiene el código de dicha función. En este caso y dado que se utilizan 5 variables lingüísticas para los antecedentes con 3 valores posibles cada una y una variable de salida para el consecuente con 5 valores posibles, habría que crear 20 ficheros para definir estas funciones. Aunque es perfectamente posible y válido hacerlo de esta manera, resulta tedioso manejar tantos ficheros. 

Una solución adecuada al problema es definir cada variable lingüística en un fichero en el que se implementan las funciones de pertenencia de sus posibles valores. Para ello se puede crear una clase para cada variable lingüística y definir la función de pertenencia de cada valor como un método estático de la misma. Esta forma de definir las variables lingüísticas tiene la ventaja de que todo el código asociado a la variable está en un mismo fichero. Además la forma de utilizar las funciones resulta muy cómoda y expresiva.

Por ejemplo, para la variable lingüística $\chi_1$ = \emph{Temperatura} se ha definido la clase \lstinline|temp| en el fichero \lstinline|lang_variables/temp.m|. Esta clase tiene los siguientes métodos estáticos:
\begin{itemize}
\item \lstinline|get_x()| : Devuelve el universo de referencia para la variable (en este caso el vector $[0,100]$).
\item \lstinline|low(t)| : Función de pertenencia para el valor ``Bajo''.
\item \lstinline|medium(t)| : Función de pertenencia para el valor ``Medio''.
\item \lstinline|high(t)| : Función de pertenencia para el valor ``Alto''.
\end{itemize}

De esta forma, para obtener el grado de pertenencia de un valor \lstinline|t| al conjunto \emph{``Temperatura Media''} basta con hacer:

\begin{lstlisting}
v = temp.medium(t);
\end{lstlisting}

Además es posible obtener un puntero a estas funciones utilizando el operador \lstinline|@|:

\begin{lstlisting}
fh = @temp.medium;
v = fh(t);
\end{lstlisting}

La variable \lstinline|fh| es un puntero a la función \lstinline|temp.medium()| y puede utilizarse de la misma forma que ésta (línea 2). Además la variable \lstinline|fh| se puede utilizar de forma similar a otros tipos de variables, es decir, se pueden crear vectores de punteros de funciones, pasarlos como parámetros a otras funciones etc. Esta propiedad se utilizará de forma intensiva a la hora de implementar los métodos estudiados en este proyecto y el conjunto de reglas.

A continuación se incluyen las definiciones de las variables lingüísticas utilizadas en la aplicación práctica de detección de incendios forestales:

\lstinputlisting[caption=$\chi_1$ - Temperatura (\lstinline|lang_variables/temp.m|),inputencoding=cp1252]{./matlab/lang_variables/temp.m}

\lstinputlisting[caption=$\chi_2$ - Humo (\lstinline|lang_variables/smoke.m|), inputencoding=cp1252]{./matlab/lang_variables/smoke.m}

\lstinputlisting[caption=$\chi_3$ - Luz (\lstinline|lang_variables/llight.m|),inputencoding=cp1252]{./matlab/lang_variables/llight.m}

\lstinputlisting[caption=$\chi_4$ - Humedad (\lstinline|lang_variables/humidity.m|),inputencoding=cp1252]{./matlab/lang_variables/humidity.m}

\lstinputlisting[caption=$\chi_5$ - Distancia (\lstinline|lang_variables/distance.m|),inputencoding=cp1252]{./matlab/lang_variables/distance.m}

\lstinputlisting[caption=$y$ - Riesgo (\lstinline|lang_variables/threat.m|),inputencoding=cp1252]{./matlab/lang_variables/threat.m}

La función \lstinline|lineal()| es una función muy simple que permite calcular el valor de una función de pertenencia lineal entre los puntos pasados en los dos primeros parámetros. Esta función toma los siguientes parámetros:

\begin{itemize}
\item \lstinline|first|: Primer punto del rango (eje x).
\item \lstinline|last|: Último punto del rango (eje x).
\item \lstinline|value|: Punto en el que se quiere calcular el valor de la función de pertenencia (eje x).
\item \lstinline|r|: Parámetro opcional. Si tiene valor 'r' se calcula la función de pertenencia ``inversa''.
\end{itemize}

Este método construye la función de pertenencia lineal que pasa entre los puntos (\lstinline|first|, 0) y (\lstinline|last|, 1) o (\lstinline|first|, 1) y (\lstinline|last|, 0) en caso de que se pase el último parámetro con valor \lstinline|'r'| y obtiene el grado de pertenencia del punto \lstinline|value| para esa función. En caso de que el valor a calcular esté fuera del rango [\lstinline|first|, \lstinline|last|], el grado de pertenencia devuelto es 0:

\lstinputlisting[caption=Función de pertenencia lineal (\lstinline|functions/lineal.m|),inputencoding=cp1252]{./matlab/functions/lineal.m}

\subsection{Conjunto de reglas}

El conjunto de reglas se ha definido en la función \lstinline|fire_detection_rules()| (\lstinline|fire_detection_rules.m|), de forma que sea sencillo obtener y utilizar este conjunto en otros scripts y funciones. Esta función devuelve un vector de estructuras tal y como se ha definido en la sección \ref{sec:matlab-rule-base}.

Este tipo de estructura tiene la ventaja de que es sencilla de implementar y de entender. Sin embargo, para conjuntos de reglas grandes (como es el caso) resulta tedioso tener que indicar el índice para cada regla y repetir una y otra vez el nombre del conjunto (\lstinline|R|) o de los campos. 

Por esta razón se ha optado por una forma de definir el conjunto de reglas más concisa y clara. El conjunto de reglas se define como una matriz, en la que cada fila corresponde a una regla y cada columna es un puntero a la función de pertenencia del valor de la variable lingüística que ocupa esa posición. Las primeras $n-1$ columnas de cada fila corresponden a los antecedentes de la regla, y la última al consecuente.

En la función  \lstinline|fire_detection_rules| (código \ref{matlab_fire_detection_rules}) se define el conjunto de reglas en la matriz \lstinline|rules| (líneas 2-335) de esta forma. Dado que el resto de funciones implementadas esperan un conjunto de reglas definido como un vector de estructuras, se transforma esta matriz en dicho vector en las líneas 340-344. De esta forma, el conjunto de reglas es muy sencillo de leer y modificar y se asemeja mucho a la tabla \ref{tab:fire-detection-rule-set}.

\lstset{linewidth=18cm}
\lstinputlisting[label=matlab_fire_detection_rules, caption=Conjunto de reglas (\lstinline|fire_detection_rules.m|) ,inputencoding=cp1252]{./matlab/fire_detection_rules.m}
\lstset{linewidth=\textwidth}
\subsection{Método de Mamdani aplicado a la detección de incendios forestales}
Con el código implementado hasta ahora ya es posible aplicar el método de Mamdani a la detección y determinación de riesgos de incendios forestales. Para ello se ha definido la función \lstinline|fire_detection_mamdani(temp, smoke, light, humidity, distance)| que recibe como parámetros las entradas escalares de temperatura, humo , etc. y devuelve el riesgo de incendio resultante al aplicar el método de Mamdani. Esta función además devuelve los valores desdifusificados de este conjunto de salida:

\lstinputlisting[caption=Método de Mamdani aplicado a la determinación de riesgo de incendios forestales (\lstinline|fire_detection_mamdani.m|),inputencoding=cp1252]{./matlab/fire_detection_mamdani.m}

\subsection{Método de interpolación basado en índices de solapamiento aplicado a la detección de incendios forestales}
Con las funciones implementadas también se puede aplicar el método de interpolación basado en índices de solapamiento a la determinación de riesgo de incendios forestales. Para ello se ha definido la función \lstinline|fire_detection_interpolation(O, T, M, t, s, l, h, d)|, similar a la ya definida para Mamdani. Esta función, además de las entradas escalares de temperatura, humo, etc. (\lstinline|t, s, l, h, d|) recibe el índice de solapamiento \lstinline|O|, la t-norma \lstinline|T| y la función de agregación \lstinline|M| utilizadas en el método. Al igual que el método de Mamdani, esta función devuelve el valor de riesgo como un conjunto difuso y sus correspondientes valores desdifusificados:

\lstinputlisting[caption=Método de interpolación basado en índices de solapamiento aplicado a la determinación de riesgo de incendios forestales (\lstinline|fire_detection_interpolation.m|),inputencoding=cp1252]{./matlab/fire_detection_interpolation.m}

\subsection{Comparativa de resultados obtenidos con el método de interpolación}
En la sección \ref{sec:fire-detection-interpolation-method} se ha presentado una comparativa de los resultados obtenidos al aplicar el método de interpolación basado en índices de solapamiento a unas entradas concretas utilizando diferentes combinaciones de t-normas e índices de solapamiento.

Para obtener estos resultados se ha implementado la función  \lstinline|fire_detection_interpolation_comparison(t, s, l, h, d)| que aplica el método de interpolación con cada combinación de t-normas e índices de solapamiento  a las entradas \lstinline|t, s, l, h, d|:

\lstinputlisting[caption=Comparativa de resultados obtenidos con el método de interpolación (\lstinline|fire_detection_interpolation_comparison.m|),inputencoding=cp1252]{./matlab/fire_detection_interpolation_comparison.m}

Esta función genera también gráficas con los resultados obtenidos, que se imprimen por pantalla y en ficheros .tikz que se pueden importar después en \LaTeX . Además vuelca los resultados en formato de tabla \LaTeX, tal y como se puede ver en la sección \ref{sec:fire-detection-interpolation-method}. De hecho, para generar esos resultados es tan sencillo como ejecutar la siguiente orden desde la línea de comandos:

\begin{lstlisting}
fire_detection_interpolation_comparison(30, 20, 500, 50, 40);
\end{lstlisting}
