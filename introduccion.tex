% !TeX spellcheck = es_ES
\chapter{Introducción}
\label{cha:introduccion}
\section{Conjuntos difusos}
En esta sección se presentan algunas definiciones y teoremas que forman la base de toda la teoría de lógica difusa y de los teoremas y métodos presentados en este trabajo.

El concepto de conjunto difuso fue introducido por L.A. Zadeh en 1965 \cite{Zadeh65}.
\begin{definition}
Dado un conjunto de referencia (o universo) \emph{U}, un conjunto difuso \emph{A} sobre \emph{U} es un conjunto tal que:\\
\begin{equation}
\{(u_{i},\mu_{A}(u_{i}))\arrowvert u_{i} \in U\}
\end{equation}
donde \begin{math}\emph{A}:\emph{U}\rightarrow[0,1]\end{math} es una función llamada \emph{función de pertenencia}\index{función de pertenencia} de \emph{A}.
\end{definition}

\subsection{T-normas y operadores de agregación}
Dada una función \begin{math}\emph{F}:[0,1]^2\rightarrow[0,1]\end{math} y dos conjuntos difusos \emph{A} y \emph{B} definidos sobre el mismo universo \emph{U}, \begin{math}A,B\in FS(U)\end{math}, denotamos como \begin{math}F(A,B)\end{math} el conjunto difuso sobre \emph{U} cuya función de pertenencia viene dada por:
\begin{equation}
\mu_{F(A,B)}(u_{i}) = F(A(u_{i}),B(u_{i}))
\end{equation}
Una clase importante de este tipo de funciones son las llamadas t-normas y t-conormas (triangular norms) \cite{klement2000triangular}.
\begin{definition}\label{def-tnorma}
Una t-norma es una operación binaria \emph{T} en el intervalo $[0,1]$ que es conmutativa, asociativa, monotona y tiene el valor \emph{1} como elemento neutro. Es decir, una función $\emph{T} : [0,1]^2 \rightarrow [0,1]$ tal que $\forall x,y,z \in [0,1]$:
\begin{enumerate}[label=(T\arabic*),ref=(T\arabic*)]
   \item $T(x,y) = T(y,x)$ (Conmutatividad)\label{statement1}
   \item $T(x,T(y,z)) = T(T(x,y),z)$ (Asociatividad)\label{statement2}
   \item $T(x,y) \leq T(x,z)$ cuando $y \leq z$ (Monotonía)\label{statement3}
   \item $T(x,1) = x$ (Elemento neutro)\label{statement4}
  \end{enumerate}
\end{definition}
Estas propiedades son suficientes para garantizar que las t-normas generalizan la conjunción clásica ($x\bigwedge y$) cuando se aplican a valores booleanos ($T(0,0)=0$,$T(0,1)=T(1,0)=0$ y $T(1,1)=1$).
Algunos ejemplos prominentes de t-normas son los siguientes:
\begin{itemize}
	\item \bfseries Mínimo o t-norma de Gödel: $T_{G}(x,y) = min\{x,y\}$
	\item \bfseries Producto: $T_{P}(x,y) = xy$
	\item \bfseries \L{}ukasiewicz: $T_{L}(x,y) = max\{x+y-1,0\}$
\end{itemize}

\begin{definition}
Una t-conorma es una operación binaria \emph{S} en el intervalo $[0,1]$ que es conmutativa, asociativa, monotona y tiene el valor \emph{0} como elemento neutro. Es decir, una función $\emph{S} : [0,1]^2 \rightarrow [0,1]$ tal que $\forall x,y,z \in [0,1]$:
\begin{enumerate}[label=(T\arabic*),ref=(S\arabic*)]
   \item $S(x,y) = S(y,x)$ (Conmutatividad)\label{statement1}
   \item $S(x,S(y,z)) = S(S(x,y),z)$ (Asociatividad)\label{statement2}
   \item $S(x,y) \leq S(x,z)$ cuando $y \leq z$ (Monotonía)\label{statement3}
   \item $S(x,0) = x$ (Elemento neutro)\label{statement4}
  \end{enumerate}
\end{definition}
De la misma forma que las t-normas (definición \ref{def-tnorma}) generalizan la conjunción clásica, las t-conormas generalizan la disyunción ($x\bigvee y$). A continuación se incluyen algunos ejemplos de t-conormas:
\begin{itemize}
	\item \bfseries Máximo: $S_{max}(x,y) = max\{x,y\}$
	\item \bfseries Suma probabilística: $S_{P}(x,y) = x + y - xy$
	\item \bfseries Suma acotada: $S_{B}(x,y) = min\{x+y,1\}$
\end{itemize}