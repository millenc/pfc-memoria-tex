% !TeX spellcheck = es_ES
\chapter{Introducción}
\label{cha:introduccion}
\section{Conjuntos difusos}
En esta sección se presentan algunas definiciones y teoremas que forman la base de toda la teoría de lógica difusa y de los teoremas y métodos presentados en este trabajo.

El concepto de conjunto difuso fue introducido por L.A. Zadeh en 1965 \cite{Zadeh65}.
\begin{definition}
Dado un conjunto de referencia (o universo) \emph{U}, un conjunto difuso \emph{A} sobre \emph{U} es un conjunto tal que:\\
\begin{equation}
\{(u_{i},\mu_{A}(u_{i}))\arrowvert u_{i} \in U\}
\end{equation}
donde \begin{math}\emph{A}:\emph{U}\rightarrow[0,1]\end{math} es una función llamada \emph{función de pertenencia}\index{función de pertenencia} de \emph{A}.
\end{definition}

\section{T-normas y operadores de agregación}\label{sec:t-norms-aggregation-operator}
Dada una función \begin{math}\emph{F}:[0,1]^2\rightarrow[0,1]\end{math} y dos conjuntos difusos \emph{A} y \emph{B} definidos sobre el mismo universo \emph{U}, \begin{math}A,B\in FS(U)\end{math}, denotamos como \begin{math}F(A,B)\end{math} el conjunto difuso sobre \emph{U} cuya función de pertenencia viene dada por:
\begin{equation}
\mu_{F(A,B)}(u_{i}) = F(A(u_{i}),B(u_{i}))
\end{equation}
Una clase importante de este tipo de funciones son las llamadas t-normas y t-conormas (triangular norms) \cite{klement2000triangular}.
\begin{definition}\label{def:tnorma}
Una t-norma es una operación binaria \emph{T} en el intervalo $[0,1]$ que es conmutativa, asociativa, monótona y tiene el valor \emph{1} como elemento neutro. Es decir, una función $\emph{T} : [0,1]^2 \rightarrow [0,1]$ tal que $\forall x,y,z \in [0,1]$:
\begin{enumerate}[label=(T\arabic*),ref=(T\arabic*)]
   \item $T(x,y) = T(y,x)$ (Conmutatividad)\label{T1}
   \item $T(x,T(y,z)) = T(T(x,y),z)$ (Asociatividad)\label{T2}
   \item $T(x,y) \leq T(x,z)$ cuando $y \leq z$ (Monotonía)\label{T3}
   \item $T(x,1) = x$ (Elemento neutro)\label{T4}
  \end{enumerate}
\end{definition}
Estas propiedades son suficientes para garantizar que las t-normas generalizan la conjunción clásica ($x\bigwedge y$) cuando se aplican a valores booleanos ($T(0,0)=0$,$T(0,1)=T(1,0)=0$ y $T(1,1)=1$).
Algunos ejemplos prominentes de t-normas son los siguientes:
\begin{itemize}
	\item \bfseries Mínimo o t-norma de Gödel: $T_{G}(x,y) = min\{x,y\}$
	\item \bfseries Producto: $T_{P}(x,y) = xy$
	\item \bfseries \L{}ukasiewicz: $T_{L}(x,y) = max\{x+y-1,0\}$
\end{itemize}

\begin{definition}
Una t-conorma es una operación binaria \emph{S} en el intervalo $[0,1]$ que es conmutativa, asociativa, monotona y tiene el valor \emph{0} como elemento neutro. Es decir, una función $\emph{S} : [0,1]^2 \rightarrow [0,1]$ tal que $\forall x,y,z \in [0,1]$:
\begin{enumerate}[label=(S\arabic*),ref=(S\arabic*)]
   \item $S(x,y) = S(y,x)$ (Conmutatividad)\label{S1}
   \item $S(x,S(y,z)) = S(S(x,y),z)$ (Asociatividad)\label{S2}
   \item $S(x,y) \leq S(x,z)$ cuando $y \leq z$ (Monotonía)\label{S3}
   \item $S(x,0) = x$ (Elemento neutro)\label{S4}
  \end{enumerate}
\end{definition}
De la misma forma que las t-normas (definición \ref{def:tnorma}) generalizan la conjunción clásica, las t-conormas generalizan la disyunción ($x\bigvee y$). A continuación se incluyen algunos ejemplos de t-conormas:
\begin{itemize}
	\item \bfseries Máximo: $S_{max}(x,y) = max\{x,y\}$
	\item \bfseries Suma probabilística: $S_{P}(x,y) = x + y - xy$
	\item \bfseries Suma acotada: $S_{B}(x,y) = min\{x+y,1\}$
\end{itemize}
Otra clase importante de funciones son los \emph{operadores de agregación} \index{operador de agregación}\cite{calvo2002aggregation, beliakov2007aggregation}.
\begin{definition}
Una función $\emph{M} : [a,b]^{n} \rightarrow [a,b]$ es un operador de agregación si es monótona y no decreciente en cada una de sus componentes y además cumple que $M(a, a, \cdots,a) = a$ y $M(b, b, \cdots,b) = b$.
\end{definition}
\begin{definition}
Una función de agregación \emph{M} se dice que es una media si cumple que:
\begin{equation}\label{def:mean-aggregation-operator}
min(x) = min(x_{1},\cdots,x_{n})\leq \emph{M}(x_{1},\cdots,x_{n}) \leq max(x_{1},\cdots,x_{n}) = max(x)
\end{equation}
\end{definition}
A continuación se incluyen algunos de los operadores de agregación más comunes:
\begin{itemize}
	\item \bfseries Media aritmética: $M(x_{1},x_{2},\cdots,x_{n}) = \frac{1}{n}\sum\limits_{i=1}^{n}x_{i}$
	\item \bfseries Media geométrica: $M(x_{1},x_{2},\cdots,x_{n}) = (\prod\limits_{i=1}^{n}x_{i})^{\frac{1}{n}}$
	\item \bfseries Media ponderada: $M_{w_{1},w_{2},\cdots,w_{n}}(x_{1},x_{2},\cdots,x_{n}) =\sum\limits_{i=1}^{n}(w_{i}\cdot x_{i})$ \normalfont tal que $0 \leq w_{i} \leq 1$ y $\sum\limits_{i=1}^{n}w_{i} = 1$.
	\item \bfseries Mediana: \normalfont Se toma el elemento central del conjunto ordenado de argumentos.
	\item \bfseries Máximo: $M(x_{1},x_{2},\cdots,x_{n}) = max(x_{1},x_{2},\cdots,x_{n})$
	\item \bfseries Mínimo: $M(x_{1},x_{2},\cdots,x_{n}) = min(x_{1},x_{2},\cdots,x_{n})$
\end{itemize}

\section{Funciones de solapamiento}\label{sec:overlap-functions}
En esta sección se presenta el concepto de \emph{función de solapamiento}\index{función de solapamiento} \cite{bustince2010overlapfunctions,bustince2012pairwisecomparisons,jurio2013propertiesoverlap}, que tendrá una gran importancia en el desarrollo de este trabajo, ya que a partir de estas funciones se construyen los índices de solapamiento, que serán utilizados en el método de interpolación presentado.
\begin{definition}
Una función de solapamiento $G_{O} : [0,1]^{2} \rightarrow [0,1]$ cumple:
\begin{enumerate}[label=(G\arabic*),ref=(G\arabic*)]
   \item $G_{O}(x,y) = G_{O}(y,x) \;\; \forall \; x,y \in [0,1]$ \label{G1}
   \item $G_{O}(x,y) = 0$ si y sólo si $x \cdot y = 0$ \label{G2}
   \item $G_{O}(x,y) = 1$ si y sólo si $x \cdot y = 1$ \label{G3}
   \item $G_{O}$ es creciente \label{G4}
   \item $G_{O}$ es continua \label{G5}
\end{enumerate}
\end{definition}
Las funciones de solapamiento generalizan los operadores de intersección tales como el mínimo o, en general, las t-normas (definidas en \ref{def:tnorma}). Además, las funciones de solapamiento son casos particulares de los operadores de agregación sin divisores por cero o divisores por uno. Algunos ejemplos de funciones de solapamiento son:\\
\\
$G_{O}(x,y) = min(x,y)$\\
$G_{O}(x,y) = \sqrt{xy}$\\
$G_{O}(x,y) = min(x^{k}y,xy^{k}), k \in ]0,1[$\\

\section{Índices de solapamiento}
\subsection{Definición y propiedades}
En esta sección se presenta la definición de índice de solapamiento y se estudian algunas de sus propiedades más importantes. Los índices de solapamiento tienen una importancia vital en el desarrollo de este trabajo, ya que son la base del método estudiado. 

El concepto de solapamiento aplicado a conjuntos difusos fue introducido por Zadeh en 1978 \cite{zadeh1978}.

\begin{definition}
Sea $A,B \in FS(U)$. La consistencia entre A y B se define como:
\begin{equation}\label{eq:zadeh-consistency}
O_{Z}(A,B) = sup_{i=1}^{n}(min(A(u_{i}),B(u_{i})))
\end{equation}
\end{definition}

Para conjuntos de referencia finitos, el supremo es en realidad el máximo elemento del conjunto.

En 1982 Dubois y Prade \cite{dubois2000} presentan la siguiente definición para el índice de solapamiento:

\begin{definition}
Un índice de solapamiento es una función $O : FS(U) \times FS(U) \rightarrow [0,1]$ tal que:
\begin{enumerate}[label=(O\arabic*),ref=(O\arabic*)]
\item $O(A,B) = 0$ si y sólo si A y B son completamente disjuntos. \label{DP01}
\item $O(A,B) = 1$, si ($A(u_{i}) = 0$ o $B(u_{i}) = 0$) o ($A(u_{i}) = 1$ o $B(u_{i}) = 1$) \label{DPO2}
\item $O(A,B) = O(B,A)$ \label{DPO3}
\item Si $B \leq C$, entonces $O(A,B) \leq O(A,C)$ \label{DPO4}
\end{enumerate}
\end{definition} \label{def:dubois-overlap-index}
La condición \ref{DPO2} en esta definición presenta la ventaja de que, si \emph{A} no es difuso, entonces $O(A,A) = 1$.

Por esta razón Dubois, Ostasiewicz y Prade imponen en \cite{dubois2000} las siguientes condiciones:
\begin{enumerate}[label=(\arabic*),ref=(\arabic*)]
\item Para todos los conjuntos difusos tales que $A(u_{i}) \le 1$ y $A(u_{i}) \le 1$ para cualquier $u_{i} \in U$, \ref{DPO2} debe ignorarse.
\item El índice ROC (\cite{dubois2000}) no satisface \ref{DPO2}.
\end{enumerate}
Debido a estas consideraciones, normalmente sólo se imponen las condiciones \ref{DP01}, \ref{DPO3}, \ref{DPO4} de la definición \ref{def:dubois-overlap-index} a los índices de solapamiento.
En \cite{bustince2013overlap} se propone la siguiente definición de índice de solapamiento:
\begin{definition}\label{def:bustince-overlap-index}
Un índice de solapamiento es una función $O : FS(U) \times FS(U) \rightarrow [0,1]$ tal que:
\begin{enumerate}[label=(O\arabic*),ref=(O\arabic*)]
\item $O(A,B) = 0$ si y sólo si A y B tienen soportes disjuntos, es decir, $A(u_{i})B(u_{i}) = 0$ para todo $u_{i} \in U$\label{BO1}
\item $O(A,B) = O(B,A)$\label{BO2}
\item Si $B \leq C$, entonces $O(A,B) \leq O(A,C)$\label{BO3}
\end{enumerate}
Un índice de solapamiento normal es un índice \emph{O} tal que:
\begin{enumerate}[label=(O4),ref=(O4)]
\item Si existe un $u_{i} \in U$ tal que $A(u_{i}) = B(u_{i}) = 1$, entonces $O(A,B) = 1$\label{BO4}
\end{enumerate}
\end{definition}
\subsection{Construcción de índices de solapamiento}
En esta sección se presenta el método de construcción de índices de solapamiento propuesto por \cite{bustince2013overlap} que se basa en operadores de agregación (sección \ref{sec:t-norms-aggregation-operator}) y funciones de solapamiento (sección \ref{sec:overlap-functions}).
\begin{theorem}
Sea $M : [0,1]^{2} \rightarrow [0,1]$ una función de agregación tal que $M(x_{1},\cdots,x_{n}) = 0$ si y sólo si $x_{1} = \cdots = x_{n} = 0$. Sea $G_{O} : [0,1]^{2} \rightarrow [0,1]$ una función de solapamiento. Entonces, la función $O: F(U) \times F(U) \rightarrow [0,1]$ definida como:
\begin{equation}\label{eq:construction-overlap-index}
O(A,B) = M(G_{O}(A(u_{1}),B(u_{1})),\cdots,G_{O}(A(u_{n}),B(u_{n})))
\end{equation}
es un índice de solapamiento en el sentido de la definición \ref{def:bustince-overlap-index}. Recíprocamente, si $G_{O}$ es una función de solapamiento y $M:[0,1]^{n} \rightarrow [0,1]$ es un operador de agregación tal que O definido por la ecuación \ref{eq:construction-overlap-index} es un índice de solapamiento, entonces $M(x_{1},\cdots,x_{n}) = 0$ si y sólo si $x_{1} = \cdots = x_{n} = 0$.
\end{theorem}
