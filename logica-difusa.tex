% !TeX spellcheck = es_ES
\chapter{Lógica difusa}
\label{cha:logica-difusa}
En esta sección se introducen las bases de la lógica difusa y los mecanismos de inferencia difusa. Además se describen algunos de los métodos clásicos de inferencia difusa basada en reglas.

\section{¿Qué es la lógica difusa?}
La lógica difusa fue introducida por Lofti A. Zadeh en 1965. Desde entonces se ha aplicado en multitud de escenarios tales como control de procesos industriales, medicina, electrónica y otros tipos de sistemas expertos. En general, la mayoría de aplicaciones de la lógica difusa son el área del control.

La motivación para el desarrollo de una lógica difusa fue expresada por Zadeh (1984, \cite{Zadeh1984}) de la siguiente manera:
\begin{quote}
\emph{``La habilidad de la mente humana para razonar en términos difusos es realmente una gran ventaja. Aunque una gran cantidad de información es captada por los sentidos en una situación determinada, de alguna manera la mente humana es capaz de descartar la mayoría de esa información y concentrarse sólo en aquello que es relevante.''}
\end{quote}
El objetivo principal de la lógica difusa es intentar dotar a las máquinas de un sistema de razonamiento aproximado similar al de los humanos, que pueda lidiar con imprecisiones y términos inexactos. Para ello, la lógica difusa se basa en la utilización de variables lingüísticas expresadas en lenguaje natural, que forman la base del razonamiento aproximado. 

Por medio del uso de variables lingüísticas se pueden modelar conceptos como \emph{``muy alto''} o \emph{``bastante caliente''} y establecer relaciones entre ellos, expresadas de forma matemática y algorítmica. Los sistemas de lógica difusa tratan la imprecisión de las entradas mediante el uso de variables lingüísticas (expresadas como conjuntos difusos), que transforman dichos valores de entrada en números difusos.

\section{Razonamiento aproximado}
El razonamiento aproximado utiliza conjuntos difusos y lógica difusa para modelar la forma de discurrir y razonar del ser humano \cite{bojadziev1995}. El razonamiento aproximado no posee la precisión y exactitud de la lógica clásica, pero es más efectivo a la hora de lidiar con imprecisiones, conceptos vagos o sistemas complejos y/o pobremente definidos.

En este proyecto se va a implementar una forma de razonamiento aproximado que utiliza un mecanismo de inferencia difusa basado en el modus ponens generalizado, que es una extensión del modus ponens clásico.

\subsection{Modus ponens clásico}
El modus ponens clásico es una proposición compuesta muy conocida en la lógica clásica. Tiene la forma:

\begin{equation}
((p \wedge (p \rightarrow q)) \rightarrow q)
\end{equation}

Para inferir el valor de verdad de \emph{q} a partir del de \emph{p} utilizando el modus ponens clásico, se utiliza una \emph{regla} de inferencia, expresada de forma simbólica utilizando el \emph{silogismo}:

\begin{equation}
\begin{array}{rl}
\text{Premisa 1:} & p \\
\text{Premisa 2:} & p \to q \\
\cline{2-2}
\text{Conclusión:} & q
\end{array}
\label{eq:inference-rule}
\end{equation}

que puede expresarse como: si la proposición \emph{p} es verdadera (premisa 1) y la proposición $p \to q$ es verdadera (premisa 2), entonces \emph{q} es verdadera (conclusión). Esto permite obtener el valor de verdad de \emph{q} a partir del de \emph{p} utilizando una regla de inferencia. Es decir, por medio de esta regla de inferencia sabemos que, si \emph{p} es verdadero, entonces \emph{q} es también verdadero.

La regla de inferencia \ref{eq:inference-rule} puede expresarse de manera más detallada como:

\begin{equation}
\begin{array}{rl}
\text{Premisa 1:} & x \text{ es } A \\
\text{Premisa 2:} & \text{IF } x \text{ es } A \text{ THEN } y \text{ es } B \\
\cline{2-2}
\text{Conclusión:} & y \text{ es } B
\end{array}
\label{eq:inference-rule-detail}
\end{equation}

donde $p = x \text{ es } A$ y $q = y \text{ es } B$.

\begin{example}
Un ejemplo de la regla de inferencia \ref{eq:inference-rule-detail} es la regla de divisibilidad por 3: \emph{``Ùn número entero es divisible por 3 si la suma de sus dígitos es múltiplo de 3''}. Puede expresarse mediante el siguiente silogismo:

\begin{equation}
\normalfont
\begin{array}{rl}
\text{Premisa 1:} & \text{\emph{La suma de los dígitos de un número es múltiplo de 3}} \\
\text{Premisa 2:} & \text{IF \emph{La suma de los dígitos de un número es múltiplo de 3} THEN \emph{El número es divisible por 3}} \\
\cline{2-2}
\text{Conclusión:} & \text{\emph{El número es divisible por 3}}
\end{array}
\label{eq:inference-rule-example}
\end{equation}

En este ejemplo se utilizan conceptos exactos (lógica clásica). Un número pertenece o no pertence al conjunto de números cuya suma de cifras es múltiplo de 3. Por lo tanto la proposición \emph{``La suma de los dígitos de un número es múltiplo de 3''} sólo puede tomar los valores \emph{Verdadero} (1) o \emph{Falso} (0). 
\end{example}

\subsection{Modus ponens generalizado}

El \emph{modus ponens generalizado} es, como su nombre índica, una generalización del modus ponens clásico (definido en la sección anterior) utilizado en la lógica difusa. Tiene la forma:

\begin{equation}
\begin{array}{rl}
\text{Premisa 1:} & p' \\
\text{Premisa 2:} & p \to q \\
\cline{2-2}
\text{Conclusión:} & q'
\end{array}
\label{eq:fuzzy-inference-rule}
\end{equation}

o lo que es lo mismo:

\begin{equation}
\begin{array}{rl}
\text{Premisa 1:} & x \text{ es } A' \\
\text{Premisa 2:} & \text{IF } x \text{ es } A \text{ THEN } y \text{ es } B \\
\cline{2-2}
\text{Conclusión:} & y \text{ es } B'
\end{array}
\label{eq:fuzzy-inference-rule-detail}
\end{equation}

En este caso las proposiciones $p'=x \text{ es } A'$, $p=x \text{ es } A$, $q=y \text{ es } B$, $q'=y \text{ es } B'$ vienen caracterizadas por los conjuntos difusos $A'$,$A$,$B$ y $B'$,que representan conceptos difusos. Esta regla de inferencia puede interpretarse de la siguiente manera: si $p \to q$ y tenemos $p'$ (aproximadamente $p$) entonces tenemos $q'$ (aproximadamente q). El modus ponens generalizado permite, por tanto, utilizar conceptos difusos y obtener el ``grado de verdad'' de la conclusión a partir de las premisas.

En la siguiente sección se define el concepto de \emph{regla difusa} (p.e. $\text{IF } x \text{ es } A \text{ THEN } y \text{ es } B$), que constituye una de las premisas en el modus ponens generalizado. En las secciones \ref{sec:mamdani-controller} y \ref{sec:interpolation-method} se definen los métodos de inferencia difusa que se van a estudiar en este proyecto. Estos métodos utilizan reglas difusas y el modus ponens generalizado para obtener las salidas ($B'$) a partir de las premisas ($A'$).

\section{Reglas difusas}
Una \emph{regla difusa IF-THEN}\footnote{Podría traducirse como SI-ENTONCES.} puede definirse de manera informal como:

\begin{definition}
Una regla difusa IF-THEN es una estructura condicional de la forma:
\begin{equation}
\text{IF [antecedentes] THEN [consecuentes]}
\end{equation},
donde [antecedentes] y [consecuentes] son proposiciones difusas.
\end{definition}

Las proposiciones difusas pueden ser \emph{simples} o \emph{compuestas}. Una proposición difusa simple es un predicado único formado por una variable lingüística \emph{x} y un valor \emph{A} de dicha variable (un conjunto difuso). Es decir, \emph{A} es un conjunto difuso (valor lingüístico) definido sobre el mismo universo de referencia \emph{U} de \emph{x}. Por ejemplo si la variable \emph{x} es la \emph{``Temperatura medida en ºC''} del ejemplo \ref{ex:lang-variable}, entonces las siguientes son proposiciones difusas sobre \emph{x}:

\begin{equation}
\text{$x$ es B}
\end{equation}
\begin{equation}
\text{$x$ es M}
\end{equation}
\begin{equation}
\text{$x$ es A}
\end{equation}

donde \emph{B}, \emph{M} y \emph{A} son los valores lingüísticos ``baja'', ``media'' y ``alta'' respectivamente. Estos valores lingüísticos vienen definidos por conjuntos difusos con funciones de pertenencia $\mu_{Baja}$, $\mu_{Media}$ y $\mu_{Alta}$.

Las proposiciones difusas simples se pueden combinar para formar predicados más complejos. Para ello se utilizan normalmente los conectivos \emph{AND} (intersección difusa) y \emph{OR} (unión difusa). Por ejemplo, se puede definir la condición \emph{``Temperatura es Alta Ó Temperatura es Media''} con la siguiente proposición difusa:

\begin{equation}
\text{$x$ es $A$ OR $x$ es $M$}
\end{equation}

donde $A$ y $M$ son los valores lingüísticos \emph{``alta''} ($\mu_{Alta}$) y \emph{``media''} ($\mu_{Media}$) respectivamente. En la misma proposición difusa se pueden utilizar variables diferentes (generalmente es así) definidas normalmente sobre universos de referencia diferentes. El consecuente generalmente suele ser una proposición difusa simple formado por una variable \emph{y} sobre \emph{V}. Por ello, se puede definir el concepto de \emph{regla difusa IF-THEN} de manera más formal como:

\begin{definition}
Una \emph{regla difusa IF-THEN} es una estructura condicional de la forma:

\begin{equation}
\text{IF $x_{1}$ es $A_{1}$ $\odot$ $x_{2}$ es $A_{2}$ $\odot$ \ldots $\odot$ $x_{n}$ es $A_{n}$ THEN $y$ es $B$}
\end{equation}
donde $x_{1} \in U_{i}$,\ldots, $x_{n} \in U_{n}$,$y \in V$ son variables lingüísticas y $\odot$ algún conectivo difuso (AND,OR,AND NOT,etc.).
\end{definition}

\section{Sistemas difusos basados en reglas}\label{sec:sistemas-difusos-basados-en-reglas}
Un enfoque muy utilizado a la hora de modelar un sistema de lógica difusa es la utilización de \emph{reglas} expresadas en un lenguaje muy próximo al natural, mediante variables lingüísticas. Estas reglas habitualmente son formuladas por un experto humano aunque pueden ser derivadas también de datos numéricos. A este tipo de sistemas difusos se les denomina \emph{sistemas difusos basados en reglas}. Los métodos estudiados en este trabajo pertenecen a este tipo de sistemas.

Existen principalmente 3 tipos de sistemas difusos \cite{wang1997}: sistemas difusos puros, sistemas difusos tipo Takagi-Sugeno-Kang (TKG) y sistemas difusos con fusificador y defusificador. A continuación se realiza una breve descripción de las características de estos sistemas.

En la figura \ref{fig:pure-fuzzy-system} se representa el esquema básico de un sistema difuso puro. En los sistemas difusos puros tanto las entradas como la salida del sistema son conjuntos difusos. El sistema difuso está dotado de un \emph{conjunto de reglas difusas} \emph{IF-THEN}, que forman la base de conocimiento sobre el dominio del problema. El \emph{sistema de inferencia difusa} combina estas reglas \emph{IF-THEN} y transforma los conjuntos difusos de entrada sobre el universo $U \subset R^{n}$ ($x_{1}.\ldots,x_{n}$) en un conjunto difuso sobre $V \subset R$ ($y$), utilizando principios de lógica difusa (modus ponens).

\begin{figure}[tb]
	\centering
	\tikzstyle{block} = [draw, fill=blue!20, rectangle, minimum height=3em, minimum width=6em]
\tikzstyle{input} = [coordinate]
\tikzstyle{output} = [coordinate]

\tikzset{
    state/.style={
    	   fill=white,
           rectangle,
           rounded corners,
           draw=black, very thick,
           minimum height=4em,
           minimum width=15em,
           inner sep=2pt,
           text centered,
           drop shadow
           },
}

\tikzstyle{materia}=[draw, fill=blue!20, text width=6.0em, text centered,
  minimum height=1.5em,drop shadow]


\shorthandoff{>}
\begin{tikzpicture}[auto, node distance=2cm,>=latex']
\node [input, name=input] {};
\node [state, right of=input,node distance=7cm,align=left] (inference-engine) {Sistema de inferencia difusa};
\node [state, above of=inference-engine] (rule-base) {Conjunto de reglas difusas};
\node [output, right of=inference-engine,node distance=7cm] (output) {};

\draw [->] (input) -- node {$x \in FS(U)$} (inference-engine);
\draw [->] (rule-base) -- node {} (inference-engine);
\draw [->] (inference-engine) -- node {$y \in FS(V)$} (output);
\end{tikzpicture}
	\caption{Diagrama de un sistema difuso puro.}
	\label{fig:pure-fuzzy-system}
\end{figure}

El principal problema de los sistemas difusos puros es que tanto las entradas como las salidas son conjuntos difusos, mientras que en aplicaciones de control industrial se utilizan generalmente valores escalares reales. Para solucionar este problema, Takagi, Sugeno y Kang propusieron un nuevo método simplificado que utiliza entradas y salidas escalares \cite{takagisugeno1985}\cite{sugenokang1988}.

El método de Takagi-Sugeno-Kang utiliza reglas de la forma:

\begin{equation}
R_{i}: \text{IF }x_{1}\text{ es }A_{1i}\text{, }x_{2}\text{ es }A_{2i}\text{ , \ldots , }x_{n}\text{ es }A_{ni}\text{ THEN } y = f_{i}(x_{1},x_{2},\ldots,x_{n})
\end{equation}

donde $A_{ij}$ representa las funciones de pertenencia asociadas a las variables lingüísticas utilizadas en los antecedentes y $f_{i}(x_{1},x_{2},\ldots,x_{n})$ representa los consecuentes. Normalmente $f_{i}$ es un polinomio en las variables de entrada, pero puede ser cualquier tipo de función mientras pueda describir la salida del modelo de forma apropiada, teniendo en cuenta las entradas. 

Como se puede ver, la principal diferencia entre un sistema difuso puro y el sistema de Takagi-Sugeno-Kang es que el consecuente de la regla (\emph{THEN}) es sustituido por una simple fórmula matemática. Este cambio supone que el método para combinar las diferentes reglas para proporcionar una salida es más sencillo. De hecho, en el método de Takagi-Sugeno-Kang se utiliza la media ponderada de los consecuentes obtenidos de las diferentes reglas.

El principal problema del método de Takagi-Sugeno-Kang es que el consecuente de las reglas es una función matemática y por lo tanto se pierde cierta capacidad de modelar el sistema utilizando el lenguaje natural. Para solucionar estos problemas, se puede modificar el sistema difuso puro, añadiendo un fusificador a la entrada de las variables y un defusificador a la salida.

\begin{figure}[tb]
	\centering
	\tikzstyle{block} = [draw, fill=blue!20, rectangle, minimum height=3em, minimum width=6em]
\tikzstyle{input} = [coordinate]
\tikzstyle{output} = [coordinate]

\tikzset{
    state/.style={
    	   fill=white,
           rectangle,
           rounded corners,
           draw=black, very thick,
           minimum height=4em,
           minimum width=15em,
           inner sep=2pt,
           text centered,
           drop shadow
           },
}

\tikzset{
    state-small/.style={
    	   fill=white,
           rectangle,
           rounded corners,
           draw=black, very thick,
           minimum height=4em,
           minimum width=5em,
           inner sep=2pt,
           text centered,
           drop shadow
           },
}

\tikzstyle{materia}=[draw, fill=blue!20, text width=6.0em, text centered,
  minimum height=1.5em,drop shadow]


\shorthandoff{>}
\begin{tikzpicture}[auto, node distance=2cm,>=latex']
\node [input, name=input] {};
\node [state-small, right of=input,node distance=2.5cm,align=left] (fuzzifier) {Fusificar};
\node [state, right of=fuzzifier,node distance=6cm,align=left] (inference-engine) {Sistema de inferencia difusa};
\node [state, above of=inference-engine] (rule-base) {Conjunto de reglas difusas};
\node [state-small, right of=inference-engine,node distance=6cm,align=left] (defuzzifier) {Defusificar};
\node [output, right of=defuzzifier,node distance=2.5cm] (output) {};

\draw [->] (input) -- node {$x \in U$} (fuzzifier);
\draw [->] (fuzzifier) -- node {$x \in FS(U)$} (inference-engine);
\draw [->] (rule-base) -- node {} (inference-engine);
\draw [->] (inference-engine) -- node {$y \in FS(V)$} (defuzzifier);
\draw [->] (defuzzifier) -- node {$y \in V$} (output);
\end{tikzpicture}
	\caption{Diagrama de un sistema difuso con fusificador y defusificador.}
	\label{fig:fuzzy-system}
\end{figure}

En la figura \ref{fig:fuzzy-system} se representa el esquema básico de un sistema difuso con fusificador y defusificador. El fusificador transforma las entradas escalares en $U$ del sistema en conjuntos difusos. Una vez realizada la inferencia difusa se obtiene un conjunto difuso como salida. El defusificador transforma el conjunto difuso de salida en una variable escalar en $V$. Existen multitud de métodos para realizar estas transformaciones y en siguientes secciones se detallarán algunos de ellos. El sistema difuso presentado en este trabajo se basa en este último tipo de sistema.

\subsection{Fusificadores}\label{sec:fusificadores}

En la sección \ref{sec:sistemas-difusos-basados-en-reglas} se ha definido un tipo de sistema difuso que utiliza un concepto llamado \emph{fusificador} para transformar las entradas escalares del sistema en conjuntos difusos. Un \emph{fusificador} puede definirse formalmente como:

\begin{definition}\label{def:fusificador}
Un \emph{fusificador} es una función $F:\mathbb{R} \rightarrow FS(U)$ que toma un valor escalar real $x^{*}$ en el universo $U$, $x^{*}\in U \subset \mathbb{R}$, y lo transforma en un conjunto difuso $A'$ sobre $U$.
\end{definition}

Para que el conjunto de salida represente de manera lo más fiel posible al valor de entrada, es necesario imponer algunas restricciones a la hora de construir el fusificador. La principal es que la función de pertenencia de $A'$ debe tener un valor alto en el punto $x^*$ (generalmente máximo, $\mu_{A'}(x^*) = \max(\mu_{A'}(\mu_{i})),\forall \mu_{i} \in U$). Además, si se considera que existe ruido o imprecisión en los valores entrada (por ejemplo imprecisiones provocadas por instrumentos de medida) se puede utilizar una función de pertenencia cuyo valor sea alto en los puntos próximos a $x^*$ y bajo (o incluso cero) en los más alejados.

Algunos de los fusificadores más utilizados son los siguientes \cite{wang1997}:

\begin{itemize}
\item\bfseries Fusificador singleton: \normalfont El \emph{fusificador singleton} (fig. \ref{fig:fuzzifier-singleton}) transforma un valor escalar real $x^* \in U$ en un conjunto difuso $A'$ en $U$, de forma que la función de pertenencia de $A'$ tiene valor 1 en el punto $x^*$ y 0 en el resto de puntos de $U$:
\begin{equation}
\mu_{A'}(x) = \begin{cases} 1 & \mbox{si } \mbox{x}=x^*, \\ 0 & \mbox{en cualquier otro caso} \end{cases}
\end{equation}
Este tipo de fusificador tiene la ventaja de ser simple y facilitar los cálculos realizados en el motor de inferencia. Sin embargo, tiene la desventaja de que no puede lidiar con ruido o imprecisiones en los valores de entrada. 
\item\bfseries Fusificador gaussiano: \normalfont El \emph{fusificador gaussiano} (fig. \ref{fig:fuzzifier-gaussian}) transforma un valor escalar real $x^* \in U$ en un conjunto difuso $A'$ en $U$, utilizando una función de pertenencia Gaussiana:
\begin{equation}
\mu_{A'}(x) = e^{-(\frac{x_1 - x_1^*}{a_1})^2}\star\ldots\star e^{-(\frac{x_n - x_n^*}{a_n})^2}
\end{equation}
donde $a_i$ son parámetros positivos y $\star$ es una t-norma, generalmente el producto algebraico o el mínimo. Este tipo de fusificador es adecuado cuando existe ruido o imprecisiones en los datos de entrada.
\item\bfseries Fusificador triangular: \normalfont El \emph{fusificador triangular} (fig. \ref{fig:fuzzifier-triangular}) transforma un valor escalar real $x^* \in U$ en un conjunto difuso $A'$ en $U$, utilizando una función de pertenencia triangular:
\begin{equation}
\mu_{A'}(x) = \begin{cases} (1 - \frac{|x_1 - x_1^*|}{b_1})\star\ldots\star  (1 - \frac{|x_n - x_n^*|}{b_n}) & \mbox{si } |x_i - x_i^*| \leq b_i, i = 1,2,\ldots,n, \\ 0 & \mbox{en cualquier otro caso} \end{cases}
\end{equation}
donde $b_i$ son parámetros positivos y $\star$ es una t-norma, generalmente el producto algebraico o el mínimo. Al igual que el fusificador gaussiano este tipo de fusificador es adecuado cuando existe ruido o imprecisiones en los datos de entrada.
\end{itemize}

\begin{figure}[H]
	\centering
	\begin{subfigure}[t]{\textwidth}
		\setlength\figureheight{4cm}
		\setlength\figurewidth{12cm}
		% This file was created by matlab2tikz v0.4.7 (commit b040dd9e3d20b4d02324400633a5d15752e58612) running on MATLAB 8.0.
% Copyright (c) 2008--2014, Nico Schlömer <nico.schloemer@gmail.com>
% All rights reserved.
% Minimal pgfplots version: 1.3
% 
\begin{tikzpicture}

\begin{axis}[%
width=\figurewidth,
height=\figureheight,
scale only axis,
xmin=0,
xmax=100,
xtick={0,20,40,60,80,100},
xticklabels={\empty},
xlabel={x},
ymin=0,
ymax=1,
ylabel={$\mu{}_{\text{A'}}\text{(x)}$},
axis x line*=bottom,
axis y line*=left
]
\addplot [color=black,solid,line width=1.5pt,forget plot]
  table[row sep=crcr]{0	0\\
1	0\\
2	0\\
3	0\\
4	0\\
5	0\\
6	0\\
7	0\\
8	0\\
9	0\\
10	0\\
11	0\\
12	0\\
13	0\\
14	0\\
15	0\\
16	0\\
17	0\\
18	0\\
19	0\\
20	0\\
21	0\\
22	0\\
23	0\\
24	0\\
25	0\\
26	0\\
27	0\\
28	0\\
29	0\\
30	0\\
31	0\\
32	0\\
33	0\\
34	0\\
35	0\\
36	0\\
37	0\\
38	0\\
39	0\\
40	0\\
41	0\\
42	0\\
43	0\\
44	0\\
45	0\\
46	0\\
47	0\\
48	0\\
49	0\\
50	0\\
51	0\\
52	0\\
53	0\\
54	0\\
55	0\\
56	0\\
57	0\\
58	0\\
59	0\\
60	0\\
61	0\\
62	0\\
63	0\\
64	0\\
65	0\\
66	0\\
67	0\\
68	0\\
69	0\\
70	0\\
71	0\\
72	0\\
73	0\\
74	0\\
75	0\\
76	0\\
77	0\\
78	0\\
79	0\\
80	0\\
81	0\\
82	0\\
83	0\\
84	0\\
85	0\\
86	0\\
87	0\\
88	0\\
89	0\\
90	0\\
91	0\\
92	0\\
93	0\\
94	0\\
95	0\\
96	0\\
97	0\\
98	0\\
99	0\\
100	0\\
};
\addplot [color=blue,line width=1.5pt,mark size=4.0pt,only marks,mark=asterisk,mark options={solid},forget plot]
  table[row sep=crcr]{40	1\\
};
\addplot [color=black,dotted,line width=1.5pt,forget plot]
  table[row sep=crcr]{40	0\\
40	1\\
};
\node[right, inner sep=0mm, text=black]
at (axis cs:42,0.95,0) {$\text{(x*,}\mu{}_{\text{A'}}\text{(x*))}$};
\end{axis}
\end{tikzpicture}%
		\caption{Fusificador singleton.}
		\label{fig:fuzzifier-singleton}
	\end{subfigure}
	
	\vspace{1 cm}
	\begin{subfigure}[t]{\textwidth}
		\setlength\figureheight{4cm}
		\setlength\figurewidth{12cm}
		% This file was created by matlab2tikz v0.4.7 (commit b040dd9e3d20b4d02324400633a5d15752e58612) running on MATLAB 8.0.
% Copyright (c) 2008--2014, Nico Schlömer <nico.schloemer@gmail.com>
% All rights reserved.
% Minimal pgfplots version: 1.3
% 
\begin{tikzpicture}

\begin{axis}[%
width=\figurewidth,
height=\figureheight,
scale only axis,
xmin=0,
xmax=100,
xtick={0,20,40,60,80,100},
xticklabels={\empty},
xlabel={x},
ymin=0,
ymax=1,
ylabel={$\mu{}_{\text{A'}}\text{(x)}$},
axis x line*=bottom,
axis y line*=left
]
\addplot [color=black,solid,line width=1.5pt,forget plot]
  table[row sep=crcr]{0	0.000335462627902512\\
1	0.000497955421503273\\
2	0.000731802418880473\\
3	0.00106476623666792\\
4	0.00153381067932446\\
5	0.00218749111818289\\
6	0.00308871540823677\\
7	0.00431784000763308\\
8	0.00597602289500594\\
9	0.00818870101437408\\
10	0.0111089965382423\\
11	0.0149207860690678\\
12	0.0198410947443703\\
13	0.0261214098539182\\
14	0.0340474547345993\\
15	0.0439369336234074\\
16	0.0561347628341337\\
17	0.071005353739637\\
18	0.0889216174593863\\
19	0.110250525304485\\
20	0.135335283236613\\
21	0.164474456577155\\
22	0.197898699083615\\
23	0.235746076555864\\
24	0.278037300453194\\
25	0.32465246735835\\
26	0.3753110988514\\
27	0.429557358210739\\
28	0.486752255959972\\
29	0.546074426639709\\
30	0.606530659712633\\
31	0.666976810858474\\
32	0.726149037073691\\
33	0.782704538241868\\
34	0.835270211411272\\
35	0.882496902584595\\
36	0.923116346386636\\
37	0.9559974818331\\
38	0.980198673306755\\
39	0.995012479192682\\
40	1\\
41	0.995012479192682\\
42	0.980198673306755\\
43	0.9559974818331\\
44	0.923116346386636\\
45	0.882496902584595\\
46	0.835270211411272\\
47	0.782704538241868\\
48	0.726149037073691\\
49	0.666976810858474\\
50	0.606530659712633\\
51	0.546074426639709\\
52	0.486752255959972\\
53	0.429557358210739\\
54	0.3753110988514\\
55	0.32465246735835\\
56	0.278037300453194\\
57	0.235746076555864\\
58	0.197898699083615\\
59	0.164474456577155\\
60	0.135335283236613\\
61	0.110250525304485\\
62	0.0889216174593863\\
63	0.071005353739637\\
64	0.0561347628341337\\
65	0.0439369336234074\\
66	0.0340474547345993\\
67	0.0261214098539182\\
68	0.0198410947443703\\
69	0.0149207860690678\\
70	0.0111089965382423\\
71	0.00818870101437408\\
72	0.00597602289500594\\
73	0.00431784000763308\\
74	0.00308871540823677\\
75	0.00218749111818289\\
76	0.00153381067932446\\
77	0.00106476623666792\\
78	0.000731802418880473\\
79	0.000497955421503273\\
80	0.000335462627902512\\
81	0.000223745793720621\\
82	0.000147748360232034\\
83	9.6593413722184e-05\\
84	6.25215037748203e-05\\
85	4.00652973929511e-05\\
86	2.54193465161992e-05\\
87	1.59667838978047e-05\\
88	9.92950430585108e-06\\
89	6.1135679663714e-06\\
90	3.72665317207867e-06\\
91	2.24905596703234e-06\\
92	1.34381227763152e-06\\
93	7.94939361534914e-07\\
94	4.65571571578309e-07\\
95	2.69957850336301e-07\\
96	1.5497531357029e-07\\
97	8.80817919646056e-08\\
98	4.9564053191725e-08\\
99	2.76124245682803e-08\\
100	1.52299797447126e-08\\
};
\addplot [color=blue,line width=1.5pt,mark size=4.0pt,only marks,mark=asterisk,mark options={solid},forget plot]
  table[row sep=crcr]{40	1\\
};
\addplot [color=black,dotted,line width=1.5pt,forget plot]
  table[row sep=crcr]{40	0\\
40	1\\
};
\node[right, inner sep=0mm, text=black]
at (axis cs:44,0.96,0) {$\text{(x*,}\mu{}_{\text{A'}}\text{(x*))}$};
\end{axis}
\end{tikzpicture}%
		\caption{Fusificador gaussiano.}
		\label{fig:fuzzifier-gaussian}
	\end{subfigure}
	
	\vspace{1 cm}
	\begin{subfigure}[t]{\textwidth}
		\setlength\figureheight{4cm}
		\setlength\figurewidth{12cm}
		% This file was created by matlab2tikz v0.4.7 (commit b040dd9e3d20b4d02324400633a5d15752e58612) running on MATLAB 8.0.
% Copyright (c) 2008--2014, Nico Schlömer <nico.schloemer@gmail.com>
% All rights reserved.
% Minimal pgfplots version: 1.3
% 
\begin{tikzpicture}

\begin{axis}[%
width=\figurewidth,
height=\figureheight,
scale only axis,
xmin=0,
xmax=100,
xtick={0,20,40,60,80,100},
xticklabels={\empty},
xlabel={x},
ymin=0,
ymax=1,
ylabel={$\mu{}_{\text{A'}}\text{(x)}$},
axis x line*=bottom,
axis y line*=left
]
\addplot [color=black,solid,line width=1.5pt,forget plot]
  table[row sep=crcr]{0	0\\
1	0\\
2	0\\
3	0\\
4	0\\
5	0\\
6	0\\
7	0\\
8	0\\
9	0\\
10	0\\
11	0\\
12	0\\
13	0\\
14	0\\
15	0\\
16	0\\
17	0\\
18	0\\
19	0\\
20	0\\
21	0.05\\
22	0.1\\
23	0.15\\
24	0.2\\
25	0.25\\
26	0.3\\
27	0.35\\
28	0.4\\
29	0.45\\
30	0.5\\
31	0.55\\
32	0.6\\
33	0.65\\
34	0.7\\
35	0.75\\
36	0.8\\
37	0.85\\
38	0.9\\
39	0.95\\
40	1\\
41	0.95\\
42	0.9\\
43	0.85\\
44	0.8\\
45	0.75\\
46	0.7\\
47	0.65\\
48	0.6\\
49	0.55\\
50	0.5\\
51	0.45\\
52	0.4\\
53	0.35\\
54	0.3\\
55	0.25\\
56	0.2\\
57	0.15\\
58	0.1\\
59	0.05\\
60	0\\
61	0\\
62	0\\
63	0\\
64	0\\
65	0\\
66	0\\
67	0\\
68	0\\
69	0\\
70	0\\
71	0\\
72	0\\
73	0\\
74	0\\
75	0\\
76	0\\
77	0\\
78	0\\
79	0\\
80	0\\
81	0\\
82	0\\
83	0\\
84	0\\
85	0\\
86	0\\
87	0\\
88	0\\
89	0\\
90	0\\
91	0\\
92	0\\
93	0\\
94	0\\
95	0\\
96	0\\
97	0\\
98	0\\
99	0\\
100	0\\
};
\addplot [color=blue,line width=1.5pt,mark size=4.0pt,only marks,mark=asterisk,mark options={solid},forget plot]
  table[row sep=crcr]{40	1\\
};
\addplot [color=black,dotted,line width=1.5pt,forget plot]
  table[row sep=crcr]{40	0\\
40	1\\
};
\node[right, inner sep=0mm, text=black]
at (axis cs:44,0.96,0) {$\text{(x*,}\mu{}_{\text{A'}}\text{(x*))}$};
\end{axis}
\end{tikzpicture}%
		\caption{Fusificador triangular.}
		\label{fig:fuzzifier-triangular}
	\end{subfigure}
		\caption{Representación de las funciones de pertenencia obtenidas con los fusificadores \emph{singleton} (\ref{fig:fuzzifier-singleton}), \emph{gaussiano} (\ref{fig:fuzzifier-gaussian}) y \emph{triangular} (\ref{fig:fuzzifier-triangular}).}
		\label{fig:fuzzifiers}
\end{figure}

\subsection{Defusificadores}\label{sec:defusificadores}
El último paso en el sistema difuso definido en \ref{sec:sistemas-difusos-basados-en-reglas} es la \emph{defusificación}, es decir, transformar el conjunto difuso, obtenido al aplicar las reglas de inferencia, en un valor escalar. Puede decirse que un defusificador realiza la operación complementaria a un fusificador (descritos en la sección \ref{sec:fusificadores}). Así pues, un defusificador puede definirse como \cite{wang1997}:

\begin{definition}
Un \emph{defusificador} es una función que transforma un conjunto difuso $B'$ en $V \subset \mathbb{R}$ (la salida del sistema de inferencia difusa) en un valor escalar $y^* \in V$.
\end{definition}

La misión de un defusificador es obtener el punto $y^*$ en $V$ que mejor representa al conjunto difuso $B'$. En este sentido, los defusificadores cumplen un rol similar a los operadores de agregación (por ejemplo la media aritmética). Se pueden utilizar diversas funciones como defusificadores, aunque generalmente se eligen aquellas que satisfacen las siguientes condiciones \cite{hellendoorn93}:

\begin{itemize}
\item\bfseries Continuidad: \normalfont Un cambio pequeño en la entrada del sistema difuso no debería causar una gran variación en la salida.
\item\bfseries Desambiguación: \normalfont El método de defusificación debe devolver un único valor para $y^*$. Es decir, no debe existir ambigüedad a la hora de seleccionar el valor para $y^*$.
\item\bfseries Plausibilidad: \normalfont El punto $y^*$ debe representar al conjunto difuso $B'$ de forma intuitiva. Es decir, el punto $y^*$ debe estar aproximadamente en el centro del soporte de $B'$ y tener un grado de pertenencia alto en $B'$.
\item\bfseries Simplicidad computacional: \normalfont La función debe ser lo más sencilla posible de calcular. Esto es especialmente importante en controladores difusos que operan en tiempo real.
\item\bfseries Método de ponderado: \normalfont El método seleccionado debe ponderar los diferentes conjuntos de salida. Este criterio depende del ámbito del problema.
\end{itemize}
A continuación se presentan algunos de los métodos de defusificación más utilizados:

\begin{itemize}
  \item\bfseries Centroide: \normalfont También conocido como \emph{centro de gravedad}, el método del \emph{centroide} fue propuesto por Sugeno en 1985 y es uno de los métodos más utilizados \cite{lee1990}. Este método obtiene el valor $y^*$ como el centro del área cubierta por la función de pertenencia de $B'$. El método del centro de gravedad puede definirse como:
  \begin{equation}
	y^* = \frac{\sum y \mu_{B'}(y)}{\sum \mu_{B'}(y)}
  \end{equation}
  Este método tiene la ventaja de que proporciona un valor plausible de forma intuitiva.
  \item\bfseries Bisector: \normalfont El método del bisector obtiene el punto por el que pasa la línea que divide la región delimitada por la función de pertenencia de $B'$ y el eje de abscisas, en dos subsecciones de igual área:
  \begin{equation}
	y^*\text{ tal que: } \sum\limits_{y=y_0}^{y=y^*}\mu_{B'}(y) = \sum\limits_{y=y^*}^{y=y_n}\mu_{B'}(y)
  \end{equation}
  El valor obtenido con este método coincide en ocasiones con el método del \emph{centroide} y también es computacionalmente costoso.
  \item\bfseries Máximo: \normalfont El defusificador del máximo obtiene el punto $y^*$ en el que la función de pertenencia $\mu_{B'}(y)$ alcanza su máximo valor. Es decir:
  \begin{equation}
	y^*\text{ tal que: } \mu_{B'}(y^*) = \max(\mu_{B'}(y_{i}))\qquad ,\forall y_{i} \in V
  \end{equation}
  Este método tiene la ventaja de ser fácil de implementar y poco costoso de calcular. Sin embargo, puede producir resultados ambiguos, puesto que pueden existir varios puntos en los que $\mu_{B'}(y)$ alcanza un valor máximo. En el caso de que existan varios puntos con valores máximos es necesario imponer algún criterio para seleccionar uno de ellos. Si se construye el conjunto:
  \begin{equation}
	hgt(B') = \{y \in V | \mu_{B'}(y) = \sup\limits_{y \in V}\mu_{B'}(y)\}
  \end{equation}
  es decir, el conjunto de todos los puntos en $V$ donde $\mu_{B'}(y)$ alcanza su valor máximo. El defusificador del máximo obtiene un elemento arbitrario $y^*$ del conjunto $hgt(B')$, es decir:
  \begin{equation}
	y^* = \text{ cualquier punto en } hgt(B')
  \end{equation}
  \item\bfseries Menor de máximos (SOM, \emph{smallest of maximum}): \normalfont Dado el conjunto $hgt(B')$ con todos los puntos donde $\mu_{B'}(y)$ alcanza su valor máximo, el defusificador \emph{menor de máximos} obtiene el menor punto $y^*$ del conjunto tal que:
  \begin{equation}
	y^* = \inf\{y \in hgt(B')\}
  \end{equation}
  \item\bfseries Mayor de máximos (LOM, \emph{largest of maximum}): \normalfont Dado el conjunto $hgt(B')$ con todos los puntos donde $\mu_{B'}(y)$ alcanza su valor máximo, el defusificador \emph{mayor de máximos} obtiene el mayor punto $y^*$ del conjunto tal que:
   \begin{equation}
  	y^* = \sup\{y \in hgt(B')\}
    \end{equation}
   \item\bfseries Media de máximos (MOM, \emph{mean of maximum}): \normalfont Dado el conjunto $hgt(B')$ con todos los puntos donde $\mu_{B'}(y)$ alcanza su valor máximo, el defusificador \emph{media de máximos} obtiene el punto medio $y^*$ del conjunto tal que:
   \begin{equation}
     	y^* = \frac{ \inf\{y \in hgt(B')\} + \sup\{y \in hgt(B')\}}{2}
   \end{equation}
\end{itemize}
En la figura \ref{fig:defuzzifiers} se presenta una función de pertenencia y los resultados de aplicar los defusificadores definidos anteriormente. En este caso, los resultados obtenidos con el método del \emph{centroide} y el \emph{bisector} coinciden, aunque no siempre tiene por qué ser así.
\begin{figure}[t]
	\centering
	\setlength\figureheight{5.5cm}
	\setlength\figurewidth{12cm}
	% This file was created by matlab2tikz v0.4.7 (commit e06548f86e222baeac9b1532d97c4b747a61abe5) running on MATLAB 8.0.
% Copyright (c) 2008--2014, Nico Schlömer <nico.schloemer@gmail.com>
% All rights reserved.
% Minimal pgfplots version: 1.3
% 
%
% defining custom colors
\definecolor{mycolor1}{rgb}{0.00000,0.75000,0.75000}%
\definecolor{mycolor2}{rgb}{0.75000,0.00000,0.75000}%
%
\begin{tikzpicture}

\begin{axis}[%
width=\figurewidth,
height=\figureheight,
scale only axis,
xmin=0,
xmax=100,
xtick={0,10,20,30,40,50,60,70,80,90,100},
xticklabels={\empty},
ymin=0,
ymax=1,
axis x line*=bottom,
axis y line*=left,
legend style={draw=black,fill=white,legend cell align=left}
]
\addplot [color=black,solid,line width=1.2pt]
  table[row sep=crcr]{0	0\\
1	0.06\\
2	0.12\\
3	0.18\\
4	0.24\\
5	0.3\\
6	0.36\\
7	0.42\\
8	0.48\\
9	0.54\\
10	0.6\\
11	0.6\\
12	0.6\\
13	0.6\\
14	0.6\\
15	0.6\\
16	0.6\\
17	0.6\\
18	0.6\\
19	0.6\\
20	0.6\\
21	0.6\\
22	0.6\\
23	0.6\\
24	0.6\\
25	0.6\\
26	0.6\\
27	0.6\\
28	0.6\\
29	0.633333333333333\\
30	0.666666666666667\\
31	0.7\\
32	0.733333333333333\\
33	0.766666666666667\\
34	0.8\\
35	0.833333333333333\\
36	0.866666666666667\\
37	0.9\\
38	0.933333333333333\\
39	0.966666666666667\\
40	1\\
41	1\\
42	1\\
43	1\\
44	1\\
45	1\\
46	1\\
47	1\\
48	1\\
49	1\\
50	1\\
51	1\\
52	1\\
53	1\\
54	1\\
55	1\\
56	1\\
57	1\\
58	1\\
59	1\\
60	1\\
61	0.95\\
62	0.9\\
63	0.85\\
64	0.8\\
65	0.75\\
66	0.7\\
67	0.65\\
68	0.6\\
69	0.55\\
70	0.5\\
71	0.45\\
72	0.4\\
73	0.4\\
74	0.4\\
75	0.4\\
76	0.4\\
77	0.4\\
78	0.4\\
79	0.4\\
80	0.4\\
81	0.4\\
82	0.4\\
83	0.4\\
84	0.4\\
85	0.4\\
86	0.4\\
87	0.4\\
88	0.4\\
89	0.4\\
90	0.4\\
91	0.36\\
92	0.32\\
93	0.28\\
94	0.24\\
95	0.2\\
96	0.16\\
97	0.12\\
98	0.08\\
99	0.04\\
100	0\\
};
\addlegendentry{$\mu\text{(x)}$};

\addplot [color=blue,line width=1.2pt,mark size=4.5pt,only marks,mark=asterisk,mark options={solid}]
  table[row sep=crcr]{47	1\\
};
\addlegendentry{centroid};

\addplot [color=black!50!green,line width=1.2pt,mark size=4.5pt,only marks,mark=+,mark options={solid}]
  table[row sep=crcr]{47	1\\
};
\addlegendentry{bisector};

\addplot [color=red,line width=1.2pt,mark size=3.2pt,only marks,mark=square,mark options={solid}]
  table[row sep=crcr]{50	1\\
};
\addlegendentry{mom};

\addplot [color=mycolor1,line width=1.2pt,mark size=3.0pt,only marks,mark=triangle,mark options={solid,rotate=180}]
  table[row sep=crcr]{40	1\\
};
\addlegendentry{som};

\addplot [color=mycolor2,line width=1.2pt,mark size=3.0pt,only marks,mark=triangle,mark options={solid}]
  table[row sep=crcr]{60	1\\
};
\addlegendentry{lom};

\end{axis}
\end{tikzpicture}%
	\caption{Resultado de aplicar los defusificadores \emph{centroide} (\textasteriskcentered), \emph{bisector}(+), \emph{media de máximos} ($\square$), \emph{menor de máximos}($\triangledown$) y \emph{mayor de máximos}($\vartriangle$).}
	\label{fig:defuzzifiers}
\end{figure}

\section{El controlador de Mamdani}\label{sec:mamdani-controller}
Uno de los métodos de inferencia difusa más utilizado es el llamado \emph{controlador de Mamdani} \cite{Mamdani1975}, propuesto por Mamdani y Assilian en 1975 para realizar el control de un motor de vapor a partir de un conjunto de reglas obtenidas de operadores humanos experimentados.

El método de Mamdani utiliza reglas difusas de la forma:

\begin{equation}
R_{i}: \text{IF }\chi_{1}\text{ es }A_{i1}\text{, }\chi_{2}\text{ es }A_{i2}\text{ , \ldots , }\chi_{n}\text{ es }A_{im}\text{ THEN } y \text{ es } B_i
\end{equation}

donde $\chi_1,\ldots,\chi_m$ son variables lingüísticas cuyos valores son conjuntos difusos $A_{ij} \subset FS(U_j)$ con $j\in\{1,\ldots,m\}$. El consecuente $y$ es también una variable lingüística sobre $Y$ ($B_i \subset FS(Y)$). Las entradas del sistema son valores escalares $x_1 \in U_1,\ldots,x_m \in U_m$. 

Cada regla se evalúa por separado y se obtiene el grado de pertenencia de cada entrada al conjunto difuso de su correspondiente antecedente, utilizando el operador mínimo para realizar la conjunción AND de los antecedentes ($k_i = \min(\mu_{Ai1}(x_1),\ldots,\mu_{Aim}(x_m))$). A este valor $k_i$ se le denomina normalmente \emph{grado de activación} de la regla. El resultado de evaluar cada regla es un conjunto difuso sobre $Y$ ($B_i'$), que se calcula ``truncando'' el consecuente con el grado de activación ($B_i' = \{(y,\min(B_i(y),k_i))|y \in Y\}$). En el caso de múltiples reglas, el conjunto difuso de salida se calcula combinando los conjuntos difusos obtenidos en cada regla utilizando el máximo ($B'(y) = \max\limits_{i=1}^{n}(B_i')$).

\begin{algorithm}
\caption{Método de Mamdani}
\label{algo:mamdani}
\DontPrintSemicolon
\KwIn{Un conjunto de reglas $R_{i}$ ($i \in \{1,\ldots,n\}$) con varios antecedentes $A_{ij}$ ($j \in \{1,\ldots,m\}$)  y entradas escalares $x_1,\ldots,x_m$.}
\KwOut{\emph{B'}.}
\For{$i \in \{1,\ldots,n\}$} {
Calcular $k_i = \min(\mu_{Ai1}(x_1),\ldots,\mu_{Aim}(x_m))$ \\
Calcular $B_i' = \{(y,\min(B_i(y),k_i))|y \in Y\}$
}
Construir $B' = \{(y, B'(y))|y \in Y\}$ dado por: \\
\centering
\nonl $B'(y) = \max\limits_{i=1}^{n}(B_i')$.\\
\Return{$B'$}\;
\end{algorithm}

La principal ventaja del método de Mamdani es que es relativamente sencillo de implementar y computacionalmente eficiente. El método de Larsen es similar, pero utilizando el producto en vez del mínimo como operador de implicación \cite{larsen1980}.

\section{Método de interpolación basado en índices de solapamiento}\label{sec:interpolation-method}
En esta sección se introduce un nuevo método para resolver sistemas basados en reglas que generaliza el método clásico de interpolación , utilizando para ello índices de solapamiento \cite{bustince2013overlap}.\\
\\
Históricamente el método más utilizado para resolver sistemas basados en reglas, es decir, para calcular el consecuente \emph{B'}, era el método de interpolación \cite{klir1987}. En este método se utiliza la consistencia de Zadeh \emph{$O_{Z}$} \cite{zadeh1978}, dada en la ecuación \ref{eq:zadeh-consistency}. Los pasos a seguir son:

\begin{algorithm}
\DontPrintSemicolon
\KwIn{Un conjunto de reglas $R_{j}$, con $j \in \{1,\ldots,n\}$, un hecho \emph{A'} y el índice de consistencia $O_{Z}$ (ec. \ref{eq:zadeh-consistency}).}
\KwOut{\emph{B'}.}
\vspace{0.4 cm}
\For{$j \in \{1,\ldots,n\}$} {
Calcular $O_{Z}(A',A_{j}) = \max\limits_{x \in X}(\min(A'(x),A_{j}(x))) $
}
Construir $B' = \{(y, B'(y))|y \in Y\}$ dado por: \\
\centering
\nonl $B'(y) = \max\limits_{j=1}^{n}(\min(B_{j}(y),O_{Z}(A', A_{j})))$.\\
\Return{$B'$}\;
\caption{Método de interpolación}
\label{algo:interpolation-method}
\end{algorithm}
Dado que el índice de consistencia $O_{Z}$ es un índice de solapamiento que cumple la definición \ref{def:dubois-overlap-index}, el algoritmo \ref{algo:interpolation-method} se puede generalizar para utilizar cualquier índice de solapamiento:

\begin{algorithm}
\DontPrintSemicolon
\KwIn{Un conjunto de reglas $R_{j}$, con $j \in \{1,\ldots,n\}$ y un hecho \emph{A'}.}
\KwOut{\emph{B'}.}
\vspace{0.4 cm}
Seleccionar un operador de agregación $M_{1}$, una función de solapamiento $G_{O}$ y un índice de solapamiento O.\\
\For{$j \in \{1,\ldots,n\}$} {
Calcular $O(A', A_{j})$
}
Construir $B' = \{(y, B'(y))|y \in Y\}$ dado por: \\
\centering
\nonl $B'(y) = \overset{n}{\underset{j=1}{M_{1}}}(G_{O}(B_{j}(y),O(A', A_{j})))$.\\
\Return{$B'$}\;
\caption{Método de interpolación generalizado}
\label{algo:overlap-interpolation-method}
\end{algorithm}

Si en el algoritmo \ref{algo:overlap-interpolation-method} utilizamos $M_{1} = \max$, $G_{O} = \min$ y $O = O_{Z}$ entonces se recupera el algoritmo \ref{algo:interpolation-method}.

En el caso de que cada regla tenga varios antecedentes, se puede generalizar el algoritmo \ref{algo:overlap-interpolation-method} de la siguiente manera:

\begin{algorithm}
\DontPrintSemicolon
\KwIn{Un conjunto de reglas $R_{j}$ con varios antecedentes, con $j \in \{1,\ldots,n\}$ y un hecho \emph{A'}.}
\KwOut{\emph{B'}.}
\vspace{0.4 cm}
Seleccionar un operador de agregación $M$, una t-norma \emph{T} y un índice de solapamiento \emph{O}.\\
\For{$i =1 \to n$} {
Calcular $O(A'_{i}, A_{i1}),\ldots,O(A'_{m}, A_{im})$\\
Calcular $k_{i} = T(O(A'_{i}, A_{i1}),\ldots,O(A'_{m}, A_{im}))$\\
Construir sobre el universo de referencia \emph{Y} el conjunto $K_{i} = \{(y,k_{i})|y \in Y\}$
}
Construir $B' = \{(y, B'(y))|y \in Y\}$ dado por: \\
\centering
\nonl $B'(y) = \overset{n}{\underset{i=1}{M}}(\min(K_{i},B_{i}))$.\\
\Return{$B'$}\;
\caption{Método de interpolación generalizado para reglas con varios antecedentes}
\label{algo:multi-overlap-interpolation-method}
\end{algorithm}

